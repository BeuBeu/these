\documentclass[a4paper, 12pt]{memoir}
\usepackage{a4wide}
\usepackage[francais]{babel}
\usepackage[latin1]{inputenc}
\usepackage{amssymb, amsfonts, amsmath}
\usepackage{hyperref}
\usepackage{memhfixc}

\pagestyle{empty}




\begin{document}
\title{Le Coq aux tomates d'arbres}
\author{Beno�t Boyer}
\date{\today}
\maketitle

\frontmatter
\tableofcontents

\mainmatter
\chapter{Introduction}
\section{Contexte, Motivations}

\section{R�sum� des travaux existants}

\section{Pr�sentation des Contributions de cette th�se}


\chapter{Pr�requis}

\section{La r��criture}

\section{Les automates d'arbres}

\section{Mod�le Checking}
\subsection{LTL}
\subsection{Automates de B�chi}

\section{Coq}

\section{La compl�tion d'automates d'arbres}
\subsection{D�finition}
\subsection{Algorithme g�n�ral}
\chapter{Preuves de propri�t�s temporelles sur TRS}

\section{Principe}% de l'approche}

\section{D�finitions}
\subsection{D�finition et S�mantique des $R$-Automates}
\subsection{Relation $\dashrightarrow_R$}

\section{Extension de la compl�tion}
%Construction d'un R-Automate
\section{D�finition de la R-LTL}

\section{V�rification de propri�t�s R-LTL sur $\dashrightarrow_R$}
\subsection{De l'automate � la structure de Kripke}

\subsection{Cas Exact}

\subsection{Quid du cas Approch�}

\section{V�rification de propri�t�s R-LTL sur $\rightarrow_R$}


\chapter{D�tections de contre-exemples et raffinement}
\section{Motivation}
\section{Principe de l'approche � la CEGAR}
\section{}
\section{}



\chapter{Certification de la compl�tion d'automates d'arbres}


\section{Introduction g�n�rale � la certification de programme}

\section{La propri�t� � v�rifier et l'approche retenue}

\section{Les automates d'arbres et la r��criture en Coq}

\section{L'automate est un post-point fixe}

\section{L'inclusion efficace d'automates}


\section{La cl�ture}
\subsection{Sans epsilon transition}

\subsection{Avec epsilon transition?}


\section{Performance, Exemples}

\chapter{Conclusions}
\section{Bilan sur les travaux}

\section{Perspectives}

% \backmatter
\appendix
\chapter{Annexes}
\end{document}



%%% Local Variables: 
%%% mode: latex
%%% TeX-master: t
%%% End: 
