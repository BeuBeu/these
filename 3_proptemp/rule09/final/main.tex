\documentclass[submission, copyright]{eptcs}
%\newcommand{\volume}{118}
%\newcommand{\anno}{2009}
\newcommand{\firstpage}{1}
%\usepackage{breakurl}


\usepackage{graphicx}
\usepackage[latin1]{inputenc}
\usepackage{amscd, amsfonts, amssymb, amsmath, amsthm}
\usepackage[curve]{xypic}
\usepackage{tikz}
%%%%%%%%%%%%%%%%%%%%%%%%%%%%%%%%%%%%%%%%%%%%%%%%%%%%%%%%%%%%%
%%Ces macros sont toutes utilisees par le document
%%%%%%%%%%%%%%%%%%%%%%%%%%%%%%%%%%%%%%%%%%%%%%%%%%%%%%%%%%%%%



% merging of states 
% la commande est deja definie dans le package xypic 
% utilise pour la figure de la paire critique...
% je l'ai remplace de maniere temporaire par \kw{merge}

% \newcommand{\merge}{Merge}
% suggestion de remplacement ?
% \newcommand{\merges}{Merge}


%Insertion de commentaires dans le PDF
%\newcommand{\comments}[1]{\paragraph{\bf Commentaire~: }{\color{blue}#1\newline}}

%%Mots clefs :
\newcommand{\kw}[1]{\mathtt{#1}}
\newcommand{\parts}[1]{\mathcal{P}(#1)}

\newcommand{\bydef}{\stackrel{\triangle}{=}}
\newcommand{\mapswith}[1]{\stackrel{#1}{\longrightarrow}}

\newcommand{\coq}{\textsf{Coq}}
\newcommand{\timbuk}{\textsf{Timbuk}}
\newcommand{\ocaml}{\textsf{OCaml}}
\newcommand{\tom}{\textsf{Tom}}
\newcommand{\scala}{\textsf{Scala}}


%%% A VOIR SI JE GARDE OU PAS
\def \N {\mathbb{N}}
%\def \norm {Norm}
\newcommand{\merge}{Merge}
\newcommand{\E}{\mathcal{E}}
\newcommand{\nr}{N}
\newcommand{\ddom}{\mathcal{D}om}
\newcommand{\dom}{\mathcal{D}om}

%Model Checking

\newcommand{\K}{\mathit K}
\newcommand{\M}{\mathit M}
\newcommand{\trans}{\Delta_\varepsilon^{\leftarrow}}
\newcommand{\Ap}{\mathcal A_p}
\newcommand{\validates}{\models}
\newcommand{\nxt}{\mathbf X}
\newcommand{\fut}{\mathbf F}
\newcommand{\gbl}{\mathbf G}
\newcommand{\unt}{\mathbf U}
\newcommand{\rel}{\mathbf R}


%Maths
\newcommand{\suchthat}{\textrm{ s.t. }}
\newcommand{\Nat}{\mathbb N}
\newcommand{\Natmod}{\Nat[\ ]}
\newfont{\amstoto}{msbm10}
%\newcommand{\NN}{\mbox{\amstoto\char'116}}
\newcommand{\NN}{\N^*}
\newcommand{\ZZ}{\mbox{\amstoto\char'132}}

%Logic
\newcommand{\equ}{\; \Longleftrightarrow \;}
%\newcommand{\validates}{\models}

%Shortcuts
\newcommand{\tn}{T}
\newcommand{\la}{\langle}
\newcommand{\ra}{\rangle}
\newcommand{\eps}{\varepsilon}

\newcommand{\iem}{\text{ième}}
\newcommand{\er}{\text{er}}
\newcommand{\et}{\mbox{ et }}
\newcommand{\st}{\mbox{ t.q. }}
\newcommand{\oute}{\mbox{ ou }}
\newcommand{\ou}{\; \vee \;}

\newcommand{\spf}{\;\; \Longrightarrow \;\;}
\newcommand{\dbf}{\;\; \Longleftrightarrow \;\;}
\newcommand{\imp}{\; \Longrightarrow \;}
\newcommand{\rimp}{\; \Longleftarrow \;}
\newcommand{\notimp}{\; \mathrel{\not\hspace*{-1mm}\Longrightarrow} \;}
\newcommand{\sep}{\; | \;}
\newcommand{\dsep}{\; || \;}


%Term Rewriting Systems
%
\newcommand{\Q}{\mathit Q}
\newcommand{\Qf}{\Q_F}
\newcommand{\arity}{ar}
%\newcommand{\Sub}{{\cal S}}
\newcommand{\F}{\mathcal{F}}
%\newcommand{\Y}{{\cal Y}}
%\newcommand{\C}{{\cal C}}
\newcommand{\D}{{\cal D}}
\newcommand{\TF}{\mathcal{T(F)}}
\newcommand{\TFX}{\mathcal{T(F, X)}}
\newcommand{\TFQ}{\mathcal{ T(F \cup \Q)}}
\newcommand{\TFQp}{\mathcal{ T(F \cup \Q')}}
\newcommand{\TFQX}{\mathcal{ T(F \cup \Q, X)}}
\newcommand{\TFXQ}{\mathcal{ T(F, X \times \Q)}}
\newcommand{\TC}{\mathcal{ T(C)}}
\newcommand{\aut}{\langle \F, \Q, \Qf, \Delta \rangle} 
%\newcommand{\B}{\mathcal{ B}}
\newcommand{\ordlexico}{\prec}
\newcommand{\bottom}{\perp}
%\newcommand{\match}{\unlhd}

\newcommand{\var}{\mathcal{ V}ar}
\newcommand{\pos}{\mathcal{ P}os}
\newcommand{\R}{\mathcal R}
\newcommand{\RE}{{\R_{/E}}}
\newcommand{\Rep}{\R ep}
\newcommand{\desc}{\R^*}
\newcommand{\descE}{\RE^*}
\newcommand{\T}{\mathcal T}
\newcommand{\X}{\mathcal X}
\newcommand{\vars}{\mathcal V}
\newcommand{\rw}{\rightarrow}
\newcommand{\trw}{\rightarrow^{\lambda}}
\newcommand{\lrw}{\longrightarrow}
\newcommand{\xrw}{\xrightarrow}
\newcommand{\rwegal}{\rw^=}

\newcommand{\nrw}{\nrightarrow}
\newcommand{\arw}{\dashrightarrow}
%\newcommand{\rwne}[1]{\rw^*_{#1}}
\newcommand{\rwne}{\rw^{\not\varepsilon}}
\newcommand{\rwned}{\rwne{\Delta}}
\newcommand{\rweq}{\rw^=}
%\newcommand{\rwtag}[1]{\stackrel{#1}{\rw}}
\newcommand{\rwc}{\twoheadrightarrow}
%\newcommand{\rwneq}{\rw^{\not =}}
%\newcommand{\rwe}{\f^\varepsilon}
\newcommand{\rwmod}{\rw_{\R/E}}


%substitutions
\newcommand{\plus}{\sqcup}
\newcommand{\bigplus}{\bigsqcup}
\newcommand{\id}{id}

%completion
\newcommand{\match}{\unlhd}
\newcommand{\matchi}{\lhd}
\newcommand{\matchb}{\dot{\unlhd}}
\newcommand{\matchbi}{\dot{\lhd}}
\newcommand{\Deps}{\upvarepsilon}
\newcommand{\Drw}{\upvarepsilon_{\R}}
\newcommand{\Deq}{\upvarepsilon_{=}}
%\newcommand{\Deps}{\upvarepsilon}
%\newcommand{\norm}{\downarrow}
\newcommand{\norm}{\kw{Norm}}
\newcommand{\slice}{\kw{Slice}}
\newcommand{\compl}{\kw{C}}
\newcommand{\widen}{\kw{W}}
\newcommand{\prune}{\kw{P}}

\newcommand{\simp}{\leadsto}
                          
%Tree Automata
\newcommand{\A}{\mathit A}
\newcommand{\B}{\mathit B}
\newcommand{\C}{\mathit C}
%\newcommand{\F}{\mathcal F}
\newcommand{\Pred}{\mathcal P}
\newcommand{\sub}{\Subset}
\newcommand{\Lang}{\mathcal{L}}
\newcommand{\f}{\rw}
\newcommand{\aaex}{{\mathit A}_{\R}}
\newcommand{\aaexeq}{{\mathit A}_{\R,E}}
\newcommand{\aapprox}{\A^*_{\R,E}}
\newcommand{\automaton}[3]{\la #1, #2, #3 \ra}


\newcommand{\rwA}{\rw_\A}
\newcommand{\rwR}{\rw_\R}
\newcommand{\rwB}{\rw_\B}


%\newcommand{\none}{\kw{none}}
\newcommand{\true}{\mathit{true}}
\newcommand{\false}{\mathit{false}}
\newcommand{\ifte}{\mathit{if}}

% \newcommand{\completion}{\kw{next}}
% \newcommand{\refinedCompletion}{\kw{completion}}
% \newcommand{\equations}{\kw{applyEquations }}
% \newcommand{\refinement}{\kw{refinement}}
% \newcommand{\update}{\kw{updateAndClean}}
% \newcommand{\automatonCleaning}{\kw{ automatonCleaning}}

% \newcommand{\badApproximations}{\kw{badApprox }}
% \newcommand{\badTerms}{\kw{Bad }}
% \newcommand{\moduloTerms}{\kw{moduloTerms }}
% \newcommand{\transToDelete}{\kw{transToDelete }}
% \newcommand{\reachableTerms}{\kw{reachableTerms }}
% \newcommand{\reachableStates}{\kw{reachableStates }}
% \newcommand{\states}{\kw{statesIn }}




%\newcommand{\sun}{\textsc{Sun Microsystem\ }}
%\newcommand{\java}{\textsc{Java\ }}
%\newcommand{\midp}{\textsc{Java MIDP\ }}
%\newcommand{\lande}{\textsc{Lande\ }}
%\newcommand{\danger}{\textsc{\textbf{Danger !}}\normalsize}
%\newcommand{\coq}{{\textit Coq}}

%%%%%%%%%%%%%%%%%%%%%%%%%%%%%%%%%%%%%%%%%%%%%%%%%%%%%%%%%%%%%%%%%%%%
\newcommand{\reff}[1]{[\textsc{Fig}. \ref{#1}]}
\newcommand{\refs}[1]{[sec. \ref{#1}]}
\newcommand{\refa}[1]{[cf. \ref{#1} p. \pageref{#1}]}

%Macros pour construire une regle d'inference small-step semantics
\newcommand{\instr}[1]{instruction_P(m, pc) = \kw{#1}}
\newcommand{\hs}[1]{\langle #1 \rangle}
\newcommand{\infer}[3]
{\dfrac{\instr{#1}}{\hs{#2}::sf \to_{#1}\hs{#3}::sf}}

% #2 = side-conditions : 
\newcommand{\inferi}[5]
{\dfrac{\instr{#1} \quad\ #2 }{#3::sf, \rho \to_{#1} #4::sf, #5}}
%%%%%%%%%%%%%%%%%%%%%%%%%%%%%%%%%%%%%%%%%%%%%%%%%%%%%%%%%%%%%%%%%%%%%

%\newtheorem{property}{Property}
%\newenvironment{property}{\theoremlike{Property}}{\par\medskip}
%\newtheorem{algorithm}[subsection]{Algorithm}
%\newtheorem{example}{Example}
%\newenvironment{example}{\theoremlike{Example}}{\par\medskip}

\newcounter{savetheorem}

%%% Local Variables: 
%%% coding: utf-8
%%% mode: latex
%%% TeX-master: "main"
%%% TeX-PDF-mode: t
%%% ispell-local-dictionary: "french"
%%% End: 

\usetikzlibrary{automata}

\newtheorem{theorem}{Theorem}
\newtheorem{lemma}{Lemma}
\newtheorem{definition}{Definition}
\newtheorem{property}{Property}
\newtheorem{example}{Example}

\newcommand{\titre}{Verifying Temporal Regular properties of
  Abstractions of Term Rewriting Systems}

\title{\titre}
 
\author{Beno�t Boyer
\institute{Universit� Rennes 1, France}
\email{Benoit.Boyer@irisa.fr}
\and
Thomas Genet
\institute{Universit� Rennes 1, France}
\email{Thomas.Genet@irisa.fr}
}
\def\titlerunning{\titre}
\def\authorrunning{Boyer \& Genet}
\begin{document}
\maketitle

\begin{abstract}
  The tree automaton completion is an algorithm used for proving safety
  properties of systems that can be modeled by a term rewriting system. This
  representation and verification technique works well for proving properties of
  infinite systems like cryptographic protocols or more recently on Java
  Bytecode programs. This algorithm computes a tree automaton which represents
  a (regular) over approximation of the set of reachable terms by rewriting
  initial terms. This approach is limited by the lack of information about
  rewriting relation between terms. Actually, terms in relation by rewriting are
  in the same equivalence class: there are recognized by the same state in the
  tree automaton.

  Our objective is to produce an automaton embedding an 
  abstraction of the rewriting relation sufficient to prove temporal
  properties of the term rewriting system.

  We propose to extend the algorithm to produce an automaton having more
  equivalence classes to distinguish a term or a subterm from its successors w.r.t. rewriting. 
  While ground transitions are used to recognize equivalence classes of terms,
  $\epsilon$-transitions represent the rewriting relation between terms.
  From the completed automaton, it is possible to automatically build a
  Kripke structure abstracting the rewriting sequence.
  States of the Kripke structure are states of the tree automaton and the
  transition relation is given by the set of $\epsilon$-transitions.
  States of the Kripke structure are labelled by the set of terms recognized
  using ground transitions. On this Kripke structure, we define the Regular
  Linear Temporal Logic (R-LTL) for expressing properties. Such properties can then 
  be checked using standard model checking algorithms. The only
  difference between LTL and R-LTL is that predicates are replaced by
  regular sets of acceptable terms.  
\end{abstract}

\section{Introduction}
% But : Verifier des proprietes temporelles sur des programmes Java
% Comment : en etendant la completion d'automates arbres pour
% prendre en compte la relation de reecriture, afin d'extraire 
% du r�sultat une structure de Kripke avec laquelle on peut
% faire du mod�le checking standard.

%\subsection{Motivations}

Our main objective is to formally verify programs or systems modeled
using Term Rewriting Systems.  In a previous
work~\cite{BoichutGJL-RTA07}, we have shown that it is possible to
translate a Java bytecode program into a Term Rewriting System (TRS).
In this case, terms model Java Virtual Machine (JVM) states and the
execution of bytecode instructions is represented by rewriting,
according to the small-step semantics of Java. An interesting point of
this approach is the possibility to classify rewriting rules. More
precisely, there is a strong relation between the position of
rewriting in a term and the semantics of the executed transition on
the corresponding state. For the case of Java bytecode, since a term
represents a JVM state, rewriting at the top-most position corresponds
to manipulations of the call stack, i.e. it simulates a method call or
method return.  On the other hand, since the left-most subterm
represents the execution context of the current method (so called
frame), rewriting at this position simulates the execution of the code
of {\em this} method. Hence, by focusing on rewriting at a particular
position, it is possible to analyse a Java program at the method call
level (inter procedural control flow) or at the instruction level
(local control flow).
The contribution of this paper is dual. First, we propose an abstract rewriting
relation to characterize the rewriting paths at a particular depth in terms.
Second, we propose an algorithm which builds a tree automaton recognizing this
relation between terms. Thus, it is possible for instance to build
a tree automaton recognizing the graph of method calls by abstracting the
rewriting relation for the top-most position of JVM terms.

The verification technique used in~\cite{BoichutGJL-RTA07}, called Tree Automata
Completion~\cite{FeuilladeGVTT-JAR04}, is able to finitely over-approximate the
set of reachable terms, i.e. the set of all reachable states of the
JVM. However, this technique lacks precision in the sense that it makes no
difference between all those reachable terms. Due to the approximation
algorithm, all reachable terms are considered as equivalent and the execution
ordering is lost. In particular, this prevents to prove temporal properties of such models. 
However, using approximations makes it possible to prove unreachability
properties of infinite state systems.

In this preliminary work, we propose to improve the Tree Automata Completion
method so as to prove temporal properties of a TRS representing a finite state
system. The first step is to refine the algorithm so as to produce a tree
automaton keeping an approximation of the rewriting relation between
terms. Then, in a second step, we propose a way to check LTL-like formulas on
this tree automaton.


\section{Preliminaries}

Comprehensive surveys can be found in~\cite{BaaderN-book98} for
rewriting, and in~\cite{TATA,GilleronTison-FI95} for tree automata
and tree language theory.

Let $\F$ be a finite set of symbols, each associated with an arity function, and
let $\X$ be a countable set of variables. $\TFX$ denotes the set of terms, and
$\TF$ denotes the set of ground terms (terms without variables). The set of
variables of a term $t$ is denoted by $\var(t)$. A substitution is a function
$\sigma$ from $\X$ into $\TFX$, which can be uniquely extended to an
endomorphism of $\TFX$. A position $p$ for a term $t$ is a word over $\NN$. The
empty sequence $\lambda$ denotes the top-most position. The set $\pos(t)$ of
positions of a term $t$ is inductively defined by:
\begin{itemize}
\item $\pos(t)= \{ \lambda\} $ if $t \in \X$
\item $\pos(f(t_1,\dots,t_n)) = \{ \lambda \} \cup \{i.p \mid 1 \leq i \leq n
  \et p \in \pos(t_i) \}$
\end{itemize}
If $p \in \pos(t)$, then $t|_p$ denotes the subterm of $t$ at position $p$ and
$t[s]_p$ denotes the term obtained by replacement of the subterm $t|_p$ at
position $p$ by the term $s$. A term rewriting system (TRS) $\R$ is a set of {\em
  rewrite rules} $l \rw r$, where $l, r \in \TFX$, $l \not \in \X$, and $\var(l)
\supseteq \var(r)$.
% A rewrite rule $l \rw r$ is {\em left-linear} if each
% variable of $l$ (resp. $r$) occurs only once in $l$.  A TRS $\R$ is left-linear
% if every rewrite rule $l \rw r$ of $\R$ is left-linear).
The TRS $\R$ induces a rewriting relation $\rw_{\R}$ on terms as follows. Let
$s, t\in \TFX$ and $l \rw r \in \R$, $s \rw^p_{\R} t$ denotes that there exists a
position $p\in\pos(t)$ and a substitution $\sigma$ such that $s|_p= l\sigma$ and
$r=s[r\sigma]_p$. Note that the rewriting position $p$ can generally be omitted,
i.e. we generally write $s \rw_{\R} t$. The reflexive transitive closure of
$\rw_{\R}$ is denoted by $\rw^*_{\R}$. The set 
of $\R$-descendants of a set of ground terms $E$ is $\desc(E) = \{t
\in \TF \sep \exists s \in E \st s \rw^*_{\R} t \}$.

% La aussi y a des choses a enlever....
The {\em verification technique} defined
in~\cite{Genet-RTA98,FeuilladeGVTT-JAR04} is based on the approximation of $\desc(E)$.
Note that $\desc(E)$ is possibly infinite: $\R$ may not terminate
and/or $E$ may be infinite. The set $\desc(E)$ is generally not
computable~\cite{GilleronTison-FI95}. However, it is possible to
over-approximate it~\cite{Genet-RTA98,FeuilladeGVTT-JAR04,Takai-RTA04}
using tree automata, i.e. a finite representation of infinite
(regular) sets of terms.  In this verification setting, the TRS $\R$
represents the system to verify, sets of terms $E$ and $Bad$ respectively 
represent the set of initial configurations and the set of ``bad''
configurations that should not be reached. Using tree automata
completion, we construct a tree automaton $\B$ whose language
$\Lang{}(\B)$ is such that $\Lang{}(\B) \supseteq \desc(E)$. If
$\Lang{}(\B)\cap Bad = \emptyset$ then this proves that $\desc(E)\cap
Bad=\emptyset$, and thus that none of the ``bad'' configurations is
reachable. We now define tree automata.

Let $\Q$ be a finite set of symbols, with arity $0$, called {\em states} such
that $\Q \cap \F= \emptyset$.  $\TFQ$ is called the set of {\em configurations}.
\begin{definition}[Transition, normalized transition, $\varepsilon$-transition]
  \label{def:normalized}
  A {\em transition} is a rewrite rule $c \f q$, where $c$ is a
  configuration i.e. $c \in \TFQ$ and $q \in \Q$. A {\em normalized
    transition} is a transition $c \f q$ where $c = f(q_1, \ldots,
  q_n)$, $f \in \F$ whose arity is $n$, and $q_1, \ldots, q_n \in \Q$.
  An {\em $\varepsilon$-transition} is a transition of the form  $q \f q'$ where $q$ and $q'$ are states. 
  % Any set of transition $\Delta \cup \{q \f q'\}$ can be
  % equivalently replaced by $\Delta \cup \{c \f q' \sep c \f q \in \Delta \}$.
\end{definition}

\begin{definition}[Bottom-up nondeterministic finite tree automaton]
  A bottom-up nondeterministic finite tree automaton (tree automaton for short)
  is a quadruple $\A= \langle \F, \Q, \Q_F,\Delta \cup \Deps \rangle$, where $\Q_F
  \subseteq \Q$, $\Delta$ is a set of normalized transitions
  and $\Deps$ is a set of $\varepsilon$-transitions.
\end{definition}

The {\em rewriting relation} on $\TFQ$ induced by the transitions of $\A$ (the
set $\Delta \cup \Deps$) is denoted by $\f_{\Delta\cup\Deps}$.  When $\Delta$ is
clear from the context, $\f_{\Delta\cup\Deps}$ will also be denoted by
$\f_{\A}$. We also introduce $\rwne_\A$ the \emph{transitive relation} which is induced by the set
$\Delta$ alone.

% Here is the definition of the recognized language, see~\cite{BoyerGJ-RR08} for examples.
% Similarly, by notation abuse, we will often note $q \in \A$ and $t\f q \in \A$
% respectively for $q \in \Q$ and $t \f q \in \Delta$.

\begin{definition}[Recognized language, canonical term]
  The tree language recognized by $\A$ in a state $q$ is $\Lang{}(\A,q) = \{t \in \TF \sep t \f^*_{\A} q \}$.
  The language recognized by $\A$ is $\Lang{}(\A) = \bigcup_{q \in \Q_F} \Lang{}(\A, q)$. A tree language is regular if
  and only if it can be recognized by a tree automaton.
  A term $t$ is a {\em canonical term} of the state $q$, if $t \rwne_\A q$.
\end{definition}


\begin{example}
   Let $\A$ be the tree automaton $\langle \F, \Q, \Q_F, \Delta \rangle$ such
   that $\F=\{f,g,a\}$, $\Q= \{q_0, q_1, q_2\}$, $\Q_F=\{q_0\}$,  $\Delta= \{f(q_0)
   \rw q_0, g(q_1) \rw q_0, a \rw q_1, b \rw q_2 \}$ and
   $\Delta_{\epsilon}=\{q_2 \rw q_1 \}$. In $\Delta$, transitions are {\em
     normalized}. A transition of the form $f(g(q_1)) \f q_0$ is not
   normalized. The term $g(a)$ is a term of $\TFQ$ (and of $\TF$) and can be
   rewritten by $\Delta$ in the following way: $g(a) \rwne_\A g(q_1)
   \rwne_\A q_0$. Hence $g(a)$ is a canonical term of $q_1$. Note also that
   $b \rw_\A q_2 \rw_\A q_1$. Hence, $\Lang{}(\A, q_1)=
   \{a, b\}$ and $\Lang{}(\A)=\Lang{}(\A, 
   q_0) = \{g(a), g(b),f(g(a)), f(f(g(b))),\ldots\}=\{f^*(g([a|b]))\}$.
   % Note also that $\Lang{}(\A, q_2)=\emptyset$ since no term of $\TF$ rewrites to
%   % $q_2$, hence $q_2$ is a dead state.
 \end{example}


%% Model Checking. framework utilis� : automate de Buchi, LTL 


\section{The Tree Automata Completion with $\varepsilon$-transitions}


% Section Rappel du principe de la completion...
\section{The Tree Regular Model Checking Problem}
\label{sec:completion}

Our objective is to verify properties of a given system. We will focus
on models of systems whose set of reachable states may be, for
modeling reasons, infinite\,\cite{WB98} -- but our solution also works
for huge finite-state systems. Our first problem is to provide a
symbolic representation to represent and manipulate possibly infinite
sets of states. The problem is undecidable and only partial solutions
exist. Here, we will use {\em Tree Regular Model Checking
  (TRMC)}\,\cite{ALRd05}.
\noindent
In TRMC, a program is a tuple $(\F, I,
Rel)$, where

\begin{itemize}
\item
$\F$ is an alphabet on which a set of terms $\TF$ can be defined;

\item
$I$ is a set of initial configurations represented by a
tree automaton $\A$, i.e. $\Lang(\A)=I$;

\item $Rel$ is a transition relation represented by a set of
  left-linear rewriting rules $\R$.
\end{itemize}

\noindent
In our setting, a program will thus be represented by the tuple $(\F,\A,\R)$.
It has been shown that the above framework can be used to represent a
wide range of applications going from cryptographic protocols to JAVA
applications. We consider reachability problems.

\begin{definition}[Reachability problem]
\label{def:reachability}
Consider a program $(\F,\A,\R)$ and $Bad$ a set of forbidden terms
$Bad$. The reachability problem consists in checking whether there
exists terms of $\desc(\Lang(\A))$ that belong to $Bad$.
\end{definition}

% We will consider the verification of {\em reachability properties},
% i.e., properties that are represented by sets of states. We say that a
% program satisfies a reachability property $\phi$ if all of its
% reachable state is included in the set of states that represent
% $\phi$. Assuming the existence of a tree automaton for representing
% $\phi$, verifying reachability properties reduces to solve the {\em
%   reachability problem}.

% \begin{definition}
% \label{def:reachability}
%   Given a programme $(\F, {\phi}_I, R)$, we consider the reachability
%   problem that consists in computing a tree automaton $\aaex^*$ for
%   the set $R^*(\phi_I)$.
% \end{definition}



\noindent
For finite-state systems, computing the set of reachable terms
($\desc(\Lang(\A))$) reduces to enumerate the terms that can be
reached from the initial set of configurations. Unfortunately, for
infinite-state systems, this enumeration may never terminate.  There
is thus also a need to ``accelerate'' the search through the state
space in order to reach, in a finite amount of time, states at
unbounded depths. Among the existing algorithms used to compute a tree
automaton representing the set of reachable terms of a system, one
finds {\em completion algorithm}. A completion algorithm is a
semi-algorithm that computes an automaton $\aaex^*$ that is possibly
an over-approximation of the set of reachable terms. In the rest of
this section we introduce the principle of completion and point its
current limits.

We say that a tree automaton $\B$ is $\R$-closed if for all terms
$s,t$ such that $s\f_\R$ and $s$ is recognized by $\B$ into state $q$ then so is
$t$. The situation is represented with the following graph.
\begin{tabular}{lc}
  \hspace{-.3cm}
  \begin{minipage}{.8\linewidth}
    It is easy to see that $\Lang(\B) \supseteq \desc(\Lang(\A))$ if
    $\B$ is $\R$-closed and $\Lang(\B)\supseteq
    \Lang(\A)$~\cite{BoyerGJ-IJCAR08}. From an algorithmic point of
    view, building a $\R$-closed $\aaex^*$ from $\A$ consists in {\em
      completing} $\A$ with new transitions.  The completion algorithm
    computes successive automata $\aaex^1,\aaex^2,\ldots$ that
    represent the effect of applying the set of rewriting rules to the
    initial automaton.
  \end{minipage}
  &
  \begin{minipage}{.2\linewidth}
    $
    \xymatrix{
      s \ar[r]_{\R}\ar[d]^{*}_{\B} & t \ar@/^1.2pc/[ld]_{*}^{\B}\\
      q & %\ar[l]^{\A_{i+1}} q'
    }
    $
  \end{minipage}
\end{tabular}

Each application of $\R$ is called a {\em completion step} and
consists in searching for {\em critical pairs} $\langle t,q \rangle$ where the above
diagram is not closed, i.e. $ s\rwR t$, $s \rw_{\A}^* q$ and $t
\not\rw_{\A}^* q$.
% remis car il faut introduire la notation \aaex^i, utilisée dans la suite.
The idea being that the algorithm solves the critical pair by
constructing from $\A$, a new tree automaton $\aaex^1$ with the
additional transitions needed to obtain $t \rw_{\aaex^1}^* q$,
representing the effect of applying $\R$. Then a similar algorithm is
applied on $\aaex^1$ to obtain $\aaex^2$, and so on until a fixpoint
$\aaex^*$ is reached.

As the language recognized by $A$ may be infinite, it is not possible to find
all the critical pairs by enumerating the terms that it recognizes. The solution
that was promoted in \cite{Genet-RTA98} consists in applying sets of
substitutions $\sigma: \X \mapsto \Q$ mapping variables of rewrite rules to
states representing infinite sets of (recognized) terms. Given a tree automaton
$\aaex^i$ and a rewrite rule $l \rw r \in \R$, to find all the critical pairs of
$l \rw r$ on $\aaex^i$, completion uses a {\em matching
  algorithm}~\cite{FeuilladeGVTT-JAR04} that produces the set of substitutions
$\sigma: \X \mapsto \Q$ and states $q\in \Q$ such that $l \sigma \rw_{\aaex^i}^*
q$ and $r\sigma \not\rw_{\aaex^i}^* q$. Solving critical pairs thus consists in
adding new transitions: $r\sigma \rw q'$ and $q' \rw q$. Those transitions may
have to be normalized to respect the definition of transitions of tree
automata. As it was shown in~\cite{Genet-RTA98}, this operation may add not only
new transitions but also new states to the automaton. In the rest of the paper,
the completion-step operation will be represented by $\compl$, i.e., the
automaton obtained by applying the completion step to $\aaex^i$ is denoted
$\compl(\aaex^i)$.

The problem is that, except for particular
classes~\cite{FeuilladeGVTT-JAR04,Genet-Habil}, the automaton
representing the set of reachable terms cannot be obtained from $A$ by
applying a finite number of completion steps and the process thus
needs to be accelerated. For doing so, one can uses an approximation
technique based on a set of equations $E$ and produce an
over-approximation of the set of reachable terms, i.e., a tree
automaton $\aapprox$ such that $\Lang(\aapprox) \supseteq
\desc(\Lang(\A))$.

To produce such an automaton, each automaton $\aaex^i$ obtained by
applying $i$ completion steps to $A$ is approximated using a
widening function $\widen$ parametrized by $E$. An equation $u=v$ is
applied to a tree automaton $\A$ as follows: for all substitution
$\sigma:\X \mapsto \Q$ and distinct states $q_1$ and $q_2$ such that
$u \sigma \rw_{\A}^* q_1$ and $v\sigma \rw_{\A}^* q_2$, states $q_1$
and $q_2$ are merged.  Completion and widening steps can be
linked, i.e. $\aaexeq^0=\A$ and $\aaexeq^{i+1}= \widen(\compl(\aaexeq^i))$, until a
$\R$-closed fixpoint $\aapprox$ is found.  In~\cite{GenetR-JSC10}, it
has been shown that, under some assumptions, the obtained automaton
recognizes no more that terms reachable by rewriting with $\R$ modulo
$E$. As a result, the approximation framework and methodology is close
to equational abstractions of~\cite{MeseguerPM-TCS08}.

% The completion technique has successfully been used for the
% verification of cryptographic
% protocols~\cite{GenetK-CADE00,GenetTTVTT-wits03}.


% Furthermore, the
% tree automata completion implementation, \timbuk, is used as a
% verification backend in AVISPA~\cite{BoichutHKO-AVIS04,avispa}. More
% recently, tree automata completion has been used for fast prototyping
% of static analyzers for Java bytecode
% programs~\cite{BoichutGJL-RTA07}. In this settings, $\R$ encodes the
% system ({\em e.g.} protocol and intruder behavior or Java virtual
% machine semantics), $\aapprox$ represents the over-approximation of
% all states of the system. Then, 


%%the verification consists in showing
%%that the intersection between $\aapprox$ and an automaton $$

\begin{example}
\label{ex:comp}
Let $\R=\{f(x) \rw f(s(s(x)))\}$, $E=\{s(s(x))=s(x)\}$, $\A=\langle \F, \Q, \Q_F,
\Delta\rangle$ be a tree automaton such that $\Q_F=\{q_0\}$ and $\Delta=\{ a \rw
q_1, f(q_1) \rw q_0\}$, i.e. $\Lang(\A)=\{f(a)\}$.
\begin{tabular}{lc}
  \hspace{-.3cm}
  \begin{minipage}{.75\linewidth}
    The first completion step finds the following critical pair: $f(q_1) \rwA^* q_0$
    and $f(s(s(q_1))) \not\rwA^* q_0$. Hence, the completion algorithm produces
    $\aaex^1=\compl(\A)$ having all transitions of $\A$ plus $\{s(q_1) \rw q_2, s(q_2) \rw q_3,
    f(q_3) \rw q_4, q_4 \rw q_0\}$ where $q_2, q_3, q_4$ are new states produced
    by normalization of $f(s(s(q_1))) \rw q_0$. Applying $\widen$ with
    the equation $s(s(x))=s(x)$ on $\aaex^1$ merges the states $q_3$ and $q_2$.
  \end{minipage}&
  \begin{minipage}{.25\linewidth}
    $\xymatrix{
      s(s(q_1)) \ar@{=}[r]\ar[d]_{\aaex^1}^{*} & s(q_1) \ar[d]_{*}^{\aaex^1}\\
      q_3 & q_2
    }
    $
  \end{minipage}
\end{tabular}




  %$s(s(q_1))
  %\rw^*_{\aaex^1} q_3$ and $s(q_1) \rw^*_{\aaex^1} q_2$. 

In~\cite{GenetR-JSC10}, $\A^1_{\R,E}=\widen(\aaex^1)$ is built from $\aaex^1$ by renaming $q_3$ by
$q_2$. The set of transitions of $\A^1_{\R,E}$ is thus $\Delta \cup \{s(q_1) \rw
q_2, s(q_2) \rw q_2, f(q_2) \rw q_4, q_4 \rw q_0\}$.  Completion stops on
$\A^1_{\R,E}$ because it is $\R$-closed, thus $\aapprox=\A^1_{\R,E}$.
Now, let us assume that $Bad=\{f(s(a)), f(s(s(a)))\}$. The first term is not in
$\desc(\Lang(\A))$ but the second is. However, those two terms are 
%However, $f(s(a))$ is not in $\desc(\Lang(\A))$ but $f(s(s(a)))$ is.Those two terms are
recognized by $\aapprox$ and there is no way to distinguish between
the two: no way to detect that the second is {\em really} reachable
nor to automatically refine the abstraction so as to reject the first
one.
\end{example}


If the intersection between $\aapprox$ and $Bad$ is not empty, then it
does not necessarily mean that the system does not satisfy the
property. There is thus the need for techniques to decide whether a
counter-example is indeed a reachable term that does not satisfy the
property or if it is a term added by the abstraction and that cannot
be reached from the set of initial states. If the latter case occurs,
one has to propose a refinement technique that will remove the
false-positive from the abstraction. Studying such techniques for
completion automata is the main objective of this paper.

%%The rest of the paper is organized as follows. In Section
%%\ref{sec:re-automaton}, we introduce $\RE$-automaton that is an
%%extended tree automata. Section \ref{sec:re-automaton} proposes a
%%completion algorithm for $\RE$-automata, while Section \ref{} shows
%%how the extended structure can be used to refine in an efficient
%%manner.




% Tree automata completion~\cite{genet-RTA98,FeuilladeGVTT-JAR04,GenetR-JSC10} is
% an algorithm for computing sets of reachable terms $\desc(I)$, given a regular
% language of initial terms $I=\Lang(\A)$. For most of the known classes of TRS
% $\R$ for which $\desc(\Lang(\A))$ is regular, the output of completion is a tree automaton
% $\aaex^*$ such that $\Lang(\aaex^*)=\desc(\Lang(\A))$. When
% $\desc(\Lang(\A))$ is not regular, it is possible to parameterize the algorithm
% by a set of equations $E$, and to compute a tree automaton $\aapprox$ 
% over-approximating reachable terms,
% i.e. $\Lang(\aapprox) \supseteq \desc(\Lang(\A))$. 



%%% Local Variables: 
%%% mode: latex
%%% TeX-PDF-mode: t
%%% TeX-master: "main"
%%% End: 


\begin{example}
  \label{the_example}
  To illustrate this result, we give a completed tree automaton for a small
  TRS. We define $\R$ as the union of the two sets of rules
  $\R_1 = \{ a \rw b,\; b\rw c \}$ and $\R_2 = \{f(c) \rw g(a),\; g(c) \rw h(a),\; h(c) \rw f(a)\}$. We define
  initial set $E=\{f(a)\}$.
  We obtain the following tree automaton fixpoint :
  {\small
  \[\aaex^*= \left\la
  \Qf = \{q_f\},\quad
  \Delta = \left\{
    \begin{array}{rcl}
      a & \rw & q_a \\
      b & \rw & q_b \\
      c & \rw & q_c \\
      f(q_a) & \rw & q_f \\
      g(q_a) & \rw & q_g \\
      h(q_a) & \rw & q_h \\
    \end{array}
  \right\}\:
  \Deps= \left\{
    \begin{array}{rcl}
      q_b & \rw & q_a \\
      q_c & \rw & q_b \\
      q_g & \rw & q_f \\
  %    q_f & \rw & q_g \\
      q_h & \rw & q_g \\
      q_f & \rw & q_h \\
    \end{array}\right\}
  \:
\right\ra
  \]
  }
  
  If we consider the transition $q_h \rw q_g$, and its canonical terms $h(a)$
  and $g(a)$ respectively, we can deduce $g(a) \arw_\R h(a)$. This is obviously an
  abstraction since we have $g(a) \rw_\R^1 g(b) \rw_\R^1 g(c) \rw_\R^\lambda h(a)$.
\end{example}



In the following, we use the notation $\arw_{\R_i}$ to specify the
relation for a relevant subset $\R_i$ of $\R$. For instance, $u
\arw_{\R_i} v$ 
denotes that there exists $w$ such that $u \rw_\R^* w$ with no rewriting
at the $\lambda$ position of $u$ and $w
\rw_{\R_i}^\lambda v$. In example~\ref{the_example},
we can say that $g(a)
\arw_{\R_2} h(a)$.


\section{From Tree Automaton to Kripke Structure}
Let $\aaex^*= \la \T(\F), \Q, \Q_F, \Delta \cup \Deps \ra$ be a complete tree
automaton, for a given TRS $\R$ and an initial language recognized by $\A$. A
Kripke structure is a four tuple $\K = (S, S_0, R, L)$ where $S$ is a set of
states, $S_0 \subseteq S$ initial states, $R \subseteq S \times S$ a left-total
transition relation and $L$ a function that labels each state with a set of
predicates which are true in that state. In our case, the set of true predicates
is a regular set of terms.

\begin{definition}[Labelling Function]
  Let $\A_P = \la \T(\F), \Q, \Delta \ra$ be the structure defined from
  $\aaex^*$ by removing $\varepsilon$-transitions and final states.
  % We knowingly omit the set of final states.
  %since it is not important.
  We define the labelling function $L : q \mapsto \la \TF, \Q, \{q\}, \Delta\ra$ as the function which associates
  to a state $q$ the automaton $\A_P$ where $q$ is the unique final state. We obviously have
  the property for all state state $q$ :
  \[\forall t \in \Lang{}(L(q)), \quad t \rwne_{\aaex^*} q\]
\end{definition}
Now, we can build the Kripke structure for the subset $\R_i$ of $\R$ on which
we want to prove some temporal properties.


\begin{definition}[Construction of a Kripke Structure]
  We build the 4-tuple $(S, S_0, R, L)$ from a tree automaton such
  that we have $S = Q$, $S_0 \subseteq S$ is a set of initial states,
  $R(q, q')$ if $q' \rw q \in \Deps$ and the labelling function $L$ as
  just defined previously.
\end{definition}

Kripke structures must have a complete relation $R$. For any state $q$
whose have no successor by $R$, we had a loop such that $R(q, q)$ holds. Note
that this is a classical transformation of Kripke structures~\cite{ClarkeGP}.
A Kripke structure is parametrized by the set $S_0$. It defines which connected 
component of $R$ we are interested to analyze. For instance, to analyze 
the abstract rewriting at the top position of terms in $\Lang{}(\aaex^*)$, we define
set $S_0 = \Q_F$ (the set of final states of $\aaex^*$), since all canonical
terms of final states are initial terms. 
For all abstract rewriting at a deeper position $p$, we need to define 
a set $Sub$ of initial subterms considered as the beginning of the rewriting
at the position $p$. Then the set $S_0$ will be defined as 
$S_0 = \{q \sep \exists t \in Sub,\; t \rwne_{\aaex^*} q\}$.
 

Kripke structure models exactly the abstract rewriting
relation $\arw_{\R_i}^*$ for the corresponding subset $\R_i \subseteq \R$.

\begin{theorem}
  Le be $\K=(S, S_0, R, L)$ a Kripke structure built from $\aaex^*$.
  For any states $s$, $s'$ such that $R(s, s')$ holds, there exists two
  terms $u \in L(s)$ and $v \in L(s')$ such that $u \arw_{\R_i} v$.
\end{theorem}

\begin{proof}
  Here, the proof is quite trivial. It is a consequence of the theorem \ref{thm:correct} which can be
  applied on the relation $R$ of the Kripke structure.
\end{proof}

In Example~\ref{the_example}, if we want to verify properties of $\R_1$
or $\R_2$, we need to consider a different subset of $\Deps$ corresponding
to the abstraction of the relation rewriting $\arw_{\R_i}$.
%%%%% The subsets are quite simple 
Figures~\ref{fig2}~and~\ref{fig3} show the Kripke structures corresponding to those
abstractions. Note that in figure~\ref{fig2}, a loop is needed on state $c$ to have a total relation
for $\K_1$.

\begin{figure}[!ht]
  \begin{minipage}{0.5\linewidth}
    \centering
    \begin{tikzpicture}[thick, initial text=]
      \tikzstyle{every node}=[font=\tiny]
      \tikzstyle{every state}=[minimum size=.8cm]
      \tikzstyle{accepting}=[accepting by double]
      \node [initial,state] (a) at (0, 0) {$q_a$}; 
      \node [state] (b) at (2, 0) {$q_b$};
      \node [state] (c) at (4, 0) {$q_c$};
      \node [] (f) at (0, 1.8) {};
      \draw[->] (a) edge (b) (b) edge (c) (c) [loop above] edge (c);
    \end{tikzpicture}
    % \includegraphics[scale=0.4]{R1}
    \caption{\label{fig2}\footnotesize Kripke structure $\K_1$ for $\arw_{\R_1}$}
  \end{minipage}
  \begin{minipage}{0.5\linewidth}
    \begin{center}
      \begin{tikzpicture}[thick, initial text=]
        \tikzstyle{every node}=[font=\tiny]
        \tikzstyle{every state}=[minimum size=.1cm]
        \tikzstyle{accepting}=[accepting by double]
        \node [initial,state] (a) at (0, 0) {$q_f$}; %{$f(a)$}; 
        \node [state] (b) at (2, 0) {$q_g$}; %{$g(a)$};
        \node [state] (c) at (1, 1.5) {$q_h$}; %{$h(a)$};
        \draw[->] (a) edge (b) (b) edge (c) (c) edge (a);
      \end{tikzpicture}
      % \includegraphics[scale=0.4]{R2}
      \caption{\label{fig3}\footnotesize Kripke structure $\K_2$ for $\arw_{\R_2}$}
    \end{center}
  \end{minipage}
\end{figure}
% \comments{REFAIRE LES SCHEMAS : REMPLACER LES TERMES PAR DES ETATS CORRESPONDANTS
% OU ETATS (TERMES)? Moi je laisserais les termes sur cet exemple car ca permet de
% comprendre mieux ou on va. Quite a mettre les etats dans la suite de l'exemple
% et expliquer qu'ils reconnaissent ces termes.}


The set $S_0$ of initial states depends of the abstract rewriting relation selected.
For example, if we want to analyze $\arw_{\R_2}$ (or $\arw_{\R_1}$), we define $S_0=\{q_f\}$ (resp. 
$S_0 = \{q_a\}$).


\section{Verification of R-LTL properties}
To express our properties, we propose to define the Regular Linear
Temporal Logic (R-LTL). R-LTL is LTL where predicates are replaced by a tree
automaton. The language of such a tree automaton
characterizes a set of admissible terms. A state $q$ of a Kripke
structure validates the atomic property $P$ characterized by a tree automaton $\A_P$
if and only if one term recognized by $L(q)$ must be recognized by $\A_P$ to satisfy the
property. More formally:

\[\K(Q,\ Q_F,\ R,\ L),\ q \models P\quad \equ\quad \Lang{}(L(q)) \cap \Lang{}(\A_P) \neq \emptyset\]

We also add the operators ($\land$, $\lor$, $\neg$, $\nxt$, $\fut$, $\gbl$, $\unt$, $\rel$)
with their standard semantics as in LTL to keep the expressiveness
of the temporal logic. More information about these operators can
be found in~\cite{ClarkeGP}. Note that temporal properties do not range over the 
rewriting relation $\rw_\R$ but over its abstraction $\arw_\R$.
It means that the semantics of the temporal operators has to be interpreted
w.r.t. this specific relation. For example, the formula $\gbl(\{f(a)\} \imp \nxt \{g(a)\})$
on $\K_2$ (for more clarity, we note predicates as sets of terms): the formula 
has to be interpreted as : for all $q$ $q'$, if $\K_2,\ q \models \{f(a)\}$ and $R(q, q')$ then
we have  $\K_2,\ q' \models \{g(a)\}$. In the rewriting interpretation the only term $u$ such
that $f(a) \arw_{\R_2} u$ is $u = g(a)$.

We use the B�chi automata framework to perform model checking. A survey of this
technique can be found in the chapter 9 of~\cite{ClarkeGP}.  LTL (or R-LTL)
formulas and Kripke structures can be translated into B�chi automata. We
construct two B�chi automata : $\B_\K$ obtained from the Kripke structure and
$\B_L$ defined by the LTL formula. Since the set of behaviors of the Kripke
structure is the language of the automaton $\B_\K$, the Kripke structure
satisfies the R-LTL formula if all its behaviors are recognized by the automaton
$\B_L$. It means checking $\Lang{}(\B_\K) \subseteq \Lang{}(\B_L)$. For this
purpose, we construct the automaton $\overline{\B_L}$ that recognizes the
language $\overline{\Lang{}(\B_L)}$ and we check the emptiness of the automaton
$\B_\cap$ that accepts the intersection of languages $\Lang{}(\B_K)$ and
$\overline{\Lang{}(\B_L)}$. If this intersection is empty, the term rewriting
system satisfies the property. This is the standard model-checking technique.

$\B_\M$ and $\B_\K$ are classically defined as 5-tuples: alphabet, states,
initial states, final states and transition relation.
Generally, the alphabet of B�chi automata is a set of predicates.
Since we use here tree automata to define predicates, the alphabet of
$\B_\K$ and $\B_L$ is $\Sigma$ the set of tree automata that can be defined over $\TF$. 
Actually, a set of behaviors is a word
 which describes a sequence of states: if $\pi=s_0s_1s_2s_3\dots$ denotes
a valid sequence of states in the Kripke structure, then the word
$\pi' = L(s_0)L(s_1)L(s_2)\dots$ is recognized by $\B_\K$. The algorithms
used to build $\B_\M$ and $\B_\K$ can be found in~\cite{ClarkeGP}.


The automaton intersection $\B_\cap$ is obtained by computing the product of $\B_\K$ by $\overline{\B_L}$.
By construction all states of $\B_\K$ have to be final. Intuitively any infinite path
over the Kripke structure must be recognized by $\B_\K$. This case allows to use a 
simpler version of the general B�chi automata product.
\begin{definition}[$\B_\K \times \overline{\B_L}$]
  The product of $\B_\K = \la \Sigma,\; \Q,\; Q_i,\; \Delta,\; \Q\ra$ by $\overline{\B_L} = \la\Sigma,\; \Q',\;\Q'_i,\; \Delta',\; F\ra$ is defined as
  \[\la \Sigma,\; \Q \times \Q',\; \Q_i \times \Q'_i,\; \Delta_\times,\;  \Q \times F \ra \]
  where $\Delta_\times$ is the set of transitions $(q_\K, q_L)
  \stackrel{(\A_\K,\A_L)}{\lrw} (q'_\K, q'_L)$ such that $q_\K
  \stackrel{\A_\K}{\lrw} q'_\K$ is a
  transition of $\B_\K$ and $q_L
  \stackrel{\A_L}{\lrw} q'_L$ is a transition of $\overline{\B_L}$. Moreover, the transition 
  is only valid if the intersection between the languages of $\A_\K$ and $\A_L$ is non
  empty as expected by the satisfiability of the R-LTL atomic formula.
  % as expected by the interpretation of the R-LTL atomic formula.
\end{definition}

Finally the emptiness of the language $\Lang{}(\B_\cap)$ can be checked using the standard algorithm
based on depth first search to check if final states are reachable.
\begin{example}
To illustrate the approach, we propose to check the formula $P =
\gbl(\{f(a)\} \imp \nxt \{g(a)\})$
on example~\ref{the_example}. The automaton $\overline{\B_L}$ (fig.~\ref{fig4}) recognizes the negation of the formula $P$
expressed as $\fut(\{f(a)\} \land \nxt \neg\{g(a)\})$ and $\B_\K$ (fig.~\ref{fig5}) recognizes the all behaviors of the Kripke structure
$\K_2$~(fig.~\ref{fig3}). The notation $\A_\alpha$ denotes the tree automaton such that its language
is described by $\alpha$ ($\A_{\neg g(a)}$ recognizes the complement of the language $\Lang{}(\A_{g(a)})$ and $\A_*$ recognizes all
term in $\TF$). Figure~\ref{fig6} shows the result of intersection $\B_\cap$ between $\B_\K$ and $\overline{\B_L}$. Only reachable
states and valid transitions (labeled by non empty tree automata intersection) are showed. Since no 
reachable states of $\B_\cap$ are final, its language is empty. It means that all behaviors of $\K_2$ satisfy $P$ : the
only successor of $f(a)$ for the relation $\arw_{\R_2}$ is $g(a)$.
\begin{figure}[!ht]
  \begin{minipage}{0.5\linewidth}
    \centering
    % \bar{\B_L}
    \begin{tikzpicture}[scale=.8,thick,initial text=]
      \tikzstyle{every node}=[font=\tiny]
      \tikzstyle{every state}=[minimum size=.1cm]
      \tikzstyle{accepting}=[accepting by double]
      % 
      \node [initial,state] (q1) at (0, 0) {$1$}; 
      \node [state] (q2) at (2, 0) {$2$};
      \node [accepting, state] (q3) at (4, 0) {$3$};
      % 
      \path[->]
      (q1)  edge [loop above] node {$\A_*$} (q1) 
      edge node [above] {$\A_{f(a)}$} (q2)
      (q2)  edge node [above] {$\overline{\A_{g(a)}}$} (q3)
      (q3)  edge [loop above] node {$\A_*$} (q3);
    \end{tikzpicture}
    \caption{\footnotesize Automaton $\overline{\B_L}$}
    \label{fig4}

    % \bar{\B_\K}
    \begin{tikzpicture}[scale=.8,thick,initial text=]
      \tikzstyle{every node}=[node distance=40,font=\tiny]
      \tikzstyle{accepting}=[accepting by double]
      \tikzstyle{every state}=[accepting, minimum size=.1cm]
      % 
      \node [initial,state] (q4)            {$4$}; 
      \node [state] (q5) [right of=q4]      {$5$};
      \node [state] (q6) [below of=q5]      {$6$};
      \node [state] (q7) [right of=q6]      {$7$};
      % 
      \path[->]
      (q4)  edge []                         node [above]    {$L(q_f)$} (q5) 
      (q5)  edge []                         node [left]     {$L(q_g)$} (q6)
      (q6)  edge []                         node [below]    {$L(q_g)$} (q7)
      (q7)  edge [bend angle=40,bend right] node [right]    {$L(q_g)$} (q5);
    \end{tikzpicture}
    \caption{\footnotesize Automaton $\B_{\K}$}
    \label{fig5}
  \end{minipage}
  \begin{minipage}{0.5\linewidth}
    \centering
    % B_\cap
    \vspace{10mm}
    \begin{tikzpicture}[scale=.8,thick,initial text=,bend angle=40]
      \tikzstyle{every node}=[node distance=50,font=\tiny]
      \tikzstyle{accepting}=[accepting by double]
      \tikzstyle{every state}=[minimum size=.1cm]
      % 
      \node [initial,state] (q14)            {$1,4$}; 
      \node [state] (q15) [right of=q14, node distance=60]     {$1,5$};
      \node [state] (q16) [below of=q15]     {$1,6$};
      \node [state] (q17) [right of=q16, node distance=60]     {$1,7$};
      \node [state] (q25) [below of=q14]     {$2,5$};
      % 
      \path[->]
      (q14) edge []                 node [above]           {$\A_* \cap L(q_f)$}       (q15)
            edge []                 node [left]            {$\A_{f(a)}\cap L(q_f)$}    (q25)
      (q15) edge []                 node [left]            {$\A_* \cap L(q_g)$}       (q16)
      (q16) edge []                 node [above]           {$\A_* \cap L(q_h)$}       (q17)
      (q17) edge [bend right]       node [right]           {$\A_* \cap L(q_f)$}       (q15)
            edge [bend left]        node [below]           {$\A_{f(a)} \cap L(q_g)$}   (q25);
    \end{tikzpicture}
    \caption{\footnotesize Automaton $\B_\cap$}
    \label{fig6}
  \end{minipage}
\end{figure}

\end{example}

\section{Conclusion, Discussion}

In this paper, we show how to improve the tree automata completion mechanism to
keep the ordering between reachable terms. This ordering was lost in the
original algorithm~\cite{FeuilladeGVTT-JAR04}. Another contribution is the
mechanism making it possible to prove LTL-like temporal properties on such
abstractions of sets of reachable terms. The work presented here only deals with
finite state systems and exact tree automata completion results. Future plans
are to extend this result so as to prove temporal properties on
over-approximations of infinite state systems. A similar objective has already
been tackled in~\cite{MeseguerPM-TCS08}. However, this was done in a pure
rewriting framework where abstractions are more heavily constrained than in tree
automata completion~\cite{FeuilladeGVTT-JAR04}. Hence, by extending LTL formula
checking on tree automata over-approximations, we hope to ease the verification
of temporal formula on infinite state systems.

% In this paper, we show how to improve the tree automata completion
% mechanism to keep the ordering between reachable terms. This ordering is lost in
% the original algorithm. Another contribution is the mechanism making it possible
% to prove LTL-like temporal properties on such abstractions of sets of reachable
% terms. In this paper, we only deal with exact tree automata completion
% results. Future plans are to extend this result so as to prove temporal
% properties on over-approximations. A similar objective has already been tackled
% in~\cite{MeseguerPM-TCS08}. However, this was done in a pure rewriting framework
% where abstractions are more heavily constrained than in tree automata
% completion~\cite{FeuilladeGVTT-JAR04}. Hence, by extending LTL formula checking
% on tree automata over-approximations, we hope to ease the verification of
% temporal formula on infinite state systems.

\section*{Acknowledgements}

Many thanks to Axel Legay and Vlad Rusu for fruitful discussions on this
work and to anonymous referees for their comments.

\bibliographystyle{eptcs} % or whatever you prefer
\bibliography{sabbrev,eureca,genet,mc}


%   In this way the tree automaton computed contains a sufficient
%   abstraction of the rewriting relation to prove temporal
%   properties. By focusing at a position $p$ of any canonical term we
%   can consider this abstraction of the rewriting sequence at the top
%   position of subterms (located at the position $p$). In this case we
%   consider only rewriting relation at the top position between the
%   between canonical subterms at position $p$.

%   This abstraction is particularly interesting for term rewriting systems,
%   where rewriting sequence at a particular position of a term can be
%   seen has a module of the global computation in the analyzed
%   system. For instance in Java Bytecode programs where terms denote
%   states, rewriting at the top position of a state only corresponds to
%   a call to a method, whereas rewriting on the greatest and left-most
%   subterm of a state only corresponds to execution of Bytecode
%   instructions of the current method.



\end{document}
%%% Local Variables: 
%%% mode: latex
%%% TeX-master: t
%%% End: 
