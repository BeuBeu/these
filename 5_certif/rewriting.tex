\section{Formalisation de la réécriture}
\label{sec:rewriting}

Le but de cette partie est de formaliser en \coq: les termes, la
réécriture, les termes atteignables et le problème d'atteignabilité
lui-même.  Premièrement on utilise les entiers binaires positifs
fournis par la librairie standard de \coq\ pour définir les ensembles de
symboles comme les variables ($\X$), les symboles de fonctions
($\F$), ou les ensembles d'états ($\Q$). 
Pour être plus explicite, on renomme les \lstinline!positive!
en \lstinline!ident!. L'ensemble des termes se définit inductivement:

\switchlstcoq
%A discuter....
%Definition F := positive.
%Definition X := positive.
\begin{lstlisting}
Definition ident := positive.

Inductive term : Set :=
| Fun : ident -> list term -> term
| State : ident -> tern
| Var : ident -> term.
\end{lstlisting}

Cet ensemble contient plus de termes que les ensembles $\TFX$, $\TF$ et $\TFQ$.
Pour être exact, il définit tous les termes de l'algèbre $\TFQX$.
Ainsi, on introduit les prédicats~\lstinline!TFX: term -> Prop!, \lstinline!TF: term -> Prop!
et~\lstinline!TFQ: term -> Prop! pour caractériser respectivement chacun des sous-ensembles.

Maintenant, le terme $f(x, a)$ sera construit comme~\lstinline!Fun 0 (Var 0::(Fun 1 nil)::nil)!
en supposant que l'on a la correspondance suivante entre les symboles,
variables et les entiers binaires $f \mapsto 0$, $a \mapsto 1$ et $x
\mapsto 0$ par exemple.  On remarque qu'il est possible d'utiliser 
la valeur $0$ pour dénoter les symboles $f$ et $x$, car les constructeurs
\lstinline!Fun! et \lstinline!Var! du type \lstinline!term! permettent la distinction:
\lstinline!Var 0! pour la variable $x$ et \lstinline!Fun 0 $\dots$! pour un terme de la forme $f(\dots)$.

\paragraph{Remarques:}
Une faiblesse de \coq\ est la génération automatique du principe
d'induction \lstinline!term_rect!. Le théorème généré ne tient pas
compte du fait que, dans le cas du constructeur $Fun$, le terme est constitué d'une
\lstinline!list term! pour former les sous-termes ce qui
nécessite souvent d'utiliser l'hypothèse d'induction:
\lstinline!term_rect! est trop faible pour prouver quoi que ce soit.
Dès lors, il est nécessaire de définir un second théorème nommé \lstinline!term_rect'! pour
construire les preuves qui nécessitent une étude de cas sur \lstinline!term!.
\begin{lstlisting}
Lemma term_rect' : 
   forall (P: term -> Type) (P0: list term -> Type),
      (forall x, P (Var x)) -> (forall q, P (State q)) ->
         (forall p l, P0 l -> P (Fun p l)) ->
         P0 nil ->
         (forall t lt, P t -> P0 lt -> P0 (t::lt)) ->
           forall t, P t.
\end{lstlisting}
Ce principe d'induction repose sur l'imbrication mutuelle de deux prédicats~\lstinline!P! et~\lstinline!P0!
portant respectivement l'un sur les termes et l'autre sur les listes de sous-termes. Le but de ce principe
d'induction est de montrer la conclusion \lstinline!forall t, P t!.
Le prédicat~\lstinline!P0! n'est là que pour fournir une aide sur la manière dont on veut raisonner pour transmettre
(le pas d'induction) la propriété~\lstinline!P! des sous-termes qui constituent~\lstinline!l! vers un terme complet de la forme \lstinline!Fun f l!.
En fait, dans la majorité des cas, il suffit de considérer une instance particulière de ce principe 
où~\lstinline!P0! est instancié par~\lstinline!fun l => forall x, In x l -> P x!:
\begin{lstlisting}
Lemma term_rec_In :
   forall (P: term -> Set),
      (forall x, P (Var x)) -> (forall q, P (State q)) ->
         (forall p l, (forall x, In x l -> P x) -> P (Fun p l)) ->
            forall t, P t.
\end{lstlisting}
Pour construire ce principe d'induction, on a besoin au préalable de montrer la décidabilité de l'égalité
sur les termes. Cette propriété ne peut-être montré qu'avec une instance particulière du principe~\lstinline!term_rect'!
où le prédicat~\lstinline!P0! exprime la décidabilité de l'égalité sur les listes de termes.
De plus, le résultat sur la décision de l'égalité est la propriété indispensable que l'on utilise couramment 
pour comparer les termes. %, dans les preuves où les algorithmes.
\begin{lstlisting}
Theorem term_eq_dec : forall u v: term, {u = v} + {u <> v}.
\end{lstlisting}
% This is required to be able to prove most of the theorems on terms.


Une règle de réécriture $l \rw r$ est représentée par un couple de termes avec une preuve
que la règle est bien formée, \textit{i.e.} un terme de preuve \coq\ qui assure que l'ensemble des variables du membre droit $r$
est un sous-ensemble des variables du membre gauche $l$.
A ce titre, la fonction \lstinline!Fv : term -> list ident! construit l'ensemble des variables d'un terme. 
% In \coq, it becomes:
\begin{lstlisting}
Inductive rule: Set :=
| Rule(l r: term) (Hsub: subseteq (Fv r)(Fv l))
                  (Hl: TFX l) (Hr: TFX r)       : rule.
\end{lstlisting}
D'autre part, la complétion manipule des ensembles de règles linéaires à gauche. On a donc introduit une deuxième définition
qui possède une hypothèse supplémentaire assurant la linéarité du membre gauche. Cette dernière  est assurée par le prédicat
\lstinline!linear: term -> Prop! qui fixe le nombre d'occurrences de chaque variable à $1$ au maximum.
\begin{lstlisting}
Inductive l_rule: Set :=
| Lrule(l r: term) (Hsub: subseteq(Fv r)(Fv l))
                   (Hl: TFX l) (Hr: TFX r)
                   (Hlin: linear l)             : lrule.
\end{lstlisting}

Dans la suite, le type \lstinline!list rule! représente un système de règles de réécriture et donc
\lstinline!list lrule! correspond à un système de règles de réécriture linéaires à gauche.
On utilise la notation \lstinline!(t @ sigma)! pour dénoter le terme résultant de 
l'application d'une substitution \lstinline!sigma! à chaque variable qui apparaît dans le terme~\lstinline!t!. 

\begin{lstlisting}
Definition substitution := ident -> option term.
\end{lstlisting}

\noindent
En \coq, la relation de réécriture \emph{"$u$ est réécrit en $v$ par $l
  \rightarrow r$"}, habituellement définie comme $\exists p \in \pos(t),\exists \sigma \
t.q.\ u|_p = l\sigma \ \land \ v = u [ r\sigma ]_p$, est divisée en deux prédicats:

\begin{itemize}
\item Le premier~\lstinline!TRew! définit la réécriture d'un terme à la racine du terme (à la position $\epsilon$). En
  fait, l'ensemble des couples de termes $(t, t')$ qui sont réécrits à la racine par la règle $l \rw r$
  peuvent être vus comme l'ensemble des termes $(l\sigma, r\sigma)$ pour toute substitution~$\sigma$.

\item 
  Le second prédicat~\lstinline!Rew! définit inductivement la relation de réécriture pour toute position d'un terme $t$
  par la règle $l \rw r$, par la réécriture à la racine de tout sous-terme de $t$ par $l \rw r$.
\end{itemize}


\begin{lstlisting}
(* Topmost rewriting : *)
Inductive TRew (x : rule) : term -> term -> Prop :=
| R_Rew :
     forall s l r Hsub Hl Hr,
        x = Rule l r Hsub Hl Hr -> TRew x (l @ s) (r @ s).
\end{lstlisting}


\begin{lstlisting}
(* Rewriting at any position of any term *)
Inductive Rew (r : rule) : term -> term -> Prop :=
| RewT : forall t t', 
     TRew r t t' -> Rew r t t'
| RewSub : forall f l l', 
     RewTerms r l l' -> Rew r (Fun f l) (Fun f l')
with RewTerms (r : rule) : list term -> list term -> Prop :=
| RewNext: forall t l l',
     RewTerms r l l' -> RewTerms r (t::l) (t::l')
| RewThis: forall l t t',
     Rew r t t' -> RewTerms r (t::l) (t'::l).
\end{lstlisting}


% De la même manière, en utilisant une définition inductive il est possible de 
% définir le \lstinline!Rew! prédicat pour la réécriture à n'importe quelle position.

Cette définition de la relation de réécriture présente la particularité
d'être facilement manipulable. On est tout de même en droit de se demander si cette
formalisation est équivalente à la définition standard. Une manière simple de répondre
consiste à montrer l'équivalence des deux définitions. On introduit donc la définition standard~\lstinline!StdRew!
et on montre l'équivalence:

\begin{lstlisting}
Definition StdRew (x: rule)(u v: term) := 
   forall l r Hsub Hl Hr, Rule l r Hsub Hl Hr = x ->
      exists p: position, exists s: substition, 
         get u p = Some (l @ s) /\ v = set u p (r @ s).

Theorem StdRew_Rew:
   forall (x: rule)(u v: term), StdRew lr u v <-> Rew lr u v.
\end{lstlisting}

La définition standard utilise les fonctions~\lstinline!get! et~\lstinline!set!
permettant respectivement d'accéder ou de remplacer le sous-terme à la postion $p$:
\lstinline!get u p = Some t! est équivalent à $u|_p = u$ et \lstinline!set u p t! équivaut à $u[t]_p$.
On remarquera qu'il n'est pas nécessaire de préciser que la position $p$ appartient à $\pos(u)$.
La fonction~\lstinline!get! est une fonction partielle (elle renvoie une valeur de type~\lstinline!option term!)
dont l'évaluation est la forme~\lstinline!Some t! si et seulement si $p \in pos(u)$:
\begin{lstlisting}
Lemma Pos_get: forall t p, Pos t p <-> exists t', get t p = Some t'.
\end{lstlisting}


Ensuite, on doit définir $\rw^*_{\R}$. En \coq, on prefèrera voir cela comme le prédicat
\lstinline{Reachable R u} qui caractérise l'ensemble des termes atteignables à partir de
$u$ par $\rightarrow^*_\R$.

\begin{lstlisting}
Inductive Reachable(R : list rule)(t : term) : term -> Prop:=
| R_refl : Reachable R t t
| R_trans : forall u v r, Reachable R t u -> In r R -> Rew r u v -> 
    Reachable R t v.
\end{lstlisting}

%%% Local Variables: 
%%% mode: latex
%%% TeX-master: "../main"
%%% End: 