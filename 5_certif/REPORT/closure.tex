\section{Formalization of closure by rewriting}
\label{sec:closure}

In this part we aim at defining formally the \lstinline!IsClosed! predicate, the
function \lstinline!closure! and prove the soundness of this function
w.r.t. \lstinline!IsClosed!. Recall that to
check if a tree automaton $\A = \langle \Q_F,\ \Delta \rangle$ is closed
w.r.t. a TRS $\R$, it is enough to prove that for all $t \in \Lang(\A)$, if
$t'$ is reachable from $t$ by $\rw_\R^*$ then $t' \in \Lang(\A)$. 
Now that we have defined in \coq\ rewriting and tree
automata, we can define more formally the \lstinline!IsClosed! predicate and recall
the \lstinline!closure_sound! theorem to prove:

\begin{lstlisting}
Definition IsClosed (R : list rule) (A : t_aut) : Prop :=
   forall q t t', IsRec A.delta q t -> Reachable R t t' -> IsRec A.delta q t'.

Theorem closure_sound:
      forall R A', closure R A' = true -> IsClosed R A'.
\end{lstlisting}


The algorithm to check closure of $\A$ by $\R$ computes for each rule $l
\rightarrow r \in \R$ the full set of the substitutions $\sigma$ s.t. $l\sigma
\rightarrow_\Delta^* q$ and then, checks that $r\sigma \rightarrow_\Delta^*
q$. Then, the correctness proof consists in showing that if \lstinline!closure!
answers true, then $\Lang(\A)$ closed by $\rw_\R$.

We now give some hints to define the \lstinline!closure! function.  First, for
all rule $l \rw r$ of $\R$, this function has to find all the substitutions
$\sigma:\X \mapsto \Q$ and all the states $q \in \Q$ such that 
$l\sigma \rightarrow_\Delta^* q$. This is what we call the 
\emph{matching-problem}. Second, this function has to check that for all the $q$
and $\sigma$ found, we have $r\sigma \rw_{\Delta}^* q$. Third, in the
correctness theorem, we have to show that all the substitutions $\sigma:
\X\mapsto \Q$ cover the set of substitutions on terms, i.e. of the form $\sigma':
\X \mapsto \TF$, and hence cover all reachable terms.

We note $l \unlhd q$ the matching problem consisting in finding all the
substitutions $\sigma:\X \mapsto \Q$ and all the states $q \in \Q$ such that
$l\sigma \rightarrow_\Delta^* q$. An algorithm solving this kind of
problems was defined in~\cite{Genet-RR97b}. It consists in normalizing
the formula $l \unlhd q$ with the following deduction rules:

\[\textrm{(Unfold) }
\dfrac {
  f(s_1, \dots, s_n)\unlhd f(q_1,\dots, q_n)
}{
  s_1 \unlhd q_1 \land \dots \land s_n \unlhd q_n
}\quad \quad
\textrm{(Clash) }
\dfrac {
  f(s_1, \dots, s_n) \unlhd g(q'_1, \dots q'_m)
}{
\bottom}
\]
\[\textrm{(Config) }
\dfrac {
  s \unlhd q
}{
  s \unlhd c_1 \lor \dots \lor s \unlhd c_k \lor \bottom
}
\textrm{if }s \notin \X \textrm{, and } \{c_i\ |\ c_i \rightarrow q \in \Delta\}_1^k. 
\]

Moreover, after each application of one of this rules,
the result is also rewritten into disjunctive normal form.
\versionlongue{ using:
\[
\dfrac{ 
  \phi_1\land (\phi_2 \lor \phi_3)
}{
  (\phi_1 \land \phi_2) \lor (\phi_1 \land \phi_3)
}\quad
\dfrac {\phi_1 \lor \bottom }{\phi_1}\quad
\dfrac {\phi_1 \land \bottom}{\bottom}
\]
}
When normalization of the initial problem is terminated, we obtain a
formula like $ \bigvee_{i=1}^{n} \phi_i$ where
$\phi_i=\bigwedge_{j=1}^{m}x^i_j \unlhd q^i_j$ such that $x^i_j \in
\X$ and $q^i_j \in \Q$. Each $\phi_i$ can be seen as a
substitution $\sigma_i = \{x^i_j \mapsto q^i_j\}$.
\versionlongue{
\begin{theorem}{\cite{Genet-RR97b}}
\label{th:completeness}
  Substitutions $\sigma_i$ obtained by the matching algorithm are the only
  substitutions s.t. $s\sigma_i \rightarrow^*_\Delta q$.
\end{theorem}}
The implementation of the \lstinline!matching! function in \coq\ is very close
to this algorithm. 
\versionlongue{We implement disjunction as lists where $\bottom$ is mapped to
\lstinline!nil!.  The \coq\ signature of the matching function is
\lstinline!Delta.t -> state -> term -> list substitution!.  For this algorithm,
the termination is bound by a syntactic argument.  We can define it easily in
\coq\ by a simple structural recursion over the \lstinline!term! which has to be
matched.}
\versionlongue{
Moreover, the soundness and completeness properties, corresponding to
Theorem~\ref{th:completeness}, can be defined in \coq\ as follows:
}
\versioncourte{Moreover, the soundness and completeness properties of this
algorithm can be defined in \coq\ as follows:
}
\begin{lstlisting}
Theorem matching_sound :
   forall D q l s, In s (matching D q l) -> IsRec D q (l @ s).

Theorem matching_complete :
   forall D q l s, IsRec D q (l @ s) -> In s (matching D q l).
\end{lstlisting}

The second part of the \lstinline!closure! function consists in verifying that
for each substitution $\sigma$ s.t. $l\sigma \rightarrow_\Delta^* q$, we also
have $r\sigma \rightarrow_\Delta^* q$.  This job is performed using the
\lstinline!all_red! function, we define, whose purpose is to check that this
property is true for all the found substitutions. Then, we only need to prove
the soundness of this function using the following \coq\ theorem:

\begin{lstlisting}
Theorem all_red_sound : 
  forall D q r sigmas, 
     all_red D q r sigmas = true -> forall s, In s sigmas -> IsRed D q (r@s).
\end{lstlisting}

By combining the \lstinline!matching! and the \lstinline!all_red! functions, 
we obtain the algorithm for checking up all critical pairs
found at state $q$ and for the rule $l \rightarrow r$. We define the combination
as:

\versionlongue{

\begin{lstlisting}
Definition closure_at_state D q l r := all_red D q r (matching D q l).
\end{lstlisting}}

\begin{lstlisting}
Theorem closure_at_state_sound : 
   forall D q l r, closure_at_state D q l r = true -> 
      (forall s, IsRed D q (l @ s) -> IsRed D q (r @ s)).
\end{lstlisting}

  Given a rule $l \rightarrow r$ and a state $q$, this algorithm answers
  \lstinline!true! if for all substitution $\sigma: \X \mapsto \Q$ s.t. $l
  \sigma \rw^*_\Delta q$ then $r\sigma \rw^*_{\Delta} q$. Now that we have
  proved this result for substitutions $\sigma:\X \mapsto \Q$, we have to prove
  that it implies the same property for substitutions $\sigma':\X \mapsto \TF$,
  this is Lemma~\ref{lem:qtotf}. On the opposite, to prove that every reachable
  term of $\TF$ will be covered by a configuration of $\TFQ$ in $\Delta$, we
  have to prove that if there exists a substitution $\sigma': \X \mapsto \TF$,
  then we can construct a corresponding substitution $\sigma: \X \mapsto \Q$,
  this is Lemma~\ref{lem:tftoq}.
\begin{lemma}
\label{lem:qtotf}
Given a term $u \in \T(\F, \X)$, $\sigma :\X \mapsto \Q$ a substitution
s.t. $u\sigma \rightarrow^*_\Delta q$, if we have a substitution $\sigma' :\X
\mapsto \TF$ s.t. $\forall x \in \ddom(\sigma): \sigma' x \in \Lang(\Delta,
\sigma x)$, then we have $u\sigma' \rw^*_{\Delta} q$ and thus $u\sigma' \in
\Lang(\Delta, q)$.
\end{lemma}

Roughly, if the left or right-hand side $u$ of a rewriting rule matches
a configuration $u\sigma \in \TFQ$ and $u\sigma
\rw^*_{\Delta} q$ then, all terms $u\sigma' \in \TF$, matched by $u$, 
are also reducible into $q$, i.e. $u\sigma' \rw^*_{\Delta} q$ and $u\sigma' \in
\Lang(\Delta,q)$.

\begin{lemma}
\label{lem:tftoq}
Given a term $u \in \T(\F, \X)$, if there exists a substitution $\sigma':\X
\mapsto \TF$ such that $u\sigma' \rw_{\Delta}^* q$, then there exists a
substitution $\sigma:\X\mapsto \Q$ s.t. $\sigma' x \in \Lang(\Delta, \sigma x)$
and $u\sigma\rw^*_{\Delta} q$.
\end{lemma}

\versionlongue{In the same way, for all term $u\sigma' \in \Lang(\Delta, q)$ that can be
matched by $u$, there exists a configuration $u\sigma$ s.t. $u\sigma'
\rightarrow^*_\Delta u \sigma$ and $u\sigma \rightarrow^*_\Delta
q$.}
Using those two lemmas, we can conclude that for all term $t\in\Lang(\Delta,q)$
rewritten in $t'$ at the topmost position by $l \rightarrow r$, then $t' \in
\Lang(\Delta, q)$. This property is easily lifted as a property of the
\lstinline!closure! function for all states of $\Q$ and using all rules of $\R$
at topmost position.
\versioncourte{
Then, it is enough to lift this property to general rewriting at any
position. Finally, the \lstinline!closure_sound! general theorem is 
shown by using a reflexive and transitive application of the last 
property.
}

\begin{lstlisting}
Theorem closure_sound_0 : 
   forall D R, closure D R = true ->
      forall q lr, In q (states D) -> In lr R ->
         forall t t', IsRec D q t -> TRew lr t t' -> IsRec D q t'.
\end{lstlisting}
The next step consists in showing the same property but using rewriting at any
position, hence proving the same theorem with \lstinline!Rew! (general
rewriting) instead of \lstinline!TRew! (topmost rewriting) between
$t$ and $t'$.

\begin{lstlisting}
Theorem closure_sound_1 : 
   forall D R, closure D R = true ->
      forall q lr, In q (states D) -> In lr R ->
         forall t t', IsRec D q t -> Rew lr t t' -> IsRec D q t'.
\end{lstlisting}
Finally, the \lstinline!closure_sound! general theorem is 
shown by using a reflexive and transitive application of this
\lstinline!closure_sound_1! in order to deal with any term $t'$ that can be
reached using the \lstinline!Reachable! predicate.

%%% Local Variables: 
%%% mode: latex
%%% TeX-master: "main"
%%% End: 
