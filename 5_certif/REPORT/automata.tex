\section{Formalization of Tree Automata}
\label{sec:automata}

The fact that the checker, to be executed, is directly extracted from the \coq\
formalization has an important consequence on the tree automata formalization. Since
the data structures used in the formalization are those that are really used for
the execution, they need to be formal {\em and} efficient. For tree automata,
instead of a naive representation, it is thus necessary to use optimized formal
data structures borrowed from~\cite{RivalGL-TPHOL01}.

\switchlstcoq

In Section~\ref{sec:rewriting}, we have represented variables $\X$ and function
symbols $\F$ by the type \lstinline!ident!. We do the same for $\Q$. We define a
tree automaton as a pair $(\Q_F, \Delta)$, where $\Q_F$ is the finite set of
final states, and $\Delta$ the finite set of normalized transitions like $f(q_1,
\dots, q_n) \rightarrow q$. In \coq, $\Q_F$ is a simple \lstinline!list ident!
but $\Delta$ is represented using the \lstinline!FMapPositive! \coq\ library of
functional mappings, where data are indexed by \lstinline!positive! numbers. In
the \lstinline!FMapPositive! structure, every transition $f(q_1, \dots, q_n)
\rightarrow q$ is encoded by a list of states $(q_1, \dots, q_n)$ indexed by $f$
in a map which is also indexed by $q$ in a second map.  This representation is
a good solution to deal efficiently, in \coq, with transition sets.
%
%

\begin{lstlisting}
Module Delta : DELTA.
   (* Transition sets : *)
   Definition config := list state.
   
   Definition t := 
        FMap.t (FMap.t (list config)).
   (* ............... *)
\end{lstlisting}

%
%
Now we can define a predicate to characterize the recognized language of a tree
automaton.  In fact, we are defining the set of ground terms that are reduced to
a state $q$ by a transition set $\Delta$. This set, which corresponds to
$\Lang(\A,q)$ if $\Delta$ is the set of transitions of $\A$, can be
constructed inductively in \coq\ using the single deduction rule:

\begin{prooftree}
  \AxiomC{$t_1 \in \Lang(\Delta, q_1)$}
  \AxiomC{\dots\dots}
  \AxiomC{$t_n \in \Lang(\Delta, q_n)$}
  \RightLabel{If $f(q_1, \dots, q_n) \rightarrow q \in \Delta$}
  \TrinaryInfC{$f(t_1,\dots, t_n) \in \Lang(\Delta, q)$}  
\end{prooftree}

In \coq, we express this statement using the inductive predicate
\lstinline!IsRec!. A term $t$ is recognized by a tree automaton $(\Q_F,
\Delta)$, if the predicate {\tt IsRec}~$\Delta$~$q$~$t$ is valid for $q \in
\Q_F$.

\begin{lstlisting}
Inductive IsRec (D: Delta.t) : state -> term -> Prop :=
  Rec_Term : forall f lt q,
      IsRec' D (Delta.get q f D) lt -> IsRec D q (Fun f lt)

with IsRec' (D: Delta.t) : list config -> list term -> Prop :=
| Rec_SubTerm : forall lt c lc, IsRec'' D c lt -> IsRec' D (c::lc) lt
| Rec_SubTerm' : forall lt c lc, IsRec' D lc lt -> IsRec' D (c::lc) lt
     
with IsRec'' (D: Delta.t) : config -> list term -> Prop :=
| Rec_Nil : IsRec'' D nil nil
| Rec_Cons : forall t q lt lq, IsRec D q t -> IsRec'' D lq lt ->
       IsRec'' D (q::lq) (t::lt).
\end{lstlisting}

%%% Local Variables: 
%%% mode: latex
%%% TeX-master: "main"
%%% End: 
