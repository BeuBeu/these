\section{An optimized inclusion checker}
\label{sec:inclusion}

In this part, we give the formal definition of the \lstinline!Included! property and
of the \lstinline!inclusion! \coq\ function used to effectively check the tree
automata inclusion. From the previous formal definitions on tree automata, we
can state the \lstinline!Included! predicate in the following way:

\begin{lstlisting}
  Definition Included (a b : t_aut) : Prop :=
    forall t q, In q a.qf -> IsRec a.delta q t ->
      exists q', In q' b.qf /\ IsRec b.delta q' t.
\end{lstlisting}


Now let us focus on the function \lstinline!inclusion! itself.  The usual
algorithm for proving inclusion of regular languages recognized by
nondeterministic bottom-up tree automata, for instance for proving $\Lang(\A)
\subseteq \Lang(\B)$, consists in proving that $\Lang(\A) \cap
\Lang(\overline{\B}) =\emptyset$, where $\overline{\B}$ is the complement
automaton for $\B$. However, the algorithm for building $\overline{\B}$ from
$\B$ is EXPTIME-complete~\cite{TATA}. This is the reason why we here define a
criterion with a better practical complexity. It is is based on a simple
syntactic comparison of transition sets, i.e. we check the inclusion of
transition sets modulo the renamings performed by the $\merge$ function.
This increases a lot the efficiency of our checker, especially by
saving memory. This is crucial to check inclusion of big tree automata (see
Section~\ref{sec:benchmarks}). This algorithm is correct but, of course, it is
not complete in general, i.e. not always able to prove that $\Lang(\A) \not
\subseteq \Lang(\B)$. However, we show in the following that, under certain
conditions on $\A$ and $\B$ which are satisfied if $\B$ is obtained by
completion of $\A$, this algorithm is also complete and thus becomes a decision
procedure.  First, we introduce the following notation:
{\small
\begin{center}
  \begin{tabular}[c]{lcp{8cm}}
    $\Gamma$    & : & induction hypothesis set\\
    $\Delta_i$  & : & transition set of the tree automaton $\mathcal{A}_i$\\
    $\{c|c \rightarrow q \in \Delta\}$ & : & configurations of $\Delta$ that are rewritten in $q$\\
    $\{c_i\}_n^m$ & : & configuration set from $c_n$ to $c_m$\\
%    $\emptyset$ & : & empty set of configurations\\
%    $[q_1, \dots, q_n]$ & : & a tuple of $n$ states\\
  \end{tabular}
\end{center}
} We formulate our inclusion problem by formulas of the form: ${\Gamma
  \vdash_{A, B} q \Subset q'}$. Such a statement stands for: under the
assumption $\Gamma$, it is possible to prove that $\Lang(A,q) \subseteq
\Lang(B,q')$.
% A term $t$ is reconnized by state $q$ a tree automaton if it exists
% one sequence of transitions of $\Delta$ that rewrites $t$ into $q$. 
% Thus, ${\Gamma \vdash_{A, B} q \Subset q'}$ states inductively for the assertion:
% if a term $t$ s.t. is rewritten into $q$ by a sequence of transitions of $\Delta_\A$ then
% it can be done by an equivalent sequence in $\Delta_\B$ which rewrites $t$ into $q'$.\\
The algorithm consists in building proof trees for those statements using the
following set of deduction rules.
% This algorithm checks only transition set inclusion (up to
% renaming). Transition set inclusion implies corresponding language
% inclusion (as proved in Section~\ref{sec:correction}). It is more
% reasonable to implement it in \coq.\\
%The algorithm checks only this property, but it is enough to imply 
%corresponding language inclusion. (as proved in Section~\ref{sec:correction}).
%It is more reasonable to implement it in \coq.\\

{\small
\[\textrm{(Induction) }
\dfrac{
  \Gamma \cup \{q \Subset q'\} \vdash_{A, B}
  \{c|c \rightarrow_{\Delta_A} q\} \Subset \{c|c \rightarrow_{\Delta_B} q'\}
}{
  \Gamma \vdash_{A, B} q \Subset q'
}\textrm{  if }(q \Subset q') \notin \Gamma
\]


\[\textrm{(Axiom) }
\dfrac{}{
  \Gamma \cup \{q \Subset q'\} \vdash_{A, B} q \Subset q'
}\quad \quad \quad \quad 
\textrm{(Empty) }
\dfrac{}{
  \Gamma \vdash_{A, B} \emptyset \Subset \{c'_j\}^m_1
}
\]



\[\textrm{(Split-l) }
\dfrac {
  \Gamma \vdash_{A, B} c_1 \Subset \{c'_j\}^m_1 \quad \dots\dots \quad
  \Gamma \vdash_{A, B}  c_n \Subset \{c'_j\}^m_1
}{
  \Gamma \vdash_{A, B} \{c_i\}^n_1 \Subset \{c'_j\}^m_1
}\]

%\[(empty)\]

\[\textrm{(Weak-r) }
\dfrac{
  \Gamma \vdash_{A, B} c \Subset c'_k
}{
  \Gamma \vdash_{A, B} c \Subset \{c'_i\}^n_1
}\textrm{ if }(1 \le k \le n)
\quad\quad\quad \quad
\textrm{(Const.) }
\dfrac{}{
  \Gamma \vdash_{A, B} a() \Subset a()
}
\]



\[\textrm{(Config) }
\dfrac{
  \Gamma \vdash_{A, B} q_1 \Subset q'_1 \quad \dots\dots \quad \Gamma \vdash_{A, B} q_n \Subset q'_n
}{
  \Gamma \vdash_{A, B} f(q_1, \dots, q_n) \Subset f(q'_1, \dots, q'_n)
}\]

%\[const\]
}
\noindent

Given $\Q_{F_\A}$ and $\Q_{F_\B}$ the sets of final states of
$\A$ and $\B$, $\#()$ a symbol of arity $1$ not occurring in $\F$,
to prove $\Lang(\A) \subseteq \Lang(\B)$, we start our deduction by the
statement: $\emptyset \vdash_{\A, \B} \{\#(q)\ |\ q \in \mathcal{Q}_{F_\A}\} \Subset
  \{\#(q)\ |\ q \in \mathcal{Q}_{F_\B}\} $
 

\begin{example}
  Let $\A$ and $\B$ be two automata s.t.:
  {\small
  \[\A = \left\{ 
    \begin{array}{rcl}
      a& \rightarrow &q_1\\
      b& \rightarrow &q_2\\
      f(q_1, q_2)&\rightarrow &{\bf q}\\
    \end{array}\right\}
  \textrm { with } \Q_{F_\A}=\{{\bf q}\}
  \textrm { and }
  \B = \left\{ 
    \begin{array}{rcl}
      a& \rightarrow &{\bf q'}\\
      b& \rightarrow &{\bf q'}\\
      f(q', q')&\rightarrow &{\bf q'}\\
    \end{array}\right\}
  \textrm { with } \Q_{F_\B}=\{{\bf q'}\}
  \]
  }
  Here we have $\Lang(\A) \subseteq \Lang(\B)$ and we can derive
  $\emptyset \vdash_{\A, \B} \#(q) \Subset \#(q')$ with the deduction rules:

  {\tiny
    \begin{prooftree}
      \AxiomC{}
      \LeftLabel{(Const.)}
      \UnaryInfC{$\{q \Subset q', q_1 \Subset q'\} \vdash_{\A, \B} a() \Subset a()$}
      \LeftLabel{(Weark-r)}
      \UnaryInfC{$ \{q \Subset q',\ q_1 \Subset q'\} \vdash_{\A, \B} a() \Subset \{ a(), b(), f(q', q') \}$}
      \LeftLabel{(Induction)}
      \UnaryInfC{$\{q \Subset q'\} \vdash_{\A, \B} q_1 \Subset q'$}
      %%%%%%%%%%%%%%%%%%%%%%%%%%%%% 
      \AxiomC{}
      \LeftLabel{(Const.)}
      \UnaryInfC{$\{q \Subset q', q_2 \Subset q'\} \vdash_{\A, \B} b() \Subset b()$}
      %\RightLabel{(Weark-r)}
      \UnaryInfC{$ \{q \Subset q',\ q_2 \Subset q'\} \vdash_{\A, \B} b() \Subset \{ a(), b(), f(q', q') \}$}
      %\RightLabel{(Induction)}
      \UnaryInfC{$\{q \Subset q'\} \vdash_{\A, \B} q_2 \Subset q'$}
      %%%%%%%%%%%%%%%%%%%%%%%%%%%%%%%%%%%%%%
      \LeftLabel{(Config)}
      \BinaryInfC{$\{q \Subset q'\} \vdash_{\A, \B} f(q_1, q_2) \Subset f(q', q') $ }
      \LeftLabel{(Weark-r)}
      \UnaryInfC{ $\{q \Subset q'\} \vdash_{\A, \B} f(q_1, q_2) \Subset \{ a(), b(), f(q', q')\} $}
      \LeftLabel{(Induction)}
      \UnaryInfC{$\emptyset \vdash_{\A, \B} q \Subset q'$}
      \LeftLabel{(Config)}
      \UnaryInfC{$\emptyset \vdash_{\A, \B} \#(q) \Subset \#(q')$}
    \end{prooftree}
  }
\end{example}

%We define the simple inclusion relation as :
%\begin{definition}
%  Let $\Delta$ and $\Delta'$ be two transition functions.\\
%  We say $\Delta \subseteq \Delta'$ if $\Delta'$ contains all rules of $\Delta$ :
%  \begin{equation}
%    \Delta \subseteq \Delta' \Longleftrightarrow \forall\ (c \rightarrow q) \in \Delta,\ (c \rightarrow q) \in \Delta'
%  \end{equation}
%\end{definition}
%Then we have trivially the property :
%\begin{theorem}
%  Given two transition function $\Delta$ and $\Delta'$,
%  \begin{equation}
%    \forall q,\ \Lang(\Delta, q) \subseteq \Lang(\Delta', q)
%  \end{equation}
%\end{theorem}

The main property we want demonstrate in \coq\
is that this syntactic criterion implies the semantic inclusion for the
considered languages in \ref{sec:automata}.

\begin{lstlisting}
Theorem inclusion_sound :
   forall A B, inclusion A B = true -> Included A B.
\end{lstlisting}


Before proving this in \coq, we need to define more formally the function
\lstinline!inclusion!. This function cannot be defined as a simple structural
recursion.  Thus \coq\ needs a termination
proof for this algorithm. Thanks to the \coq\ feature \lstinline!Function!, it is
possible to define the algorithm using a measure function and provide a proof
that its value decreases at each recursive call to ensure the termination.

\versioncourte{\subsection{Termination, Soundness, Completeness}}
\versionlongue{\subsection{Termination}}
\label{sec:termination}
Termination of deduction rules can be proved by defining a measure function $\mu$ on statements of he form $\Gamma \vdash_{\A,\B} \alpha \Subset \beta$.
%We use the finiteness property of $\Gamma$ that is finite and the size of terms on the right of $\vdash$.
The $\Gamma$ relation can be seen as a subset of $\mathcal{Q}_A \times \mathcal{Q}_B$ which is a finite set. All tree automata have
a finite number of states.
Then the statement measure $\mu({\Gamma \vdash_{A, B} \alpha \unlhd \beta})$ is defined as tuple $(\mu_1(\Gamma), \mu_2(\alpha)+\mu_2(\beta))$
where:
{\small
\[\left[
  \begin{array}{lcl}
    \mu_1(\Gamma) & = & | \Q_A \times \Q_B | - |\Gamma|\\
    \mu_2(x) &=&
    \left\{\begin{array}{l}
        (m+1-n) \textrm{ if } x = \{c_i\}^m_n\\
        1 \textrm{ if } x = f(q_1, \dots, q_n),\\
        0 \textrm{ otherwise}
      \end{array}\right.\\
  \end{array}\right.
\]}

Then we define the ordering $\ll$ by the lexicographic combination of the usual 
order $<$ on natural numbers for $\mu_1$ and $\mu_2$.% and $\mu_3$:
\versionlongue{
\[(x, y) \ll (x', y') \Longleftrightarrow
\bigvee\left\{
\begin{array}{l}
  x < x' \\
  x = x' \land y < y'\\
  %x = x' \land y = y' \land z < z'
\end{array}\right.
\]
}
Since $<$ is well founded, the lexicographic combination $\ll$ is also well
founded.

\begin{theorem}{(Termination)}
  At each deduction step, the measure decreases strictly:
  \[
  \dfrac{
    \Gamma \vdash_{A, B} \alpha \Subset \beta
  }{
    \Gamma' \vdash_{A, B} \alpha' \Subset \beta'
  }
  \Longrightarrow \mu({\Gamma \vdash_{A, B} \alpha \Subset \beta}) \ll \mu({\Gamma' \vdash_{A, B} \alpha' \Subset \beta'})
  \]
\end{theorem}
\versioncourte{
\begin{proof}
See~\cite{BoyerGJ-RR08}, for details.
\end{proof}
}
\annexe{
\begin{proof}
  The following array summarizes for each derivation rule what component of the tuple
  proves that $\mu$ decreases between conclusion and premises of the rule:

  \[\begin{array}[h]{l|c|c|}%c|}
    \footnotesize
    & \mu_1 & \mu_2 \\ \hline % & \mu_3\\ \hline
    \textrm{Induction}   & \strut \mu_1(\Gamma) < \mu_1(\Gamma') & - \\ \hline
    \textrm{Split-l} & \mu_1(\Gamma) = \mu_1(\Gamma') & 
    %\begin{array}{l}
      \mu_2(c_i) = 1 < \mu_2(\{c_i\}_1^n)\\
     % \mu_2((\{c_i\}_2^n) < \mu_2(\{c_i\}_1^n)\\
    %\end{array}\\ \hline

    \textrm{Weak-r} & \mu_1(\Gamma) = \mu_1(\Gamma') & \mu_2(c_k) < \mu_2(\{c_i\}_1^n)\\ \hline
    \textrm{Config} & \mu_1(\Gamma) = \mu_1(\Gamma') & 
    \begin{array}{l}
      \mu_2(f(\dots, q_i, \dots)) = \mu_2(f(\dots, q'_i, \dots)) = 1\\
      \mu_2(q_i) = \mu_2(q'_i) = 0 \textrm{ thus } 2 > 0
    \end{array}\\ \hline
    %\textrm{Tuple} & \mu_1(\Gamma) = \mu_1(\Gamma') & 0 = 0 &
    %\begin{array}{l}
    %  \mu_3([q_1, \dots, q_n]) > \mu_3(q1)\\
    %  \mu_3([q_1, \dots, q_n]) > \mu_3([q_2, \dots, q_n])\\
    %\end{array}\\ \hline
\end{array} \]
For the {\tt Split-l} (resp. {\tt Weak-r}) rule, we consider $n > 1$ to have a set $\alpha = \{c_i\}_1^n$ (resp. $\beta$) with
at least two elements. If ($n = 1$) then this rule does not apply on the current statement
${\Gamma \vdash \alpha \Subset \beta}$.\\

\end{proof}
}
\versionlongue{
\begin{theorem}
  When $\mu(\Gamma \vdash_{A, B} \alpha \Subset \beta) = (0, 0)$, we have a statement as Axiom or Nil:
  the current proof derivation is completed.
\end{theorem}
\begin{proof}
  From  $\mu(\Gamma \vdash_{A, B} \alpha \Subset \beta) = (0, 0)$ we deduce immediately:
  \begin{enumerate}
  \item $\mu_1(\Gamma) = 0$ implies $\Gamma = \Q_A \times \Q_B$ 
  \item $\mu_2(\alpha) = \mu_2(\beta) = 0$ implies $\alpha$ and $\beta$ are both either a state or empty set.
  \end{enumerate}
  Thus the statement may be:
  \begin{itemize}
  \item $\Gamma \vdash_{A, B} \emptyset \Subset \emptyset$ is the case
    of Empty rule : proof derivation is ended.
  \item $\Gamma \vdash_{A, B} q \Subset q'$ : we can use the fact
    $\Gamma = \Q_A \times \Q_B$, thus $(q, q') \in \Q_A\times\Q_B \imp
    (q \Subset q') \in \Gamma$. This case matches with Axiom rule that
    terminates the proof derivation.
  \end{itemize}
\end{proof}
}
\archive{
In \coq, do not need not exactly define all the measure as such. In fact,
a part of the algorithm can be expressed by structural recursion. In
particular, we use \lstinline!list! to represent configurations set as
for configurations which are expressed by \lstinline!list! of
\lstinline!state!. Since list covering is structural recursive, it is
also for configurations set. Then we have implemented this algorithm
as a single big recursive function \lstinline!inclusion_aux : Delta.t -> Delta.t -> gamma -> state -> state -> bool!  where
\lstinline!gamma! stands for a \lstinline!list! of \lstinline!state * state!. For a such function we only need $\mu_1$ and
we have reduce the \coq\ proof termination at one case.}

\versionlongue{\subsection{Soundness}}
\label{sec:correction}

\versioncourte{
\begin{theorem}{(Soundness)}
  \label{thm:soundness}
  For all tree automata $\A$ and $\B$, if there exists $\prod$ a proof tree
  of $\emptyset \vdash_{\A, \B} q \Subset q'$ then we have
  $\Lang(\Delta_\A, q) \subseteq \Lang(\Delta_\B, q')$
\end{theorem}

\begin{proof}
This can be done by an induction on the size of the term of 
$\Lang(\Delta_\A, q)$. See~\cite{BoyerGJ-RR08} for details.
\end{proof}
}

\annexe{
  Before proving it, we have to define and prove the following theorem.
  \begin{theorem}{(Cut in $\Subset$-proof trees)}
    \label{thm:cut}
    For all tree automata $\A$ and $\B$, if there exists $\prod$ a proof tree
    of $\Gamma \vdash_{\A, \B} q \Subset q'$, and a proof tree of 
    $\Gamma \cup \{q \Subset q'\}\vdash_{\A, \B} q_a \Subset q_b$
    then exists also a proof tree of $\Gamma \vdash_{\A, \B} q_a \Subset q_b$.
  \end{theorem}
  \begin{proof}
    We proceed by induction on $\mu(\Gamma)$.
    
    \noindent
    If $\mu(\Gamma) = 0$, we have immediately $\Q_\A \times \Q_\B =
    \Gamma$. Hence,
    since $q_a \Subset q_b \in \Gamma$, 
    we can prove $\Gamma \vdash_{\A,\B} q_a \Subset q_b$ using the Axiom rule.

    \medskip
    \noindent
    Now, as induction hypothesis, let us assume that $\forall \Gamma\ 
    s.t.\ \mu(\Gamma) = n$, $\forall q\ q'$, if there exists a proof tree $\prod$ of
    $\Gamma \vdash_{\A, \B} q \Subset q'$ and if for all $q_a, q_b$ there exists a
    proof tree of $\Gamma \cup \{q \Subset q'\}\vdash_{\A, \B} q_a \Subset q_b$ then we
    have also a proof tree of $\Gamma \vdash_{\A, \B} q_a \Subset q_b$.  Now, we
    aim at proving that this property is true for $\Gamma$ such that $\mu(\Gamma)
    = n+1$.
  
    \medskip
    \noindent
    Let us consider the proof tree of the second hypothesis $\Gamma \cup \{q \Subset
    q'\}\vdash_{\A, \B} q_a \Subset q_b$.  Firstly, if the proof tree is built
    using the Axiom rule we have $(q_a \Subset q_b) \in \Gamma \cup \{(q \Subset
    q')\}$.  Two cases are possible:
    \begin{itemize}
    \item either $(q_a \Subset q_b) \in \Gamma$, and then we build the proof of 
      $\Gamma \vdash_{\A,\B} q_a \Subset q_b$ using the Axiom rule.
      
    \item or $q = q_a$ and $q' = q_b$, and then the goal $\Gamma \vdash_{\A,\B} q_a
      \Subset q_b$ is equivalent to $\Gamma \vdash_{\A, \B} q \Subset q'$ whose
      proof tree is $\prod$.
    \end{itemize}
    
    \noindent
    Secondly, if the proof tree of $\Gamma \cup \{q \Subset q'\}\vdash_{\A, \B} q_a
    \Subset q_b$ is built using the Induction rule, then we have: 

    \medskip
    \newcommand{\env}{\Gamma \cup \{q_a \Subset q_b\} \cup \{q \Subset q'\} \vdash_{\A, \B}}
    
    \centerline{
      \begin{minipage}{21cm}
        {\tiny
          \begin{prooftree}
            \AxiomC{\small $\prod_{c_1}$}
            \UnaryInfC{$\env c_1 \Subset c'_{k_1}$}
            \LeftLabel{(Weark-r)}
            \UnaryInfC{$\env 
              c_1 \Subset  \{c_k'| c_k' \rightarrow_\B q_b\}_1^m$}
            % Pointillets du milieu
            \AxiomC{\small \dots\dots}
            \AxiomC{\small $\prod_{c_n}$}
            \UnaryInfC{$\env c_n \Subset c'_{k_n}$}
            \RightLabel{(Weark-r)}
            \UnaryInfC{$\env
              c_n \Subset  \{c_k'| c_k' \rightarrow_\B q_b\}_1^m$}
            \LeftLabel{(Split-l)}
            \TrinaryInfC{$\env
              \{c_i| c_i \rightarrow_\A q_a\}_1^n \Subset  \{c_i'| c_i'\rightarrow_\B q_b\}_1^m$}
            \LeftLabel{(Induction)}
            \UnaryInfC{$\Gamma \cup \{q \Subset q'\} \vdash_{\A, \B} q_a \Subset q_b$}
          \end{prooftree}
        }
      \end{minipage}}
    
    \medskip
    \noindent
    Where each $\prod_{c_i}$ has the following form (assuming
    that $c_i = f(q_{i_1}, \dots, q_{i_n})$ and $c'_{k_i} = f
    (q'_{i_1}, \dots, q'_{i_n})$ ):
    
    {\tiny
      \begin{prooftree}
        \AxiomC{\small $\prod_{i_1}$}
        \UnaryInfC{$\env q_{i_1} \Subset q'_{i_1}$}
        \AxiomC{\small \dots\dots}
        \AxiomC{\small $\prod_{i_n}$}
        \UnaryInfC{$\env q_{i_n} \Subset q'_{i_n}$}
        \LeftLabel{(Config)}
        \TrinaryInfC{$\env f(q_{i_1}, \dots, q_{i_n}) \Subset f (q'_{i_1}, \dots, q'_{i_n})$}
      \end{prooftree}
    }
    
    If we try to build the proof tree of our goal $\Gamma \vdash_{\A,\B} q_a
    \Subset q_b$, it necessarily begins in the same way except that $\{q \Subset
    q'\}$ will not appear in the left-hand side of statements. Each branch of
    this tree will end by a statement of the form $\Gamma \cup \{q_a \Subset
    q_b\} \vdash_{\A, \B} q_{i_j} \Subset q'_{i_j}$. Now to conclude the proof,
    we have to find proof trees $\prod'_{i_j}$ for all those statements. We know
    that there exists proof trees $\prod_{i_j}$ for all statements $\Gamma \cup \{q
    \Subset q'\} \cup \{q_a \Subset q_b\} \vdash_{\A, \B} q_{i_j} \Subset
    q'_{i_j}$.  We can use the induction hypothesis on $\prod_{i_j}$ to obtain
    $\prod'_{i_j}$ as follows:
    \begin{itemize}

    \item Since $\mu(\Gamma) = n + 1$, then $\mu(\Gamma\cup \{q_a \Subset q_b\}) = n$
      

    \item Since $\prod$ is a proof of $\Gamma \vdash_{\A, \B} q \Subset q'$, it is also a proof of
      $\Gamma \cup \{q_a \Subset q_b\}\vdash_{\A, \B} q \Subset q'$.
      
    \item Each $\prod_{i_j}$ is a proof of $\env  q_{i_j} \Subset q'_{i_j}$
    \end{itemize}
    
    \noindent
    Using induction, we deduce that for all $i,j$ there exist proof trees
    $\prod'_{i_j}$ of $\Gamma \cup \{q_a \Subset q_b\} \vdash_{\A, \B} q_{i_j}
    \Subset q'_{i_j}$.  This ends the proof tree of our goal $\Gamma \vdash_{\A,
      \B} q_a \Subset q_b$.
  \end{proof}
  
  \begin{theorem}{(Soundness)}
    \label{thm:soundness}
    For all tree automata $\A$ and $\B$, if there exists $\prod$ a proof tree
    of $\emptyset \vdash_{\A, \B} q \Subset q'$ then we have $\Lang(\A, q) \subseteq \Lang(\B, q')$
  \end{theorem}
  
  \begin{proof}
    We prove that $\forall t$, $t \in \Lang(\A, q) \imp t \in \Lang(\B, q')$ by
    induction on $t$. Let $t = f(t_1, \dots, t_n)$. We assume that the property is
    true for each subterm $t_i$, i.e. for all $q_i, q'_i$ s.t. if there exists a
    proof tree $\prod_i$ of $\emptyset \vdash_{\A, \B} q_i \Subset q'_i$ then $t_i
    \rightarrow^*_{\A} q \imp t \rightarrow^*_{\B} q_i'$.  Since $t=f(t_1, \dots,
    t_n) \in \Lang(\A, q)$, then for each subterm $t_i$, we know that there exists
    $q_1, \ldots, q_n$ such that $t_i \in \Lang(\A, q_i)$ and $f(q_1, \dots, q_n)
    \rightarrow q \in \A$. Besides this, by unfolding $\prod$ the proof tree of
    $\emptyset \vdash_{\A, \B} q \Subset q'$, we can deduce that for each
    transition like $f(q_1, \dots, q_n) \rightarrow q \in \A$, there exists $f(q'_1,
    \dots, q'_n) \rightarrow q' \in \B$ s.t. we have a proof tree $\prod_i$ of
    $\{q \Subset q'\} \vdash_{\A, \B} q_i \Subset q'_i$. 
    Since $f(q_1, \ldots, q_n) \rw q \in \A$, we obtain that $f(q'_1, \dots, q'_n)
    \rightarrow q' \in \B$ and a proof $\prod_i$ for $\{q \Subset q'\} \vdash_{\A,
      \B} q_i \Subset q'_i$. To conclude that $f(t_1, \dots, t_n) \in \Lang(\B,
    q')$ we just have to prove that $t_i \in \Lang(\B, q'_i)$. Note that we have a
    proof tree $\prod_i$ for $\{q \Subset q'\} \vdash_{\A,\B} q_i \Subset q'_i$
    and that to apply the induction hypothesis we need a proof tree for $\emptyset
    \vdash_{\A,\B} q_i \Subset q'_i$. Using the Theorem~\ref{thm:cut} on $\prod$
    and $\prod_i$, we can deduce the existence of $\prod'_i$ the proof tree of
    $\emptyset \vdash_{\A, \B} q_i \Subset q'_i$. Then using induction hypothesis
    on $t'_i, q_i, q'_i$ and $\prod'_i$,
    we obtain that for each $t_i \in \Lang(\A, q_i)$, we also have $t_i \in
    \Lang(\B, q'_i)$. Finally, since $f(q'_1, \ldots, q'_n)\rw q' \in \B$, we
    obtain that $t=f(t_1, \dots, t_n) \in \Lang(\B, q')$.
  \end{proof}
}


\versionlongue{\subsection{Completeness}}
\label{sec:completness}

As said above, the described algorithm is not complete in general.
However, we show that it is complete for tree automata produced by
completion. In particular if $\A_\R^k$ is obtained after $k$
completion step from $\A^0$ then we can build a proof $\prod$ for the
statement $\emptyset \vdash_{\A^0, \A^k_{\R}} \{\#(q)\ |\ q \in
\mathcal{Q}_{F_0}\} \Subset \{\#(q')\ |\ q'\in
\mathcal{Q}_{F_k}\}$. Recall that the tree automaton produced by the
$k^{th}$ step of completion is noted $\mathcal{A}_k = \langle
\mathcal{F}, \mathcal{Q}_k, \mathcal{Q}_{F_k}, \Delta_k\rangle$.
\versioncourte{ The tree automata completion performs two main
  operations at each step of calculus: \emph{normalization} and
  \emph{state merging}.  In the case of normalization, the language
  inclusion can simply be proved using transition set inclusion. With
  the state merging operation, set inclusion is not enough because it
  implies transition merging too.  This is the reason why we have to
  define a new ordered relation preserved by each operation.  }
\versionlongue{
First, let us introduce some new ordered relation on tree automata:
}
\begin{definition}
    \label{eq:prop0}
  Given $\A$, $\B$ two tree automata, $\sqsubseteq$ is the reflexive and
  transitive relation defined as follows: $\A \sqsubseteq \B$ if there exists a function
  $\varrho$ that renames states of $\A$ into states of $\B$ and such that 
  all renamed rules $\Delta_\A$ are contained in $\Delta_\B$:
  
  \begin{equation}
    \A \sqsubseteq \B \Longleftrightarrow
    \exists \varrho : \mathcal{Q_\A} \rightarrow \mathcal{Q_\B},
    \ \varrho(\Delta_\A) \subseteq \Delta_\B\ \wedge \ \varrho(\Q_{F_\A}) \subseteq \Q_{F_\B}
  \end{equation}
\end{definition}


\annexe{
We need to extend the renaming $\varrho$ to the structures and sets used in the following:
\begin{itemize}
\item $\varrho(\{q_i\}_1^n)$ stands for $\{\varrho(q_i)\}_1^n$

\item $\varrho(c) = \left\{\begin{array}{ll}
      f (\varrho(c_1), \dots, \varrho(c_n)) &\textrm{ if } c = f(c_1, \dots, c_n)\\
      c &\textrm{ if } c \in \F_{0}\\
      \varrho(q) &\textrm{ if } c = q\in \Q\\
    \end{array}\right.$
  
\item $\varrho(c \rightarrow q)$ stands for $\varrho(c) \rightarrow \varrho(q)$


\item $\varrho(\Delta)$ stands for $\{ \varrho(c\rightarrow q)\ |\ c \rightarrow q \in \Delta \}$.
\end{itemize}
}
\versionlongue{
In tree automata completion, two main operations can be
performed at each steps of calculus: \emph{normalization} and
\emph{state merging}. 
The following lemma shows that those two operations ensure $\sqsubseteq$.
}

\begin{lemma}
  Given a tree automaton $\A$,
  
  \[
  \begin{array}{llll}
    1. & \mbox{ if } \A' = \A \cup \norm(r\sigma \rw q) & \mbox{ then } &\A \sqsubseteq \A' \\
    2. & \mbox{ if } \A' = \merge(\A, q_1,q_2) &\mbox{ then }& \A \sqsubseteq \A'  \\
  \end{array}
  \]
\end{lemma}
\versioncourte{
\begin{proof}
  For details, see~\cite{BoyerGJ-RR08}.
\end{proof}
}

\annexe{
\begin{proof}
  \begin{enumerate}
  \item This is easy to show since we trivially have $\Delta_{A'} \supseteq
    \Delta_A$ whatever $r\sigma$ or $q$ may be.
    Then by choosing $\varrho = id$, we have immediately the conclusion $\A
    \sqsubseteq \A'$.

  \item Let $\Delta_\A$ be the transition set of $\A$. Let $q_i$ and $q_j$ be
    the two states to merge.  We can apply to $\Delta_\A$
    a renaming function $\varrho$ which has the same behavior than state merging
    with regard to $q_1 = q_2$:
    \[
    \varrho(q) = \left\{
      \begin{array}{l}
        \textrm{if }(q\ =\ q_2)\\
        \quad \quad q_1\\
        \textrm{else}\\
        \quad \quad q
      \end{array}\right.
    \]
    
    So state merging builds $\Delta_{\A'} = \varrho(\Delta_\A)$ and by
    Definition~\ref{eq:prop0} we have trivially $\Delta_A \sqsubseteq
    \Delta_{\A'}$.
  \end{enumerate}
\end{proof}
}


\begin{theorem}
  Given a tree automaton $\mathcal{A}^0$, a TRS $\R$ and an
  equation set $\mathcal{E}$, after $k$ completion steps we obtain
  $\A_\R^k$ such that $\A^0 \sqsubseteq \A_\R^k$.
\end{theorem}
\versioncourte{
\begin{proof}
  Since we have proved that $\sqsubseteq$ is preserved
  by $\norm$ and $\merge$ functions, it is also the case for every completion 
  step between $\A^k_\R$ and $\A^{k+1}_\R$, i.e $\A^k_\R \sqsubseteq \A^{k+1}_\R$.
  Then, the theorem can be deduced using the reflexivity and transitivity of
  $\sqsubseteq$. See~\cite{BoyerGJ-RR08}.
\end{proof}
}

\annexe{
\begin{proof}
  By induction on $k$:
  
  \begin{itemize}
  \item Since $\sqsubseteq$ is reflexive, we have trivially $\A^0
    \sqsubseteq \A^0$.
  \item Let $\mathcal{A}_k$ be a tree automaton obtained after $k$ completion
    steps such that $\A^0 \sqsubseteq \A_\R^k$. By definition of completion
    $\A_\R^{k+1}$ is built from $\A_\R^k$ by applying successively normalization
    and merge. We thus have $\A_\R^k \sqsubseteq \A_\R^{k+1}$.  
%Let us apply one
%    more step of completion. After normalization of $\A_\R^k$ we have $\A_\R^k
%    \sqsubseteq \A'$; then we merge $\A'$ with set $\mathcal{E}$ to obtain $\A'
%    \sqsubseteq \A_\R^{k+1}$. 
    By transitivity of $\sqsubseteq$, from $\A^0 \sqsubseteq \A_\R^k$ and
    $\A_\R^k \sqsubseteq \A_\R^{k+1}$ we deduce immediately that $\A^0
    \sqsubseteq \A_\R^{k+1}$.
  \end{itemize}
\end{proof}
}
\versioncourte{
Now, we define the completness property as the following:
}
\begin{theorem}{(Completeness)}
  Given two tree automata $\A$ and $\B$ if $\A \sqsubseteq \B$ then there
  exists $\prod$ a proof of statement $\emptyset \vdash_{\A, \B}
  \{\#(q_f)\ |\  q_f \in \mathcal{Q}_{F_A}\} \Subset  \{\#(q'_f)\ |\  q'_f \in \mathcal{Q}_{F_B}\}$.
\end{theorem}
\versioncourte{
\begin{proof}
For details, see~\cite{BoyerGJ-RR08}.
\end{proof}
}
\annexe{
\begin{proof}
  By definition of $\A \sqsubseteq \B$, we can deduce that there exists a renaming $\varrho$.\\
  First we prove by induction on the proof tree we have for all $\Gamma$
  and $q$, $\Gamma \vdash_{\A, \B} q \Subset \varrho(q)$:\\
  The hypothesis induction is $\forall \Gamma,\ q,\ q_i,$
  \begin{prooftree}
    \AxiomC{$\prod_i$}
    \UnaryInfC{$\Gamma \cup \{q \Subset \varrho(q)\} \vdash_{\A, \B} q_i \Subset \varrho(q_i)$}
  \end{prooftree}

  We want to construct a proof tree for $\Gamma \vdash_{\A, \B} q \Subset \varrho (q)$.\\
  Two cases are possible:
  \begin{itemize}
  \item 
    if $q \Subset \varrho(q) \in \Gamma$ then we can conclude immediately:
    \begin{prooftree}
      \AxiomC{}
      \LeftLabel{(Axiom)}
      \UnaryInfC{$\Gamma' \cup  \{q \Subset \varrho(q)\} \vdash_{\A, \B} q \Subset \varrho(q)$}
    \end{prooftree}

  \item 
    Otherwise, we need to apply Induction rule to obtain the following tree:
  
    \newcommand{\env}{\Gamma \cup \{q \Subset \varrho(q)\} \vdash_{\A, \B}}
    {\tiny
      \begin{prooftree}
        \AxiomC{\small $\prod_{c_1}$}
        \UnaryInfC{$\env 
          c_1 \Subset  \{c_k'| c_k' \rightarrow \varrho(q)\}_1^m$}
        % Pointillets du milieu
        \AxiomC{\small \dots\dots}
        \AxiomC{\small $\prod_{c_n}$}
        \UnaryInfC{$\env
          c_n \Subset  \{c_k'| c_k' \rightarrow \varrho(q)\}_1^m$}
        \LeftLabel{(Split-l)}
        \TrinaryInfC{$\env
          \{c_i| c_i \rightarrow q\}_1^n \Subset  \{c_k'| c_k'\rightarrow \varrho(q)\}_1^m$}
        \LeftLabel{(Induction)}
        \UnaryInfC{$\Gamma \vdash_{\A, \B} q \Subset \varrho(q)$}
      \end{prooftree}
    } From hypothesis $\varrho(\Delta_A) \subseteq \Delta_B$ for each
    rule $c \rightarrow q$ of $\Delta_A$, we have $\varrho(c\rightarrow q) \in \Delta_\B$.
    \[\textrm{Thus for all } (c\rightarrow q)\in \Delta_\A,\textrm{ we have } \varrho(c) \in \{c_k' | c_k' \rightarrow \varrho(q)\}_1^m \]
    For each $c_i= f_i(q_{i_1}, \dots, q_{i_n})$ we can construct the corresponding tree
    $\prod_{c_i}$ which his each branch is concluded by $\prod_{i_j}$
    an instance of induction hypothesis for the corresponding state
    $q_{i_j}$: {\tiny
      \begin{prooftree}
        \AxiomC{\small $\prod_{i_1}$}
        \UnaryInfC{$\env q_{i_1} \Subset \varrho(q_{i_1}) $}
        \AxiomC{\small \dots\dots}
        \AxiomC{\small $\prod_{i_n}$}
        \UnaryInfC{$\env q_{i_n} \Subset \varrho(q_{i_n}) $}
        %%%%%%%%%%%%%%%%%%%%%%% 
        \LeftLabel{(Config)}
        \TrinaryInfC{$\env c_i \Subset  \varrho(c_i) $}
        \LeftLabel{(Weak-r)}
        \UnaryInfC{$\env c_i \Subset  \{c_k'| c_k' \rightarrow \varrho(q)\}_1^m$}
      \end{prooftree}
    }
  \end{itemize}

  Now, we have for all $\Gamma$ and $q \in \Q_\A$ there exists a proof
  tree $\prod_q$ for all statement $\Gamma \vdash_{\A, \B} q \Subset \varrho(q)$.\\
  In particular, this is true for $\Gamma = \emptyset$ all $q$ of
  $\Q_{F_\A}$.  Since we have $\A \sqsubseteq \B \imp
  \varrho(\Q_{F_\A}) \subseteq \Q_{F_\B}$, we can build a proof tree
  as:

  {\small
    \begin{prooftree}
      \AxiomC{$\prod_{q_{f_1}}$}
      \UnaryInfC{$\emptyset \vdash_{\A, \B} q_{f_1} \Subset \varrho(q_{f_1})$}
      %\AxiomC{$\emptyset \vdash_{\A, \B} [\ ] \Subset [\ ]$}
      \LeftLabel{(Config)}
      \UnaryInfC{$\emptyset \vdash_{\A, \B} \#(q_{f_1}) \Subset \#(\varrho(q_{f_1}))$}
      \LeftLabel{(Weak-r)}
      \UnaryInfC{$\emptyset \vdash_{\A, \B} q_{f_1} \Subset \Q_{F_\B}$}
      \AxiomC{\small \dots\dots}
      \AxiomC{$\prod_{q_{f_n}}$}
      \UnaryInfC{$\emptyset \vdash_{\A, \B} q_{f_1} \Subset \varrho(q_{f_1})$}
      %\AxiomC{$\emptyset \vdash_{\A, \B} [\ ] \Subset [\ ]$}
      \RightLabel{(Config)}
      \UnaryInfC{$\emptyset \vdash_{\A, \B} \#(q_{f_n}) \Subset \#(\varrho(q_{f_n}))$}
      \RightLabel{(Weak-r)}
      \UnaryInfC{$\emptyset \vdash_{\A, \B} \#(q_{f_n}) \Subset \{\#(q)\ |\ \Q_{F_\B}\}$}
      \LeftLabel{(Split-l)}
      \TrinaryInfC{$\emptyset \vdash_{\A, \B} \{\#(q)\ |\ q \in \Q_{F_\A}\} \Subset \{\#(q)\ |\ q \in \Q_{F_\B}\}$}
    \end{prooftree}}
\end{proof}
}


Thus, we can ensure that for an automaton $\A_\R^k$ obtained by $k$
completion steps from $\mathcal{A}^0$, there exists a proof $\prod$ of
the statement $\emptyset \vdash_{\A^0, \A^k_{\R}} \{\#(q)\ |\ q \in
\mathcal{Q}_{F_{0}} \Subset \{\#(q')\ |\ q'\in \mathcal{Q}_{F_{k}}\}$.
This can be obtained by a simple combination of the two previous
theorems.

%\begin{theorem}
%  Given a tree automaton $\mathcal{A}^0$, a TRS $\R$ and an equation set
%  $\mathcal{E}$, after $k$ completion steps we obtain $\A_\R^k$ such that
%\end{theorem}


\versionlongue{
\paragraph{Remark about restrictions:}
To obtain a decision procedure for all tree automata (and not only
those obtained by completion), the "Weak-r" rule
would have to be modified.  This version is too weak; it does not take all
possible cases for union construction into account.

\begin{example}
  Let $\A$ and $\B$ be two automata s.t.:
  \[\A = \left\{ 
    \begin{array}{rcl}
      a& \rightarrow &q_1\\
      b& \rightarrow &q_2\\
      c& \rightarrow &q_2\\
      f(q_1, q_2)&\rightarrow &{\bf q}\\
    \end{array}\right\}
  \textrm { and }
  \B = \left\{ 
    \begin{array}{rcl}
      a& \rightarrow &q_1'\\
      b& \rightarrow &q_2'\\
      c& \rightarrow &q_3'\\
      f(q_1', q_2')&\rightarrow &{\bf q'}\\
      f(q_1', q_3')&\rightarrow &{\bf q'}\\
    \end{array}\right\}
  \]
  Here we have $\Lang(\A, q) = \Lang(\B, q')$ but $\emptyset \vdash_{\B, \A} q' \Subset q$ is clearly derivable
  whereas $\emptyset \vdash_{\A, \B} q \Subset q'$ is not.
\end{example}
}

\subsection{Complexity}

As said in section~\ref{section:objectives}, the standard algorithm for checking
the inclusion $\Lang(\A) \subseteq \Lang(\B)$ is based on computing the
complement automaton $\overline{\B}$. However, for non deterministic tree
automata the size~\cite{TATA} of $\overline{\B}$ can be exponentially greater
than the size of $\B$. The algorithm that we have proposed above has not this
drawback and use only a memory size that is polynomial w.r.t. to the automata
sizes.

Let $|\Q|$ be the maximum number of states in tree automata $\A$ and $\B$.  The
proof trees built using the deduction rules we gave are of height at most
$|\Q|^2$. This is due to the rules 'Induction' and 'Axiom' ensuring that every
inclusion problem $q \Subset q'$ will be analyzed only once per branch.  Since
$q\in \Q_\A$ and $q' \in \Q_\B$ and we know that the cardinal of $\Q_\A$ and
$\Q_\B$ is bounded by $|\Q|$, the length of branch is at most bounded by
$|\Q|^2$.  Since, the \lstinline!inclusion! function only constructs one branch
of the proof tree at a time, the memory usage is thus bounded by $|\Q|^2$ and
thus polynomial.

However, the time complexity of a straightforward implementation of this
algorithm is exponential. Indeed, even if each couple $q \Subset q'$ is
considered only once on each branch, the number of branches is exponential
w.r.t. $\Q$ and the same couple $q \Subset q'$ may be analyzed once per
branch. A very simple optimization of this algorithm is to table the result of
the analysis of each $q \Subset q'$ and make it available to cut similar proof
branches. Using this optimization, every couple is considered {\em only once}
for the whole proof tree. Assuming that read and write operations of the table
can be performed in constant time, this leads to an overall polynomial time
complexity bounded by $|\Q|^2$. Nevertheless, the current \coq\ implementation is
based on the non tabled version for two reasons.  First, the proof of theorem
\lstinline!inclusion_sound! is more difficult on a tabled version of the \coq\
\lstinline!inclusion! function. Second, on test cases, it appears that avoiding
the exponential blow-up of memory was critical but that practical performances
of the, potentially, exponential time algorithm are sufficient.



%%% Local Variables: 
%%% mode: latex
%%% TeX-master: "main"
%%% End: 
