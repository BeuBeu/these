\documentclass[a4paper,10pt]{llncs}
\usepackage[latin1]{inputenc}
\usepackage[english]{babel}
\usepackage[curve]{xypic}
\usepackage{stmaryrd}
\usepackage{amsmath, amssymb, amsfonts}
\usepackage{bussproofs}
\setlength{\oddsidemargin}{15mm}
\setlength{\evensidemargin}{15mm} 
\setlength{\textwidth}{135mm}
\setlength{\textheight}{214mm} 
\setlength{\topmargin}{5mm}

\input{rapinclude}

%All i need to have fun with Caml and Coq :)
\RequirePackage{alltt}
\RequirePackage{listings}
\def\lstlanguagefiles{lstlangcoq.sty,lstlangcaml.sty}
\lstloadlanguages{Coq,Caml} %Caml
\newcommand{\switchlstcoq}{
\lstset{language=Coq,flexiblecolumns=false,mathescape=true}
\lstset{keywordstyle={\bfseries}}
\lstset{commentstyle=\it,basicstyle=\small\tt}%,keywordstyle=\bfseries}
\lstset{literate={=>}{{$\Rightarrow$}}1
{\\Abstate}{{\Abstate}}1
{forall}{{$\forall$\hspace{-1ex}}}1
{exists}{{$\exists$\hspace{-1ex}}}1
{not}{{$\neg$}}1
{<=<}{{$\subseteq$}}2
{>=}{{$\ge$}}1
{>->}{{$\rightarrowtail$}}2{->}{{$\to$}}2 {/\\}{{$\land$}}1 {~}{{$<AC>$}}1}
\lstset{escapeinside={(*@}{@*)}} % In the text as an example:
}

\renewcommand{\ttdefault}{pcr}


\begin{document}
\versioncourte{\title{Certifying a Tree Automata Completion Checker\thanks{For
      reviewer's convenience, proofs of~\cite{BoyerGJ-RR08} are given as an
      appendix.}}}  \versionlongue{\title{Certifying a Tree Automata Completion
    Checker}} \author{Beno�t Boyer \and Thomas Genet \and Thomas Jensen}

\institute{
  IRISA / Universit\'e de Rennes\,1 / CNRS \\
  Campus de Beaulieu \\
  F-35042 Rennes Cedex \\
  {\tt {bboyer,genet,jensen}@irisa.fr} 
}
\maketitle


\begin{abstract}
  Tree automata completion is a technique for the verification of infinite state
  systems. It has already been used for the verification of cryptographic
  protocols and the prototyping of Java static analyzers.  However, as for many
  other verification techniques, the correctness of the associated tool becomes
  more and more difficult to guarantee. It is due to the size of the
  implementation that constantly grows and due to optimizations which
  are necessary to scale up the efficiency of the tool to verify real-size
  systems. In this paper, we define and develop a checker for tree automata
  produced by completion. The checker is defined using \coq\ and its
  implementation is automatically extracted from its formal specification. Using
  extraction gives a checker that can be run independently of the \coq\
  environment. A specific algorithm for tree automata inclusion checking have
  been defined so as to avoid the exponential blow up. The obtained checker is
  certified in \coq, independent of the implementation of completion, usable
  with any approximation performed during completion, small and fast. Some
  benchmarks are given to show how efficient the tool is.
\end{abstract}

%each file contains a section :


\input{intro}

\input{preliminaries}


% Section Rappel du principe de la completion...
\section{The Tree Regular Model Checking Problem}
\label{sec:completion}

Our objective is to verify properties of a given system. We will focus
on models of systems whose set of reachable states may be, for
modeling reasons, infinite\,\cite{WB98} -- but our solution also works
for huge finite-state systems. Our first problem is to provide a
symbolic representation to represent and manipulate possibly infinite
sets of states. The problem is undecidable and only partial solutions
exist. Here, we will use {\em Tree Regular Model Checking
  (TRMC)}\,\cite{ALRd05}.
\noindent
In TRMC, a program is a tuple $(\F, I,
Rel)$, where

\begin{itemize}
\item
$\F$ is an alphabet on which a set of terms $\TF$ can be defined;

\item
$I$ is a set of initial configurations represented by a
tree automaton $\A$, i.e. $\Lang(\A)=I$;

\item $Rel$ is a transition relation represented by a set of
  left-linear rewriting rules $\R$.
\end{itemize}

\noindent
In our setting, a program will thus be represented by the tuple $(\F,\A,\R)$.
It has been shown that the above framework can be used to represent a
wide range of applications going from cryptographic protocols to JAVA
applications. We consider reachability problems.

\begin{definition}[Reachability problem]
\label{def:reachability}
Consider a program $(\F,\A,\R)$ and $Bad$ a set of forbidden terms
$Bad$. The reachability problem consists in checking whether there
exists terms of $\desc(\Lang(\A))$ that belong to $Bad$.
\end{definition}

% We will consider the verification of {\em reachability properties},
% i.e., properties that are represented by sets of states. We say that a
% program satisfies a reachability property $\phi$ if all of its
% reachable state is included in the set of states that represent
% $\phi$. Assuming the existence of a tree automaton for representing
% $\phi$, verifying reachability properties reduces to solve the {\em
%   reachability problem}.

% \begin{definition}
% \label{def:reachability}
%   Given a programme $(\F, {\phi}_I, R)$, we consider the reachability
%   problem that consists in computing a tree automaton $\aaex^*$ for
%   the set $R^*(\phi_I)$.
% \end{definition}



\noindent
For finite-state systems, computing the set of reachable terms
($\desc(\Lang(\A))$) reduces to enumerate the terms that can be
reached from the initial set of configurations. Unfortunately, for
infinite-state systems, this enumeration may never terminate.  There
is thus also a need to ``accelerate'' the search through the state
space in order to reach, in a finite amount of time, states at
unbounded depths. Among the existing algorithms used to compute a tree
automaton representing the set of reachable terms of a system, one
finds {\em completion algorithm}. A completion algorithm is a
semi-algorithm that computes an automaton $\aaex^*$ that is possibly
an over-approximation of the set of reachable terms. In the rest of
this section we introduce the principle of completion and point its
current limits.

We say that a tree automaton $\B$ is $\R$-closed if for all terms
$s,t$ such that $s\f_\R$ and $s$ is recognized by $\B$ into state $q$ then so is
$t$. The situation is represented with the following graph.
\begin{tabular}{lc}
  \hspace{-.3cm}
  \begin{minipage}{.8\linewidth}
    It is easy to see that $\Lang(\B) \supseteq \desc(\Lang(\A))$ if
    $\B$ is $\R$-closed and $\Lang(\B)\supseteq
    \Lang(\A)$~\cite{BoyerGJ-IJCAR08}. From an algorithmic point of
    view, building a $\R$-closed $\aaex^*$ from $\A$ consists in {\em
      completing} $\A$ with new transitions.  The completion algorithm
    computes successive automata $\aaex^1,\aaex^2,\ldots$ that
    represent the effect of applying the set of rewriting rules to the
    initial automaton.
  \end{minipage}
  &
  \begin{minipage}{.2\linewidth}
    $
    \xymatrix{
      s \ar[r]_{\R}\ar[d]^{*}_{\B} & t \ar@/^1.2pc/[ld]_{*}^{\B}\\
      q & %\ar[l]^{\A_{i+1}} q'
    }
    $
  \end{minipage}
\end{tabular}

Each application of $\R$ is called a {\em completion step} and
consists in searching for {\em critical pairs} $\langle t,q \rangle$ where the above
diagram is not closed, i.e. $ s\rwR t$, $s \rw_{\A}^* q$ and $t
\not\rw_{\A}^* q$.
% remis car il faut introduire la notation \aaex^i, utilisée dans la suite.
The idea being that the algorithm solves the critical pair by
constructing from $\A$, a new tree automaton $\aaex^1$ with the
additional transitions needed to obtain $t \rw_{\aaex^1}^* q$,
representing the effect of applying $\R$. Then a similar algorithm is
applied on $\aaex^1$ to obtain $\aaex^2$, and so on until a fixpoint
$\aaex^*$ is reached.

As the language recognized by $A$ may be infinite, it is not possible to find
all the critical pairs by enumerating the terms that it recognizes. The solution
that was promoted in \cite{Genet-RTA98} consists in applying sets of
substitutions $\sigma: \X \mapsto \Q$ mapping variables of rewrite rules to
states representing infinite sets of (recognized) terms. Given a tree automaton
$\aaex^i$ and a rewrite rule $l \rw r \in \R$, to find all the critical pairs of
$l \rw r$ on $\aaex^i$, completion uses a {\em matching
  algorithm}~\cite{FeuilladeGVTT-JAR04} that produces the set of substitutions
$\sigma: \X \mapsto \Q$ and states $q\in \Q$ such that $l \sigma \rw_{\aaex^i}^*
q$ and $r\sigma \not\rw_{\aaex^i}^* q$. Solving critical pairs thus consists in
adding new transitions: $r\sigma \rw q'$ and $q' \rw q$. Those transitions may
have to be normalized to respect the definition of transitions of tree
automata. As it was shown in~\cite{Genet-RTA98}, this operation may add not only
new transitions but also new states to the automaton. In the rest of the paper,
the completion-step operation will be represented by $\compl$, i.e., the
automaton obtained by applying the completion step to $\aaex^i$ is denoted
$\compl(\aaex^i)$.

The problem is that, except for particular
classes~\cite{FeuilladeGVTT-JAR04,Genet-Habil}, the automaton
representing the set of reachable terms cannot be obtained from $A$ by
applying a finite number of completion steps and the process thus
needs to be accelerated. For doing so, one can uses an approximation
technique based on a set of equations $E$ and produce an
over-approximation of the set of reachable terms, i.e., a tree
automaton $\aapprox$ such that $\Lang(\aapprox) \supseteq
\desc(\Lang(\A))$.

To produce such an automaton, each automaton $\aaex^i$ obtained by
applying $i$ completion steps to $A$ is approximated using a
widening function $\widen$ parametrized by $E$. An equation $u=v$ is
applied to a tree automaton $\A$ as follows: for all substitution
$\sigma:\X \mapsto \Q$ and distinct states $q_1$ and $q_2$ such that
$u \sigma \rw_{\A}^* q_1$ and $v\sigma \rw_{\A}^* q_2$, states $q_1$
and $q_2$ are merged.  Completion and widening steps can be
linked, i.e. $\aaexeq^0=\A$ and $\aaexeq^{i+1}= \widen(\compl(\aaexeq^i))$, until a
$\R$-closed fixpoint $\aapprox$ is found.  In~\cite{GenetR-JSC10}, it
has been shown that, under some assumptions, the obtained automaton
recognizes no more that terms reachable by rewriting with $\R$ modulo
$E$. As a result, the approximation framework and methodology is close
to equational abstractions of~\cite{MeseguerPM-TCS08}.

% The completion technique has successfully been used for the
% verification of cryptographic
% protocols~\cite{GenetK-CADE00,GenetTTVTT-wits03}.


% Furthermore, the
% tree automata completion implementation, \timbuk, is used as a
% verification backend in AVISPA~\cite{BoichutHKO-AVIS04,avispa}. More
% recently, tree automata completion has been used for fast prototyping
% of static analyzers for Java bytecode
% programs~\cite{BoichutGJL-RTA07}. In this settings, $\R$ encodes the
% system ({\em e.g.} protocol and intruder behavior or Java virtual
% machine semantics), $\aapprox$ represents the over-approximation of
% all states of the system. Then, 


%%the verification consists in showing
%%that the intersection between $\aapprox$ and an automaton $$

\begin{example}
\label{ex:comp}
Let $\R=\{f(x) \rw f(s(s(x)))\}$, $E=\{s(s(x))=s(x)\}$, $\A=\langle \F, \Q, \Q_F,
\Delta\rangle$ be a tree automaton such that $\Q_F=\{q_0\}$ and $\Delta=\{ a \rw
q_1, f(q_1) \rw q_0\}$, i.e. $\Lang(\A)=\{f(a)\}$.
\begin{tabular}{lc}
  \hspace{-.3cm}
  \begin{minipage}{.75\linewidth}
    The first completion step finds the following critical pair: $f(q_1) \rwA^* q_0$
    and $f(s(s(q_1))) \not\rwA^* q_0$. Hence, the completion algorithm produces
    $\aaex^1=\compl(\A)$ having all transitions of $\A$ plus $\{s(q_1) \rw q_2, s(q_2) \rw q_3,
    f(q_3) \rw q_4, q_4 \rw q_0\}$ where $q_2, q_3, q_4$ are new states produced
    by normalization of $f(s(s(q_1))) \rw q_0$. Applying $\widen$ with
    the equation $s(s(x))=s(x)$ on $\aaex^1$ merges the states $q_3$ and $q_2$.
  \end{minipage}&
  \begin{minipage}{.25\linewidth}
    $\xymatrix{
      s(s(q_1)) \ar@{=}[r]\ar[d]_{\aaex^1}^{*} & s(q_1) \ar[d]_{*}^{\aaex^1}\\
      q_3 & q_2
    }
    $
  \end{minipage}
\end{tabular}




  %$s(s(q_1))
  %\rw^*_{\aaex^1} q_3$ and $s(q_1) \rw^*_{\aaex^1} q_2$. 

In~\cite{GenetR-JSC10}, $\A^1_{\R,E}=\widen(\aaex^1)$ is built from $\aaex^1$ by renaming $q_3$ by
$q_2$. The set of transitions of $\A^1_{\R,E}$ is thus $\Delta \cup \{s(q_1) \rw
q_2, s(q_2) \rw q_2, f(q_2) \rw q_4, q_4 \rw q_0\}$.  Completion stops on
$\A^1_{\R,E}$ because it is $\R$-closed, thus $\aapprox=\A^1_{\R,E}$.
Now, let us assume that $Bad=\{f(s(a)), f(s(s(a)))\}$. The first term is not in
$\desc(\Lang(\A))$ but the second is. However, those two terms are 
%However, $f(s(a))$ is not in $\desc(\Lang(\A))$ but $f(s(s(a)))$ is.Those two terms are
recognized by $\aapprox$ and there is no way to distinguish between
the two: no way to detect that the second is {\em really} reachable
nor to automatically refine the abstraction so as to reject the first
one.
\end{example}


If the intersection between $\aapprox$ and $Bad$ is not empty, then it
does not necessarily mean that the system does not satisfy the
property. There is thus the need for techniques to decide whether a
counter-example is indeed a reachable term that does not satisfy the
property or if it is a term added by the abstraction and that cannot
be reached from the set of initial states. If the latter case occurs,
one has to propose a refinement technique that will remove the
false-positive from the abstraction. Studying such techniques for
completion automata is the main objective of this paper.

%%The rest of the paper is organized as follows. In Section
%%\ref{sec:re-automaton}, we introduce $\RE$-automaton that is an
%%extended tree automata. Section \ref{sec:re-automaton} proposes a
%%completion algorithm for $\RE$-automata, while Section \ref{} shows
%%how the extended structure can be used to refine in an efficient
%%manner.




% Tree automata completion~\cite{genet-RTA98,FeuilladeGVTT-JAR04,GenetR-JSC10} is
% an algorithm for computing sets of reachable terms $\desc(I)$, given a regular
% language of initial terms $I=\Lang(\A)$. For most of the known classes of TRS
% $\R$ for which $\desc(\Lang(\A))$ is regular, the output of completion is a tree automaton
% $\aaex^*$ such that $\Lang(\aaex^*)=\desc(\Lang(\A))$. When
% $\desc(\Lang(\A))$ is not regular, it is possible to parameterize the algorithm
% by a set of equations $E$, and to compute a tree automaton $\aapprox$ 
% over-approximating reachable terms,
% i.e. $\Lang(\aapprox) \supseteq \desc(\Lang(\A))$. 



%%% Local Variables: 
%%% mode: latex
%%% TeX-PDF-mode: t
%%% TeX-master: "main"
%%% End: 


% \section{The need for a certified checker in \coq}
%\label{sec:intro_checker}
\switchlstcoq 

\section{A result checker for tree automata completion}
\label{section:objectives}
%This part has to describe precisely the contribution of the paper.

\archive{The main question about \timbuk\ results is \emph{can we trust them?}
If the tree automata completion outputs an incorrect fixpoint
automaton, the analysis may validate an unsafe program.  Moreover,
\timbuk\ is a complex tool. It contains more than 11000 lines
of Ocaml. Thus it is difficult to be sure that \timbuk\ is bug free".
We can only rely on testing w.r.t. a base of test cases to remove as many
bugs as possible.
}

By moving the certification problem from the completion algorithm to
the checker, the certification problem consists in proving the following
\coq\ theorem:
\begin{lstlisting}
Theorem sound_checker :
      forall A A' R, checker A R A' = true -> ApproxReachable A R A'.
\end{lstlisting}
where \lstinline!ApproxReachable! is a \coq\ predicate that describes
the Soundness Property: \emph{$\Lang(A')$ contains all terms reachable
  by rewriting terms of $\Lang(A)$ with $\R$}, i.e. $\Lang(\A')
\supseteq \desc(\Lang(\A))$. 
To state formally this predicate in \coq, we need to
give a \coq\ axiomatization of Term Rewriting Systems and of Tree Automata. It is
given in Section~\ref{sec:rewriting}.
Given two automata $\A$, $\A'$ and a TRS $\R$ the checker 
verifies that $\Lang(\A')\supseteq \R^*(\Lang(\A))$ or
(\lstinline!ApproxReachable A R A'!) in \coq. To perform this, we need to check the two
following properties:

\begin{itemize}
\item \lstinline!Included!: inclusion of initial set in the fixpoint: $\Lang(\A) \subseteq \Lang(\A')$.

\item \lstinline!IsClosed!: $\A'$ is closed by rewriting with $\R$: For all $l \rightarrow
  r \in \R$ and all $t \in \Lang(\A')$, if $t$ is rewritten in $t'$ by the rule
  $l \rightarrow r$ then $t' \in \Lang(\A')$. 
% Trop detaill� pour la partie objectives...
% To prove this property, we need
%   verify that for each substitution $\sigma:\X \mapsto \Q$ and state $q$ of
%   $\A'$, if $l\sigma \rw_{\A'}^* q$ then we have $r\sigma \rw_{\A'}^* q$,
%     i.e. prove that the critical pair $(l\sigma \rightarrow q,\ l\sigma
%     \rightarrow r\sigma)$ is joinable.
\end{itemize}
For each item, we provide a \coq\ function and its correctness theorem: function
{\tt inclusion} is dedicated to inclusion checking and function {\tt closure}
checks if a tree automaton is closed by rewriting.  We also give the theorem
used to deduce \lstinline!ApproxReachable A R A'! from \lstinline!Included A A'! and \lstinline!IsClosed R A'!:
\begin{lstlisting}
Theorem inclusion_sound:
      forall A A', inclusion A A' = true -> Included A A'.

Theorem closure_sound:
      forall R A', closure R A' = true -> IsClosed R A'.

Theorem Included_IsClosed_ApproxReachable:
      forall A A' R, Included A A' -> IsClosed R A' -> ApproxReachable A R A'.
\end{lstlisting}


Note that, in this paper we focus on the proof of $\Lang(\A')\supseteq
\R^*(\Lang(\A))$.  However, to prove the unreachability property, the emptiness
of the intersection between $\Lang(\A')$ and the bad term set has also to be
verified. Since the formalization in \coq\ of the intersection and emptiness
decision are close to their standard definition~\cite{TATA}, and since they have
already been covered by~\cite{RivalGL-TPHOL01}, they are not be detailed in this
paper.





%%% Local Variables: 
%%% mode: latex
%%% TeX-master: "main"
%%% End: 


\section{Formalization of Term Rewriting Systems}
\label{sec:rewriting}

The aim of this part is to formalize in \coq: terms, term rewriting systems,
reachable terms and the reachability problem itself.  Firstly we use the
positive integers provided by the \coq's standard library to define symbol sets
like variables ($\X$) or function symbols ($\F$). We rename \lstinline!positive! into
\lstinline!ident! to be more explicit. Then, we define term set $\TFX$ using inductive
types:

\switchlstcoq
%A discuter....
%Definition F := positive.
%Definition X := positive.
\begin{lstlisting}
Definition ident := positive.

Inductive term : Set :=
| Fun : ident -> list term -> term
| Var : ident -> term.
\end{lstlisting}

\versionlongue{
Now, the term $f(x, a)$ will be written \lstinline!Fun 0 (Var 0::(Fun 1 nil)::nil)!  assuming that we have the corresponding mapping between between
symbols, variables and positive integers $f \mapsto 0$, $a \mapsto 1$ and $x
\mapsto 0$ for example.  Note that it is possible to attach the value $0$
to $f$ and $x$, since the \lstinline!term!'s constructors \lstinline!Fun! and
\lstinline!Var! allows to differentiate between variable and function symbols.
}

\versionlongue{
\paragraph{Remarks:}
\begin{itemize}
\item
  Since equality is decidable, we can easily define term equality as a
  decidable relation. Afterward, this is very useful to define functions
  where term comparison is required.
\item
  A bad point for \coq\ is the induction principle
  \lstinline!term_rect! automatically generated which is too weak.
  To prove properties in the following, we need a more efficient theorem
  named \lstinline!term_rect'!. This is required to be able to prove
  most of the theorems on terms.
\end{itemize}
}

A rewrite rule $l \rw r$ is represented by a pair of terms with a
well-definition proof, i.e. a \coq\ proof that the set of variables of $r$ is a
subset of the set of variables of $l$. The function
\lstinline!Fv : term -> list ident! builds the set of variables for a term. 
% In \coq, it becomes:
\begin{lstlisting}
Inductive rule : Set :=
| Rule (l r : term)(H : subseteq (Fv r) (Fv l)) : rule.
\end{lstlisting}

In the following, \lstinline!list rule! type represents a
TRS. In \coq\ we use \lstinline!(t @ sigma)!  to denote the term resulting of
the application of a substitution \lstinline!sigma! to each variable that occurs
in a term \lstinline!t!. 

\begin{lstlisting}
Definition substitution := ident -> option term.
\end{lstlisting}

In \coq, the rewriting relation \emph{"$u$ is rewritten in $v$ by $l
  \rightarrow r$"},
commonly defined by $\exists \sigma \
s.t.\ u|_p = l\sigma \ \land \ v = u [ r\sigma ]_p$, is split into
two predicates:
\begin{itemize}
\item The first one defines the rewriting of a term at the topmost position. In
  fact, the set of term pairs $(t, t')$ which are rewritten at the top most by
  the rule can be seen as the set of term pairs $(l\sigma, r\sigma)$ for any
  substitution $\sigma$.

\item 
  The second one just defines inductively the rewriting relation at any
  position of a term $t$ by a rule $l \rightarrow r$, by the topmost
  rewriting of any subterm of $t$ by $l \rightarrow r$.
\end{itemize}

\begin{lstlisting}
(* Topmost rewriting : *)
Inductive TRew (x : rule) : term -> term -> Prop :=
| R_Rew :
  forall s l r (H : subseteq (Fv r) (Fv l)),
    x = Rule l r H -> TRew x (l @ s) (r @ s).
\end{lstlisting}


\begin{lstlisting}
(* Rewrite at any position of term *)
Inductive Rew (r : rule) : term -> term -> Prop :=
| Rew1 : forall t t', 
    TRew r t t' -> Rew r t t'
| Rew2 : forall f l l',
    Rargs r l l' -> Rew r (Fun f l) (Fun f l')
\end{lstlisting}

\begin{lstlisting}
with Rargs (r : rule):list term->list term->Prop:=
| Ra1 : forall t t' l,
    Rew r t t' -> Rargs r (t::l) (t'::l)
| Ra2 : forall t l l',
    Rargs r l l' -> Rargs r (t::l) (t::l').
\end{lstlisting}


\versioncourte{Similarly, using an inductive definition it is possible to
  define the \lstinline!Rew! predicate for rewriting at any position.  } Then we
have to define $\rw^*_{\R}$. In \coq, we prefer to see it as the predicate
\lstinline{Reachable R u} that characterizes the set of reachable terms from
$u$ by $\rightarrow^*_\R$.

\begin{lstlisting}
Inductive Reachable(R : list rule)(t : term) : term -> Prop:=
| R_refl : Reachable R t t
| R_trans : forall u v r, Reachable R t u -> In r R -> Rew r u v -> 
    Reachable R t v.
\end{lstlisting}

%%% Local Variables: 
%%% mode: latex
%%% TeX-master: "main"
%%% End: 

\input{automata}

\input{inclusion}

\input{closure}

\section{Benchmarks}
\label{sec:benchmarks}

From the \coq\ formal specification, we have extracted an Ocaml checker
implementation. In the following table, we have collected several benchmarks.
For each test, we
give the size of the two tree automata (initial $\A^0$ and completed $\A_\R^*$)
as number of transitions/number of states.  For each TRS $\R$ we give the number
of rules.  The 'CS' column gives the number of completion steps necessary to
complete $\A^0$ into $\A^*_{\R}$ and 'CT' gives the completion time.  The 'CKT'
column gives the time for the checker to certify the $\A^*_{\R}$ and the 'CKM'
gives the memory usage. The important thing to observe here is that, the
completion time is very long (sometimes more than 24 hours), the {\em checking}
of the corresponding automaton is always fast (a matter of seconds).

The four tests are Java programs translated into term rewriting systems using
the technique detailed in~\cite{BoichutGJL-RTA07}. All of them are completed
using \timbuk\ except the example {\tt List2.java} which has been completed using a completion
tool under development by Yohan Boichut and Emilie Balland.  This shows that the
completed automaton produced by a new and fully optimzed tool is also accepted
by our checker.  The {\tt List1.java} and {\tt List2.java} corresponds to the
same Java program but with slightly different encoding into TRS and
approximations. The {\tt Ex\_poly.java} is the example given
in~\cite{BoichutGJL-RTA07} and the {\tt Bad\_Fixp} is the same problem as {\tt
  Ex\_poly.java} except that the completed automaton $\A^*_{\R}$ has been
intentionally corrupted. Thus, this is thus not a valid fixpoint and rejected by
the checker.

\begin{center}
\begin{tabular}{l|c|c|c|c|c|c|c|c}
  Name & $\A^0$ & $\A^*_\R$ & $\R$ & CS & CT & CKT & CKM \\ \hline
  
  {\tt List1.java} & 118/82 & 422/219 & 228 & 180 & $\approx$ 3 days & 0,9s & 2,3 Mo \\ \hline
  %Yoh
  {\tt List2.java} & 1/1 & 954/364 & 308 & 473 & 1h30 & 2,2s & 3,1 Mo \\ \hline
  %Res_
  {\tt Ex\_poly.java} & 88/45 & 951/352 & 264 & 161 & $\approx$ 1 day & 2,5s & 3,3 Mo \\ \hline
  %Res_ modifi�
  {\tt Bad\_Fixp} & 88/45 & 949/352 & 264 & 161 & $\approx$ 1 day & 1,6s & 3,2 Mo \\ \hline

%  NSPK & 26tr. 12 states & 243 tr. 12st. & 13 & 6 & 8'/21.4s & 5 Mo \\ \hline
\end{tabular}
\end{center}
%%% Local Variables: 
%%% mode: latex
%%% TeX-master: "main"
%%% End: 


%\section{Complete certificate}
%\label{sec:certificate}
%Peut-�tre une partie pour expliquer comment termine-t-on la preuve 
%pour obtenir le certificat.
\section{Conclusion and further research}
\label{sec:conclusion}
In this paper we have defined a \coq\ checker for tree automata completion.  The
first characteristic of the work presented here is that the checker does not
validate a specific implementation of completion but, instead, the result. As a
consequence, the checker remains valid even if the implementation of the
completion algorithm changes or is optimized. A second salient feature is that
the code of the checker is directly generated from the correctness proof of its
verified \coq\ specification through the \coq\ extraction mechanism. Third, we
have payed particular attention to the formalization of the checker in order not
to lose efficiency to obtain the certification. We have defined a specific
inclusion algorithm in order to avoid the usual exponential blow-up obtained
with the standard inclusion algorithm.  We have defined the \coq\ formal
specification so that it was possible to extract an independent OCaml
checker. Finally, we made an extensive use of efficient formal data structures
leading to more complex proof but also to faster extracted checker.  An
extension for non left-linear TRS, which are sometimes necessary for specifying
cryptographic protocols, is under development.  Since many different kind of
analysis can be expressed as reachability problems over tree automata, and since
verification of completed automata revealed to be very efficient, we aim at
using this technique in a PCC architecture where tree automata are viewed as
program certificates. At last, note that even if this checker is external to
\coq, we can use the correction proof of the checker and the \coq\ reflexivity
mechanism to lift-up the external verification into a proof in the \coq\
system. This can be necessary to perform efficient unreachability proofs on
rewriting systems in \coq\ using an external completion tool.


\bibliographystyle{alpha}
\bibliography{sabbrev,eureca,genet}

\newpage
\versioncourte{
\appendix
\section{Proofs}
%\annexe
%\newpage
\setcounter{theorem}{0}
\setcounter{lemma}{0}
\begin{theorem}{(Termination)}
  At each deduction step, the measure decreases strictly:
  \[
  \dfrac{
    \Gamma \vdash_{\A, \B} \alpha \Subset \beta
  }{
    \Gamma' \vdash_{\A, \B} \alpha' \Subset \beta'
  }
  \Longrightarrow \mu({\Gamma \vdash_{\A, \B} \alpha \Subset \beta}) \ll \mu({\Gamma' \vdash_{\A, \B} \alpha' \Subset \beta'})
  \]
\end{theorem}
\begin{proof}
  The following array summarizes for each derivation rule what component of the tuple
  proves that $\mu$ decreases between conclusion and premises of the rule:
  
  \[\begin{array}[h]{l|c|c|}%c|}
    \footnotesize
    & \mu_1 & \mu_2 \\ \hline % & \mu_3\\ \hline
    \textrm{Induction}   & \strut \mu_1(\Gamma) < \mu_1(\Gamma') & - \\ \hline
    \textrm{Split-l} & \mu_1(\Gamma) = \mu_1(\Gamma')
                     & \mu_2(c_i) = 1 < \mu_2(\{c_i\}_1^n)\\ \hline
    \textrm{Weak-r} & \mu_1(\Gamma) = \mu_1(\Gamma') 
                    & \mu_2(c_k) < \mu_2(\{c_i\}_1^n)\\ \hline
    \textrm{Config} & \mu_1(\Gamma) = \mu_1(\Gamma') & 
    \begin{array}{l}
      \mu_2(f(\dots, q_i, \dots)) = \mu_2(f(\dots, q'_i, \dots)) = 1\\
      \mu_2(q_i) = \mu_2(q'_i) = 0 \textrm{ thus } 2 > 0
    \end{array}\\ \hline
  \end{array} \]
  For the Split-l (resp. Weak-r) rule, we consider $n > 1$ to have a set $\alpha = \{c_i\}_1^n$ (resp. $\beta$) with
  at least two elements. If ($n = 1$) then this rule does not apply on the current statement
  ${\Gamma \vdash \alpha \Subset \beta}$.\\

\end{proof}

\begin{theorem}{(Cut in $\Subset$-proof trees)}
  \label{thm:cut}
  For all tree automata $\A$ and $\B$, if there exists $\prod$ a proof tree
  of $\Gamma \vdash_{\A, \B} q \Subset q'$, and a proof tree of 
  $\Gamma \cup \{q \Subset q'\}\vdash_{\A, \B} q_a \Subset q_b$
  then exists also a proof tree of $\Gamma \vdash_{\A, \B} q_a \Subset q_b$.
\end{theorem}
\begin{proof}
  We proceed by induction on $\mu(\Gamma)$.

  \noindent
  If $\mu(\Gamma) = 0$, we have immediately $\Q_\A \times \Q_\B =
  \Gamma$. Hence,
  since $q_a \Subset q_b \in \Gamma$, 
  we can prove $\Gamma \vdash_{\A,\B} q_a \Subset q_b$ using the Axiom rule.

  \medskip
  \noindent
  Now, as induction hypothesis, let us assume that $\forall \Gamma\ 
  s.t.\ \mu(\Gamma) = n$, $\forall q\ q'$, if there exists a proof tree $\prod$ of
  $\Gamma \vdash_{\A, \B} q \Subset q'$ and if for all $q_a, q_b$ there exists a
  proof tree of $\Gamma \cup \{q \Subset q'\}\vdash_{\A, \B} q_a \Subset q_b$ then we
  have also a proof tree of $\Gamma \vdash_{\A, \B} q_a \Subset q_b$.  Now, we
  aim at proving that this property is true for $\Gamma$ such that $\mu(\Gamma)
  = n+1$.
  
  \medskip
  \noindent
  Let us consider the proof tree of the second hypothesis $\Gamma \cup \{q \Subset
  q'\}\vdash_{\A, \B} q_a \Subset q_b$.  Firstly, if the proof tree is built
  using the Axiom rule we have $(q_a \Subset q_b) \in \Gamma \cup \{(q \Subset
  q')\}$.  Two cases are possible:
  \begin{itemize}
  \item either $(q_a \Subset q_b) \in \Gamma$, and then we build the proof of 
    $\Gamma \vdash_{\A,\B} q_a \Subset q_b$ using the Axiom rule.

  \item or $q = q_a$ and $q' = q_b$, and then the goal $\Gamma \vdash_{\A,\B} q_a
    \Subset q_b$ is equivalent to $\Gamma \vdash_{\A, \B} q \Subset q'$ whose
    proof tree is $\prod$.
  \end{itemize}

  \noindent
  Secondly, if the proof tree of $\Gamma \cup \{q \Subset q'\}\vdash_{\A, \B} q_a
  \Subset q_b$ is built using the Induction rule, then we have: 

\medskip
  \newcommand{\env}{\Gamma \cup \{q_a \Subset q_b\} \cup \{q \Subset q'\} \vdash_{\A, \B}}

\centerline{
  \begin{minipage}{21cm}
    {\tiny
      \begin{prooftree}
        \AxiomC{\small $\prod_{c_1}$}
        \UnaryInfC{$\env c_1 \Subset c'_{k_1}$}
        \LeftLabel{(Weark-r)}
        \UnaryInfC{$\env 
          c_1 \Subset  \{c_k'| c_k' \rightarrow_\B q_b\}_1^m$}
        % Pointillets du milieu
        \AxiomC{\small \dots\dots}
        \AxiomC{\small $\prod_{c_n}$}
        \UnaryInfC{$\env c_n \Subset c'_{k_n}$}
        \RightLabel{(Weark-r)}
        \UnaryInfC{$\env
          c_n \Subset  \{c_k'| c_k' \rightarrow_\B q_b\}_1^m$}
        \LeftLabel{(Split-l)}
        \TrinaryInfC{$\env
          \{c_i| c_i \rightarrow_\A q_a\}_1^n \Subset  \{c_i'| c_i'\rightarrow_\B q_b\}_1^m$}
        \LeftLabel{(Induction)}
        \UnaryInfC{$\Gamma \cup \{q \Subset q'\} \vdash_{\A, \B} q_a \Subset q_b$}
      \end{prooftree}
    }
  \end{minipage}}

\medskip
    \noindent
    Where each $\prod_{c_i}$ has the following form (assuming
    that $c_i = f(q_{i_1}, \dots, q_{i_n})$ and $c'_{k_i} = f
    (q'_{i_1}, \dots, q'_{i_n})$ ):
    
    {\tiny
      \begin{prooftree}
        \AxiomC{\small $\prod_{i_1}$}
        \UnaryInfC{$\env q_{i_1} \Subset q'_{i_1}$}
        \AxiomC{\small \dots\dots}
        \AxiomC{\small $\prod_{i_n}$}
        \UnaryInfC{$\env q_{i_n} \Subset q'_{i_n}$}
        \LeftLabel{(Config)}
        \TrinaryInfC{$\env f(q_{i_1}, \dots, q_{i_n}) \Subset f (q'_{i_1}, \dots, q'_{i_n})$}
      \end{prooftree}
    }
    
    If we try to build the proof tree of our goal $\Gamma \vdash_{\A,\B} q_a
    \Subset q_b$, it necessarily begins in the same way except that $\{q \Subset
    q'\}$ will not appear in the left-hand side of statements. Each branch of
    this tree will end by a statement of the form $\Gamma \cup \{q_a \Subset
    q_b\} \vdash_{\A, \B} q_{i_j} \Subset q'_{i_j}$. Now to conclude the proof,
    we have to find proof trees $\prod'_{i_j}$ for all those statements. We know
    that there exists proof trees $\prod_{i_j}$ for all statements $\Gamma \cup \{q
    \Subset q'\} \cup \{q_a \Subset q_b\} \vdash_{\A, \B} q_{i_j} \Subset
    q'_{i_j}$.  We can use the induction hypothesis on $\prod_{i_j}$ to obtain
    $\prod'_{i_j}$ as follows:
    \begin{itemize}

    \item Since $\mu(\Gamma) = n + 1$, then $\mu(\Gamma\cup \{q_a \Subset q_b\}) = n$
      

    \item Since $\prod$ is a proof of $\Gamma \vdash_{\A, \B} q \Subset q'$, it is also a proof of
      $\Gamma \cup \{q_a \Subset q_b\}\vdash_{\A, \B} q \Subset q'$.

    \item Each $\prod_{i_j}$ is a proof of $\env  q_{i_j} \Subset q'_{i_j}$
    \end{itemize}

    \noindent
    Using induction, we deduce that for all $i,j$ there exist proof trees
    $\prod'_{i_j}$ of $\Gamma \cup \{q_a \Subset q_b\} \vdash_{\A, \B} q_{i_j}
    \Subset q'_{i_j}$.  This ends the proof tree of our goal $\Gamma \vdash_{\A,
      \B} q_a \Subset q_b$.
  \end{proof}



\begin{theorem}{(Soundness)}
  \label{thm:soundness}
  For all tree automata $\A$ and $\B$, if there exists $\prod$ a proof tree
  of $\emptyset \vdash_{\A, \B} q \Subset q'$ then we have $\Lang(\A, q) \subseteq \Lang(\B, q')$
\end{theorem}

\begin{proof}
  We prove that $\forall t$, $t \in \Lang(\A, q) \imp t \in \Lang(\B, q')$ by
  induction on $t$. Let $t = f(t_1, \dots, t_n)$. We assume that the property is
  true for each subterm $t_i$, i.e. for all $q_i, q'_i$ s.t. if there exists a
  proof tree $\prod_i$ of $\emptyset \vdash_{\A, \B} q_i \Subset q'_i$ then $t_i
  \rightarrow^*_{\A} q \imp t \rightarrow^*_{\B} q_i'$.  Since $t=f(t_1, \dots,
  t_n) \in \Lang(\A, q)$, then for each subterm $t_i$, we know that there exists
  $q_1, \ldots, q_n$ such that $t_i \in \Lang(\A, q_i)$ and $f(q_1, \dots, q_n)
  \rightarrow q \in \A$. Besides this, by unfolding $\prod$ the proof tree of
  $\emptyset \vdash_{\A, \B} q \Subset q'$, we can deduce that for each
  transition like $f(q_1, \dots, q_n) \rightarrow q \in \A$, there exists $f(q'_1,
  \dots, q'_n) \rightarrow q' \in \B$ s.t. we have a proof tree $\prod_i$ of
  $\{q \Subset q'\} \vdash_{\A, \B} q_i \Subset q'_i$. 
  Since $f(q_1, \ldots, q_n) \rw q \in \A$, we obtain that $f(q'_1, \dots, q'_n)
  \rightarrow q' \in \B$ and a proof $\prod_i$ for $\{q \Subset q'\} \vdash_{\A,
    \B} q_i \Subset q'_i$. To conclude that $f(t_1, \dots, t_n) \in \Lang(\B,
  q')$ we just have to prove that $t_i \in \Lang(\B, q'_i)$. Note that we have a
  proof tree $\prod_i$ for $\{q \Subset q'\} \vdash_{\A,\B} q_i \Subset q'_i$
  and that to apply the induction hypothesis we need a proof tree for $\emptyset
  \vdash_{\A,\B} q_i \Subset q'_i$. Using the Theorem~\ref{thm:cut} on $\prod$
  and $\prod_i$, we can deduce the existence of $\prod'_i$ the proof tree of
  $\emptyset \vdash_{\A, \B} q_i \Subset q'_i$. Then using induction hypothesis
  on $t'_i, q_i, q'_i$ and $\prod'_i$,
  we obtain that for each $t_i \in \Lang(\A, q_i)$, we also have $t_i \in
  \Lang(\B, q'_i)$. Finally, since $f(q'_1, \ldots, q'_n)\rw q' \in \B$, we
  obtain that $t=f(t_1, \dots, t_n) \in \Lang(\B, q')$.
  
\end{proof}

\annexe{
We need to extend the renaming $\varrho$ to the structures and sets used in the following:
\begin{itemize}
\item $\varrho(\{q_i\}_1^n)$ stands for $\{\varrho(q_i)\}_1^n$

\item $\varrho(c) = \left\{\begin{array}{ll}
      f (\varrho(c_1), \dots, \varrho(c_n)) &\textrm{ if } c = f(c_1, \dots, c_n)\\
      c &\textrm{ if } c \in \F_{0}\\
      \varrho(q) &\textrm{ if } c = q\in \Q\\
    \end{array}\right.$
  
\item $\varrho(c \rightarrow q)$ stands for $\varrho(c) \rightarrow \varrho(q)$

  
\item $\varrho(\Delta)$ stands for $\{ \varrho(c\rightarrow q)\ |\ c \rightarrow q \in \Delta \}$.
\end{itemize}
}
\begin{lemma}
\label{lem:renaming}
  Given a tree automaton $\A$,
  
  \[
  \begin{array}{llll}
    1. & \mbox{ if } \A' = \A \cup \norm(r\sigma \rw q) & \mbox{ then } &\A \sqsubseteq \A' \\
    2. & \mbox{ if } \A' = \merge(\A, q_1,q_2) &\mbox{ then }& \A \sqsubseteq \A'  \\
  \end{array}
  \]
\end{lemma}

\begin{proof}
  \begin{enumerate}
  \item This is easy to show since we trivially have $\Delta_{\A'} \supseteq
    \Delta_\A$ whatever $r\sigma$ or $q$ may be.
    Then by choosing $\varrho = id$, we have immediately the conclusion $\A
    \sqsubseteq \A'$.
    
  \item Let $\Delta_\A$ be the transition set of $\A$. Let $q_i$ and $q_j$ be
    the two states to merge.  We can apply to $\Delta_\A$
    a renaming function $\varrho$ which has the same behavior than state merging
    with regard to $q_1 = q_2$:
    \[
    \varrho(q) = \left\{
      \begin{array}{l}
        \textrm{if }(q\ =\ q_2)\\
        \quad \quad q_1\\
        \textrm{else}\\
        \quad \quad q
      \end{array}\right.
    \]
    
    So state merging builds $\Delta_{\A'} = \varrho(\Delta_\A)$ and by
    Definition~\ref{eq:prop0} we have trivially $\Delta_\A \sqsubseteq
    \Delta_{\A'}$.
  \end{enumerate}
\end{proof}


\begin{theorem}
  Given a tree automaton $\mathcal{A}^0$, a TRS $\R$ and an
  equation set $\mathcal{E}$, after $k$ completion steps we obtain
  $\A_\R^k$ such that $\A^0 \sqsubseteq \A_\R^k$.
\end{theorem}

\begin{proof}
  By induction on $k$:
  
  \begin{itemize}
  \item Since $\sqsubseteq$ is reflexive, we have trivially $\A^0
    \sqsubseteq \A^0$.
  \item Let $\mathcal{A}_k$ be a tree automaton obtained after $k$ completion
    steps such that $\A^0 \sqsubseteq \A_\R^k$. By definition of completion
    $\A_\R^{k+1}$ is built from $\A_\R^k$ by applying successively normalization
    and merge. Using Lemma~\ref{lem:renaming}, we have $\A_\R^k \sqsubseteq
    \A_\R^{k+1}$.  By transitivity of $\sqsubseteq$, from $\A^0 \sqsubseteq
    \A_\R^k$ and $\A_\R^k \sqsubseteq \A_\R^{k+1}$ we deduce immediately that
    $\A^0 \sqsubseteq \A_\R^{k+1}$.
  \end{itemize}
\end{proof}

\begin{theorem}{(Completeness)}
  Given two tree automata $\A$ and $\B$ if $\A \sqsubseteq \B$ then it
  exists $\prod$ a proof of statement $\emptyset \vdash_{\A, \B}
  \{\#(q_f)\ |\  q_f \in \mathcal{Q}_{F_A}\} \Subset  \{\#(q'_f)\ |\  q'_f \in \mathcal{Q}_{F_B}\}$.
\end{theorem}

\begin{proof}
  By definition of $\A \sqsubseteq \B$, we can deduce that there exists a
  renaming $\varrho$ s.t. $\varrho(\Delta_\A) \subseteq \varrho(\Delta_\B)$ and
  $\varrho(\Q_{F_\A}) \subseteq \varrho(\Q_{F_\B})$. The proof is done by
  induction on $\mu_1(\Gamma)$. Recall that $\mu_1(\Gamma)= |\Q_\A \times \Q_\B|
  - |\Gamma|$. Assuming that $\A \sqsubseteq \B$, we want to prove that for all
  $q$ there exists a proof tree $\prod$ for $\Gamma \vdash_{\A,\B} q \Subset
  \varrho(q)$. 
  We assume that $\forall q_i$, there exists a proof tree $\prod_i$
  of $\Gamma \cup \{ q \Subset \varrho(q) \} \vdash_{\A, \B} q_i \Subset
  \varrho(q_i)$. Now, we aim at proving that there exists a proof tree for
  the statement: $\Gamma \vdash_{\A, \B} q \Subset \varrho(q)$.

  \begin{itemize}
  \item if $q \Subset \varrho(q) \in \Gamma$ then the proof tree
    is trivial:
  \begin{prooftree}
    \AxiomC{}
    \LeftLabel{(Axiom)}
    \UnaryInfC{$\Gamma  \vdash_{\A, \B} q \Subset \varrho(q)$}
  \end{prooftree}
   
\item if $q \Subset \varrho(q) \not \in \Gamma$ then we need to
  apply Induction rule to obtain the following tree:
  
    \newcommand{\env}{\Gamma \cup \{q \Subset \varrho(q)\} \vdash_{\A, \B}}
    {\tiny
      \begin{prooftree}
        \AxiomC{\small $\prod_{c_1}$}
        \UnaryInfC{$\env 
          c_1 \Subset  \{c_k'| c_k' \rightarrow \varrho(q)\}_1^m$}
        % Pointillets du milieu
        \AxiomC{\small \dots\dots}
        \AxiomC{\small $\prod_{c_n}$}
        \UnaryInfC{$\env
          c_n \Subset  \{c_k'| c_k' \rightarrow \varrho(q)\}_1^m$}
        \LeftLabel{(Split-l)}
        \TrinaryInfC{$\env
          \{c_i| c_i \rightarrow q\}_1^n \Subset  \{c_k'| c_k'\rightarrow \varrho(q)\}_1^m$}
        \LeftLabel{(Induction)}
        \UnaryInfC{$\Gamma \vdash_{\A, \B} q \Subset \varrho(q)$}
      \end{prooftree}
    } From hypothesis $\varrho(\Delta_\A) \subseteq \Delta_\B$ for each rule $c
    \rightarrow q$ of $\Delta_A$, we have $\varrho(c\rightarrow q) \in
    \Delta_\B$. Thus for all $(c\rightarrow q)\in \Delta_\A$, we have
    $\varrho(c) \in \{c_k' | c_k' \rightarrow \varrho(q)\}_1^m$.  Then for each
    $c_i$, the proof tree $\prod_{c_i}$ is built in a similar way. Let us detail
    it for a particular $c_i= f(q_{i_1}, \dots, q_{i_n})$. We can construct the
    corresponding tree $\prod_{c_i}$ whose proof tree is concluded by
    $\prod_{i_j}$ an instance of induction hypothesis for the corresponding
    state $q_{i_j}$: {\tiny
      \begin{prooftree}
        \AxiomC{\small $\prod_{i_1}$}
        \UnaryInfC{$\env q_{i_1} \Subset \varrho(q_{i_1}) $}
        \AxiomC{\small \dots\dots}
        \AxiomC{\small $\prod_{i_n}$}
        \UnaryInfC{$\env q_{i_n} \Subset \varrho(q_{i_n}) $}
        %%%%%%%%%%%%%%%%%%%%%%% 
        \LeftLabel{(Config)}
        \TrinaryInfC{$\env c_i \Subset  \varrho(c_i) $}
        \LeftLabel{(Weak-r)}
        \UnaryInfC{$\env c_i \Subset  \{c_k'| c_k' \rightarrow \varrho(q)\}_1^m$}
      \end{prooftree}
    }
  \end{itemize}

  Now, we have shown that for all $\Gamma$ and $q \in \Q_\A$ there exists a proof
  tree $\prod$ for all statement $\Gamma \vdash_{\A, \B} q \Subset \varrho(q)$.
  In particular, this is true for $\Gamma = \emptyset$ all $q$ of
  $\Q_{F_\A}$.  Since we have $\A \sqsubseteq \B \imp
  \varrho(\Q_{F_\A}) \subseteq \Q_{F_\B}$, we can build a proof tree
  as:

  {\small
    \begin{prooftree}
      \AxiomC{$\prod_{q_{f_1}}$}
      \UnaryInfC{$\emptyset \vdash_{\A, \B} q_{f_1} \Subset \varrho(q_{f_1})$}
      % \AxiomC{$\emptyset \vdash_{\A, \B} [\ ] \Subset [\ ]$}
      \LeftLabel{(Config)}
      \UnaryInfC{$\emptyset \vdash_{\A, \B} \#(q_{f_1}) \Subset \#(\varrho(q_{f_1}))$}
      \LeftLabel{(Weak-r)}
      \UnaryInfC{$\emptyset \vdash_{\A, \B} q_{f_1} \Subset \Q_{F_\B}$}
      \AxiomC{\small \dots\dots}
      \AxiomC{$\prod_{q_{f_n}}$}
      \UnaryInfC{$\emptyset \vdash_{\A, \B} q_{f_1} \Subset \varrho(q_{f_1})$}
      % \AxiomC{$\emptyset \vdash_{\A, \B} [\ ] \Subset [\ ]$}
      \RightLabel{(Config)}
      \UnaryInfC{$\emptyset \vdash_{\A, \B} \#(q_{f_n}) \Subset \#(\varrho(q_{f_n}))$}
      \RightLabel{(Weak-r)}
      \UnaryInfC{$\emptyset \vdash_{\A, \B} \#(q_{f_n}) \Subset \{\#(q)\ |\ \Q_{F_\B}\}$}
      \LeftLabel{(Split-l)}
      \TrinaryInfC{$\emptyset \vdash_{\A, \B} \{\#(q)\ |\ q \in \Q_{F_\A}\} \Subset \{\#(q)\ |\ q \in \Q_{F_\B}\}$}
    \end{prooftree}}
\end{proof}



%%% Local Variables: 
%%% mode: latex
%%% TeX-master: "main"
%%% End: 
}
\end{document}
