\documentclass[a4paper,10pt]{llncs}
\usepackage[latin1]{inputenc}
\usepackage[english]{babel}
\usepackage[curve]{xypic}
\usepackage{stmaryrd}
\usepackage{amsmath, amssymb, amsfonts}
\usepackage{bussproofs}
\setlength{\oddsidemargin}{15mm}
\setlength{\evensidemargin}{15mm} 
\setlength{\textwidth}{135mm}
\setlength{\textheight}{214mm} 
\setlength{\topmargin}{5mm}

%\newtheorem{theorem}{Theorem}
%\newtheorem{definition}{Definition}
%\newtheorem{proposition}{Proposition}
%\newtheorem{example}{Example}
%\newtheorem{lemma}{Lemma}
%\newtheorem{corollary}{Corollary}


%\newcommand{\article}[1]{#1}
%\newcommand{\report}[1]{}

\newcommand{\nr}{N}
\newcommand{\E}{\mathcal{E}}
\newcommand{\T}{\mathcal{T}}
\newcommand{\A}{\mathcal{A}}
\newcommand{\Q}{\mathcal{Q}}
\newcommand{\f}{\rightarrow}
\newcommand{\fg}{\leftarrow}

\newcommand{\versionlongue}[1]{}
\newcommand{\versioncourte}[1]{#1}
\newcommand{\annexe}[1]{}
\newcommand{\archive}[1]{}

\newcommand{\rwD}{\rightarrow^*_{\Delta}}

\def \N {\mathbb{N}}
\def \norm {Norm}
\def \merge {Merge}

\def \R {\mathcal{R}}
\def \card {{\rm \mathcal{C}ard}}
\newcommand{\desc}{\R^*}



\newcommand{\mytt}{\tt \small}

%-------- Symboles speciaux
\newcommand{\lmulti}{\{\!\!\{}
\newcommand{\rmulti}{\}\!\!\}}
\newcommand{\timbuk}{{\sf Timbuk}}
%Benoit : \coq commande
\newcommand{\coq}{{\sf Coq}}
\newcommand{\relU}{\mathrel{\mbox{U}}}
\newcommand{\bracl}{\llbracket}
\newcommand{\bracr}{\rrbracket}
\newcommand{\ncirc}[1]{\rlap{$\bigcirc$}
                        \hspace*{1mm}\rlap{{\small
                                #1}}\hspace*{2.2mm}}
 

\newcommand{\uinter}[1]{\vspace*{-2mm}{\begin{alltt} {\scriptsize #1} \end{alltt}}\vspace*{-2mm}}
\newcommand{\finter}[1]{\vspace*{-2mm}{
        \begin{alltt}
        {\scriptsize
        \input{#1}}
        \end{alltt}}\vspace*{-2mm}}


\newfont{\amstoto}{msbm10}
\newcommand{\NN}{\mbox{\amstoto\char'116}}
\newcommand{\ZZ}{\mbox{\amstoto\char'132}}
\newcommand{\arity}{ar}
\newcommand{\Sub}{{\cal S}}
\newcommand{\F}{\mathcal{F}}
\newcommand{\X}{{\cal X}}
\newcommand{\Y}{{\cal Y}}
\newcommand{\C}{{\cal C}}
\newcommand{\D}{{\cal D}}
\newcommand{\TF}{{\cal T(F)}}
\newcommand{\TFX}{{\cal T(F, X)}}
\newcommand{\TFQ}{{\cal T(F \cup Q)}}
\newcommand{\TFQp}{{\cal T(F \cup Q')}}
\newcommand{\TFQX}{{\cal T(F \cup Q, X)}}
\newcommand{\TFXQ}{{\cal T(F, X \times Q)}}
\newcommand{\TC}{{\cal T(C)}}
\newcommand{\Qf}{{\cal Q}_f}
\newcommand{\aut}{\langle \F, \Q, \Qf, \Delta \rangle} 
\newcommand{\B}{{\cal B}}
\newcommand{\ordlexico}{\prec}
\newcommand{\bottom}{\perp}
\newcommand{\match}{\unlhd}

\newcommand{\fleche}{\mapsto}
\newcommand{\send}{\hookrightarrow}
\newcommand{\cale}[1]{\vrule height #1 depth 5pt width 0pt}
\newcommand{\deduc}[2]{$ \textstyle #1 \mbox{\cale{0mm}} \over
{\textstyle #2 \mbox{\cale{5mm}}}$}

\newcommand{\Decompose}{{\bf Decompose}}
\newcommand{\Clash}{{\bf Clash}}
\newcommand{\Configuration}{{\bf Configuration}}
\newcommand{\Distributivity}{{\bf Distributivity}}
\newcommand{\Simplifyone}{{\bf Simplify~1}}
\newcommand{\Simplifytwo}{{\bf Simplify~2}}


\newcommand{\mycomment}[1]{}
\newcommand{\et}{\mbox{ and }}
\newcommand{\st}{\mbox{ s.t. }}
\newcommand{\oute}{\mbox{ or }}
\newcommand{\ou}{\; \vee \;}

\newcommand{\spf}{\;\; \Longrightarrow \;\;}
\newcommand{\dbf}{\;\; \Longleftrightarrow \;\;}
\newcommand{\imp}{\; \Longrightarrow \;}
\newcommand{\sep}{\; | \;}
\newcommand{\dsep}{\; || \;}




\def\build#1_#2^#3{\mathrel{
    \mathop{\kern 0pt#1}\limits_{#2}^{#3}}}

\def\CM#1{\build\hbox to 10mm {\rightarrowfill}_{}^{CM#1}}

%  #2
% --->
%  #1   #3


\newcommand{\rwl}[3]{\mathrel{{\build{\longrightarrow}_{#1}^{#2}}\mskip-2mu_{#3}}}
\newcommand{\rws}[3]{\mathrel{{\build{\rightarrow}_{#1}^{#2}}\mskip-2mu_{#3}}}
% Benoit : incompatible avec le package amsmath.sty
%\newcommand{\xrightarrow}[1]{\rwl{}{#1}{}}


%-------- Des fleches de reecriture...


\newcommand{\rw}{\rightarrow}
\newcommand{\lrwr}{\mathrel{\mathrel{_{\R}}\joinrel\leftarrow}}
\newcommand{\lrwA}{\mathrel{\mathrel{_{\A}}\joinrel\leftarrow}}
\newcommand{\rwR}{\rws{}{}{\R}}
\newcommand{\rwA}{\rws{}{}{\A}}
\newcommand{\rwB}{\rws{}{}{\B}}


\newcommand{\pos}{{\cal P}os}
\newcommand{\posf}{\pos_{\F}} 
\newcommand{\posx}{\pos_{\X}} 
\newcommand{\var}{{\cal V}ar}
\newcommand{\ddom}{{\cal D}om}
\newcommand{\ran}{{\cal R}an}
\newcommand{\Root}{{\cal R}oot}
\newcommand{\Card}{{\cal C}ard}

%\newcommand{\calC}{{\cal C}}
%\newcommand{\calD}{{\cal D}}
%\newcommand{\calP}{{\cal P}}

%\newcommand{\kk}{\leadsto}
%\newcommand{\match}{\unlhd}
%\newcommand{\Cterm}{Q-term}
%\newcommand{\Cterms}{Q-terms}
\newcommand{\Qsubst}{$\Q$-substitution}
\newcommand{\Qsubsts}{$\Q$-substitutions}
%\newcommand{\arw}{\leadsto}

\newcommand{\Norm}{Norm}
\newcommand{\States}{States}

%\newcommand{\infini}{\omega}
\newcommand{\comp}{{\cal C}_{\nr, \R}}
\newcommand{\aapprox}{{\cal A}_{\nr, \R}}
\newcommand{\aaex}{{\cal A}_{\R}}


\newcommand{\Lang}{{\cal L}}

\newcommand{\exskip}{\medskip}


%%% Local Variables: 
%%% mode: latex
%%% TeX-master: "main"
%%% End: 

%All i need to have fun with Caml and Coq :)
\RequirePackage{alltt}
\RequirePackage{listings}
\def\lstlanguagefiles{lstlangcoq.sty,lstlangcaml.sty}
\lstloadlanguages{Coq,Caml} %Caml
\newcommand{\switchlstcoq}{
\lstset{language=Coq,flexiblecolumns=false,mathescape=true}
\lstset{keywordstyle={\bfseries}}
\lstset{commentstyle=\it,basicstyle=\small\tt}%,keywordstyle=\bfseries}
\lstset{literate={=>}{{$\Rightarrow$}}1
{\\Abstate}{{\Abstate}}1
{forall}{{$\forall$\hspace{-1ex}}}1
{exists}{{$\exists$\hspace{-1ex}}}1
{not}{{$\neg$}}1
{<=<}{{$\subseteq$}}2
{>=}{{$\ge$}}1
{>->}{{$\rightarrowtail$}}2{->}{{$\to$}}2 {/\\}{{$\land$}}1 {~}{{$<AC>$}}1}
\lstset{escapeinside={(*@}{@*)}} % In the text as an example:
}

\renewcommand{\ttdefault}{pcr}


\begin{document}
\versioncourte{\title{Certifying a Tree Automata Completion Checker\thanks{For
      reviewer's convenience, proofs of~\cite{BoyerGJ-RR08} are given as an
      appendix.}}}  \versionlongue{\title{Certifying a Tree Automata Completion
    Checker}} \author{Beno�t Boyer \and Thomas Genet \and Thomas Jensen}

\institute{
  IRISA / Universit\'e de Rennes\,1 / CNRS \\
  Campus de Beaulieu \\
  F-35042 Rennes Cedex \\
  {\tt {bboyer,genet,jensen}@irisa.fr} 
}
\maketitle


\begin{abstract}
  Tree automata completion is a technique for the verification of infinite state
  systems. It has already been used for the verification of cryptographic
  protocols and the prototyping of Java static analyzers.  However, as for many
  other verification techniques, the correctness of the associated tool becomes
  more and more difficult to guarantee. It is due to the size of the
  implementation that constantly grows and due to optimizations which
  are necessary to scale up the efficiency of the tool to verify real-size
  systems. In this paper, we define and develop a checker for tree automata
  produced by completion. The checker is defined using \coq\ and its
  implementation is automatically extracted from its formal specification. Using
  extraction gives a checker that can be run independently of the \coq\
  environment. A specific algorithm for tree automata inclusion checking have
  been defined so as to avoid the exponential blow up. The obtained checker is
  certified in \coq, independent of the implementation of completion, usable
  with any approximation performed during completion, small and fast. Some
  benchmarks are given to show how efficient the tool is.
\end{abstract}

%each file contains a section :


\section{Introduction}
\label{sec:introduction}


\versionlongue{
In the field of infinite system verification, three of the most successful
techniques are: assisted proof, abstract interpretation and symbolic/abstract
model-checking. In all those techniques, the verification relies on specific
softwares that may, themselves, contain bugs. On the one hand, a proof assistant
like \coq~\cite{coqart} avoids this problem since any proof, on any exotic domain, is
finally checked using a small unique certified kernel.  On the other hand, proof
assistants like \coq\ offer poor automation when compared with fully
automatic tools like static analyzers or model-checkers. However, the efforts
achieved on the two last ones, to obtain a better automation and efficiency, have a
great impact on their reliability. Static analyzers and model-checkers are
usually huge, drastically optimized programs whose safety is not proved but
essentially ``demonstrated'' by an extensive use.
}

Static program analysis is one of the cornerstones of software
verification and is increasingly used to protect computing devices
from malicious or mal-functioning code. However, program verifiers are
themselves complex programs and a single error may jeopardize the
entire trust chain of which they form part. 
Efforts have been made to certify static
analyzers~\cite{KleinN-TCS03,BartheD-FASE04,CacheraJPR-TCS05} or to certify the results obtained by static
analyzers~\cite{LetouzeyT-TPHOL00,BessonJP-TCS06} in \coq\ in order to increase confidence
in the analyzers. 
%The latter is, in fact, more related
%to proof carrying code~\cite{Necula-POPL97}, where a complex untrusted
%verification software produce a certificate that is checked, afterwards, by a
%small trusted checker.
In this paper, we instantiate the general framework used
in~\cite{BessonJP-TCS06} to the particular case of analyzing term
rewriting systems by  tree automata
completion~\cite{Genet-RTA98,FeuilladeGVTT-JAR04}. Given a term rewriting
system, the tree automata completion is a technique for over-approximating the
set of terms reachable by rewriting in order to prove the
unreachability of certain ``bad'' states that violates a given
security property. This technique has already been used to
prove security properties on cryptographic protocols~\cite{GenetK-CADE00},
\cite{GenetTTVTT-wits03,BoichutHKO-AVIS04,avispa,ZuninoD-FOSSACS06} and, more
recently, to prototype static analyzers on Java byte
code~\cite{BoichutGJL-RTA07}.

In this paper, we show how to mechanize the proof, within the proof
assistant \coq, that the tree automaton produced by completion
recognizes an over-approximation of all reachable terms. 
 \coq\ is based on constructive logic (Calculus of
Inductive Constructions) and it is possible to 
%Thanks to Curry-Howard isomorphism, proofs and
%functionnal programs are very closed: if we have the proof that there exists an
%algorithm solving a problem, \coq\ is abble to 
extract an Ocaml or Haskell
function implementing exactly the algorithm whose specification has
been 
expressed in \coq. The extracted code is thus a {\em certified}
implementation of the specification given in the \coq\ formalism.  Extracted
programs are standalone and do not require the \coq\ environment to be
executed. For details about the extraction 
%and reflection 
mechanisms, readers can refer to~\cite{coqart}.

%Our objective could be to specify formally the full completion and extract a
%certified algorithm from the \coq\ specification. However, the more complex the
%algorithm, the harder is the proof. 
A specific challenge in the work reported here has been how to marry
constructive logic and efficiency. Previous case studies with tree
automata completion, on cryptographic
protocols~\cite{GenetTTVTT-wits03} and on Java
bytecode~\cite{BoichutGJL-RTA07} show that we need an efficient
completion algorithm to verify properties of real models. 
% the implementation of completion has to be drastically optimized. 
For instance, the current implementation of completion (called
\timbuk~\cite{timbuk-site}) is based on imperative data structures like hash
tables whereas \coq\ allows only pure functional structures.  A second problem
is the termination of completion. Since \coq\ can only deal with total
functions, functions must be proved terminating for any
computation. In general, such a property cannot be guaranteed on completion
because it mainly depends on term rewriting system and approximation equations
given initially.

For these two reasons, there is little hope to specify and certify an
efficient and purely functional version of the completion
algorithm. Instead, we
have adopted a solution based on a result-checking approach.  
% the "Proof Carrying Code" approach\cite{Necula-POPL97}. 
It consists of building a smaller program (called the
\emph{checker}) - certified in \coq\ -  that checks if the tree automaton computed by \timbuk\ is
sound. In this paper, we restrict to the case of left-linear term rewriting
systems which revealed to be sufficient for verifying Java
programs~\cite{BoichutGJL-RTA07}. However, a checker dealing with general term
rewriting systems like completion does in~\cite{FeuilladeGVTT-JAR04} is under
development.

%
%
\archive{Note that this objective is similar
to the one of Roberto Zunino in~\cite{Zunino}. However, our we aim at tackling
this goal in a different way. Whereas~\cite{Zunino} produce the text of a \coq\
proof certifying a specific result, we extract a certified efficient checker (in
Ocaml~\cite{Ocaml}) from the \coq\ specification using the \coq\ extraction
mechanism.
%
%
Our aim is to extract a certified efficient checker (in Ocaml~\cite{Ocaml}) from
the specification using the \coq\ extraction mechanism~\cite{coqart}.  As a
result, a lot of efforts have to be payed to the definition of formal {\em and}
efficient data structures: terms, tree automata, etc. This is crucial
for verification of the Java byte code for instance because the tree
automata to certify are huge. However, those efforts are worth since, once
extracted, the certified Ocaml checker is fully independent of the \coq\
environment: easier to run, less greedy with memory and faster.
}
%
%
\archive{This has several consequences.  First, a lot of effort has to be
  payed to the definition of formal {\em and} efficient data structures: terms,
  tree automata, etc. This is crucial because, for verification purposes like
  the Java byte code for instance, the tree automata to certify are
  huge. Second, since the obtained checker is an Ocaml program, it is no longer
  executed in \coq., which is a big difference with~\cite{zunino}. This, again,
  improves the efficiency but not only. In~\cite{zunino}, given a tree
  automaton, the tool generates a \coq proof text that has to be verified
  afterward in \coq.  The problem here is that there is no success guarantee:
  the proof text may be rejected by \coq\ (without indicating if it is because the
  property is false or because the proof text is ill-formed), or \coq\ may even
  not terminate on this proof. In our case, the tree
automaton is checked by an Ocaml program that {\em always terminates}, decides
if the tree automaton is a valid fixpoint of tree automata completion and
certify unreachability.  A great advantage of this approach using an external
%
%
The extracted checker, we define here, is able to {\em decide} if the given
tree automaton is a valid fixpoint of tree automata completion, and also certify
unreachability.  A great advantage of using an external checker is that it is
totally independent of the completion tool we use. In particular, the
implementation of the completion tool can be optimized as necessary: as long as
it outputs a tree automaton, this result can be certified by our checker.

}

The closest work to ours is the one done by X.~Rival and
J.~Goubault-Larrecq~\cite{RivalGL-TPHOL01}. They have designed a library to
manipulate tree automata in~\coq\ and proposed some optimized formal data
structures that we reuse. However, we aim at dealing with larger tree automata
than those used in their benchmarks. Moreover, we need some other tools which
are not provided by the library as for example a specific algorithm to check
inclusion. 
\versionlongue{Inclusion checking may be done using closure operators
(i.e. intersection and complementation) and emptiness checking but, as shown in
Section~\ref{sec:inclusion}, this way of performing inclusion checking has a bad
complexity.}

This paper is organized as follows. Rewriting and tree automata are reviewed in
Section~\ref{sec:preliminaries} and tree automata completion in
Section~\ref{section:completion}. Section~\ref{section:objectives} states the main
functions to define, inclusion and closure test, and the corresponding theorems
to prove. Section~\ref{sec:rewriting} and Section~\ref{sec:automata} give the
\coq\ formalization of rewriting and of tree automata,
respectively. 
The core of the checker consists of two algorithms: an optimized automata inclusion
test, defined in Section~\ref{sec:inclusion}, and a procedure for
checking that an automaton is \emph{closed} under rewriting w.r.t.~a
given term rewriting system, defined in  Section~\ref{sec:closure}. 
An important feature of the inclusion checker is that, while it is
not complete for all tree automata, we can  prove that it is complete
for the class of tree automata generated by the completion algorithm. 
Section~\ref{sec:benchmarks} gives some details about the performances of the
checker in practice. Finally, we conclude and list some on-going research on this
subject. 

%%% Local Variables: 
%%% mode: latex
%%% TeX-master: "main"
%%% End: 

\section{Preliminaries}
\label{sec:preliminaries}

Comprehensive surveys can be found in~\cite{BaaderN-book98} for
rewriting, and in~\cite{TATA,GilleronTison-FI95} for tree automata
and tree language theory.

Let $\F$ be a finite set of symbols, each associated with an arity function, and
let $\X$ be a countable set of variables. $\TFX$ denotes the set of terms, and
$\TF$ denotes the set of ground terms (terms without variables). The set of
variables of a term $t$ is denoted by $\var(t)$. A substitution is a function
$\sigma$ from $\X$ into $\TFX$, which can be extended uniquely to an
endomorphism of $\TFX$. A position $p$ for a term $t$ is a word over $\NN$. The
empty sequence $\epsilon$ denotes the top-most position. The set $\pos(t)$ of
positions of a term $t$ is inductively defined by:
\begin{itemize}
\item $\pos(t)= \{ \epsilon\} $ if $t \in \X$
\item $\pos(f(t_1,\dots,t_n)) = \{ \epsilon \} \cup \{i.p \mid 1 \leq i \leq n
  \et p \in \pos(t_i) \}$
\end{itemize}
If $p \in \pos(t)$, then $t|_p$ denotes the subterm of $t$ at position $p$ and
$t[s]_p$ denotes the term obtained by replacement of the subterm $t|_p$ at
position $p$ by the term $s$. A term rewriting system $\R$ is a set of {\em
  rewrite rules} $l \rw r$, where $l, r \in \TFX$, $l \not \in \X$, and $\var(l)
\supseteq \var(r)$. A rewrite rule $l \rw r$ is {\em left-linear} if each
variable of $l$ (resp. $r$) occurs only once in $l$.  A TRS $\R$ is left-linear
if every rewrite rule $l \rw r$ of $\R$ is left-linear).  The TRS $\R$ induces a
rewriting relation $\rw_{\R}$ on terms whose reflexive transitive closure is
denoted by $\rw^{\star}_{\R}$. The set of $\R$-descendants of a set of ground
terms $E$ is $\desc(E) = \{t \in \TF \sep \exists s \in E \st s \rw^{\star}_{\R}
t \}$.

The {\em verification technique} defined
in~\cite{Genet-RTA98,FeuilladeGVTT-JAR04} is based on $\desc(E)$.  Note that
$\desc(E)$ is possibly infinite: $\R$ may not terminate and/or $E$ may be
infinite. The set $\desc(E)$ is generally not
computable~\cite{GilleronTison-FI95}. However, it is possible to
over-approximate it~\cite{Genet-RTA98,FeuilladeGVTT-JAR04,Takai-RTA04} using
tree automata, i.e. a finite representation of infinite (regular) sets of terms.
In this verification setting, the TRS $\R$ represents the system to verify, sets
of terms $E$ and $Bad$ represent respectively the set of initial configurations
and the set of ``bad'' configurations that should not be reached.  Then, using
tree automata completion, we construct a tree automaton $\B$ whose language
$\Lang(\B)$ is such that $\Lang(\B) \supseteq \desc(E)$. Then if $\Lang(\B)\cap
Bad = \emptyset$ then this proves that $\desc(E)\cap Bad=\emptyset$, and thus
that none of the ``bad'' configurations is reachable. We now define tree
automata.

Let $\Q$ be a finite set of symbols, with arity $0$, called {\em states} such
that $\Q \cap \F= \emptyset$.  $\TFQ$ is called the set of {\em configurations}.
\begin{definition}[Transition and normalized transition]
  \label{def:normalized}
  A {\em transition} is a rewrite rule $c \f q$, where $c$ is a configuration
  i.e. $c \in \TFQ$ and $q \in \Q$. A {\em normalized transition} is a
  transition $c \f q$ where $c = f(q_1, \ldots, q_n)$, $f \in \F$ whose arity is
  $n$, and $q_1, \ldots, q_n \in \Q$.
\end{definition}

% An epsilon transition is a transition of the form $q \f q'$ where $q$ and $q'$
% are states. Any set of transition $\Delta \cup \{q \f q'\}$ can be
% equivalently replaced by $\Delta \cup \{c \f q' \sep c \f q \in \Delta \}$.

\begin{definition}[Bottom-up nondeterministic finite tree automaton]
  A bottom-up nondeterministic finite tree automaton (tree automaton for short)
  is a quadruple $\A= \langle \F, \Q, \Q_F,\Delta \rangle$, where $\Q_F
  \subseteq \Q$ and $\Delta$ is a set of normalized transitions.
\end{definition}

The {\em rewriting relation} on $\TFQ$ induced by the transitions of $\A$ (the
set $\Delta$) is denoted by $\f_{\Delta}$.  When $\Delta$ is clear from the
context, $\f_{\Delta}$ will also be denoted by $\f_{\A}$.
% Similarly, by notation abuse, we will often note $q \in \A$ and $t\f q \in \A$
% respectively for $q \in \Q$ and $t \f q \in \Delta$.

\begin{definition}[Recognized language]
  The tree language recognized by $\A$ in a state $q$ is $\Lang(\A,q) = \{t \in
  \TF \sep t \f^{\star}_{\A} q \}$.  The language recognized by $\A$ is
  $\Lang(\A) = \bigcup_{q \in \Q_F} \Lang(\A, q)$. A tree language is regular if
  and only if it can be recognized by a tree automaton.
  % A state $q$ is a {\em dead state} if $\Lang(\A, q)= \emptyset$.
\end{definition}


 \begin{example}
   Let $\A$ be the tree automaton $\langle \F, \Q, \Q_F, \Delta \rangle$ such
   that $\F=\{f,g,a\}$, $\Q= \{q_0, q_1\}$, $\Q_F=\{q_0\}$ and $\Delta= \{f(q_0)
   \f q_0, g(q_1) \f q_0, a \f q_1 \}$. In $\Delta$ transitions are {\em
     normalized}. A transition of the form $f(g(q_1)) \f q_0$ is not
   normalized. The term $g(a)$ is a term of $\TFQ$ (and of $\TF$) and can be
   rewritten by $\Delta$ in the following way: $g(a) \f_{\Delta} g(q_1)
   \f_{\Delta} q_0$. Note that $\Lang(\A, q_1)= \{a\}$ and $\Lang(\A)=\Lang(\A,
   q_0) = \{g(a),f(g(a)), f(f(g(a))),\ldots\}=\{f^{\star}(g(a))\}$.
   % Note also that $\Lang(\A, q_2)=\emptyset$ since no term of $\TF$ rewrites to
%   % $q_2$, hence $q_2$ is a dead state.
 \end{example}

%%% Local Variables: 
%%% mode: latex
%%% TeX-master: "main"
%%% End: 


% Version light du papier de TACC

% \section{Tree Automata Completion}
%\label{section:completion}

Given a tree automaton $\A$ and a TRS $\R$, the tree automata completion
algorithm, proposed in~\cite{Genet-RTA98,FeuilladeGVTT-JAR04}, computes a \emph{tree complete
automaton} $\aaex^*$ such that $\Lang{}(\aaex^*)=\desc(\Lang{}(\A))$ when it is
possible (for some of the classes of TRSs where an exact computation is
possible, see~\cite{FeuilladeGVTT-JAR04}), and such that $\Lang{}(\aaex^*)
\supseteq \desc(\Lang{}(\A))$ otherwise. 
In this paper, we only consider the exact case.

The tree automata completion with $\varepsilon$-transtions works as follow.
From $\A=\aaex^0$ completion builds a sequence $\aaex^0.\aaex^1\ldots\aaex^k$ of automata such that if
$s\in\Lang{}(\aaex^i)$ and $s\f_{\R} t$ then $t\in\Lang{}(\aaex^{i+1})$. Transitions of $\aaex^i$ are denoted by the set
$\Delta^i \cup \Deps^i$. Since for every tree automaton, there exists a
deterministic tree automaton recognizing the same language, we can assume
that initially $A$ has the following properties:

\begin{property}[$\rwne$ deterministic]
  \label{prop:deterministic}
  If $\Delta$ contains two normalized transitions of the form 
  $f(q_1, \dots, q_n) \rw q$ and $f(q_1, \dots, q_n) \rw q'$, it means $q = q'$. 
  This ensures that the rewriting relation $\rwne$ is deterministic.
\end{property}

\begin{property}
  \label{prop:wellinitial}
  For all state $q$ there is at most one normalized transition $f(q_1, \dots, q_n) \rw q$
  in $\Delta$. This ensures that if we have $t \rwne q$ and $t' \rwne q$ then $t = t'$.
\end{property}

If we find a fixpoint automaton $\aaex^k$ such that $\desc(\Lang{}(\aaex^k)) =
\Lang{}(\aaex^k)$, then we note $\aaex^*=\aaex^k$ 
and we have $\Lang{}(\aaex^*) \supseteq \desc(\Lang{}(\aaex^0))$~\cite{FeuilladeGVTT-JAR04}.
% , or $\Lang{}(\aaex^*)\supseteq
%\desc(\Lang{}(\A))$ if $\R$ is not in one class of~\cite{FeuilladeGVTT-JAR04}.
To build $\aaex^{i+1}$ from $\aaex^{i}$, we achieve a \textit{completion step}
which consists of finding \textit{critical pairs} between $\f_{\R}$ and
$\f_{\aaex^i}$. To define the notion of critical pair, we extend the definition
of substitutions to the terms of $\TFQ$. For a substitution $\sigma:\X\mapsto\Q$ and
a rule $l\f r \in \R$, a critical pair is an instance $l\sigma $ of $l$ such
that there exists $q\in\Q$ satisfying $l\sigma \f^*_{\aaex^i}q$ and $l\sigma
\f_{\R} r\sigma$. Note that since
$\R$, $\aaex^i$ and the set $\Q$ of states of $\aaex^i$ are finite, there is only a finite
number of critical pairs. For every critical pair detected between $\R$ and
$\aaex^i$ such that we do not have a state $q$' for which $r\sigma \rwne_{\aaex^i}q'$ and $q' \rw q \in \Deps^i$, the
tree automaton $\aaex^{i+1}$ is constructed by adding new transitions $r\sigma \rwne q'$ to $\Delta^i$
and $q' \rw q$ to $\Deps^i$ such that $\aaex^{i+1}$ recognizes $r\sigma$ in $q$, i.e. $r\sigma \f^*_{\aaex^{i+1}} q$, see
Figure~\ref{fig:cp}.
%\vspace*{-5mm}
\begin{figure}[!ht]
  {\small
    \[
    \xymatrix{
      l\sigma \ar[r]_-{\R}\ar[d]^-{*}_-{\aaex^i} & r\sigma \ar[d]_-{\not\varepsilon}^{\aaex^{i+1}}\\
      q & q' \ar[l]^-{\aaex^{i+1}}
    }
    \]}
  \vspace*{-7mm}
  \caption{\footnotesize A critical pair solved \label{fig:cp}
  }
\end{figure}
%\vspace*{-3mm}
%%%%%%%
It is important to note that we consider the critical pair only if the
last step of the reduction $l\sigma \f^*_{\aaex^i}q$, is the last step of rewriting is not a $\varepsilon$-transition.
Without this condition, the completion computes the transitive closure of the
expected relation $\Deps$, and thus looses precision. %waste of information ?
%%%%%%
The transition $r \sigma \f q'$ is not necessarily a normalized
transition of the form $f(q_1, \ldots, q_n) \f q'$ and so it has to be normalized
first. Instead of adding $r\sigma \rw q'$ we add $\norm(r\sigma \rw q')$ to
transitions of $\Delta^i$.
Here is the $\norm$ function used to normalize transitions. Note that, in 
this function, transitions are normalized using new states of $\Q_{new}$.
%As we will see in Lemma~\ref{lemma:approx}, this has no effect on the
%safety of the normalization but only on its precision.

\begin{definition} [$\norm$] Let $\A=\la \F, \Q, \Qf, \Delta\cup\Deps\ra$ be a tree automaton, $\Q_{new}$ a
  set of {\em new} states such that $\Q\cap \Q_{new} = \emptyset$, $s \in \TFQ$ and $q'\in
  \Q$.
  The normalization of the transition $s \rw q'$ is done in two mutually inductive steps.
  The first step denoted by $\norm(s \rw q'\sep\Delta)$, we rewrite $s$ by $\Delta$ until rewriting 
  is impossible: we obtain a unique configuration $t$ if $\Delta$ respects the property~\ref{prop:deterministic}.
  The second step $\norm'$ is inductively defined by:
  % TODO : A REVOIR 
  \begin{itemize}
%  \item 
%    $\norm'(t \rw q')= \emptyset$ if $t\in\Q$,
  \item
    $\norm'(f(t_1, \ldots, t_n) \rw q\sep\Delta)= \Delta \cup \{f(t_1, \ldots,
    t_n) \rw q\}$ if $\forall i = 1\ldots n:\ t_i \in \Q$
  \item 
    $\norm'(f(t_1, \ldots, t_n) \rw q \sep \Delta)= \norm(f(t_1, \ldots, q_i,\ldots, t_n) \rw q\sep \norm'(t_i\rw q_i\sep \Delta)\ )$
    where $t_i$ is subterm s.t. $t_i \in \TFQ\setminus \Q$ and $q_i \in \Q_{new}$.
  \end{itemize}
\end{definition}

 \begin{lemma}
   \label{lem:welldefined}
   If the property \ref{prop:deterministic} holds for $\aaex^i$ then it holds also for $\aaex^{i+1}$.
 \end{lemma}

 \begin{proof}[Intuition]
   The determinism of $\rwne$ is preserved by $\Delta$, since when a new set of transitions
   is added to $\Delta$ for a subterm $t_i$, we rewrite all other subterms $t_j$ with the new $\Delta$ until rewriting is impossible 
   before resuming the normalization. Then, if we try to add to $\Delta$ a transition $f(q_1, \dots, q_n) \rw q$
   though there exists a transition $f(q_1, \dots, q_n) \rw q'\in \Delta$, it means that the configuration $f(q_1, \dots, q_n)$ 
   can be rewritten by $\Delta$. This is a contradiction : when we resume the normalization all subterms $t_i$ can not be rewritten 
   by the current $\Delta$. So, we never add a such transition to $\Delta$. The normalization produces a new set of transitions $\Delta$
   that preserves the property \ref{prop:deterministic}.
 \end{proof}

It is very important to remark that the transition $q'\rw q$ in Figure~\ref{fig:cp}
creates an order between the language recognized by $q$ and the one recognized by
$q'$.  Intuitively, we know that for all substitution $\sigma' : \X \rw \TF$ such that $l\sigma'$ is
a term recognized by $q$, it is rewritten by $\R$ into a canonical term ($r\sigma'$) of $q'$.
By duality, the term $r\sigma'$ has a parent ($l\sigma'$) in the state $q$.
Extending this reasoning, $\Deps$ defines a relation between canonical
terms. This relation follows rewriting steps at the top position and forgets
rewriting in the subterms.

\begin{definition}[$\arw$]
  Let $\R$ be a TRS. For all terms $u$ $v$, we have $u \arw_{\R} v$ iff there exists
  $w$ such that $u \rw_\R^* w$, $w \trw_{\R} v$ and there is not
  rewriting on top position $\lambda$ on the sequence denoted by $u
  \rw_\R^* w$.
  
\end{definition}
%A d�tailler

In the following, we show that the completion builds a tree automaton where
the set $\Deps$ is an \emph{abstraction} $\arw_{\R_i}$ of the rewriting relation $\rw_\R$, for
any relevant set $\R_i$.


\begin{theorem}[Correctness]
  \label{thm:correct}
  Let be $\aaex^*$ a complete tree automaton %obtained from $\R$ and $\A_0$,
  such that $q'\rw q$ is a $\varepsilon$-transition of $\aaex^*$. Then, for all canonical
  terms $u$ $v$ of states $q$ and $q'$ respectively s.t. $q'\rw q$, we have :
  \vspace*{-5mm}
  
  \[\xymatrix{ 
    u \ar[d]^-{\not\varepsilon}_-{\aaex^*} \ar@{-->}[r]_-{\R}
                &v \ar[d]^-{\not\varepsilon}_-{\aaex^*}\\ 
    q &q' \ar[l] }
  \]
\end{theorem}

First, we have to prove that the property \ref{prop:deterministic} is preserved by completion.
To prove theorem \ref{thm:correct}, we need a stronger lemma.

\begin{lemma}[]
  \label{lem:correct}
  Let be $\aaex^*$ a complete tree automaton, $q$ a state of $\aaex^*$ and $v\in\Lang{}(\aaex^*,q)$.
  Then, for all canonical term $u$ of $q$, we have $u \rw_\R^* v$. 
\end{lemma}

\begin{proof}[Proof sketch]
  
  The proof is done by induction on the number of  completion steps
  to reach the post-fixpoint $\aaex^*$ : we are going to show that
  if $\aaex^i$ respects the property of lemma~\ref{lem:correct},
  then $\aaex^{i+1}$ also does.
  
  The initial $\aaex^0$ respects the expected property~: we consider
  any state $q$ and a canonical term $t$ of $q$: since no completion
  step was done, $\aaex^0$ has no $\varepsilon$-transitions. It means
  that for all term $t'\rwne q$. Thanks to the property
  \ref{prop:wellinitial}, we have $t = t'$ and obviously $t \rw^*_\R
  t'$.

  Now, we consider the normalization of a transition of the form $r\sigma \rwne q'$
  such that $l\sigma \rw^*_{\aaex^i} q$ with $\Delta$ the ground transition set and $\Deps$ the $\varepsilon$-transition set of $\aaex^i$.
  We show that the property is true for all new states (including $q'$). 
  Then, in a second time, we will show that it is true for state $q$,
  if we add the second transition of completion: $q'\rw q$. 
  
  %
  Let us focus on the normalization of $\norm'(r\sigma \rw q'\sep \Delta)$ where for
  any existing state $q$ and for all $u\ v \in \TF$ such that $v \rw_{\Delta\cup\Deps} q$ and $u \rw_{\Delta} q$, we have $u \rw_\R^* v$.
  We show that for all $t \in \TFQ$, if we have $\Delta' = \norm'(t \rw q'\sep \Delta)$, for all $u\ v \in \TF$ such that $v \rw_{\Delta'\cup\Deps} q'$ and $u \rw_{\Delta'} q$, we have $u \rw_\R^* v$. 
  The induction is done on the %decreasing
  number of symbols of $\F$ used to build $t$.

  First case $\norm'(t \rw q \sep \Delta)$ where $t = f(q_1,\dots, q_n)$ : we define $\Delta'$ by adding the transition $f(q_1, \dots, q_n) \rw q$
  to $\Delta$, where $q$ is a new state. Then, for all substitutions $\sigma' : \Q \mapsto \TF$ such that $t\sigma' \rw_{\Delta\cup\Deps} q$, and all 
  substitutions $\sigma'' : \Q \mapsto \TF$ such that $t\sigma'' \rw_{\Delta'} q$ we aim at proving that $t\sigma''
  \rw_\R^* t\sigma'$. Since each state $q_i$
  is already defined, using the hypothesis on $\Delta$ we deduce that $\sigma''(q_i) \rw^*_\R \sigma'(q_i)$. This implies that $t\sigma'' \rw_\R^* t\sigma'$, the property 
  also holds for $\Delta'$.

  Second case $\norm'(t \rw q \sep \Delta)$ where $t = f(t_1,\ldots,t_n)$: we select $t_i$ a subterm of $t$, obviously the number
  of symbols is strictly lower to the number of symbols of $t$.
  By induction, for the normalization of $\norm'(t_i \rw q_i\sep \Delta)$ we have a new 
  set $\Delta'$ that respects the expected property. Then, we normalize $t$ into $t' = f(t'_1, \dots, q_i, \dots, t'_n)$, 
  the term obtained after rewriting with $\Delta'$ thanks to $\norm$. Since $t_i \not\in \Q$, the number of
  symbols of $\F$ in $t' = f(t_1, \dots, q_i, \dots, t_n)$ is strictly smaller than the number of symbols of $ \F$ in $t$. Note 
  that rewriting $t'$ with $\Delta'$ can only decrease the number of symbols of $\F$ in $t'$.
  Since $t'$ has a decreasing number of symbols and $\Delta'$ respects the property we can deduce by induction
  that we have $\Delta'' = \norm'(t'\rw q\sep \Delta')$ such that for all $v \rw_{\Delta''\cup\Deps} q'$ and $u \rw_{\Delta''} q$, $u \rw_\R^* v$.
  
  So, we conclude that the normalization $\norm'(r\sigma \rw q'\sep \Delta)$ computes $\Delta'$ the set of ground transitions for $\aaex^{i+1}$.
  For all terms $u$ $v$ such that $u \rw_{\Delta'\cup\Deps} q'$ and $u \rw_{\Delta'} q'$ we have $u \rw_\R^* v$. 

  Now, let us consider the second added transition $q' \rw q$ to $\Deps$, all canonical terms
  $r\sigma''$ of $q'$, and all terms $l\sigma''' \in
  \Lang{}(\aaex^i, q)$ such that $l\sigma''' \rw_\R r\sigma'''$ and
  $r\sigma''' = r\sigma''$.  By hypothesis on $\aaex^i$, we know that every canonical term $u$ of $q$
  we have $u \rw_\R^*
  l\sigma'''$. By transitivity, we have $u \rw_\R^* r\sigma''$.  The
  last step consists in proving that for all terms of all states of
  $\aaex^{i+1}$, the property holds: this can be done by induction on
  the depth of the recognized terms.
\end{proof}

The theorem \ref{thm:correct} is shown by considering the introduction of the
transition $q' \rw q$. By construction, there exists a substitution $\sigma : \X \mapsto \Q$ and a rule
$l \rw r \in \R$ such that we have $l\sigma \rw^*_{\aaex^*} q$ and $r\sigma \rwne_{\aaex^*} q'$. We consider all substitution  
$\sigma' : \X \mapsto \TF$ such that for each variable $x \in \vars(l)$, $\sigma'(x)$ is a canonical term
of the state $\sigma(x)$. Obviously, using the result of the lemma \ref{lem:correct},
for all canonical term $u$ of $q$ we have $u \rw^*_\R l\sigma'$. Since the last step of rewriting 
in the reduction $l\sigma \rw^*_{\aaex^*} q$ is not a $\varepsilon$-transition, we also deduce that $l\sigma'$ is not produced
by a rewriting at the top position of $u$ whereas it is the case for $r\sigma'$ and we have $u \arw_{\R} r\sigma'$.

 
\begin{theorem}[Completeness]
  Let $\aaex^*$ be a complete tree automaton, %obtained from $\R$ and $\A_0$,
  $q,q'$ states of $\aaex^*$ and $u,v \in \TF$ such that $u$ is a canonical term of $q$
  and $v$ is a canonical term of $q'$. If $u \arw_\R v$ then there exists a $\varepsilon$-transition $q' \rw q$ in $\aaex^*$.
\end{theorem}
\begin{proof}[Proof sketch]
  By definition of $u \arw_\R v$ there exists a term $w$ such that $u \rw_\R^* w$ and
  and there exists a rule $l \rw r \in \R$ and a substitution $\sigma : \X \mapsto \TF$ such that 
  $w = l\sigma$ and $v = r\sigma$.
  Since $\aaex^*$ is a complete tree automaton, it is closed by rewriting. This means 
  that any term obtained by rewriting any term of $\Lang{}(\aaex^*, q)$ is also in $\Lang{}(\aaex^*, q)$. This
  property is true in particular for the terms $u$ and $w$. 
  Since $w$ is rewritten in $q$ by transitions of $\aaex^*$, we can define
  a second substitution $\sigma' : \X \mapsto \Q$ such that $l\sigma \rw^*_{\aaex^*} l\sigma' \rw^*_{\aaex^*} q$.
  Using again the closure property of $\aaex^*$, we know that the critical pair $l\sigma' \rw_\R r\sigma'$
  and $l\sigma' \rw^*_{\aaex^*} q$ is solved by adding the transitions $r\sigma' \rwne_{\aaex^*} q''$ and $q'' \rw q$. Since the property \ref{prop:deterministic}
  is preserved by completion steps, we can deduce that $q'' = q'$ which means $q' \rw q$.
\end{proof}
%Bon il faut mettre du baratin ici, la preuve ne marche pas...


%When using only new states to normalize all the new transitions occurring in all
%the completion steps, completion is as precise as possible.

% However, doing so, completion is likely not to terminate (because of general undecidability
% results~\cite{GilleronTison-FI95}).  Enforcing termination of completion can be
% easily done by bounding the set of new states to be used with $\norm$ during the
% whole completion. We then obtain a finite tree automaton over-approximating the
% set of reachable states. The fact that normalizing with any set of states (new
% or not) is {\em safe} is guaranteed by the following simple lemma. For the
% general safety theorem of completion see~\cite{FeuilladeGVTT-JAR04}.

% \begin{lemma}
% \label{lemma:approx}
% For all tree automaton $\A=\aut$, $t\in \TFQ\setminus \Q$ and $q\in\Q$, if $\Pi=\norm(t \rw
% q)$ whatever the states chosen in $\norm(t \rw q)$ we have $t \rw^*_{\Pi} q$.
% \end{lemma}
% \begin{proof}
% This can be done by a simple induction on transitions~\cite{FeuilladeGVTT-JAR04}.
% %to normalize, see~\cite{FeuilladeGVTT-JAR04}.
% \end{proof}

% To let the user of completion guide the approximation, we use two different
% tools: a set $\nr$ of {\em normalization rules} (see~\cite{FeuilladeGVTT-JAR04})
% and a set $\E$ of {\em approximation equations}. Rules and equations can be
% either defined by hand so as to prove a complex
% property~\cite{GenetTTVTT-wits03}, or generated automatically when the property
% is more standard~\cite{BoichutHKO-AVIS04}. Normalization rules can be seen as a
% specific strategy for normalizing new transitions using the $\norm$
% function. We have seen that Lemma~\ref{lemma:approx} is enough to
% guarantee that the chosen normalization strategy has no impact on the safety of
% completion. Similarly, for our checker, we will see in Section~\ref{sec:closure}
% that the related \coq\ safety proof can be carried out independently of the
% normalization strategy (i.e. set $N$ of normalization rules).  On the opposite, the
% effect of approximation equations is more complex and has to be studied more
%%carefully.


% On the one side, normalization
% rules define which states are to be used to normalize a transition with
% $\norm$. When using $\nr$ to guide the normalization, we note $\norm_{\nr}$ the
% normalization function.  On the other side, approximation equations define some
% approximated equivalence classes. Equations of $\E$ are applied directly on
% $\A^{i+1}_{\R}$ to merge together the states whose recognized terms are in the
% same equivalence class w.r.t. $\E$.
%
% For all $s,l_1, \ldots, l_n \in\TFQX$ and for all
% $x,x_1, \ldots, x_n \in \Q\cup\X$, the general form for a normalization rule
% is:
% \[[s \rw x] \rw [l_1 \rw x_1, \ldots, l_n \rw x_n]\] where the expression $[s
% \rw x]$ is a pattern to be matched with the new transitions $t \rw q'$ obtained
% by completion. The expression $ [l_1 \rw x_1, \ldots, l_n \rw x_n]$ is a set of
% rules used to normalize $t$. To normalize a transition of the form $t \rw q'$,
% we match $s$ with $t$ and $x$ with $q'$, obtain a substitution $\sigma$ from the
% matching and then we normalize $t$ with the rewrite system $\{l_1\sigma \rw
% x_1\sigma, \ldots, l_n \sigma \rw x_n\sigma\}$. Furthermore, if $\forall
% i=1\ldots n: x_i\in \Q$ or $x_i \in \var(l_i) \cup \var(s) \cup \{x\}$ then
% since $\sigma: \X \mapsto \Q$, $x_1 \sigma, \ldots, x_n\sigma$ are necessarily
% states. 
% If a transition cannot be fully normalized using approximation rules
% $\nr$, normalization is finished using some new states, see Example
% \ref{example:approx}. 
% %We denote by $\norm_\nr(t \rw q')$ the set of transitions
% %obtained by the normalization of $t \rw q'$ by normalization rules $\nr$.
% An approximation equation is of the form $u=v$ where $u,v\in\TFX$.  Let $\sigma:
% \X \mapsto \Q$ be a substitution such that $u\sigma \rw_{\A_{\R}^{i+1}} q$,
% $v\sigma \rw_{\A_{\R}^{i+1}} q'$ and $q\neq q'$, see
% Figure~\ref{fig:merge}. Then, we know that there exists some terms recognized by
% $q$ and some recognized by $q'$ which are equivalent modulo $\E$. A correct
% over-approximation of $\aaex^{i+1}$ consists in applying the $\merge$ function to
% it, i.e. replace $\aaex^{i+1}$ by $\merge(\aaex^{i+1},q, q')$, as long as an
% approximation equation of $\E$ applies. The $\merge$ function, defined below,
% merges states in a tree automaton.  See~\cite{BoyerGJ-RR08} for examples of
% completion and approximation.
% \begin{definition}[$\merge$]
%   Let $\A= \langle \F, \Q, \Q_F, \Delta \rangle$ be a tree automaton and
%   $q_1,q_2$ be two states of $\A$. We denote by $\merge(\A,q_1, q_2)$ the tree
%   automaton where every occurrence of $q_2$ is replaced by $q_1$ in $\Q$, $\Q_F$
%   and in every left-hand side and right-hand side of every transition of
%   $\Delta$.
% \end{definition}

% The following examples illustrate completion and how to carry out an
% approximation, using equations, when the language $\desc(\Lang(\A)) $ is not
% regular.

% \label{example:merge}
% Let $\R=\{g(x,y) \rw g(f(x),f(y))\} $ and let $\A$ be the tree automaton such
% that $\Q_F=\{q_f\}$ and $\Delta=\{a \rw q_a, g(q_a,q_a)\rw q_f\}$. Hence
% $\Lang(\A)= \{g(a,a)\}$ and $\desc(\Lang(\A))=\{g(f^n(a),f^n(a))~|~n\geq
% 0\}$. Let $\E=\{f(x)=x\}$ be the set of approximation equations. During the
% first completion step on $\aaex^0=\A$,  we find $\sigma=\{x \mapsto q_a\}$ and
% the following critical pair

% {\small
% $$
% \xymatrix{
%   g(q_a,q_a) \ar[r]_{\R}\ar[d]^{*}_{\aaex^0} & g(f(q_a),f(q_a)) \ar@/^1.2pc/[ld]_{*}^{\aaex^{1}}\\
%   q_f & %\ar[l]^{\A_{i+1}} q'
% }
% $$}

% Hence, we have to add the transition $g(f(q_a),f(q_a)) \rw q_f$ to $\aaex^0$ to
% obtain $\aaex^1$. This transitions can be normalized in the following way:
% $\norm(g(f(q_a),f(q_a)) \rw q_f)=\{g(q_1, q_2) \rw q_f, f(q_a)\rw q_1,f(q_a)\rw
% q_2 \}$ where $q_1$ and $q_2$ are new states. Those new states and transitions
% are added to $\aaex^0$ to obtain $\aaex^1$. On this tree automaton, we can apply
% the equation $f(x)=x$ of $\E$ with the substitution $\sigma=\{x\mapsto q_a\}$:

% {\small
% $$\xymatrix{
% f(q_a) \ar@{=}[r]_{\E}\ar[d]_{\aaex^{1}}^{*} & q_a \ar[d]_{*}^{\aaex^{1}}\\
% q_1 & q_a
% }
% $$}

% Hence, we can replace $\aaex^1$ by
% $\merge(\aaex^1,q_1,q_a)$ where $\Delta$ is $\{ a \rw q_1, g(q_1,q_1)\rw q_f,
% g(q_1, q_2) \rw q_f, f(q_1) \rw q_1, f(q_1) \rw q_2\}$. Similarly, in this last
% tree automaton, we have 

% {\small $$\xymatrix{
% f(q_1) \ar@{=}[r]_{\E}\ar[d]_{\aaex^{1}}^{*} & q_1 \ar[d]_{*}^{\aaex^{1}}\\
% q_2 & q_1
% }
% $$}

% and we can thus apply $\merge(\aaex^1, q_2, q_1)$. Finally, the value of
% $\Delta$ for $\aaex^1$ approximated by $\E$ is $\{a \rw q_2, g(q_2,q_2)\rw q_f,
% f(q_2) \rw q_2 \}$. Now, the only critical pair that can be found on $\aaex^1$ is
% joinable:

% {\small
% $$
% \xymatrix{
%   g(q_2,q_2) \ar[r]_{\R}\ar[d]^{*}_{\aaex^1} & g(f(q_2),f(q_2)) \ar@/^1.2pc/[ld]_{*}^{\aaex^{1}}\\
%   q_f & %\ar[l]^{\A_{i+1}} q'
% }
% $$}

% Hence, we have $\aaex^*=\aaex^1$ and
% $\Lang(\aaex^*)=\{g(f^n(a),f^m(a))~|~n,m\geq 0\}$ which is an over-approximation
% of $\desc(\Lang(\A))$.
% \end{example}

% The tree automata completion algorithm and the approximation mechanism are
% implemented in the \timbuk~\cite{timbuk-site} tool. On the previous example, once
% the fixpoint automaton $\aaex^*$ has been computed, it is possible to check
% whether some terms are reachable, i.e. recognized by $\aaex^*$ or not. This
% can be done using tree automata 
% intersections~\cite{FeuilladeGVTT-JAR04}. 



%%% Local Variables: 
%%% mode: latex
%%% TeX-master: "main"
%%% End: 


% \section{The need for a certified checker in \coq}
%\label{sec:intro_checker}
\switchlstcoq 

\section{A result checker for tree automata completion}
\label{section:objectives}
%This part has to describe precisely the contribution of the paper.

\archive{The main question about \timbuk\ results is \emph{can we trust them?}
If the tree automata completion outputs an incorrect fixpoint
automaton, the analysis may validate an unsafe program.  Moreover,
\timbuk\ is a complex tool. It contains more than 11000 lines
of Ocaml. Thus it is difficult to be sure that \timbuk\ is bug free".
We can only rely on testing w.r.t. a base of test cases to remove as many
bugs as possible.
}

By moving the certification problem from the completion algorithm to
the checker, the certification problem consists in proving the following
\coq\ theorem:
\begin{lstlisting}
Theorem sound_checker :
      forall A A' R, checker A R A' = true -> ApproxReachable A R A'.
\end{lstlisting}
where \lstinline!ApproxReachable! is a \coq\ predicate that describes
the Soundness Property: \emph{$\Lang(A')$ contains all terms reachable
  by rewriting terms of $\Lang(A)$ with $\R$}, i.e. $\Lang(\A')
\supseteq \desc(\Lang(\A))$. 
To state formally this predicate in \coq, we need to
give a \coq\ axiomatization of Term Rewriting Systems and of Tree Automata. It is
given in Section~\ref{sec:rewriting}.
Given two automata $\A$, $\A'$ and a TRS $\R$ the checker 
verifies that $\Lang(\A')\supseteq \R^*(\Lang(\A))$ or
(\lstinline!ApproxReachable A R A'!) in \coq. To perform this, we need to check the two
following properties:

\begin{itemize}
\item \lstinline!Included!: inclusion of initial set in the fixpoint: $\Lang(\A) \subseteq \Lang(\A')$.

\item \lstinline!IsClosed!: $\A'$ is closed by rewriting with $\R$: For all $l \rightarrow
  r \in \R$ and all $t \in \Lang(\A')$, if $t$ is rewritten in $t'$ by the rule
  $l \rightarrow r$ then $t' \in \Lang(\A')$. 
% Trop detaill� pour la partie objectives...
% To prove this property, we need
%   verify that for each substitution $\sigma:\X \mapsto \Q$ and state $q$ of
%   $\A'$, if $l\sigma \rw_{\A'}^* q$ then we have $r\sigma \rw_{\A'}^* q$,
%     i.e. prove that the critical pair $(l\sigma \rightarrow q,\ l\sigma
%     \rightarrow r\sigma)$ is joinable.
\end{itemize}
For each item, we provide a \coq\ function and its correctness theorem: function
{\tt inclusion} is dedicated to inclusion checking and function {\tt closure}
checks if a tree automaton is closed by rewriting.  We also give the theorem
used to deduce \lstinline!ApproxReachable A R A'! from \lstinline!Included A A'! and \lstinline!IsClosed R A'!:
\begin{lstlisting}
Theorem inclusion_sound:
      forall A A', inclusion A A' = true -> Included A A'.

Theorem closure_sound:
      forall R A', closure R A' = true -> IsClosed R A'.

Theorem Included_IsClosed_ApproxReachable:
      forall A A' R, Included A A' -> IsClosed R A' -> ApproxReachable A R A'.
\end{lstlisting}


Note that, in this paper we focus on the proof of $\Lang(\A')\supseteq
\R^*(\Lang(\A))$.  However, to prove the unreachability property, the emptiness
of the intersection between $\Lang(\A')$ and the bad term set has also to be
verified. Since the formalization in \coq\ of the intersection and emptiness
decision are close to their standard definition~\cite{TATA}, and since they have
already been covered by~\cite{RivalGL-TPHOL01}, they are not be detailed in this
paper.





%%% Local Variables: 
%%% mode: latex
%%% TeX-master: "main"
%%% End: 


\section{Formalization of Term Rewriting Systems}
\label{sec:rewriting}

The aim of this part is to formalize in \coq: terms, term rewriting systems,
reachable terms and the reachability problem itself.  Firstly we use the
positive integers provided by the \coq's standard library to define symbol sets
like variables ($\X$) or function symbols ($\F$). We rename \lstinline!positive! into
\lstinline!ident! to be more explicit. Then, we define term set $\TFX$ using inductive
types:

\switchlstcoq
%A discuter....
%Definition F := positive.
%Definition X := positive.
\begin{lstlisting}
Definition ident := positive.

Inductive term : Set :=
| Fun : ident -> list term -> term
| Var : ident -> term.
\end{lstlisting}

\versionlongue{
Now, the term $f(x, a)$ will be written \lstinline!Fun 0 (Var 0::(Fun 1 nil)::nil)!  assuming that we have the corresponding mapping between between
symbols, variables and positive integers $f \mapsto 0$, $a \mapsto 1$ and $x
\mapsto 0$ for example.  Note that it is possible to attach the value $0$
to $f$ and $x$, since the \lstinline!term!'s constructors \lstinline!Fun! and
\lstinline!Var! allows to differentiate between variable and function symbols.
}

\versionlongue{
\paragraph{Remarks:}
\begin{itemize}
\item
  Since equality is decidable, we can easily define term equality as a
  decidable relation. Afterward, this is very useful to define functions
  where term comparison is required.
\item
  A bad point for \coq\ is the induction principle
  \lstinline!term_rect! automatically generated which is too weak.
  To prove properties in the following, we need a more efficient theorem
  named \lstinline!term_rect'!. This is required to be able to prove
  most of the theorems on terms.
\end{itemize}
}

A rewrite rule $l \rw r$ is represented by a pair of terms with a
well-definition proof, i.e. a \coq\ proof that the set of variables of $r$ is a
subset of the set of variables of $l$. The function
\lstinline!Fv : term -> list ident! builds the set of variables for a term. 
% In \coq, it becomes:
\begin{lstlisting}
Inductive rule : Set :=
| Rule (l r : term)(H : subseteq (Fv r) (Fv l)) : rule.
\end{lstlisting}

In the following, \lstinline!list rule! type represents a
TRS. In \coq\ we use \lstinline!(t @ sigma)!  to denote the term resulting of
the application of a substitution \lstinline!sigma! to each variable that occurs
in a term \lstinline!t!. 

\begin{lstlisting}
Definition substitution := ident -> option term.
\end{lstlisting}

In \coq, the rewriting relation \emph{"$u$ is rewritten in $v$ by $l
  \rightarrow r$"},
commonly defined by $\exists \sigma \
s.t.\ u|_p = l\sigma \ \land \ v = u [ r\sigma ]_p$, is split into
two predicates:
\begin{itemize}
\item The first one defines the rewriting of a term at the topmost position. In
  fact, the set of term pairs $(t, t')$ which are rewritten at the top most by
  the rule can be seen as the set of term pairs $(l\sigma, r\sigma)$ for any
  substitution $\sigma$.

\item 
  The second one just defines inductively the rewriting relation at any
  position of a term $t$ by a rule $l \rightarrow r$, by the topmost
  rewriting of any subterm of $t$ by $l \rightarrow r$.
\end{itemize}

\begin{lstlisting}
(* Topmost rewriting : *)
Inductive TRew (x : rule) : term -> term -> Prop :=
| R_Rew :
  forall s l r (H : subseteq (Fv r) (Fv l)),
    x = Rule l r H -> TRew x (l @ s) (r @ s).
\end{lstlisting}


\begin{lstlisting}
(* Rewrite at any position of term *)
Inductive Rew (r : rule) : term -> term -> Prop :=
| Rew1 : forall t t', 
    TRew r t t' -> Rew r t t'
| Rew2 : forall f l l',
    Rargs r l l' -> Rew r (Fun f l) (Fun f l')
\end{lstlisting}

\begin{lstlisting}
with Rargs (r : rule):list term->list term->Prop:=
| Ra1 : forall t t' l,
    Rew r t t' -> Rargs r (t::l) (t'::l)
| Ra2 : forall t l l',
    Rargs r l l' -> Rargs r (t::l) (t::l').
\end{lstlisting}


\versioncourte{Similarly, using an inductive definition it is possible to
  define the \lstinline!Rew! predicate for rewriting at any position.  } Then we
have to define $\rw^*_{\R}$. In \coq, we prefer to see it as the predicate
\lstinline{Reachable R u} that characterizes the set of reachable terms from
$u$ by $\rightarrow^*_\R$.

\begin{lstlisting}
Inductive Reachable(R : list rule)(t : term) : term -> Prop:=
| R_refl : Reachable R t t
| R_trans : forall u v r, Reachable R t u -> In r R -> Rew r u v -> 
    Reachable R t v.
\end{lstlisting}

%%% Local Variables: 
%%% mode: latex
%%% TeX-master: "main"
%%% End: 

\section{Formalisation des automates d'arbres}
\label{sec:automata}

Le fait que le $\kw{checker}$, qui doit être exécuté, est directement extrait de la formalisation \coq\
contraint la formalisation des automates d'arbres. Comme les structures de données
utilisées dans la formalisation sont celles qui sont réellement utilisées
lors de l'exécution, elles doivent être {\em efficaces} d'un point de vue algorithmique.
Pour les automates d'arbres, au lieu d'une représentation naïve, il est nécessaire d'utiliser 
une formalisation de la structure de données proposée dans~\cite{RivalGL-TPHOL01} pour manipuler de 
façon optimale les automates d'arbres.

\switchlstcoq


Dans la section~\ref{sec:rewriting}, on a représenté les variables $\X$ et les symboles de 
de fonctions $\F$ par le type \lstinline!ident!. On fait la même chose pour définir les états $\Q$.
On définit un automate comme un couple $(\Q_F, \Delta)$, où $\Q_F$ est l'ensemble fini des 
états finals, et $\Delta$ l'ensemble fini des transitions normalisées comme $f(q_1, \dots, q_n) \rightarrow q$.
En \coq, $\Q_F$ est une simple \lstinline!list ident! alors que $\Delta$ est représentée en utilisant les \lstinline!FMapPositive!
de la librairie \coq. Il s'agit d'une implémentation des tables d'associations fonctionnelles, où les données sont indexées par \lstinline!positive!.
Les \lstinline!positive! sont en fait une représentation binaire des entiers strictement positifs.
Dans la structure de \lstinline!FMapPositive!, chaque transition $f(q_1, \dots, q_n) \rightarrow q$ est encodée par une liste d'états $(q_1, \dots, q_n)$ indexée par $f$
dans une première table qui est ensuite indexée par l'état $q$ dans une seconde table. Cette représentation est une bonne 
solution pour manipuler efficacement les ensembles de transtions en \coq.
%
%

\begin{lstlisting}
Module Delta : DELTA.
   (* Transition sets : *)
   Definition config := list state.
   
   Definition t := 
        FMap.t (FMap.t (list config)).
   (* ............... *)
\end{lstlisting}

%
%
Ensuite, on peut définir un prédicat pour caractériser le langage reconnu par un automate d'arbres.
En fait, il s'agit de définir l'ensemble des termes clos qui sont réduits (réécrits) dans un état 
$q$ par les transitions de $\Delta$. Cet ensemble, qui correspond à $\Lang(\Delta,q)$ si $\Delta$ 
est l'ensemble des transitions de $\A$, peut-être construit inductivement en \coq\ en utilisant l'unique règle de déduction:

\begin{prooftree}
  \AxiomC{$t_1 \in \Lang(\Delta, q_1)$}
  \AxiomC{\dots\dots}
  \AxiomC{$t_n \in \Lang(\Delta, q_n)$}
  \RightLabel{Si $f(q_1, \dots, q_n) \rightarrow q \in \Delta$}
  \TrinaryInfC{$f(t_1,\dots, t_n) \in \Lang(\Delta, q)$}  
\end{prooftree}

En \coq, on exprime cette proposition en utilisant le prédicate inductif
\lstinline!IsRec!. Un terme $t$ est reconnu par un automate d'arbre $(\Q_F, \Delta)$, si
le prédicat \texttt{IsRec}~$\Delta$~$q$~$t$ est valide pour $q \in \Q_F$.

\begin{lstlisting}
Inductive IsRec (D: Delta.t) : state -> term -> Prop :=
  Rec_Term : forall f lt q,
      IsRec' D (Delta.get q f D) lt -> IsRec D q (Fun f lt)

with IsRec' (D: Delta.t) : list config -> list term -> Prop :=
| Rec_SubTerm : forall lt c lc, IsRec'' D c lt -> IsRec' D (c::lc) lt
| Rec_SubTerm' : forall lt c lc, IsRec' D lc lt -> IsRec' D (c::lc) lt
     
with IsRec'' (D: Delta.t) : config -> list term -> Prop :=
| Rec_Nil : IsRec'' D nil nil
| Rec_Cons : forall t q lt lq, IsRec D q t -> IsRec'' D lq lt ->
       IsRec'' D (q::lq) (t::lt).
\end{lstlisting}

Il est encore légitime de se demander quel crédit accorder à une telle spécification
qui n'a plus beaucoup de point commun avec la théorie des automates d'arbres classiques.
C'est pour cette raison que l'on décide de montrer l'équivalence entre ce formalisme 
et la définition classique telle que présentée dans le chapitre~\ref{chap:preliminaires}.
On définit alors un automate comme un couple composé de l'ensemble des états finaux et 
et l'ensemble des transitions normalisées. Les ensembles sont représentés par des listes.
\begin{lstlisting}
Inductive transition : Set :=
| Ground (l r: term):
      forall f lt q (Hl: l = Fun f lt) (Hr: r = State q)
                    (Hnorm: forall t, t \in lt -> exists q', t = State q') : Transition.

(* Conversion  Delta.t vers un ensemble de transitions *)
Definition trans_of: Delta.t -> list transitions.
    (* ... Body ... *)

(* L'execution $t \rw_\A t'$ *)
Definition Run (t: transition)(u v: term) := 
   forall l r f lt q Hl Hr Hnorm, Ground l r f lt q Hl Hr Hnorm = t ->
      exists p: position,
         get u p = Some l /\ v = set u p r.

(* Cloture de Run $t \rw^*_\A t'$ *)
Inductive StdRun (l: list transition): term -> term -> Prop :=
| RunRefl: forall u, StdRun l u
| RunStep: forall u v w t, StdRun l u v -> t \in l -> Run t v w -> StdRun l u w.


(* Equivalence: *)
Theorem StdRun_IsRec:
   forall $\Delta$ q t,
      IsRec $\Delta$ q t <-> StdRun (trans_of $\Delta$) t (state q).
\end{lstlisting}
Bien que la formalisation soit plus simple que dans la librairie \coq\ d'automates déterministes 
descendants~\cite{RivalGL-TPHOL01}, la preuve qu'elle soit équivalente à la formalisation théorique
est un argument irréfutable pour se convaincre de la correction de l'approche.
Enfin, il existe aussi une librairie récente d'automates d'arbres pour Isabelle\cite{TA-isabelle2010}
basée sur une implémentation similaire à celle-ci. La principale différence est l'utilisation des 
arbres "rouges et noirs" pour construire les tables d'associations plutôt que des arbres simples.
Cela permet de maintenir des structures avec une bonne complexité pour tout accès notamment pour
contruire les unions et intersections d'automates. Pour le $\kw{checker}$, le seul algorithme nécessaire
est la décision du vide de l'intersection entre deux automates, pour vérifier les propriétés. Or il n'est pas
nécessaire de construire explicitement l'intersection, on peut simplement rechercher si il existe une exécution commune 
entre les deux automates d'arbres considérés.


%%% Local Variables: 
%%% mode: latex
%%% TeX-master: "../main"
%%% End: 


\section{An optimized inclusion checker}
\label{sec:inclusion}

In this part, we give the formal definition of the \lstinline!Included! property and
of the \lstinline!inclusion! \coq\ function used to effectively check the tree
automata inclusion. From the previous formal definitions on tree automata, we
can state the \lstinline!Included! predicate in the following way:

\begin{lstlisting}
  Definition Included (a b : t_aut) : Prop :=
    forall t q, In q a.qf -> IsRec a.delta q t ->
      exists q', In q' b.qf /\ IsRec b.delta q' t.
\end{lstlisting}


Now let us focus on the function \lstinline!inclusion! itself.  The usual
algorithm for proving inclusion of regular languages recognized by
nondeterministic bottom-up tree automata, for instance for proving $\Lang(\A)
\subseteq \Lang(\B)$, consists in proving that $\Lang(\A) \cap
\Lang(\overline{\B}) =\emptyset$, where $\overline{\B}$ is the complement
automaton for $\B$. However, the algorithm for building $\overline{\B}$ from
$\B$ is EXPTIME-complete~\cite{TATA}. This is the reason why we here define a
criterion with a better practical complexity. It is is based on a simple
syntactic comparison of transition sets, i.e. we check the inclusion of
transition sets modulo the renamings performed by the $\merge$ function.
This increases a lot the efficiency of our checker, especially by
saving memory. This is crucial to check inclusion of big tree automata (see
Section~\ref{sec:benchmarks}). This algorithm is correct but, of course, it is
not complete in general, i.e. not always able to prove that $\Lang(\A) \not
\subseteq \Lang(\B)$. However, we show in the following that, under certain
conditions on $\A$ and $\B$ which are satisfied if $\B$ is obtained by
completion of $\A$, this algorithm is also complete and thus becomes a decision
procedure.  First, we introduce the following notation:
{\small
\begin{center}
  \begin{tabular}[c]{lcp{8cm}}
    $\Gamma$    & : & induction hypothesis set\\
    $\Delta_i$  & : & transition set of the tree automaton $\mathcal{A}_i$\\
    $\{c|c \rightarrow q \in \Delta\}$ & : & configurations of $\Delta$ that are rewritten in $q$\\
    $\{c_i\}_n^m$ & : & configuration set from $c_n$ to $c_m$\\
%    $\emptyset$ & : & empty set of configurations\\
%    $[q_1, \dots, q_n]$ & : & a tuple of $n$ states\\
  \end{tabular}
\end{center}
} We formulate our inclusion problem by formulas of the form: ${\Gamma
  \vdash_{A, B} q \Subset q'}$. Such a statement stands for: under the
assumption $\Gamma$, it is possible to prove that $\Lang(A,q) \subseteq
\Lang(B,q')$.
% A term $t$ is reconnized by state $q$ a tree automaton if it exists
% one sequence of transitions of $\Delta$ that rewrites $t$ into $q$. 
% Thus, ${\Gamma \vdash_{A, B} q \Subset q'}$ states inductively for the assertion:
% if a term $t$ s.t. is rewritten into $q$ by a sequence of transitions of $\Delta_\A$ then
% it can be done by an equivalent sequence in $\Delta_\B$ which rewrites $t$ into $q'$.\\
The algorithm consists in building proof trees for those statements using the
following set of deduction rules.
% This algorithm checks only transition set inclusion (up to
% renaming). Transition set inclusion implies corresponding language
% inclusion (as proved in Section~\ref{sec:correction}). It is more
% reasonable to implement it in \coq.\\
%The algorithm checks only this property, but it is enough to imply 
%corresponding language inclusion. (as proved in Section~\ref{sec:correction}).
%It is more reasonable to implement it in \coq.\\

{\small
\[\textrm{(Induction) }
\dfrac{
  \Gamma \cup \{q \Subset q'\} \vdash_{A, B}
  \{c|c \rightarrow_{\Delta_A} q\} \Subset \{c|c \rightarrow_{\Delta_B} q'\}
}{
  \Gamma \vdash_{A, B} q \Subset q'
}\textrm{  if }(q \Subset q') \notin \Gamma
\]


\[\textrm{(Axiom) }
\dfrac{}{
  \Gamma \cup \{q \Subset q'\} \vdash_{A, B} q \Subset q'
}\quad \quad \quad \quad 
\textrm{(Empty) }
\dfrac{}{
  \Gamma \vdash_{A, B} \emptyset \Subset \{c'_j\}^m_1
}
\]



\[\textrm{(Split-l) }
\dfrac {
  \Gamma \vdash_{A, B} c_1 \Subset \{c'_j\}^m_1 \quad \dots\dots \quad
  \Gamma \vdash_{A, B}  c_n \Subset \{c'_j\}^m_1
}{
  \Gamma \vdash_{A, B} \{c_i\}^n_1 \Subset \{c'_j\}^m_1
}\]

%\[(empty)\]

\[\textrm{(Weak-r) }
\dfrac{
  \Gamma \vdash_{A, B} c \Subset c'_k
}{
  \Gamma \vdash_{A, B} c \Subset \{c'_i\}^n_1
}\textrm{ if }(1 \le k \le n)
\quad\quad\quad \quad
\textrm{(Const.) }
\dfrac{}{
  \Gamma \vdash_{A, B} a() \Subset a()
}
\]



\[\textrm{(Config) }
\dfrac{
  \Gamma \vdash_{A, B} q_1 \Subset q'_1 \quad \dots\dots \quad \Gamma \vdash_{A, B} q_n \Subset q'_n
}{
  \Gamma \vdash_{A, B} f(q_1, \dots, q_n) \Subset f(q'_1, \dots, q'_n)
}\]

%\[const\]
}
\noindent

Given $\Q_{F_\A}$ and $\Q_{F_\B}$ the sets of final states of
$\A$ and $\B$, $\#()$ a symbol of arity $1$ not occurring in $\F$,
to prove $\Lang(\A) \subseteq \Lang(\B)$, we start our deduction by the
statement: $\emptyset \vdash_{\A, \B} \{\#(q)\ |\ q \in \mathcal{Q}_{F_\A}\} \Subset
  \{\#(q)\ |\ q \in \mathcal{Q}_{F_\B}\} $
 

\begin{example}
  Let $\A$ and $\B$ be two automata s.t.:
  {\small
  \[\A = \left\{ 
    \begin{array}{rcl}
      a& \rightarrow &q_1\\
      b& \rightarrow &q_2\\
      f(q_1, q_2)&\rightarrow &{\bf q}\\
    \end{array}\right\}
  \textrm { with } \Q_{F_\A}=\{{\bf q}\}
  \textrm { and }
  \B = \left\{ 
    \begin{array}{rcl}
      a& \rightarrow &{\bf q'}\\
      b& \rightarrow &{\bf q'}\\
      f(q', q')&\rightarrow &{\bf q'}\\
    \end{array}\right\}
  \textrm { with } \Q_{F_\B}=\{{\bf q'}\}
  \]
  }
  Here we have $\Lang(\A) \subseteq \Lang(\B)$ and we can derive
  $\emptyset \vdash_{\A, \B} \#(q) \Subset \#(q')$ with the deduction rules:

  {\tiny
    \begin{prooftree}
      \AxiomC{}
      \LeftLabel{(Const.)}
      \UnaryInfC{$\{q \Subset q', q_1 \Subset q'\} \vdash_{\A, \B} a() \Subset a()$}
      \LeftLabel{(Weark-r)}
      \UnaryInfC{$ \{q \Subset q',\ q_1 \Subset q'\} \vdash_{\A, \B} a() \Subset \{ a(), b(), f(q', q') \}$}
      \LeftLabel{(Induction)}
      \UnaryInfC{$\{q \Subset q'\} \vdash_{\A, \B} q_1 \Subset q'$}
      %%%%%%%%%%%%%%%%%%%%%%%%%%%%% 
      \AxiomC{}
      \LeftLabel{(Const.)}
      \UnaryInfC{$\{q \Subset q', q_2 \Subset q'\} \vdash_{\A, \B} b() \Subset b()$}
      %\RightLabel{(Weark-r)}
      \UnaryInfC{$ \{q \Subset q',\ q_2 \Subset q'\} \vdash_{\A, \B} b() \Subset \{ a(), b(), f(q', q') \}$}
      %\RightLabel{(Induction)}
      \UnaryInfC{$\{q \Subset q'\} \vdash_{\A, \B} q_2 \Subset q'$}
      %%%%%%%%%%%%%%%%%%%%%%%%%%%%%%%%%%%%%%
      \LeftLabel{(Config)}
      \BinaryInfC{$\{q \Subset q'\} \vdash_{\A, \B} f(q_1, q_2) \Subset f(q', q') $ }
      \LeftLabel{(Weark-r)}
      \UnaryInfC{ $\{q \Subset q'\} \vdash_{\A, \B} f(q_1, q_2) \Subset \{ a(), b(), f(q', q')\} $}
      \LeftLabel{(Induction)}
      \UnaryInfC{$\emptyset \vdash_{\A, \B} q \Subset q'$}
      \LeftLabel{(Config)}
      \UnaryInfC{$\emptyset \vdash_{\A, \B} \#(q) \Subset \#(q')$}
    \end{prooftree}
  }
\end{example}

%We define the simple inclusion relation as :
%\begin{definition}
%  Let $\Delta$ and $\Delta'$ be two transition functions.\\
%  We say $\Delta \subseteq \Delta'$ if $\Delta'$ contains all rules of $\Delta$ :
%  \begin{equation}
%    \Delta \subseteq \Delta' \Longleftrightarrow \forall\ (c \rightarrow q) \in \Delta,\ (c \rightarrow q) \in \Delta'
%  \end{equation}
%\end{definition}
%Then we have trivially the property :
%\begin{theorem}
%  Given two transition function $\Delta$ and $\Delta'$,
%  \begin{equation}
%    \forall q,\ \Lang(\Delta, q) \subseteq \Lang(\Delta', q)
%  \end{equation}
%\end{theorem}

The main property we want demonstrate in \coq\
is that this syntactic criterion implies the semantic inclusion for the
considered languages in \ref{sec:automata}.

\begin{lstlisting}
Theorem inclusion_sound :
   forall A B, inclusion A B = true -> Included A B.
\end{lstlisting}


Before proving this in \coq, we need to define more formally the function
\lstinline!inclusion!. This function cannot be defined as a simple structural
recursion.  Thus \coq\ needs a termination
proof for this algorithm. Thanks to the \coq\ feature \lstinline!Function!, it is
possible to define the algorithm using a measure function and provide a proof
that its value decreases at each recursive call to ensure the termination.

\versioncourte{\subsection{Termination, Soundness, Completeness}}
\versionlongue{\subsection{Termination}}
\label{sec:termination}
Termination of deduction rules can be proved by defining a measure function $\mu$ on statements of he form $\Gamma \vdash_{\A,\B} \alpha \Subset \beta$.
%We use the finiteness property of $\Gamma$ that is finite and the size of terms on the right of $\vdash$.
The $\Gamma$ relation can be seen as a subset of $\mathcal{Q}_A \times \mathcal{Q}_B$ which is a finite set. All tree automata have
a finite number of states.
Then the statement measure $\mu({\Gamma \vdash_{A, B} \alpha \unlhd \beta})$ is defined as tuple $(\mu_1(\Gamma), \mu_2(\alpha)+\mu_2(\beta))$
where:
{\small
\[\left[
  \begin{array}{lcl}
    \mu_1(\Gamma) & = & | \Q_A \times \Q_B | - |\Gamma|\\
    \mu_2(x) &=&
    \left\{\begin{array}{l}
        (m+1-n) \textrm{ if } x = \{c_i\}^m_n\\
        1 \textrm{ if } x = f(q_1, \dots, q_n),\\
        0 \textrm{ otherwise}
      \end{array}\right.\\
  \end{array}\right.
\]}

Then we define the ordering $\ll$ by the lexicographic combination of the usual 
order $<$ on natural numbers for $\mu_1$ and $\mu_2$.% and $\mu_3$:
\versionlongue{
\[(x, y) \ll (x', y') \Longleftrightarrow
\bigvee\left\{
\begin{array}{l}
  x < x' \\
  x = x' \land y < y'\\
  %x = x' \land y = y' \land z < z'
\end{array}\right.
\]
}
Since $<$ is well founded, the lexicographic combination $\ll$ is also well
founded.

\begin{theorem}{(Termination)}
  At each deduction step, the measure decreases strictly:
  \[
  \dfrac{
    \Gamma \vdash_{A, B} \alpha \Subset \beta
  }{
    \Gamma' \vdash_{A, B} \alpha' \Subset \beta'
  }
  \Longrightarrow \mu({\Gamma \vdash_{A, B} \alpha \Subset \beta}) \ll \mu({\Gamma' \vdash_{A, B} \alpha' \Subset \beta'})
  \]
\end{theorem}
\versioncourte{
\begin{proof}
See~\cite{BoyerGJ-RR08}, for details.
\end{proof}
}
\annexe{
\begin{proof}
  The following array summarizes for each derivation rule what component of the tuple
  proves that $\mu$ decreases between conclusion and premises of the rule:

  \[\begin{array}[h]{l|c|c|}%c|}
    \footnotesize
    & \mu_1 & \mu_2 \\ \hline % & \mu_3\\ \hline
    \textrm{Induction}   & \strut \mu_1(\Gamma) < \mu_1(\Gamma') & - \\ \hline
    \textrm{Split-l} & \mu_1(\Gamma) = \mu_1(\Gamma') & 
    %\begin{array}{l}
      \mu_2(c_i) = 1 < \mu_2(\{c_i\}_1^n)\\
     % \mu_2((\{c_i\}_2^n) < \mu_2(\{c_i\}_1^n)\\
    %\end{array}\\ \hline

    \textrm{Weak-r} & \mu_1(\Gamma) = \mu_1(\Gamma') & \mu_2(c_k) < \mu_2(\{c_i\}_1^n)\\ \hline
    \textrm{Config} & \mu_1(\Gamma) = \mu_1(\Gamma') & 
    \begin{array}{l}
      \mu_2(f(\dots, q_i, \dots)) = \mu_2(f(\dots, q'_i, \dots)) = 1\\
      \mu_2(q_i) = \mu_2(q'_i) = 0 \textrm{ thus } 2 > 0
    \end{array}\\ \hline
    %\textrm{Tuple} & \mu_1(\Gamma) = \mu_1(\Gamma') & 0 = 0 &
    %\begin{array}{l}
    %  \mu_3([q_1, \dots, q_n]) > \mu_3(q1)\\
    %  \mu_3([q_1, \dots, q_n]) > \mu_3([q_2, \dots, q_n])\\
    %\end{array}\\ \hline
\end{array} \]
For the {\tt Split-l} (resp. {\tt Weak-r}) rule, we consider $n > 1$ to have a set $\alpha = \{c_i\}_1^n$ (resp. $\beta$) with
at least two elements. If ($n = 1$) then this rule does not apply on the current statement
${\Gamma \vdash \alpha \Subset \beta}$.\\

\end{proof}
}
\versionlongue{
\begin{theorem}
  When $\mu(\Gamma \vdash_{A, B} \alpha \Subset \beta) = (0, 0)$, we have a statement as Axiom or Nil:
  the current proof derivation is completed.
\end{theorem}
\begin{proof}
  From  $\mu(\Gamma \vdash_{A, B} \alpha \Subset \beta) = (0, 0)$ we deduce immediately:
  \begin{enumerate}
  \item $\mu_1(\Gamma) = 0$ implies $\Gamma = \Q_A \times \Q_B$ 
  \item $\mu_2(\alpha) = \mu_2(\beta) = 0$ implies $\alpha$ and $\beta$ are both either a state or empty set.
  \end{enumerate}
  Thus the statement may be:
  \begin{itemize}
  \item $\Gamma \vdash_{A, B} \emptyset \Subset \emptyset$ is the case
    of Empty rule : proof derivation is ended.
  \item $\Gamma \vdash_{A, B} q \Subset q'$ : we can use the fact
    $\Gamma = \Q_A \times \Q_B$, thus $(q, q') \in \Q_A\times\Q_B \imp
    (q \Subset q') \in \Gamma$. This case matches with Axiom rule that
    terminates the proof derivation.
  \end{itemize}
\end{proof}
}
\archive{
In \coq, do not need not exactly define all the measure as such. In fact,
a part of the algorithm can be expressed by structural recursion. In
particular, we use \lstinline!list! to represent configurations set as
for configurations which are expressed by \lstinline!list! of
\lstinline!state!. Since list covering is structural recursive, it is
also for configurations set. Then we have implemented this algorithm
as a single big recursive function \lstinline!inclusion_aux : Delta.t -> Delta.t -> gamma -> state -> state -> bool!  where
\lstinline!gamma! stands for a \lstinline!list! of \lstinline!state * state!. For a such function we only need $\mu_1$ and
we have reduce the \coq\ proof termination at one case.}

\versionlongue{\subsection{Soundness}}
\label{sec:correction}

\versioncourte{
\begin{theorem}{(Soundness)}
  \label{thm:soundness}
  For all tree automata $\A$ and $\B$, if there exists $\prod$ a proof tree
  of $\emptyset \vdash_{\A, \B} q \Subset q'$ then we have
  $\Lang(\Delta_\A, q) \subseteq \Lang(\Delta_\B, q')$
\end{theorem}

\begin{proof}
This can be done by an induction on the size of the term of 
$\Lang(\Delta_\A, q)$. See~\cite{BoyerGJ-RR08} for details.
\end{proof}
}

\annexe{
  Before proving it, we have to define and prove the following theorem.
  \begin{theorem}{(Cut in $\Subset$-proof trees)}
    \label{thm:cut}
    For all tree automata $\A$ and $\B$, if there exists $\prod$ a proof tree
    of $\Gamma \vdash_{\A, \B} q \Subset q'$, and a proof tree of 
    $\Gamma \cup \{q \Subset q'\}\vdash_{\A, \B} q_a \Subset q_b$
    then exists also a proof tree of $\Gamma \vdash_{\A, \B} q_a \Subset q_b$.
  \end{theorem}
  \begin{proof}
    We proceed by induction on $\mu(\Gamma)$.
    
    \noindent
    If $\mu(\Gamma) = 0$, we have immediately $\Q_\A \times \Q_\B =
    \Gamma$. Hence,
    since $q_a \Subset q_b \in \Gamma$, 
    we can prove $\Gamma \vdash_{\A,\B} q_a \Subset q_b$ using the Axiom rule.

    \medskip
    \noindent
    Now, as induction hypothesis, let us assume that $\forall \Gamma\ 
    s.t.\ \mu(\Gamma) = n$, $\forall q\ q'$, if there exists a proof tree $\prod$ of
    $\Gamma \vdash_{\A, \B} q \Subset q'$ and if for all $q_a, q_b$ there exists a
    proof tree of $\Gamma \cup \{q \Subset q'\}\vdash_{\A, \B} q_a \Subset q_b$ then we
    have also a proof tree of $\Gamma \vdash_{\A, \B} q_a \Subset q_b$.  Now, we
    aim at proving that this property is true for $\Gamma$ such that $\mu(\Gamma)
    = n+1$.
  
    \medskip
    \noindent
    Let us consider the proof tree of the second hypothesis $\Gamma \cup \{q \Subset
    q'\}\vdash_{\A, \B} q_a \Subset q_b$.  Firstly, if the proof tree is built
    using the Axiom rule we have $(q_a \Subset q_b) \in \Gamma \cup \{(q \Subset
    q')\}$.  Two cases are possible:
    \begin{itemize}
    \item either $(q_a \Subset q_b) \in \Gamma$, and then we build the proof of 
      $\Gamma \vdash_{\A,\B} q_a \Subset q_b$ using the Axiom rule.
      
    \item or $q = q_a$ and $q' = q_b$, and then the goal $\Gamma \vdash_{\A,\B} q_a
      \Subset q_b$ is equivalent to $\Gamma \vdash_{\A, \B} q \Subset q'$ whose
      proof tree is $\prod$.
    \end{itemize}
    
    \noindent
    Secondly, if the proof tree of $\Gamma \cup \{q \Subset q'\}\vdash_{\A, \B} q_a
    \Subset q_b$ is built using the Induction rule, then we have: 

    \medskip
    \newcommand{\env}{\Gamma \cup \{q_a \Subset q_b\} \cup \{q \Subset q'\} \vdash_{\A, \B}}
    
    \centerline{
      \begin{minipage}{21cm}
        {\tiny
          \begin{prooftree}
            \AxiomC{\small $\prod_{c_1}$}
            \UnaryInfC{$\env c_1 \Subset c'_{k_1}$}
            \LeftLabel{(Weark-r)}
            \UnaryInfC{$\env 
              c_1 \Subset  \{c_k'| c_k' \rightarrow_\B q_b\}_1^m$}
            % Pointillets du milieu
            \AxiomC{\small \dots\dots}
            \AxiomC{\small $\prod_{c_n}$}
            \UnaryInfC{$\env c_n \Subset c'_{k_n}$}
            \RightLabel{(Weark-r)}
            \UnaryInfC{$\env
              c_n \Subset  \{c_k'| c_k' \rightarrow_\B q_b\}_1^m$}
            \LeftLabel{(Split-l)}
            \TrinaryInfC{$\env
              \{c_i| c_i \rightarrow_\A q_a\}_1^n \Subset  \{c_i'| c_i'\rightarrow_\B q_b\}_1^m$}
            \LeftLabel{(Induction)}
            \UnaryInfC{$\Gamma \cup \{q \Subset q'\} \vdash_{\A, \B} q_a \Subset q_b$}
          \end{prooftree}
        }
      \end{minipage}}
    
    \medskip
    \noindent
    Where each $\prod_{c_i}$ has the following form (assuming
    that $c_i = f(q_{i_1}, \dots, q_{i_n})$ and $c'_{k_i} = f
    (q'_{i_1}, \dots, q'_{i_n})$ ):
    
    {\tiny
      \begin{prooftree}
        \AxiomC{\small $\prod_{i_1}$}
        \UnaryInfC{$\env q_{i_1} \Subset q'_{i_1}$}
        \AxiomC{\small \dots\dots}
        \AxiomC{\small $\prod_{i_n}$}
        \UnaryInfC{$\env q_{i_n} \Subset q'_{i_n}$}
        \LeftLabel{(Config)}
        \TrinaryInfC{$\env f(q_{i_1}, \dots, q_{i_n}) \Subset f (q'_{i_1}, \dots, q'_{i_n})$}
      \end{prooftree}
    }
    
    If we try to build the proof tree of our goal $\Gamma \vdash_{\A,\B} q_a
    \Subset q_b$, it necessarily begins in the same way except that $\{q \Subset
    q'\}$ will not appear in the left-hand side of statements. Each branch of
    this tree will end by a statement of the form $\Gamma \cup \{q_a \Subset
    q_b\} \vdash_{\A, \B} q_{i_j} \Subset q'_{i_j}$. Now to conclude the proof,
    we have to find proof trees $\prod'_{i_j}$ for all those statements. We know
    that there exists proof trees $\prod_{i_j}$ for all statements $\Gamma \cup \{q
    \Subset q'\} \cup \{q_a \Subset q_b\} \vdash_{\A, \B} q_{i_j} \Subset
    q'_{i_j}$.  We can use the induction hypothesis on $\prod_{i_j}$ to obtain
    $\prod'_{i_j}$ as follows:
    \begin{itemize}

    \item Since $\mu(\Gamma) = n + 1$, then $\mu(\Gamma\cup \{q_a \Subset q_b\}) = n$
      

    \item Since $\prod$ is a proof of $\Gamma \vdash_{\A, \B} q \Subset q'$, it is also a proof of
      $\Gamma \cup \{q_a \Subset q_b\}\vdash_{\A, \B} q \Subset q'$.
      
    \item Each $\prod_{i_j}$ is a proof of $\env  q_{i_j} \Subset q'_{i_j}$
    \end{itemize}
    
    \noindent
    Using induction, we deduce that for all $i,j$ there exist proof trees
    $\prod'_{i_j}$ of $\Gamma \cup \{q_a \Subset q_b\} \vdash_{\A, \B} q_{i_j}
    \Subset q'_{i_j}$.  This ends the proof tree of our goal $\Gamma \vdash_{\A,
      \B} q_a \Subset q_b$.
  \end{proof}
  
  \begin{theorem}{(Soundness)}
    \label{thm:soundness}
    For all tree automata $\A$ and $\B$, if there exists $\prod$ a proof tree
    of $\emptyset \vdash_{\A, \B} q \Subset q'$ then we have $\Lang(\A, q) \subseteq \Lang(\B, q')$
  \end{theorem}
  
  \begin{proof}
    We prove that $\forall t$, $t \in \Lang(\A, q) \imp t \in \Lang(\B, q')$ by
    induction on $t$. Let $t = f(t_1, \dots, t_n)$. We assume that the property is
    true for each subterm $t_i$, i.e. for all $q_i, q'_i$ s.t. if there exists a
    proof tree $\prod_i$ of $\emptyset \vdash_{\A, \B} q_i \Subset q'_i$ then $t_i
    \rightarrow^*_{\A} q \imp t \rightarrow^*_{\B} q_i'$.  Since $t=f(t_1, \dots,
    t_n) \in \Lang(\A, q)$, then for each subterm $t_i$, we know that there exists
    $q_1, \ldots, q_n$ such that $t_i \in \Lang(\A, q_i)$ and $f(q_1, \dots, q_n)
    \rightarrow q \in \A$. Besides this, by unfolding $\prod$ the proof tree of
    $\emptyset \vdash_{\A, \B} q \Subset q'$, we can deduce that for each
    transition like $f(q_1, \dots, q_n) \rightarrow q \in \A$, there exists $f(q'_1,
    \dots, q'_n) \rightarrow q' \in \B$ s.t. we have a proof tree $\prod_i$ of
    $\{q \Subset q'\} \vdash_{\A, \B} q_i \Subset q'_i$. 
    Since $f(q_1, \ldots, q_n) \rw q \in \A$, we obtain that $f(q'_1, \dots, q'_n)
    \rightarrow q' \in \B$ and a proof $\prod_i$ for $\{q \Subset q'\} \vdash_{\A,
      \B} q_i \Subset q'_i$. To conclude that $f(t_1, \dots, t_n) \in \Lang(\B,
    q')$ we just have to prove that $t_i \in \Lang(\B, q'_i)$. Note that we have a
    proof tree $\prod_i$ for $\{q \Subset q'\} \vdash_{\A,\B} q_i \Subset q'_i$
    and that to apply the induction hypothesis we need a proof tree for $\emptyset
    \vdash_{\A,\B} q_i \Subset q'_i$. Using the Theorem~\ref{thm:cut} on $\prod$
    and $\prod_i$, we can deduce the existence of $\prod'_i$ the proof tree of
    $\emptyset \vdash_{\A, \B} q_i \Subset q'_i$. Then using induction hypothesis
    on $t'_i, q_i, q'_i$ and $\prod'_i$,
    we obtain that for each $t_i \in \Lang(\A, q_i)$, we also have $t_i \in
    \Lang(\B, q'_i)$. Finally, since $f(q'_1, \ldots, q'_n)\rw q' \in \B$, we
    obtain that $t=f(t_1, \dots, t_n) \in \Lang(\B, q')$.
  \end{proof}
}


\versionlongue{\subsection{Completeness}}
\label{sec:completness}

As said above, the described algorithm is not complete in general.
However, we show that it is complete for tree automata produced by
completion. In particular if $\A_\R^k$ is obtained after $k$
completion step from $\A^0$ then we can build a proof $\prod$ for the
statement $\emptyset \vdash_{\A^0, \A^k_{\R}} \{\#(q)\ |\ q \in
\mathcal{Q}_{F_0}\} \Subset \{\#(q')\ |\ q'\in
\mathcal{Q}_{F_k}\}$. Recall that the tree automaton produced by the
$k^{th}$ step of completion is noted $\mathcal{A}_k = \langle
\mathcal{F}, \mathcal{Q}_k, \mathcal{Q}_{F_k}, \Delta_k\rangle$.
\versioncourte{ The tree automata completion performs two main
  operations at each step of calculus: \emph{normalization} and
  \emph{state merging}.  In the case of normalization, the language
  inclusion can simply be proved using transition set inclusion. With
  the state merging operation, set inclusion is not enough because it
  implies transition merging too.  This is the reason why we have to
  define a new ordered relation preserved by each operation.  }
\versionlongue{
First, let us introduce some new ordered relation on tree automata:
}
\begin{definition}
    \label{eq:prop0}
  Given $\A$, $\B$ two tree automata, $\sqsubseteq$ is the reflexive and
  transitive relation defined as follows: $\A \sqsubseteq \B$ if there exists a function
  $\varrho$ that renames states of $\A$ into states of $\B$ and such that 
  all renamed rules $\Delta_\A$ are contained in $\Delta_\B$:
  
  \begin{equation}
    \A \sqsubseteq \B \Longleftrightarrow
    \exists \varrho : \mathcal{Q_\A} \rightarrow \mathcal{Q_\B},
    \ \varrho(\Delta_\A) \subseteq \Delta_\B\ \wedge \ \varrho(\Q_{F_\A}) \subseteq \Q_{F_\B}
  \end{equation}
\end{definition}


\annexe{
We need to extend the renaming $\varrho$ to the structures and sets used in the following:
\begin{itemize}
\item $\varrho(\{q_i\}_1^n)$ stands for $\{\varrho(q_i)\}_1^n$

\item $\varrho(c) = \left\{\begin{array}{ll}
      f (\varrho(c_1), \dots, \varrho(c_n)) &\textrm{ if } c = f(c_1, \dots, c_n)\\
      c &\textrm{ if } c \in \F_{0}\\
      \varrho(q) &\textrm{ if } c = q\in \Q\\
    \end{array}\right.$
  
\item $\varrho(c \rightarrow q)$ stands for $\varrho(c) \rightarrow \varrho(q)$


\item $\varrho(\Delta)$ stands for $\{ \varrho(c\rightarrow q)\ |\ c \rightarrow q \in \Delta \}$.
\end{itemize}
}
\versionlongue{
In tree automata completion, two main operations can be
performed at each steps of calculus: \emph{normalization} and
\emph{state merging}. 
The following lemma shows that those two operations ensure $\sqsubseteq$.
}

\begin{lemma}
  Given a tree automaton $\A$,
  
  \[
  \begin{array}{llll}
    1. & \mbox{ if } \A' = \A \cup \norm(r\sigma \rw q) & \mbox{ then } &\A \sqsubseteq \A' \\
    2. & \mbox{ if } \A' = \merge(\A, q_1,q_2) &\mbox{ then }& \A \sqsubseteq \A'  \\
  \end{array}
  \]
\end{lemma}
\versioncourte{
\begin{proof}
  For details, see~\cite{BoyerGJ-RR08}.
\end{proof}
}

\annexe{
\begin{proof}
  \begin{enumerate}
  \item This is easy to show since we trivially have $\Delta_{A'} \supseteq
    \Delta_A$ whatever $r\sigma$ or $q$ may be.
    Then by choosing $\varrho = id$, we have immediately the conclusion $\A
    \sqsubseteq \A'$.

  \item Let $\Delta_\A$ be the transition set of $\A$. Let $q_i$ and $q_j$ be
    the two states to merge.  We can apply to $\Delta_\A$
    a renaming function $\varrho$ which has the same behavior than state merging
    with regard to $q_1 = q_2$:
    \[
    \varrho(q) = \left\{
      \begin{array}{l}
        \textrm{if }(q\ =\ q_2)\\
        \quad \quad q_1\\
        \textrm{else}\\
        \quad \quad q
      \end{array}\right.
    \]
    
    So state merging builds $\Delta_{\A'} = \varrho(\Delta_\A)$ and by
    Definition~\ref{eq:prop0} we have trivially $\Delta_A \sqsubseteq
    \Delta_{\A'}$.
  \end{enumerate}
\end{proof}
}


\begin{theorem}
  Given a tree automaton $\mathcal{A}^0$, a TRS $\R$ and an
  equation set $\mathcal{E}$, after $k$ completion steps we obtain
  $\A_\R^k$ such that $\A^0 \sqsubseteq \A_\R^k$.
\end{theorem}
\versioncourte{
\begin{proof}
  Since we have proved that $\sqsubseteq$ is preserved
  by $\norm$ and $\merge$ functions, it is also the case for every completion 
  step between $\A^k_\R$ and $\A^{k+1}_\R$, i.e $\A^k_\R \sqsubseteq \A^{k+1}_\R$.
  Then, the theorem can be deduced using the reflexivity and transitivity of
  $\sqsubseteq$. See~\cite{BoyerGJ-RR08}.
\end{proof}
}

\annexe{
\begin{proof}
  By induction on $k$:
  
  \begin{itemize}
  \item Since $\sqsubseteq$ is reflexive, we have trivially $\A^0
    \sqsubseteq \A^0$.
  \item Let $\mathcal{A}_k$ be a tree automaton obtained after $k$ completion
    steps such that $\A^0 \sqsubseteq \A_\R^k$. By definition of completion
    $\A_\R^{k+1}$ is built from $\A_\R^k$ by applying successively normalization
    and merge. We thus have $\A_\R^k \sqsubseteq \A_\R^{k+1}$.  
%Let us apply one
%    more step of completion. After normalization of $\A_\R^k$ we have $\A_\R^k
%    \sqsubseteq \A'$; then we merge $\A'$ with set $\mathcal{E}$ to obtain $\A'
%    \sqsubseteq \A_\R^{k+1}$. 
    By transitivity of $\sqsubseteq$, from $\A^0 \sqsubseteq \A_\R^k$ and
    $\A_\R^k \sqsubseteq \A_\R^{k+1}$ we deduce immediately that $\A^0
    \sqsubseteq \A_\R^{k+1}$.
  \end{itemize}
\end{proof}
}
\versioncourte{
Now, we define the completness property as the following:
}
\begin{theorem}{(Completeness)}
  Given two tree automata $\A$ and $\B$ if $\A \sqsubseteq \B$ then there
  exists $\prod$ a proof of statement $\emptyset \vdash_{\A, \B}
  \{\#(q_f)\ |\  q_f \in \mathcal{Q}_{F_A}\} \Subset  \{\#(q'_f)\ |\  q'_f \in \mathcal{Q}_{F_B}\}$.
\end{theorem}
\versioncourte{
\begin{proof}
For details, see~\cite{BoyerGJ-RR08}.
\end{proof}
}
\annexe{
\begin{proof}
  By definition of $\A \sqsubseteq \B$, we can deduce that there exists a renaming $\varrho$.\\
  First we prove by induction on the proof tree we have for all $\Gamma$
  and $q$, $\Gamma \vdash_{\A, \B} q \Subset \varrho(q)$:\\
  The hypothesis induction is $\forall \Gamma,\ q,\ q_i,$
  \begin{prooftree}
    \AxiomC{$\prod_i$}
    \UnaryInfC{$\Gamma \cup \{q \Subset \varrho(q)\} \vdash_{\A, \B} q_i \Subset \varrho(q_i)$}
  \end{prooftree}

  We want to construct a proof tree for $\Gamma \vdash_{\A, \B} q \Subset \varrho (q)$.\\
  Two cases are possible:
  \begin{itemize}
  \item 
    if $q \Subset \varrho(q) \in \Gamma$ then we can conclude immediately:
    \begin{prooftree}
      \AxiomC{}
      \LeftLabel{(Axiom)}
      \UnaryInfC{$\Gamma' \cup  \{q \Subset \varrho(q)\} \vdash_{\A, \B} q \Subset \varrho(q)$}
    \end{prooftree}

  \item 
    Otherwise, we need to apply Induction rule to obtain the following tree:
  
    \newcommand{\env}{\Gamma \cup \{q \Subset \varrho(q)\} \vdash_{\A, \B}}
    {\tiny
      \begin{prooftree}
        \AxiomC{\small $\prod_{c_1}$}
        \UnaryInfC{$\env 
          c_1 \Subset  \{c_k'| c_k' \rightarrow \varrho(q)\}_1^m$}
        % Pointillets du milieu
        \AxiomC{\small \dots\dots}
        \AxiomC{\small $\prod_{c_n}$}
        \UnaryInfC{$\env
          c_n \Subset  \{c_k'| c_k' \rightarrow \varrho(q)\}_1^m$}
        \LeftLabel{(Split-l)}
        \TrinaryInfC{$\env
          \{c_i| c_i \rightarrow q\}_1^n \Subset  \{c_k'| c_k'\rightarrow \varrho(q)\}_1^m$}
        \LeftLabel{(Induction)}
        \UnaryInfC{$\Gamma \vdash_{\A, \B} q \Subset \varrho(q)$}
      \end{prooftree}
    } From hypothesis $\varrho(\Delta_A) \subseteq \Delta_B$ for each
    rule $c \rightarrow q$ of $\Delta_A$, we have $\varrho(c\rightarrow q) \in \Delta_\B$.
    \[\textrm{Thus for all } (c\rightarrow q)\in \Delta_\A,\textrm{ we have } \varrho(c) \in \{c_k' | c_k' \rightarrow \varrho(q)\}_1^m \]
    For each $c_i= f_i(q_{i_1}, \dots, q_{i_n})$ we can construct the corresponding tree
    $\prod_{c_i}$ which his each branch is concluded by $\prod_{i_j}$
    an instance of induction hypothesis for the corresponding state
    $q_{i_j}$: {\tiny
      \begin{prooftree}
        \AxiomC{\small $\prod_{i_1}$}
        \UnaryInfC{$\env q_{i_1} \Subset \varrho(q_{i_1}) $}
        \AxiomC{\small \dots\dots}
        \AxiomC{\small $\prod_{i_n}$}
        \UnaryInfC{$\env q_{i_n} \Subset \varrho(q_{i_n}) $}
        %%%%%%%%%%%%%%%%%%%%%%% 
        \LeftLabel{(Config)}
        \TrinaryInfC{$\env c_i \Subset  \varrho(c_i) $}
        \LeftLabel{(Weak-r)}
        \UnaryInfC{$\env c_i \Subset  \{c_k'| c_k' \rightarrow \varrho(q)\}_1^m$}
      \end{prooftree}
    }
  \end{itemize}

  Now, we have for all $\Gamma$ and $q \in \Q_\A$ there exists a proof
  tree $\prod_q$ for all statement $\Gamma \vdash_{\A, \B} q \Subset \varrho(q)$.\\
  In particular, this is true for $\Gamma = \emptyset$ all $q$ of
  $\Q_{F_\A}$.  Since we have $\A \sqsubseteq \B \imp
  \varrho(\Q_{F_\A}) \subseteq \Q_{F_\B}$, we can build a proof tree
  as:

  {\small
    \begin{prooftree}
      \AxiomC{$\prod_{q_{f_1}}$}
      \UnaryInfC{$\emptyset \vdash_{\A, \B} q_{f_1} \Subset \varrho(q_{f_1})$}
      %\AxiomC{$\emptyset \vdash_{\A, \B} [\ ] \Subset [\ ]$}
      \LeftLabel{(Config)}
      \UnaryInfC{$\emptyset \vdash_{\A, \B} \#(q_{f_1}) \Subset \#(\varrho(q_{f_1}))$}
      \LeftLabel{(Weak-r)}
      \UnaryInfC{$\emptyset \vdash_{\A, \B} q_{f_1} \Subset \Q_{F_\B}$}
      \AxiomC{\small \dots\dots}
      \AxiomC{$\prod_{q_{f_n}}$}
      \UnaryInfC{$\emptyset \vdash_{\A, \B} q_{f_1} \Subset \varrho(q_{f_1})$}
      %\AxiomC{$\emptyset \vdash_{\A, \B} [\ ] \Subset [\ ]$}
      \RightLabel{(Config)}
      \UnaryInfC{$\emptyset \vdash_{\A, \B} \#(q_{f_n}) \Subset \#(\varrho(q_{f_n}))$}
      \RightLabel{(Weak-r)}
      \UnaryInfC{$\emptyset \vdash_{\A, \B} \#(q_{f_n}) \Subset \{\#(q)\ |\ \Q_{F_\B}\}$}
      \LeftLabel{(Split-l)}
      \TrinaryInfC{$\emptyset \vdash_{\A, \B} \{\#(q)\ |\ q \in \Q_{F_\A}\} \Subset \{\#(q)\ |\ q \in \Q_{F_\B}\}$}
    \end{prooftree}}
\end{proof}
}


Thus, we can ensure that for an automaton $\A_\R^k$ obtained by $k$
completion steps from $\mathcal{A}^0$, there exists a proof $\prod$ of
the statement $\emptyset \vdash_{\A^0, \A^k_{\R}} \{\#(q)\ |\ q \in
\mathcal{Q}_{F_{0}} \Subset \{\#(q')\ |\ q'\in \mathcal{Q}_{F_{k}}\}$.
This can be obtained by a simple combination of the two previous
theorems.

%\begin{theorem}
%  Given a tree automaton $\mathcal{A}^0$, a TRS $\R$ and an equation set
%  $\mathcal{E}$, after $k$ completion steps we obtain $\A_\R^k$ such that
%\end{theorem}


\versionlongue{
\paragraph{Remark about restrictions:}
To obtain a decision procedure for all tree automata (and not only
those obtained by completion), the "Weak-r" rule
would have to be modified.  This version is too weak; it does not take all
possible cases for union construction into account.

\begin{example}
  Let $\A$ and $\B$ be two automata s.t.:
  \[\A = \left\{ 
    \begin{array}{rcl}
      a& \rightarrow &q_1\\
      b& \rightarrow &q_2\\
      c& \rightarrow &q_2\\
      f(q_1, q_2)&\rightarrow &{\bf q}\\
    \end{array}\right\}
  \textrm { and }
  \B = \left\{ 
    \begin{array}{rcl}
      a& \rightarrow &q_1'\\
      b& \rightarrow &q_2'\\
      c& \rightarrow &q_3'\\
      f(q_1', q_2')&\rightarrow &{\bf q'}\\
      f(q_1', q_3')&\rightarrow &{\bf q'}\\
    \end{array}\right\}
  \]
  Here we have $\Lang(\A, q) = \Lang(\B, q')$ but $\emptyset \vdash_{\B, \A} q' \Subset q$ is clearly derivable
  whereas $\emptyset \vdash_{\A, \B} q \Subset q'$ is not.
\end{example}
}

\subsection{Complexity}

As said in section~\ref{section:objectives}, the standard algorithm for checking
the inclusion $\Lang(\A) \subseteq \Lang(\B)$ is based on computing the
complement automaton $\overline{\B}$. However, for non deterministic tree
automata the size~\cite{TATA} of $\overline{\B}$ can be exponentially greater
than the size of $\B$. The algorithm that we have proposed above has not this
drawback and use only a memory size that is polynomial w.r.t. to the automata
sizes.

Let $|\Q|$ be the maximum number of states in tree automata $\A$ and $\B$.  The
proof trees built using the deduction rules we gave are of height at most
$|\Q|^2$. This is due to the rules 'Induction' and 'Axiom' ensuring that every
inclusion problem $q \Subset q'$ will be analyzed only once per branch.  Since
$q\in \Q_\A$ and $q' \in \Q_\B$ and we know that the cardinal of $\Q_\A$ and
$\Q_\B$ is bounded by $|\Q|$, the length of branch is at most bounded by
$|\Q|^2$.  Since, the \lstinline!inclusion! function only constructs one branch
of the proof tree at a time, the memory usage is thus bounded by $|\Q|^2$ and
thus polynomial.

However, the time complexity of a straightforward implementation of this
algorithm is exponential. Indeed, even if each couple $q \Subset q'$ is
considered only once on each branch, the number of branches is exponential
w.r.t. $\Q$ and the same couple $q \Subset q'$ may be analyzed once per
branch. A very simple optimization of this algorithm is to table the result of
the analysis of each $q \Subset q'$ and make it available to cut similar proof
branches. Using this optimization, every couple is considered {\em only once}
for the whole proof tree. Assuming that read and write operations of the table
can be performed in constant time, this leads to an overall polynomial time
complexity bounded by $|\Q|^2$. Nevertheless, the current \coq\ implementation is
based on the non tabled version for two reasons.  First, the proof of theorem
\lstinline!inclusion_sound! is more difficult on a tabled version of the \coq\
\lstinline!inclusion! function. Second, on test cases, it appears that avoiding
the exponential blow-up of memory was critical but that practical performances
of the, potentially, exponential time algorithm are sufficient.



%%% Local Variables: 
%%% mode: latex
%%% TeX-master: "main"
%%% End: 


\section{Formalization of closure by rewriting}
\label{sec:closure}

In this part we aim at defining formally the \lstinline!IsClosed! predicate, the
function \lstinline!closure! and prove the soundness of this function
w.r.t. \lstinline!IsClosed!. Recall that to
check if a tree automaton $\A = \langle \Q_F,\ \Delta \rangle$ is closed
w.r.t. a TRS $\R$, it is enough to prove that for all $t \in \Lang(\A)$, if
$t'$ is reachable from $t$ by $\rw_\R^*$ then $t' \in \Lang(\A)$. 
Now that we have defined in \coq\ rewriting and tree
automata, we can define more formally the \lstinline!IsClosed! predicate and recall
the \lstinline!closure_sound! theorem to prove:

\begin{lstlisting}
Definition IsClosed (R : list rule) (A : t_aut) : Prop :=
   forall q t t', IsRec A.delta q t -> Reachable R t t' -> IsRec A.delta q t'.

Theorem closure_sound:
      forall R A', closure R A' = true -> IsClosed R A'.
\end{lstlisting}


The algorithm to check closure of $\A$ by $\R$ computes for each rule $l
\rightarrow r \in \R$ the full set of the substitutions $\sigma$ s.t. $l\sigma
\rightarrow_\Delta^* q$ and then, checks that $r\sigma \rightarrow_\Delta^*
q$. Then, the correctness proof consists in showing that if \lstinline!closure!
answers true, then $\Lang(\A)$ closed by $\rw_\R$.

We now give some hints to define the \lstinline!closure! function.  First, for
all rule $l \rw r$ of $\R$, this function has to find all the substitutions
$\sigma:\X \mapsto \Q$ and all the states $q \in \Q$ such that 
$l\sigma \rightarrow_\Delta^* q$. This is what we call the 
\emph{matching-problem}. Second, this function has to check that for all the $q$
and $\sigma$ found, we have $r\sigma \rw_{\Delta}^* q$. Third, in the
correctness theorem, we have to show that all the substitutions $\sigma:
\X\mapsto \Q$ cover the set of substitutions on terms, i.e. of the form $\sigma':
\X \mapsto \TF$, and hence cover all reachable terms.

We note $l \unlhd q$ the matching problem consisting in finding all the
substitutions $\sigma:\X \mapsto \Q$ and all the states $q \in \Q$ such that
$l\sigma \rightarrow_\Delta^* q$. An algorithm solving this kind of
problems was defined in~\cite{Genet-RR97b}. It consists in normalizing
the formula $l \unlhd q$ with the following deduction rules:

\[\textrm{(Unfold) }
\dfrac {
  f(s_1, \dots, s_n)\unlhd f(q_1,\dots, q_n)
}{
  s_1 \unlhd q_1 \land \dots \land s_n \unlhd q_n
}\quad \quad
\textrm{(Clash) }
\dfrac {
  f(s_1, \dots, s_n) \unlhd g(q'_1, \dots q'_m)
}{
\bottom}
\]
\[\textrm{(Config) }
\dfrac {
  s \unlhd q
}{
  s \unlhd c_1 \lor \dots \lor s \unlhd c_k \lor \bottom
}
\textrm{if }s \notin \X \textrm{, and } \{c_i\ |\ c_i \rightarrow q \in \Delta\}_1^k. 
\]

Moreover, after each application of one of this rules,
the result is also rewritten into disjunctive normal form.
\versionlongue{ using:
\[
\dfrac{ 
  \phi_1\land (\phi_2 \lor \phi_3)
}{
  (\phi_1 \land \phi_2) \lor (\phi_1 \land \phi_3)
}\quad
\dfrac {\phi_1 \lor \bottom }{\phi_1}\quad
\dfrac {\phi_1 \land \bottom}{\bottom}
\]
}
When normalization of the initial problem is terminated, we obtain a
formula like $ \bigvee_{i=1}^{n} \phi_i$ where
$\phi_i=\bigwedge_{j=1}^{m}x^i_j \unlhd q^i_j$ such that $x^i_j \in
\X$ and $q^i_j \in \Q$. Each $\phi_i$ can be seen as a
substitution $\sigma_i = \{x^i_j \mapsto q^i_j\}$.
\versionlongue{
\begin{theorem}{\cite{Genet-RR97b}}
\label{th:completeness}
  Substitutions $\sigma_i$ obtained by the matching algorithm are the only
  substitutions s.t. $s\sigma_i \rightarrow^*_\Delta q$.
\end{theorem}}
The implementation of the \lstinline!matching! function in \coq\ is very close
to this algorithm. 
\versionlongue{We implement disjunction as lists where $\bottom$ is mapped to
\lstinline!nil!.  The \coq\ signature of the matching function is
\lstinline!Delta.t -> state -> term -> list substitution!.  For this algorithm,
the termination is bound by a syntactic argument.  We can define it easily in
\coq\ by a simple structural recursion over the \lstinline!term! which has to be
matched.}
\versionlongue{
Moreover, the soundness and completeness properties, corresponding to
Theorem~\ref{th:completeness}, can be defined in \coq\ as follows:
}
\versioncourte{Moreover, the soundness and completeness properties of this
algorithm can be defined in \coq\ as follows:
}
\begin{lstlisting}
Theorem matching_sound :
   forall D q l s, In s (matching D q l) -> IsRec D q (l @ s).

Theorem matching_complete :
   forall D q l s, IsRec D q (l @ s) -> In s (matching D q l).
\end{lstlisting}

The second part of the \lstinline!closure! function consists in verifying that
for each substitution $\sigma$ s.t. $l\sigma \rightarrow_\Delta^* q$, we also
have $r\sigma \rightarrow_\Delta^* q$.  This job is performed using the
\lstinline!all_red! function, we define, whose purpose is to check that this
property is true for all the found substitutions. Then, we only need to prove
the soundness of this function using the following \coq\ theorem:

\begin{lstlisting}
Theorem all_red_sound : 
  forall D q r sigmas, 
     all_red D q r sigmas = true -> forall s, In s sigmas -> IsRed D q (r@s).
\end{lstlisting}

By combining the \lstinline!matching! and the \lstinline!all_red! functions, 
we obtain the algorithm for checking up all critical pairs
found at state $q$ and for the rule $l \rightarrow r$. We define the combination
as:

\versionlongue{

\begin{lstlisting}
Definition closure_at_state D q l r := all_red D q r (matching D q l).
\end{lstlisting}}

\begin{lstlisting}
Theorem closure_at_state_sound : 
   forall D q l r, closure_at_state D q l r = true -> 
      (forall s, IsRed D q (l @ s) -> IsRed D q (r @ s)).
\end{lstlisting}

  Given a rule $l \rightarrow r$ and a state $q$, this algorithm answers
  \lstinline!true! if for all substitution $\sigma: \X \mapsto \Q$ s.t. $l
  \sigma \rw^*_\Delta q$ then $r\sigma \rw^*_{\Delta} q$. Now that we have
  proved this result for substitutions $\sigma:\X \mapsto \Q$, we have to prove
  that it implies the same property for substitutions $\sigma':\X \mapsto \TF$,
  this is Lemma~\ref{lem:qtotf}. On the opposite, to prove that every reachable
  term of $\TF$ will be covered by a configuration of $\TFQ$ in $\Delta$, we
  have to prove that if there exists a substitution $\sigma': \X \mapsto \TF$,
  then we can construct a corresponding substitution $\sigma: \X \mapsto \Q$,
  this is Lemma~\ref{lem:tftoq}.
\begin{lemma}
\label{lem:qtotf}
Given a term $u \in \T(\F, \X)$, $\sigma :\X \mapsto \Q$ a substitution
s.t. $u\sigma \rightarrow^*_\Delta q$, if we have a substitution $\sigma' :\X
\mapsto \TF$ s.t. $\forall x \in \ddom(\sigma): \sigma' x \in \Lang(\Delta,
\sigma x)$, then we have $u\sigma' \rw^*_{\Delta} q$ and thus $u\sigma' \in
\Lang(\Delta, q)$.
\end{lemma}

Roughly, if the left or right-hand side $u$ of a rewriting rule matches
a configuration $u\sigma \in \TFQ$ and $u\sigma
\rw^*_{\Delta} q$ then, all terms $u\sigma' \in \TF$, matched by $u$, 
are also reducible into $q$, i.e. $u\sigma' \rw^*_{\Delta} q$ and $u\sigma' \in
\Lang(\Delta,q)$.

\begin{lemma}
\label{lem:tftoq}
Given a term $u \in \T(\F, \X)$, if there exists a substitution $\sigma':\X
\mapsto \TF$ such that $u\sigma' \rw_{\Delta}^* q$, then there exists a
substitution $\sigma:\X\mapsto \Q$ s.t. $\sigma' x \in \Lang(\Delta, \sigma x)$
and $u\sigma\rw^*_{\Delta} q$.
\end{lemma}

\versionlongue{In the same way, for all term $u\sigma' \in \Lang(\Delta, q)$ that can be
matched by $u$, there exists a configuration $u\sigma$ s.t. $u\sigma'
\rightarrow^*_\Delta u \sigma$ and $u\sigma \rightarrow^*_\Delta
q$.}
Using those two lemmas, we can conclude that for all term $t\in\Lang(\Delta,q)$
rewritten in $t'$ at the topmost position by $l \rightarrow r$, then $t' \in
\Lang(\Delta, q)$. This property is easily lifted as a property of the
\lstinline!closure! function for all states of $\Q$ and using all rules of $\R$
at topmost position.
\versioncourte{
Then, it is enough to lift this property to general rewriting at any
position. Finally, the \lstinline!closure_sound! general theorem is 
shown by using a reflexive and transitive application of the last 
property.
}

\begin{lstlisting}
Theorem closure_sound_0 : 
   forall D R, closure D R = true ->
      forall q lr, In q (states D) -> In lr R ->
         forall t t', IsRec D q t -> TRew lr t t' -> IsRec D q t'.
\end{lstlisting}
The next step consists in showing the same property but using rewriting at any
position, hence proving the same theorem with \lstinline!Rew! (general
rewriting) instead of \lstinline!TRew! (topmost rewriting) between
$t$ and $t'$.

\begin{lstlisting}
Theorem closure_sound_1 : 
   forall D R, closure D R = true ->
      forall q lr, In q (states D) -> In lr R ->
         forall t t', IsRec D q t -> Rew lr t t' -> IsRec D q t'.
\end{lstlisting}
Finally, the \lstinline!closure_sound! general theorem is 
shown by using a reflexive and transitive application of this
\lstinline!closure_sound_1! in order to deal with any term $t'$ that can be
reached using the \lstinline!Reachable! predicate.

%%% Local Variables: 
%%% mode: latex
%%% TeX-master: "main"
%%% End: 


\section{Benchmarks}
\label{sec:benchmarks}

From the \coq\ formal specification, we have extracted an Ocaml checker
implementation. In the following table, we have collected several benchmarks.
For each test, we
give the size of the two tree automata (initial $\A^0$ and completed $\A_\R^*$)
as number of transitions/number of states.  For each TRS $\R$ we give the number
of rules.  The 'CS' column gives the number of completion steps necessary to
complete $\A^0$ into $\A^*_{\R}$ and 'CT' gives the completion time.  The 'CKT'
column gives the time for the checker to certify the $\A^*_{\R}$ and the 'CKM'
gives the memory usage. The important thing to observe here is that, the
completion time is very long (sometimes more than 24 hours), the {\em checking}
of the corresponding automaton is always fast (a matter of seconds).

The four tests are Java programs translated into term rewriting systems using
the technique detailed in~\cite{BoichutGJL-RTA07}. All of them are completed
using \timbuk\ except the example {\tt List2.java} which has been completed using a completion
tool under development by Yohan Boichut and Emilie Balland.  This shows that the
completed automaton produced by a new and fully optimzed tool is also accepted
by our checker.  The {\tt List1.java} and {\tt List2.java} corresponds to the
same Java program but with slightly different encoding into TRS and
approximations. The {\tt Ex\_poly.java} is the example given
in~\cite{BoichutGJL-RTA07} and the {\tt Bad\_Fixp} is the same problem as {\tt
  Ex\_poly.java} except that the completed automaton $\A^*_{\R}$ has been
intentionally corrupted. Thus, this is thus not a valid fixpoint and rejected by
the checker.

\begin{center}
\begin{tabular}{l|c|c|c|c|c|c|c|c}
  Name & $\A^0$ & $\A^*_\R$ & $\R$ & CS & CT & CKT & CKM \\ \hline
  
  {\tt List1.java} & 118/82 & 422/219 & 228 & 180 & $\approx$ 3 days & 0,9s & 2,3 Mo \\ \hline
  %Yoh
  {\tt List2.java} & 1/1 & 954/364 & 308 & 473 & 1h30 & 2,2s & 3,1 Mo \\ \hline
  %Res_
  {\tt Ex\_poly.java} & 88/45 & 951/352 & 264 & 161 & $\approx$ 1 day & 2,5s & 3,3 Mo \\ \hline
  %Res_ modifi�
  {\tt Bad\_Fixp} & 88/45 & 949/352 & 264 & 161 & $\approx$ 1 day & 1,6s & 3,2 Mo \\ \hline

%  NSPK & 26tr. 12 states & 243 tr. 12st. & 13 & 6 & 8'/21.4s & 5 Mo \\ \hline
\end{tabular}
\end{center}
%%% Local Variables: 
%%% mode: latex
%%% TeX-master: "main"
%%% End: 


%\section{Complete certificate}
%\label{sec:certificate}
%Peut-�tre une partie pour expliquer comment termine-t-on la preuve 
%pour obtenir le certificat.
\section{Conclusion and further research}
\label{sec:conclusion}
In this paper we have defined a \coq\ checker for tree automata completion.  The
first characteristic of the work presented here is that the checker does not
validate a specific implementation of completion but, instead, the result. As a
consequence, the checker remains valid even if the implementation of the
completion algorithm changes or is optimized. A second salient feature is that
the code of the checker is directly generated from the correctness proof of its
verified \coq\ specification through the \coq\ extraction mechanism. Third, we
have payed particular attention to the formalization of the checker in order not
to lose efficiency to obtain the certification. We have defined a specific
inclusion algorithm in order to avoid the usual exponential blow-up obtained
with the standard inclusion algorithm.  We have defined the \coq\ formal
specification so that it was possible to extract an independent OCaml
checker. Finally, we made an extensive use of efficient formal data structures
leading to more complex proof but also to faster extracted checker.  An
extension for non left-linear TRS, which are sometimes necessary for specifying
cryptographic protocols, is under development.  Since many different kind of
analysis can be expressed as reachability problems over tree automata, and since
verification of completed automata revealed to be very efficient, we aim at
using this technique in a PCC architecture where tree automata are viewed as
program certificates. At last, note that even if this checker is external to
\coq, we can use the correction proof of the checker and the \coq\ reflexivity
mechanism to lift-up the external verification into a proof in the \coq\
system. This can be necessary to perform efficient unreachability proofs on
rewriting systems in \coq\ using an external completion tool.


\bibliographystyle{alpha}
\bibliography{sabbrev,eureca,genet}

\newpage
\versioncourte{
\appendix
\section{Proofs}
%\annexe
%\newpage
\setcounter{theorem}{0}
\setcounter{lemma}{0}
\begin{theorem}{(Termination)}
  At each deduction step, the measure decreases strictly:
  \[
  \dfrac{
    \Gamma \vdash_{\A, \B} \alpha \Subset \beta
  }{
    \Gamma' \vdash_{\A, \B} \alpha' \Subset \beta'
  }
  \Longrightarrow \mu({\Gamma \vdash_{\A, \B} \alpha \Subset \beta}) \ll \mu({\Gamma' \vdash_{\A, \B} \alpha' \Subset \beta'})
  \]
\end{theorem}
\begin{proof}
  The following array summarizes for each derivation rule what component of the tuple
  proves that $\mu$ decreases between conclusion and premises of the rule:
  
  \[\begin{array}[h]{l|c|c|}%c|}
    \footnotesize
    & \mu_1 & \mu_2 \\ \hline % & \mu_3\\ \hline
    \textrm{Induction}   & \strut \mu_1(\Gamma) < \mu_1(\Gamma') & - \\ \hline
    \textrm{Split-l} & \mu_1(\Gamma) = \mu_1(\Gamma')
                     & \mu_2(c_i) = 1 < \mu_2(\{c_i\}_1^n)\\ \hline
    \textrm{Weak-r} & \mu_1(\Gamma) = \mu_1(\Gamma') 
                    & \mu_2(c_k) < \mu_2(\{c_i\}_1^n)\\ \hline
    \textrm{Config} & \mu_1(\Gamma) = \mu_1(\Gamma') & 
    \begin{array}{l}
      \mu_2(f(\dots, q_i, \dots)) = \mu_2(f(\dots, q'_i, \dots)) = 1\\
      \mu_2(q_i) = \mu_2(q'_i) = 0 \textrm{ thus } 2 > 0
    \end{array}\\ \hline
  \end{array} \]
  For the Split-l (resp. Weak-r) rule, we consider $n > 1$ to have a set $\alpha = \{c_i\}_1^n$ (resp. $\beta$) with
  at least two elements. If ($n = 1$) then this rule does not apply on the current statement
  ${\Gamma \vdash \alpha \Subset \beta}$.\\

\end{proof}

\begin{theorem}{(Cut in $\Subset$-proof trees)}
  \label{thm:cut}
  For all tree automata $\A$ and $\B$, if there exists $\prod$ a proof tree
  of $\Gamma \vdash_{\A, \B} q \Subset q'$, and a proof tree of 
  $\Gamma \cup \{q \Subset q'\}\vdash_{\A, \B} q_a \Subset q_b$
  then exists also a proof tree of $\Gamma \vdash_{\A, \B} q_a \Subset q_b$.
\end{theorem}
\begin{proof}
  We proceed by induction on $\mu(\Gamma)$.

  \noindent
  If $\mu(\Gamma) = 0$, we have immediately $\Q_\A \times \Q_\B =
  \Gamma$. Hence,
  since $q_a \Subset q_b \in \Gamma$, 
  we can prove $\Gamma \vdash_{\A,\B} q_a \Subset q_b$ using the Axiom rule.

  \medskip
  \noindent
  Now, as induction hypothesis, let us assume that $\forall \Gamma\ 
  s.t.\ \mu(\Gamma) = n$, $\forall q\ q'$, if there exists a proof tree $\prod$ of
  $\Gamma \vdash_{\A, \B} q \Subset q'$ and if for all $q_a, q_b$ there exists a
  proof tree of $\Gamma \cup \{q \Subset q'\}\vdash_{\A, \B} q_a \Subset q_b$ then we
  have also a proof tree of $\Gamma \vdash_{\A, \B} q_a \Subset q_b$.  Now, we
  aim at proving that this property is true for $\Gamma$ such that $\mu(\Gamma)
  = n+1$.
  
  \medskip
  \noindent
  Let us consider the proof tree of the second hypothesis $\Gamma \cup \{q \Subset
  q'\}\vdash_{\A, \B} q_a \Subset q_b$.  Firstly, if the proof tree is built
  using the Axiom rule we have $(q_a \Subset q_b) \in \Gamma \cup \{(q \Subset
  q')\}$.  Two cases are possible:
  \begin{itemize}
  \item either $(q_a \Subset q_b) \in \Gamma$, and then we build the proof of 
    $\Gamma \vdash_{\A,\B} q_a \Subset q_b$ using the Axiom rule.

  \item or $q = q_a$ and $q' = q_b$, and then the goal $\Gamma \vdash_{\A,\B} q_a
    \Subset q_b$ is equivalent to $\Gamma \vdash_{\A, \B} q \Subset q'$ whose
    proof tree is $\prod$.
  \end{itemize}

  \noindent
  Secondly, if the proof tree of $\Gamma \cup \{q \Subset q'\}\vdash_{\A, \B} q_a
  \Subset q_b$ is built using the Induction rule, then we have: 

\medskip
  \newcommand{\env}{\Gamma \cup \{q_a \Subset q_b\} \cup \{q \Subset q'\} \vdash_{\A, \B}}

\centerline{
  \begin{minipage}{21cm}
    {\tiny
      \begin{prooftree}
        \AxiomC{\small $\prod_{c_1}$}
        \UnaryInfC{$\env c_1 \Subset c'_{k_1}$}
        \LeftLabel{(Weark-r)}
        \UnaryInfC{$\env 
          c_1 \Subset  \{c_k'| c_k' \rightarrow_\B q_b\}_1^m$}
        % Pointillets du milieu
        \AxiomC{\small \dots\dots}
        \AxiomC{\small $\prod_{c_n}$}
        \UnaryInfC{$\env c_n \Subset c'_{k_n}$}
        \RightLabel{(Weark-r)}
        \UnaryInfC{$\env
          c_n \Subset  \{c_k'| c_k' \rightarrow_\B q_b\}_1^m$}
        \LeftLabel{(Split-l)}
        \TrinaryInfC{$\env
          \{c_i| c_i \rightarrow_\A q_a\}_1^n \Subset  \{c_i'| c_i'\rightarrow_\B q_b\}_1^m$}
        \LeftLabel{(Induction)}
        \UnaryInfC{$\Gamma \cup \{q \Subset q'\} \vdash_{\A, \B} q_a \Subset q_b$}
      \end{prooftree}
    }
  \end{minipage}}

\medskip
    \noindent
    Where each $\prod_{c_i}$ has the following form (assuming
    that $c_i = f(q_{i_1}, \dots, q_{i_n})$ and $c'_{k_i} = f
    (q'_{i_1}, \dots, q'_{i_n})$ ):
    
    {\tiny
      \begin{prooftree}
        \AxiomC{\small $\prod_{i_1}$}
        \UnaryInfC{$\env q_{i_1} \Subset q'_{i_1}$}
        \AxiomC{\small \dots\dots}
        \AxiomC{\small $\prod_{i_n}$}
        \UnaryInfC{$\env q_{i_n} \Subset q'_{i_n}$}
        \LeftLabel{(Config)}
        \TrinaryInfC{$\env f(q_{i_1}, \dots, q_{i_n}) \Subset f (q'_{i_1}, \dots, q'_{i_n})$}
      \end{prooftree}
    }
    
    If we try to build the proof tree of our goal $\Gamma \vdash_{\A,\B} q_a
    \Subset q_b$, it necessarily begins in the same way except that $\{q \Subset
    q'\}$ will not appear in the left-hand side of statements. Each branch of
    this tree will end by a statement of the form $\Gamma \cup \{q_a \Subset
    q_b\} \vdash_{\A, \B} q_{i_j} \Subset q'_{i_j}$. Now to conclude the proof,
    we have to find proof trees $\prod'_{i_j}$ for all those statements. We know
    that there exists proof trees $\prod_{i_j}$ for all statements $\Gamma \cup \{q
    \Subset q'\} \cup \{q_a \Subset q_b\} \vdash_{\A, \B} q_{i_j} \Subset
    q'_{i_j}$.  We can use the induction hypothesis on $\prod_{i_j}$ to obtain
    $\prod'_{i_j}$ as follows:
    \begin{itemize}

    \item Since $\mu(\Gamma) = n + 1$, then $\mu(\Gamma\cup \{q_a \Subset q_b\}) = n$
      

    \item Since $\prod$ is a proof of $\Gamma \vdash_{\A, \B} q \Subset q'$, it is also a proof of
      $\Gamma \cup \{q_a \Subset q_b\}\vdash_{\A, \B} q \Subset q'$.

    \item Each $\prod_{i_j}$ is a proof of $\env  q_{i_j} \Subset q'_{i_j}$
    \end{itemize}

    \noindent
    Using induction, we deduce that for all $i,j$ there exist proof trees
    $\prod'_{i_j}$ of $\Gamma \cup \{q_a \Subset q_b\} \vdash_{\A, \B} q_{i_j}
    \Subset q'_{i_j}$.  This ends the proof tree of our goal $\Gamma \vdash_{\A,
      \B} q_a \Subset q_b$.
  \end{proof}



\begin{theorem}{(Soundness)}
  \label{thm:soundness}
  For all tree automata $\A$ and $\B$, if there exists $\prod$ a proof tree
  of $\emptyset \vdash_{\A, \B} q \Subset q'$ then we have $\Lang(\A, q) \subseteq \Lang(\B, q')$
\end{theorem}

\begin{proof}
  We prove that $\forall t$, $t \in \Lang(\A, q) \imp t \in \Lang(\B, q')$ by
  induction on $t$. Let $t = f(t_1, \dots, t_n)$. We assume that the property is
  true for each subterm $t_i$, i.e. for all $q_i, q'_i$ s.t. if there exists a
  proof tree $\prod_i$ of $\emptyset \vdash_{\A, \B} q_i \Subset q'_i$ then $t_i
  \rightarrow^*_{\A} q \imp t \rightarrow^*_{\B} q_i'$.  Since $t=f(t_1, \dots,
  t_n) \in \Lang(\A, q)$, then for each subterm $t_i$, we know that there exists
  $q_1, \ldots, q_n$ such that $t_i \in \Lang(\A, q_i)$ and $f(q_1, \dots, q_n)
  \rightarrow q \in \A$. Besides this, by unfolding $\prod$ the proof tree of
  $\emptyset \vdash_{\A, \B} q \Subset q'$, we can deduce that for each
  transition like $f(q_1, \dots, q_n) \rightarrow q \in \A$, there exists $f(q'_1,
  \dots, q'_n) \rightarrow q' \in \B$ s.t. we have a proof tree $\prod_i$ of
  $\{q \Subset q'\} \vdash_{\A, \B} q_i \Subset q'_i$. 
  Since $f(q_1, \ldots, q_n) \rw q \in \A$, we obtain that $f(q'_1, \dots, q'_n)
  \rightarrow q' \in \B$ and a proof $\prod_i$ for $\{q \Subset q'\} \vdash_{\A,
    \B} q_i \Subset q'_i$. To conclude that $f(t_1, \dots, t_n) \in \Lang(\B,
  q')$ we just have to prove that $t_i \in \Lang(\B, q'_i)$. Note that we have a
  proof tree $\prod_i$ for $\{q \Subset q'\} \vdash_{\A,\B} q_i \Subset q'_i$
  and that to apply the induction hypothesis we need a proof tree for $\emptyset
  \vdash_{\A,\B} q_i \Subset q'_i$. Using the Theorem~\ref{thm:cut} on $\prod$
  and $\prod_i$, we can deduce the existence of $\prod'_i$ the proof tree of
  $\emptyset \vdash_{\A, \B} q_i \Subset q'_i$. Then using induction hypothesis
  on $t'_i, q_i, q'_i$ and $\prod'_i$,
  we obtain that for each $t_i \in \Lang(\A, q_i)$, we also have $t_i \in
  \Lang(\B, q'_i)$. Finally, since $f(q'_1, \ldots, q'_n)\rw q' \in \B$, we
  obtain that $t=f(t_1, \dots, t_n) \in \Lang(\B, q')$.
  
\end{proof}

\annexe{
We need to extend the renaming $\varrho$ to the structures and sets used in the following:
\begin{itemize}
\item $\varrho(\{q_i\}_1^n)$ stands for $\{\varrho(q_i)\}_1^n$

\item $\varrho(c) = \left\{\begin{array}{ll}
      f (\varrho(c_1), \dots, \varrho(c_n)) &\textrm{ if } c = f(c_1, \dots, c_n)\\
      c &\textrm{ if } c \in \F_{0}\\
      \varrho(q) &\textrm{ if } c = q\in \Q\\
    \end{array}\right.$
  
\item $\varrho(c \rightarrow q)$ stands for $\varrho(c) \rightarrow \varrho(q)$

  
\item $\varrho(\Delta)$ stands for $\{ \varrho(c\rightarrow q)\ |\ c \rightarrow q \in \Delta \}$.
\end{itemize}
}
\begin{lemma}
\label{lem:renaming}
  Given a tree automaton $\A$,
  
  \[
  \begin{array}{llll}
    1. & \mbox{ if } \A' = \A \cup \norm(r\sigma \rw q) & \mbox{ then } &\A \sqsubseteq \A' \\
    2. & \mbox{ if } \A' = \merge(\A, q_1,q_2) &\mbox{ then }& \A \sqsubseteq \A'  \\
  \end{array}
  \]
\end{lemma}

\begin{proof}
  \begin{enumerate}
  \item This is easy to show since we trivially have $\Delta_{\A'} \supseteq
    \Delta_\A$ whatever $r\sigma$ or $q$ may be.
    Then by choosing $\varrho = id$, we have immediately the conclusion $\A
    \sqsubseteq \A'$.
    
  \item Let $\Delta_\A$ be the transition set of $\A$. Let $q_i$ and $q_j$ be
    the two states to merge.  We can apply to $\Delta_\A$
    a renaming function $\varrho$ which has the same behavior than state merging
    with regard to $q_1 = q_2$:
    \[
    \varrho(q) = \left\{
      \begin{array}{l}
        \textrm{if }(q\ =\ q_2)\\
        \quad \quad q_1\\
        \textrm{else}\\
        \quad \quad q
      \end{array}\right.
    \]
    
    So state merging builds $\Delta_{\A'} = \varrho(\Delta_\A)$ and by
    Definition~\ref{eq:prop0} we have trivially $\Delta_\A \sqsubseteq
    \Delta_{\A'}$.
  \end{enumerate}
\end{proof}


\begin{theorem}
  Given a tree automaton $\mathcal{A}^0$, a TRS $\R$ and an
  equation set $\mathcal{E}$, after $k$ completion steps we obtain
  $\A_\R^k$ such that $\A^0 \sqsubseteq \A_\R^k$.
\end{theorem}

\begin{proof}
  By induction on $k$:
  
  \begin{itemize}
  \item Since $\sqsubseteq$ is reflexive, we have trivially $\A^0
    \sqsubseteq \A^0$.
  \item Let $\mathcal{A}_k$ be a tree automaton obtained after $k$ completion
    steps such that $\A^0 \sqsubseteq \A_\R^k$. By definition of completion
    $\A_\R^{k+1}$ is built from $\A_\R^k$ by applying successively normalization
    and merge. Using Lemma~\ref{lem:renaming}, we have $\A_\R^k \sqsubseteq
    \A_\R^{k+1}$.  By transitivity of $\sqsubseteq$, from $\A^0 \sqsubseteq
    \A_\R^k$ and $\A_\R^k \sqsubseteq \A_\R^{k+1}$ we deduce immediately that
    $\A^0 \sqsubseteq \A_\R^{k+1}$.
  \end{itemize}
\end{proof}

\begin{theorem}{(Completeness)}
  Given two tree automata $\A$ and $\B$ if $\A \sqsubseteq \B$ then it
  exists $\prod$ a proof of statement $\emptyset \vdash_{\A, \B}
  \{\#(q_f)\ |\  q_f \in \mathcal{Q}_{F_A}\} \Subset  \{\#(q'_f)\ |\  q'_f \in \mathcal{Q}_{F_B}\}$.
\end{theorem}

\begin{proof}
  By definition of $\A \sqsubseteq \B$, we can deduce that there exists a
  renaming $\varrho$ s.t. $\varrho(\Delta_\A) \subseteq \varrho(\Delta_\B)$ and
  $\varrho(\Q_{F_\A}) \subseteq \varrho(\Q_{F_\B})$. The proof is done by
  induction on $\mu_1(\Gamma)$. Recall that $\mu_1(\Gamma)= |\Q_\A \times \Q_\B|
  - |\Gamma|$. Assuming that $\A \sqsubseteq \B$, we want to prove that for all
  $q$ there exists a proof tree $\prod$ for $\Gamma \vdash_{\A,\B} q \Subset
  \varrho(q)$. 
  We assume that $\forall q_i$, there exists a proof tree $\prod_i$
  of $\Gamma \cup \{ q \Subset \varrho(q) \} \vdash_{\A, \B} q_i \Subset
  \varrho(q_i)$. Now, we aim at proving that there exists a proof tree for
  the statement: $\Gamma \vdash_{\A, \B} q \Subset \varrho(q)$.

  \begin{itemize}
  \item if $q \Subset \varrho(q) \in \Gamma$ then the proof tree
    is trivial:
  \begin{prooftree}
    \AxiomC{}
    \LeftLabel{(Axiom)}
    \UnaryInfC{$\Gamma  \vdash_{\A, \B} q \Subset \varrho(q)$}
  \end{prooftree}
   
\item if $q \Subset \varrho(q) \not \in \Gamma$ then we need to
  apply Induction rule to obtain the following tree:
  
    \newcommand{\env}{\Gamma \cup \{q \Subset \varrho(q)\} \vdash_{\A, \B}}
    {\tiny
      \begin{prooftree}
        \AxiomC{\small $\prod_{c_1}$}
        \UnaryInfC{$\env 
          c_1 \Subset  \{c_k'| c_k' \rightarrow \varrho(q)\}_1^m$}
        % Pointillets du milieu
        \AxiomC{\small \dots\dots}
        \AxiomC{\small $\prod_{c_n}$}
        \UnaryInfC{$\env
          c_n \Subset  \{c_k'| c_k' \rightarrow \varrho(q)\}_1^m$}
        \LeftLabel{(Split-l)}
        \TrinaryInfC{$\env
          \{c_i| c_i \rightarrow q\}_1^n \Subset  \{c_k'| c_k'\rightarrow \varrho(q)\}_1^m$}
        \LeftLabel{(Induction)}
        \UnaryInfC{$\Gamma \vdash_{\A, \B} q \Subset \varrho(q)$}
      \end{prooftree}
    } From hypothesis $\varrho(\Delta_\A) \subseteq \Delta_\B$ for each rule $c
    \rightarrow q$ of $\Delta_A$, we have $\varrho(c\rightarrow q) \in
    \Delta_\B$. Thus for all $(c\rightarrow q)\in \Delta_\A$, we have
    $\varrho(c) \in \{c_k' | c_k' \rightarrow \varrho(q)\}_1^m$.  Then for each
    $c_i$, the proof tree $\prod_{c_i}$ is built in a similar way. Let us detail
    it for a particular $c_i= f(q_{i_1}, \dots, q_{i_n})$. We can construct the
    corresponding tree $\prod_{c_i}$ whose proof tree is concluded by
    $\prod_{i_j}$ an instance of induction hypothesis for the corresponding
    state $q_{i_j}$: {\tiny
      \begin{prooftree}
        \AxiomC{\small $\prod_{i_1}$}
        \UnaryInfC{$\env q_{i_1} \Subset \varrho(q_{i_1}) $}
        \AxiomC{\small \dots\dots}
        \AxiomC{\small $\prod_{i_n}$}
        \UnaryInfC{$\env q_{i_n} \Subset \varrho(q_{i_n}) $}
        %%%%%%%%%%%%%%%%%%%%%%% 
        \LeftLabel{(Config)}
        \TrinaryInfC{$\env c_i \Subset  \varrho(c_i) $}
        \LeftLabel{(Weak-r)}
        \UnaryInfC{$\env c_i \Subset  \{c_k'| c_k' \rightarrow \varrho(q)\}_1^m$}
      \end{prooftree}
    }
  \end{itemize}

  Now, we have shown that for all $\Gamma$ and $q \in \Q_\A$ there exists a proof
  tree $\prod$ for all statement $\Gamma \vdash_{\A, \B} q \Subset \varrho(q)$.
  In particular, this is true for $\Gamma = \emptyset$ all $q$ of
  $\Q_{F_\A}$.  Since we have $\A \sqsubseteq \B \imp
  \varrho(\Q_{F_\A}) \subseteq \Q_{F_\B}$, we can build a proof tree
  as:

  {\small
    \begin{prooftree}
      \AxiomC{$\prod_{q_{f_1}}$}
      \UnaryInfC{$\emptyset \vdash_{\A, \B} q_{f_1} \Subset \varrho(q_{f_1})$}
      % \AxiomC{$\emptyset \vdash_{\A, \B} [\ ] \Subset [\ ]$}
      \LeftLabel{(Config)}
      \UnaryInfC{$\emptyset \vdash_{\A, \B} \#(q_{f_1}) \Subset \#(\varrho(q_{f_1}))$}
      \LeftLabel{(Weak-r)}
      \UnaryInfC{$\emptyset \vdash_{\A, \B} q_{f_1} \Subset \Q_{F_\B}$}
      \AxiomC{\small \dots\dots}
      \AxiomC{$\prod_{q_{f_n}}$}
      \UnaryInfC{$\emptyset \vdash_{\A, \B} q_{f_1} \Subset \varrho(q_{f_1})$}
      % \AxiomC{$\emptyset \vdash_{\A, \B} [\ ] \Subset [\ ]$}
      \RightLabel{(Config)}
      \UnaryInfC{$\emptyset \vdash_{\A, \B} \#(q_{f_n}) \Subset \#(\varrho(q_{f_n}))$}
      \RightLabel{(Weak-r)}
      \UnaryInfC{$\emptyset \vdash_{\A, \B} \#(q_{f_n}) \Subset \{\#(q)\ |\ \Q_{F_\B}\}$}
      \LeftLabel{(Split-l)}
      \TrinaryInfC{$\emptyset \vdash_{\A, \B} \{\#(q)\ |\ q \in \Q_{F_\A}\} \Subset \{\#(q)\ |\ q \in \Q_{F_\B}\}$}
    \end{prooftree}}
\end{proof}



%%% Local Variables: 
%%% mode: latex
%%% TeX-master: "main"
%%% End: 
}
\end{document}
