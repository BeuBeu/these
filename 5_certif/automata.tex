\section{Formalisation des automates d'arbres}
\label{sec:automata}

Le fait que le $\kw{checker}$, qui doit être exécuté, est directement extrait de la formalisation \coq\
contraint la formalisation des automates d'arbres. Comme les structures de données
utilisées dans la formalisation sont celles qui sont réellement utilisées
lors de l'exécution, elles doivent être {\em efficaces} d'un point de vue algorithmique.
Pour les automates d'arbres, au lieu d'une représentation naïve, il est nécessaire d'utiliser 
une formalisation de la structure de données proposée dans~\cite{RivalGL-TPHOL01} pour manipuler de 
façon optimale les automates d'arbres.

\switchlstcoq


Dans la section~\ref{sec:rewriting}, on a représenté les variables $\X$ et les symboles de 
de fonctions $\F$ par le type \lstinline!ident!. On fait la même chose pour définir les états $\Q$.
On définit un automate comme un couple $(\Q_F, \Delta)$, où $\Q_F$ est l'ensemble fini des 
états finals, et $\Delta$ l'ensemble fini des transitions normalisées comme $f(q_1, \dots, q_n) \rightarrow q$.
En \coq, $\Q_F$ est une simple \lstinline!list ident! alors que $\Delta$ est représentée en utilisant les \lstinline!FMapPositive!
de la librairie \coq. Il s'agit d'une implémentation des tables d'associations fonctionnelles, où les données sont indexées par \lstinline!positive!.
Les \lstinline!positive! sont en fait une représentation binaire des entiers strictement positifs.
Dans la structure de \lstinline!FMapPositive!, chaque transition $f(q_1, \dots, q_n) \rightarrow q$ est encodée par une liste d'états $(q_1, \dots, q_n)$ indexée par $f$
dans une première table qui est ensuite indexée par l'état $q$ dans une seconde table. Cette représentation est une bonne 
solution pour manipuler efficacement les ensembles de transtions en \coq.
%
%

\begin{lstlisting}
Module Delta : DELTA.
   (* Transition sets : *)
   Definition config := list state.
   
   Definition t := 
        FMap.t (FMap.t (list config)).
   (* ............... *)
\end{lstlisting}

%
%
Ensuite, on peut définir un prédicat pour caractériser le langage reconnu par un automate d'arbres.
En fait, il s'agit de définir l'ensemble des termes clos qui sont réduits (réécrits) dans un état 
$q$ par les transitions de $\Delta$. Cet ensemble, qui correspond à $\Lang(\Delta,q)$ si $\Delta$ 
est l'ensemble des transitions de $\A$, peut-être construit inductivement en \coq\ en utilisant l'unique règle de déduction:

\begin{prooftree}
  \AxiomC{$t_1 \in \Lang(\Delta, q_1)$}
  \AxiomC{\dots\dots}
  \AxiomC{$t_n \in \Lang(\Delta, q_n)$}
  \RightLabel{Si $f(q_1, \dots, q_n) \rightarrow q \in \Delta$}
  \TrinaryInfC{$f(t_1,\dots, t_n) \in \Lang(\Delta, q)$}  
\end{prooftree}

En \coq, on exprime cette proposition en utilisant le prédicate inductif
\lstinline!IsRec!. Un terme $t$ est reconnu par un automate d'arbre $(\Q_F, \Delta)$, si
le prédicat \texttt{IsRec}~$\Delta$~$q$~$t$ est valide pour $q \in \Q_F$.

\begin{lstlisting}
Inductive IsRec (D: Delta.t) : state -> term -> Prop :=
  Rec_Term : forall f lt q,
      IsRec' D (Delta.get q f D) lt -> IsRec D q (Fun f lt)

with IsRec' (D: Delta.t) : list config -> list term -> Prop :=
| Rec_SubTerm : forall lt c lc, IsRec'' D c lt -> IsRec' D (c::lc) lt
| Rec_SubTerm' : forall lt c lc, IsRec' D lc lt -> IsRec' D (c::lc) lt
     
with IsRec'' (D: Delta.t) : config -> list term -> Prop :=
| Rec_Nil : IsRec'' D nil nil
| Rec_Cons : forall t q lt lq, IsRec D q t -> IsRec'' D lq lt ->
       IsRec'' D (q::lq) (t::lt).
\end{lstlisting}

Il est encore légitime de se demander quel crédit accorder à une telle spécification
qui n'a plus beaucoup de point commun avec la théorie des automates d'arbres classiques.
C'est pour cette raison que l'on décide de montrer l'équivalence entre ce formalisme 
et la définition classique telle que présentée dans le chapitre~\ref{chap:preliminaires}.
On définit alors un automate comme un couple composé de l'ensemble des états finaux et 
et l'ensemble des transitions normalisées. Les ensembles sont représentés par des listes.
\begin{lstlisting}
Inductive transition : Set :=
| Ground (l r: term):
      forall f lt q (Hl: l = Fun f lt) (Hr: r = State q)
                    (Hnorm: forall t, t \in lt -> exists q', t = State q') : Transition.

(* Conversion  Delta.t vers un ensemble de transitions *)
Definition trans_of: Delta.t -> list transitions.
    (* ... Body ... *)

(* L'execution $t \rw_\A t'$ *)
Definition Run (t: transition)(u v: term) := 
   forall l r f lt q Hl Hr Hnorm, Ground l r f lt q Hl Hr Hnorm = t ->
      exists p: position,
         get u p = Some l /\ v = set u p r.

(* Cloture de Run $t \rw^*_\A t'$ *)
Inductive StdRun (l: list transition): term -> term -> Prop :=
| RunRefl: forall u, StdRun l u
| RunStep: forall u v w t, StdRun l u v -> t \in l -> Run t v w -> StdRun l u w.


(* Equivalence: *)
Theorem StdRun_IsRec:
   forall $\Delta$ q t,
      IsRec $\Delta$ q t <-> StdRun (trans_of $\Delta$) t (state q).
\end{lstlisting}
Bien que la formalisation soit plus simple que dans la librairie \coq\ d'automates déterministes 
descendants~\cite{RivalGL-TPHOL01}, la preuve qu'elle soit équivalente à la formalisation théorique
est un argument irréfutable pour se convaincre de la correction de l'approche.
Enfin, il existe aussi une librairie récente d'automates d'arbres pour Isabelle\cite{TA-isabelle2010}
basée sur une implémentation similaire à celle-ci. La principale différence est l'utilisation des 
arbres "rouges et noirs" pour construire les tables d'associations plutôt que des arbres simples.
Cela permet de maintenir des structures avec une bonne complexité pour tout accès notamment pour
contruire les unions et intersections d'automates. Pour le $\kw{checker}$, le seul algorithme nécessaire
est la décision du vide de l'intersection entre deux automates, pour vérifier les propriétés. Or il n'est pas
nécessaire de construire explicitement l'intersection, on peut simplement rechercher si il existe une exécution commune 
entre les deux automates d'arbres considérés.


%%% Local Variables: 
%%% mode: latex
%%% TeX-master: "../main"
%%% End: 
