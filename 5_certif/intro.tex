\section{Introduction}
\label{sec:introduction}


\versionlongue{
In the field of infinite system verification, three of the most successful
techniques are: assisted proof, abstract interpretation and symbolic/abstract
model-checking. In all those techniques, the verification relies on specific
softwares that may, themselves, contain bugs. On the one hand, a proof assistant
like \coq~\cite{coqart} avoids this problem since any proof, on any exotic domain, is
finally checked using a small unique certified kernel.  On the other hand, proof
assistants like \coq\ offer poor automation when compared with fully
automatic tools like static analyzers or model-checkers. However, the efforts
achieved on the two last ones, to obtain a better automation and efficiency, have a
great impact on their reliability. Static analyzers and model-checkers are
usually huge, drastically optimized programs whose safety is not proved but
essentially ``demonstrated'' by an extensive use.
}

Static program analysis is one of the cornerstones of software
verification and is increasingly used to protect computing devices
from malicious or mal-functioning code. However, program verifiers are
themselves complex programs and a single error may jeopardize the
entire trust chain of which they form part. 
Efforts have been made to certify static
analyzers~\cite{KleinN-TCS03,BartheD-FASE04,CacheraJPR-TCS05} or to certify the results obtained by static
analyzers~\cite{LetouzeyT-TPHOL00,BessonJP-TCS06} in \coq\ in order to increase confidence
in the analyzers. 
%The latter is, in fact, more related
%to proof carrying code~\cite{Necula-POPL97}, where a complex untrusted
%verification software produce a certificate that is checked, afterwards, by a
%small trusted checker.
In this paper, we instantiate the general framework used
in~\cite{BessonJP-TCS06} to the particular case of analyzing term
rewriting systems by  tree automata
completion~\cite{Genet-RTA98,FeuilladeGVTT-JAR04}. Given a term rewriting
system, the tree automata completion is a technique for over-approximating the
set of terms reachable by rewriting in order to prove the
unreachability of certain ``bad'' states that violates a given
security property. This technique has already been used to
prove security properties on cryptographic protocols~\cite{GenetK-CADE00},
\cite{GenetTTVTT-wits03,BoichutHKO-AVIS04,avispa,ZuninoD-FOSSACS06} and, more
recently, to prototype static analyzers on Java byte
code~\cite{BoichutGJL-RTA07}.

In this paper, we show how to mechanize the proof, within the proof
assistant \coq, that the tree automaton produced by completion
recognizes an over-approximation of all reachable terms. 
 \coq\ is based on constructive logic (Calculus of
Inductive Constructions) and it is possible to 
%Thanks to Curry-Howard isomorphism, proofs and
%functionnal programs are very closed: if we have the proof that there exists an
%algorithm solving a problem, \coq\ is abble to 
extract an Ocaml or Haskell
function implementing exactly the algorithm whose specification has
been 
expressed in \coq. The extracted code is thus a {\em certified}
implementation of the specification given in the \coq\ formalism.  Extracted
programs are standalone and do not require the \coq\ environment to be
executed. For details about the extraction 
%and reflection 
mechanisms, readers can refer to~\cite{coqart}.

%Our objective could be to specify formally the full completion and extract a
%certified algorithm from the \coq\ specification. However, the more complex the
%algorithm, the harder is the proof. 
A specific challenge in the work reported here has been how to marry
constructive logic and efficiency. Previous case studies with tree
automata completion, on cryptographic
protocols~\cite{GenetTTVTT-wits03} and on Java
bytecode~\cite{BoichutGJL-RTA07} show that we need an efficient
completion algorithm to verify properties of real models. 
% the implementation of completion has to be drastically optimized. 
For instance, the current implementation of completion (called
\timbuk~\cite{timbuk-site}) is based on imperative data structures like hash
tables whereas \coq\ allows only pure functional structures.  A second problem
is the termination of completion. Since \coq\ can only deal with total
functions, functions must be proved terminating for any
computation. In general, such a property cannot be guaranteed on completion
because it mainly depends on term rewriting system and approximation equations
given initially.

For these two reasons, there is little hope to specify and certify an
efficient and purely functional version of the completion
algorithm. Instead, we
have adopted a solution based on a result-checking approach.  
% the "Proof Carrying Code" approach\cite{Necula-POPL97}. 
It consists of building a smaller program (called the
\emph{checker}) - certified in \coq\ -  that checks if the tree automaton computed by \timbuk\ is
sound. In this paper, we restrict to the case of left-linear term rewriting
systems which revealed to be sufficient for verifying Java
programs~\cite{BoichutGJL-RTA07}. However, a checker dealing with general term
rewriting systems like completion does in~\cite{FeuilladeGVTT-JAR04} is under
development.

%
%
\archive{Note that this objective is similar
to the one of Roberto Zunino in~\cite{Zunino}. However, our we aim at tackling
this goal in a different way. Whereas~\cite{Zunino} produce the text of a \coq\
proof certifying a specific result, we extract a certified efficient checker (in
Ocaml~\cite{Ocaml}) from the \coq\ specification using the \coq\ extraction
mechanism.
%
%
Our aim is to extract a certified efficient checker (in Ocaml~\cite{Ocaml}) from
the specification using the \coq\ extraction mechanism~\cite{coqart}.  As a
result, a lot of efforts have to be payed to the definition of formal {\em and}
efficient data structures: terms, tree automata, etc. This is crucial
for verification of the Java byte code for instance because the tree
automata to certify are huge. However, those efforts are worth since, once
extracted, the certified Ocaml checker is fully independent of the \coq\
environment: easier to run, less greedy with memory and faster.
}
%
%
\archive{This has several consequences.  First, a lot of effort has to be
  payed to the definition of formal {\em and} efficient data structures: terms,
  tree automata, etc. This is crucial because, for verification purposes like
  the Java byte code for instance, the tree automata to certify are
  huge. Second, since the obtained checker is an Ocaml program, it is no longer
  executed in \coq., which is a big difference with~\cite{zunino}. This, again,
  improves the efficiency but not only. In~\cite{zunino}, given a tree
  automaton, the tool generates a \coq proof text that has to be verified
  afterward in \coq.  The problem here is that there is no success guarantee:
  the proof text may be rejected by \coq\ (without indicating if it is because the
  property is false or because the proof text is ill-formed), or \coq\ may even
  not terminate on this proof. In our case, the tree
automaton is checked by an Ocaml program that {\em always terminates}, decides
if the tree automaton is a valid fixpoint of tree automata completion and
certify unreachability.  A great advantage of this approach using an external
%
%
The extracted checker, we define here, is able to {\em decide} if the given
tree automaton is a valid fixpoint of tree automata completion, and also certify
unreachability.  A great advantage of using an external checker is that it is
totally independent of the completion tool we use. In particular, the
implementation of the completion tool can be optimized as necessary: as long as
it outputs a tree automaton, this result can be certified by our checker.

}

The closest work to ours is the one done by X.~Rival and
J.~Goubault-Larrecq~\cite{RivalGL-TPHOL01}. They have designed a library to
manipulate tree automata in~\coq\ and proposed some optimized formal data
structures that we reuse. However, we aim at dealing with larger tree automata
than those used in their benchmarks. Moreover, we need some other tools which
are not provided by the library as for example a specific algorithm to check
inclusion. 
\versionlongue{Inclusion checking may be done using closure operators
(i.e. intersection and complementation) and emptiness checking but, as shown in
Section~\ref{sec:inclusion}, this way of performing inclusion checking has a bad
complexity.}

This paper is organized as follows. Rewriting and tree automata are reviewed in
Section~\ref{sec:preliminaries} and tree automata completion in
Section~\ref{section:completion}. Section~\ref{section:objectives} states the main
functions to define, inclusion and closure test, and the corresponding theorems
to prove. Section~\ref{sec:rewriting} and Section~\ref{sec:automata} give the
\coq\ formalization of rewriting and of tree automata,
respectively. 
The core of the checker consists of two algorithms: an optimized automata inclusion
test, defined in Section~\ref{sec:inclusion}, and a procedure for
checking that an automaton is \emph{closed} under rewriting w.r.t.~a
given term rewriting system, defined in  Section~\ref{sec:closure}. 
An important feature of the inclusion checker is that, while it is
not complete for all tree automata, we can  prove that it is complete
for the class of tree automata generated by the completion algorithm. 
Section~\ref{sec:benchmarks} gives some details about the performances of the
checker in practice. Finally, we conclude and list some on-going research on this
subject. 

%%% Local Variables: 
%%% mode: latex
%%% TeX-master: "main"
%%% End: 