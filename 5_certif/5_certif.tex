
\chapter{Certification de la complétion d'automates d'arbres}
\label{chap:certif}


\switchlstcoq 

\section{Un validateur de résultat pour la complétion d'automates d'arbres}
\label{sec:objectives}
%This part has to describe precisely the contribution of the paper.


La certification d'un algorithme dans un assistant de preuve, est une tâche difficile dans la mesure
où la preuve doit généralememt être entièrement construite à la main. Cela peut devenir d'autant plus ardu que l'algorithme implémente
des optimisations complexes basées sur des structures de données
avancées ou des astuces algorithmiques. C'est souvent une nécessité  si l'on cherche à certifier une implémentation réaliste,
c'est à dire dont les performances supportent un passage à l'échelle. 
Dans le cas de la complétion d'automates d'arbres, les difficultés rencontrées lors du passage à l'échelle ont
conduit les différents développements de l'outil \timbuk\ vers une implémentation qui est actuellement bien éloignée du formalisme initial.
Bien qu'impromptue, des bugs ont été et peuvent être découverts mais la détection est d'autant plus difficile que les automates
obtenus sont de taille importante. Même si chaque nouveau bug supprimé est l'occasion d'enrichir le jeu de tests, on ne peut que
s'interroger quant à l'empirisme d'une telle approche.
Peut-on avoir {\em confiance} en un outil tel que \timbuk? La question est d'autant plus cruciale, lorsque cet outil sert
la validation d'un système, comme dans les chapitres précédents.
L'objet de la certification est donc d'augmenter le degré de confiance que l'on peut accorder à \timbuk.
On souhaiterait donc prouver que l'implémentation de \timbuk\ calcule bien un post point-fixe, soit 
le théorème de correction suivant la propriété:

\[\forall \A\ \A'\ \R,\quad \A' = \kw{completion}(\A,\ \R) \imp \Lang(\A') \supseteq \R^*(\Lang(\A))\]

Si la question est légitime, il existe plusieurs manières de répondre à cette question. Allant des différentes
techniques d'analyse statique jusqu'aux assistants de preuve, la preuve d'un programme de plus de 11000 lignes
de code en \ocaml, reste un challenge important. D'autre part, une telle preuve de correction est liée 
à une implémentation particulière de l'algorithme. Ce qui signifie que tout apport d'une nouvelle optimisation
rend caduque la partie de la preuve concernée par la modification, remettant en cause la correction de toute
l'implémentation.

Une solution consiste à contourner le problème en transférant le problème de la certification en un problème
de certification du résultat. Plutôt que de montrer que toute exécution de la fonction $\kw{completion}$ donne un résultat
correct, on propose de fournir un validateur, appelé $\kw{checker}$ dans le reste du chapitre, qui vérifie si le résultat correspond
à la spécification attendue. L'avantage réside dans le fait que le $\kw{checker}$ est un programme plus simple, qui ne nécessite pas d'être modifié
tant que la spécification de l'outil ne change pas! On veut alors montrer la propriété suivante:
\[\forall \A\ \A'\ \R,\quad \kw{checker}(\A,\ \R,\ \A') = \kw{true} \imp \Lang(\A') \supseteq \R^*(\Lang(\A))\]

On obtient un certificat du résultat dès lors que l'étape de la validation se conclut par un succès.
Cependant comme l'étape de validation est requise après chaque exécution, il est nécessaire que le coût 
de la vérification puisse être négligeable ou au moins raisonnable par rapport au temps de calcul.
En effet, si le coût de la vérification est plus important que le gain apporté par les optimisations 
de l'implémentation, il est évident que les optimisations deviennent inefficaces, dans le cas
où l'on veut un certificat.


En transposant le problème de la certification de l'algorithme de la complétion vers 
l'assistant de preuve \coq, le problème de certification revient alors à montrer le théorème suivant:

\begin{lstlisting}
Theorem sound_checker :
      forall A A' R, checker A R A' = true -> AReachable A R A'.
\end{lstlisting}
où \lstinline!AReachable! est un prédicat \coq\  qui décrit
la propriété de correction: \emph{$\Lang(A')$ contient tous les termes atteignables
  par réécriture des termes de $\Lang(A)$ avec $\R$}, \textit{i.e.} $\Lang(\A')
\supseteq \desc(\Lang(\A))$. 
Pour établir formellement ce prédicat en \coq, on a besoin de donner une formalisation
des systèmes de réécriture et des automates d'arbres en \coq\ (\textit{cf.} Section~\ref{sec:rewriting}).
Pour deux automates $\A$, $\A'$ donnés et un système de réécriture $\R$, on veut vérifier 
que $\Lang(\A')\supseteq \R^*(\Lang(\A))$ soit (\lstinline!AReachable A R A'!) en \coq.
Pour réaliser cela, on a besoin de vérifier les deux propriétés suivantes:

\begin{itemize}
\item \lstinline!Included!: que tous les termes de l'ensemble initial sont présents  dans le point fixe: $\Lang(\A) \subseteq \Lang(\A')$.

\item \lstinline!IsClosed!: $\A'$ est clos par réécriture avec $\R$: pour toute règle $l \rightarrow
  r \in \R$ et tout terme $t \in \Lang(\A')$, si $t$ peut se réécrire en un terme $t'$ par la règle
  $l \rightarrow r$ alors on a $t' \in \Lang(\A')$. 
% Trop detaillé pour la partie objectives...
% To prove this property, we need
%   verify that for each substitution $\sigma:\X \mapsto \Q$ and state $q$ of
%   $\A'$, if $l\sigma \rw_{\A'}^* q$ then we have $r\sigma \rw_{\A'}^* q$,
%     i.e. prove that the critical pair $(l\sigma \rightarrow q,\ l\sigma
%     \rightarrow r\sigma)$ is joinable.
\end{itemize}
Pour chacun des items, on procède de la même façon, en fournissant une fonction \coq\ avec son théorème \coq.
La fonction \texttt{inclusion} est dédiée à la vérification de l'inclusion et la fonction \texttt{closure}
vérifie si un automate d'arbres est clos par réécriture. 
\begin{lstlisting}
Theorem inclusion_sound:
      forall A A', inclusion A A' = true -> Included A A'.

Theorem closure_sound:
      forall R A', closure R A' = true -> IsClosed R A'.
\end{lstlisting}

Le théorème qui permet de déduire \lstinline!AReachable A R A'! à partir de \lstinline!Included A A'! et \lstinline!IsClosed R A'!:
\begin{lstlisting}
Theorem Included_IsClosed_Reachable:
      forall A A' R, Included A A' -> IsClosed R A' -> AReachable A R A'.
\end{lstlisting}


On se concentre sur la preuve de  $\Lang(\A')\supseteq \R^*(\Lang(\A))$. 
Cependant, pour prouver la propriété de non-atteignabilité, la vacuité de l'intersection
entre $\Lang(\A')$ et l'ensemble des termes interdits doit aussi être vérifiée.
La formalisation en \coq\ de l'intersection et la décision du vide sont proches
de leur definition standard~\cite{TATA}, et leur implémentation
\coq\ a déjà été traitée dans~\cite{RivalGL-TPHOL01}.
D'autre part, cette version du $\kw{checker}$ est destinée à la version de la complétion où 
la résolution des paires critiques ne faisaient pas intervenir les $\varepsilon$-transitions.
On considère donc que le résultat $\kw{completion}(\A,\ \R)$ est donc un automate d'arbres 
sans $\varepsilon$-transitions. 
%Cependant 
% il n'est donc pas nécessaire de les they are not be detailed in this
% paper.

\section{Formalization of Term Rewriting Systems}
\label{sec:rewriting}

The aim of this part is to formalize in \coq: terms, term rewriting systems,
reachable terms and the reachability problem itself.  Firstly we use the
positive integers provided by the \coq's standard library to define symbol sets
like variables ($\X$) or function symbols ($\F$). We rename \lstinline!positive! into
\lstinline!ident! to be more explicit. Then, we define term set $\TFX$ using inductive
types:

\switchlstcoq
%A discuter....
%Definition F := positive.
%Definition X := positive.
\begin{lstlisting}
Definition ident := positive.

Inductive term : Set :=
| Fun : ident -> list term -> term
| Var : ident -> term.
\end{lstlisting}

\versionlongue{
Now, the term $f(x, a)$ will be written \lstinline!Fun 0 (Var 0::(Fun 1 nil)::nil)!  assuming that we have the corresponding mapping between between
symbols, variables and positive integers $f \mapsto 0$, $a \mapsto 1$ and $x
\mapsto 0$ for example.  Note that it is possible to attach the value $0$
to $f$ and $x$, since the \lstinline!term!'s constructors \lstinline!Fun! and
\lstinline!Var! allows to differentiate between variable and function symbols.
}

\versionlongue{
\paragraph{Remarks:}
\begin{itemize}
\item
  Since equality is decidable, we can easily define term equality as a
  decidable relation. Afterward, this is very useful to define functions
  where term comparison is required.
\item
  A bad point for \coq\ is the induction principle
  \lstinline!term_rect! automatically generated which is too weak.
  To prove properties in the following, we need a more efficient theorem
  named \lstinline!term_rect'!. This is required to be able to prove
  most of the theorems on terms.
\end{itemize}
}

A rewrite rule $l \rw r$ is represented by a pair of terms with a
well-definition proof, i.e. a \coq\ proof that the set of variables of $r$ is a
subset of the set of variables of $l$. The function
\lstinline!Fv : term -> list ident! builds the set of variables for a term. 
% In \coq, it becomes:
\begin{lstlisting}
Inductive rule : Set :=
| Rule (l r : term)(H : subseteq (Fv r) (Fv l)) : rule.
\end{lstlisting}

In the following, \lstinline!list rule! type represents a
TRS. In \coq\ we use \lstinline!(t @ sigma)!  to denote the term resulting of
the application of a substitution \lstinline!sigma! to each variable that occurs
in a term \lstinline!t!. 

\begin{lstlisting}
Definition substitution := ident -> option term.
\end{lstlisting}

In \coq, the rewriting relation \emph{"$u$ is rewritten in $v$ by $l
  \rightarrow r$"},
commonly defined by $\exists \sigma \
s.t.\ u|_p = l\sigma \ \land \ v = u [ r\sigma ]_p$, is split into
two predicates:
\begin{itemize}
\item The first one defines the rewriting of a term at the topmost position. In
  fact, the set of term pairs $(t, t')$ which are rewritten at the top most by
  the rule can be seen as the set of term pairs $(l\sigma, r\sigma)$ for any
  substitution $\sigma$.

\item 
  The second one just defines inductively the rewriting relation at any
  position of a term $t$ by a rule $l \rightarrow r$, by the topmost
  rewriting of any subterm of $t$ by $l \rightarrow r$.
\end{itemize}

\begin{lstlisting}
(* Topmost rewriting : *)
Inductive TRew (x : rule) : term -> term -> Prop :=
| R_Rew :
  forall s l r (H : subseteq (Fv r) (Fv l)),
    x = Rule l r H -> TRew x (l @ s) (r @ s).
\end{lstlisting}


\begin{lstlisting}
(* Rewrite at any position of term *)
Inductive Rew (r : rule) : term -> term -> Prop :=
| Rew1 : forall t t', 
    TRew r t t' -> Rew r t t'
| Rew2 : forall f l l',
    Rargs r l l' -> Rew r (Fun f l) (Fun f l')
\end{lstlisting}

\begin{lstlisting}
with Rargs (r : rule):list term->list term->Prop:=
| Ra1 : forall t t' l,
    Rew r t t' -> Rargs r (t::l) (t'::l)
| Ra2 : forall t l l',
    Rargs r l l' -> Rargs r (t::l) (t::l').
\end{lstlisting}


\versioncourte{Similarly, using an inductive definition it is possible to
  define the \lstinline!Rew! predicate for rewriting at any position.  } Then we
have to define $\rw^*_{\R}$. In \coq, we prefer to see it as the predicate
\lstinline{Reachable R u} that characterizes the set of reachable terms from
$u$ by $\rightarrow^*_\R$.

\begin{lstlisting}
Inductive Reachable(R : list rule)(t : term) : term -> Prop:=
| R_refl : Reachable R t t
| R_trans : forall u v r, Reachable R t u -> In r R -> Rew r u v -> 
    Reachable R t v.
\end{lstlisting}

%%% Local Variables: 
%%% mode: latex
%%% TeX-master: "main"
%%% End: 

\input{5_certif/automata}


\section{L'inclusion efficace d'automates}
%\section{An optimized inclusion checker}
\label{sec:inclusion}

Dans cette section, on donne la définition formelle de la propriété \lstinline!Included! ainsi
que la fonction \lstinline!inclusion!  utilisée pour vérifier efficacement l'inclusion d'automates d'arbres.
En se basant sur les définitions précédentes sur les automates d'arbres, on peut établir
le prédicat \lstinline!Included! de la manière suivante:

\begin{lstlisting}
  Definition Included (a b : t_aut) : Prop :=
    forall t q, q \in a.qf -> IsRec a.delta q t ->
      exists q', q' \in b.qf /\ IsRec b.delta q' t.
\end{lstlisting}


Ensuite on se concentre sur la fonction \lstinline!inclusion! elle-même. 
L'algorithme usuel pour montrer l'inclusion de langages réguliers reconnus
par des automates d'arbres non-déterministes et ascendants, par exemple
pour montrer $\Lang(\A) \subseteq \Lang(\B)$, consiste à montrer que $\Lang(\A) \cap
\Lang(\overline{\B}) =\emptyset$, où $\overline{\B}$ est l'automate qui reconnaît le complément 
le langage de l'automate $\B$. Cependant, l'algorithme pour construire $\overline{\B}$ à partir
$\B$ est EXPTIME-complete~\cite{TATA}. Cela est d'autant peu raisonnable que l'automate $\B$ est
le point fixe calculé par la complétion, il s'agit d'un automate particulièrement gros.
Pour cette raison, on propose de contourner le problème en définissant un critère dont
la complexité est meilleure en pratique. Il est basé sur une simple comparaison syntaxique
des ensembles de transitions, \textit{i.e.} on vérifie l'inclusion des ensembles de transitions
modulo le renommage qui a pu être réalisé par la fonction $\widen$ qui fusionne des états.
Cette technique améliore considérablement l'efficacité du $\kw{checker}$, spécialement la consommation
mémoire. Or il est crucial de pouvoir vérifier l'inclusion de gros automates d'arbres (\textit{cf.}
la section~\ref{sec:benchmarks}). Cet algorithme est correct mais, bien sûr, il n'est pas 
complet en général. Il n'est donc pas toujours capable de montrer $\Lang(\A) \subseteq \Lang(\B)$.
Cependant, on montre par la suite, sous certaines hypothèses sur $\A$ et $\B$, qui sont satisfaites
si $\B$ est obtenu par la complétion de l'automate $\A$, que cet algorithme est complet et redevient une procédure de
décision. En premier lieu, on introduit la notation suivante:
{\small
\begin{center}
  \begin{tabular}[c]{lcp{10cm}}
    $\Gamma$    & : & l'ensemble d'hypothèses d'induction\\
    $\Delta_i$  & : & l'ensemble de transitions de l'automate d'arbres $\mathcal{A}_i$\\
    $\{c|c \rightarrow q \in \Delta\}$ & : & l'ensemble des membres gauches (configurations) des transitions de $\Delta$ qui produisent l'état $q$\\
    $\{c_i\}_n^m$ & : & l'ensemble des configurations $c_i$ pour $i$ compris entre $n$ et $m$\\
%    $\emptyset$ & : & empty set of configurations\\
%    $[q_1, \dots, q_n]$ & : & a tuple of $n$ states\\
  \end{tabular}
\end{center}
} Le problème d'inclusion se formule alors de la sorte: ${\Gamma
  \vdash_{A, B} q \Subset q'}$. Une telle formulation signifie que sous les hypothèses
de $\Gamma$,  il est possible de montrer que $\Lang(A,q) \subseteq \Lang(B,q')$.
% A term $t$ is reconnized by state $q$ a tree automaton if it exists
% one sequence of transitions of $\Delta$ that rewrites $t$ into $q$. 
% Thus, ${\Gamma \vdash_{A, B} q \Subset q'}$ states inductively for the assertion:
% if a term $t$ s.t. is rewritten into $q$ by a sequence of transitions of $\Delta_\A$ then
% it can be done by an equivalent sequence in $\Delta_\B$ which rewrites $t$ into $q'$.\\

L'algorithme tente de construire un arbre de preuves permettant de conclure sur le problème
formulé au moyen des règles de déduction suivantes.

% This algorithm checks only transition set inclusion (up to
% renaming). Transition set inclusion implies corresponding language
% inclusion (as proved in Section~\ref{sec:correction}). It is more
% reasonable to implement it in \coq.\\
%The algorithm checks only this property, but it is enough to imply 
%corresponding language inclusion. (as proved in Section~\ref{sec:correction}).
%It is more reasonable to implement it in \coq.\\

{\small
\[\textrm{(Induction) }
\dfrac{
  \Gamma \cup \{q \Subset q'\} \vdash_{A, B}
  \{c|c \rightarrow_{\Delta_A} q\} \Subset \{c|c \rightarrow_{\Delta_B} q'\}
}{
  \Gamma \vdash_{A, B} q \Subset q'
}\textrm{  si }(q \Subset q') \notin \Gamma
\]


\[\textrm{(Axiom) }
\dfrac{}{
  \Gamma \cup \{q \Subset q'\} \vdash_{A, B} q \Subset q'
}\quad \quad \quad \quad 
\textrm{(Empty) }
\dfrac{}{
  \Gamma \vdash_{A, B} \emptyset \Subset \{c'_j\}^m_1
}
\]



\[\textrm{(Split-l) }
\dfrac {
  \Gamma \vdash_{A, B} c_1 \Subset \{c'_j\}^m_1 \quad \dots\dots \quad
  \Gamma \vdash_{A, B}  c_n \Subset \{c'_j\}^m_1
}{
  \Gamma \vdash_{A, B} \{c_i\}^n_1 \Subset \{c'_j\}^m_1
}\]

%\[(empty)\]

\[\textrm{(Weak-r) }
\dfrac{
  \Gamma \vdash_{A, B} c \Subset c'_k
}{
  \Gamma \vdash_{A, B} c \Subset \{c'_i\}^n_1
}\textrm{ si }(1 \le k \le n)
\quad\quad\quad \quad
\textrm{(Const.) }
\dfrac{}{
  \Gamma \vdash_{A, B} a() \Subset a()
}
\]

\[\textrm{(Config) }
\dfrac{
  \Gamma \vdash_{A, B} q_1 \Subset q'_1 \quad \dots\dots \quad \Gamma \vdash_{A, B} q_n \Subset q'_n
}{
  \Gamma \vdash_{A, B} f(q_1, \dots, q_n) \Subset f(q'_1, \dots, q'_n)
}\]

%\[const\]
}
\noindent

Si on considère $\Q_{F_\A}$ et $\Q_{F_\B}$ comme les ensembles d'états finals de $\A$ et $\B$, 
et $\#()$ un symbole particulier d'arité $1$ qui n'est pas dans $\F$.
Pour prouver $\Lang(\A) \subseteq \Lang(\B)$, il suffit de démarrer le raisonnement à partir de la formulation
$\emptyset \vdash_{\A, \B} \{\#(q)\ |\ q \in \mathcal{Q}_{F_\A}\} \Subset \{\#(q)\ |\ q \in \mathcal{Q}_{F_\B}\} $.
 

\begin{example}
  Soient $\A$ et $\B$ deux automates tels que:
  {\small
  \[\A = \left\{ 
    \begin{array}{rcl}
      a& \rightarrow &q_1\\
      b& \rightarrow &q_2\\
      f(q_1, q_2)&\rightarrow &{\textbf q}\\
    \end{array}\right\}
  \textrm { with } \Q_{F_\A}=\{{\textbf q}\}
  \textrm { and }
  \B = \left\{ 
    \begin{array}{rcl}
      a& \rightarrow &{\textbf q'}\\
      b& \rightarrow &{\textbf q'}\\
      f(q', q')&\rightarrow &{\textbf q'}\\
    \end{array}\right\}
  \textrm { with } \Q_{F_\B}=\{{\textbf q'}\}
  \]
  }
  On voit facilement que $\Lang(\A) \subseteq \Lang(\B)$ et on peut construire un arbre de preuve pour conclure à 
  $\emptyset \vdash_{\A, \B} \#(q) \Subset \#(q')$ au moyen des règles de déduction:

  {\tiny
    \begin{prooftree}
      \AxiomC{}
      \LeftLabel{(Const.)}
      \UnaryInfC{$\{q \Subset q', q_1 \Subset q'\} \vdash_{\A, \B} a() \Subset a()$}
      \LeftLabel{(Weark-r)}
      \UnaryInfC{$ \{q \Subset q',\ q_1 \Subset q'\} \vdash_{\A, \B} a() \Subset \{ a(), b(), f(q', q') \}$}
      \LeftLabel{(Induction)}
      \UnaryInfC{$\{q \Subset q'\} \vdash_{\A, \B} q_1 \Subset q'$}
      %%%%%%%%%%%%%%%%%%%%%%%%%%%%% 
      \AxiomC{}
      \LeftLabel{(Const.)}
      \UnaryInfC{$\{q \Subset q', q_2 \Subset q'\} \vdash_{\A, \B} b() \Subset b()$}
      %\RightLabel{(Weark-r)}
      \UnaryInfC{$ \{q \Subset q',\ q_2 \Subset q'\} \vdash_{\A, \B} b() \Subset \{ a(), b(), f(q', q') \}$}
      %\RightLabel{(Induction)}
      \UnaryInfC{$\{q \Subset q'\} \vdash_{\A, \B} q_2 \Subset q'$}
      %%%%%%%%%%%%%%%%%%%%%%%%%%%%%%%%%%%%%%
      \LeftLabel{(Config)}
      \BinaryInfC{$\{q \Subset q'\} \vdash_{\A, \B} f(q_1, q_2) \Subset f(q', q') $ }
      \LeftLabel{(Weark-r)}
      \UnaryInfC{ $\{q \Subset q'\} \vdash_{\A, \B} f(q_1, q_2) \Subset \{ a(), b(), f(q', q')\} $}
      \LeftLabel{(Induction)}
      \UnaryInfC{$\emptyset \vdash_{\A, \B} q \Subset q'$}
      \LeftLabel{(Config)}
      \UnaryInfC{$\emptyset \vdash_{\A, \B} \#(q) \Subset \#(q')$}
    \end{prooftree}
  }
\end{example}

%We define the simple inclusion relation as :
%\begin{definition}
%  Let $\Delta$ and $\Delta'$ be two transition functions.\\
%  We say $\Delta \subseteq \Delta'$ if $\Delta'$ contains all rules of $\Delta$ :
%  \begin{equation}
%    \Delta \subseteq \Delta' \Longleftrightarrow \forall\ (c \rightarrow q) \in \Delta,\ (c \rightarrow q) \in \Delta'
%  \end{equation}
%\end{definition}
%Then we have trivially the property :
%\begin{theorem}
%  Given two transition function $\Delta$ and $\Delta'$,
%  \begin{equation}
%    \forall q,\ \Lang(\Delta, q) \subseteq \Lang(\Delta', q)
%  \end{equation}
%\end{theorem}

La principale propriété que l'on veut montrer en \coq\ est que le critère syntaxique
implique l'inclusion sémantique c'est à dire l'inclusion de langages tel que formulée
au début de la section~\ref{sec:inclusion}.

\begin{lstlisting}
Theorem inclusion_sound :
   forall A B, inclusion A B = true -> Included A B.
\end{lstlisting}

Avant de prouver ce résultat en \coq, on doit définir plus formellement la fonction
\lstinline!inclusion!. Cette fonction ne peut être définie comme une simple récursion structurelle.
Or en \coq, la théorie sous-jacente repose sur la propriété de normalisation forte qui impose à toute
fonction définie de terminer. Ainsi, cet algorithme dont la terminaison n'est pas immédiate
nécessite de fournir à \coq\ une preuve de terminaison.
Initialement définie avec l'extension \lstinline!Function! de \coq, à cause d'un certain nombre de limitations,
la fonction \lstinline!inclusion! est actuellement construite en utilisant le langage de tactiques. C'est un théorème
qui exprime la décidabilité de l'ensemble des règles de déduction pour tout but de la forme $\Gamma \vdash_{\A, \B} q \Subset q'$,
but dénoté en \coq\ par le prédicat \lstinline!Istate A.(Delta) B.(Delta) Gamma q q'!. Comme \coq\ repose sur une logique constructive,
la preuve contient l'algorithme qui permet la décision.

\begin{lstlisting}
(* Decidability of Predicate Istate *)
Lemma inclusion :
   forall A B G q q',
      {Istate A.(Delta) B.(Delta) G q q'} + {~ Istate A.(Delta) B.(Delta) G q q'}.

(* New soundness Theorem *)
Theorem inclusion_sound:
   forall A B G q q', Istate A.(Delta) B.(Delta) G q q' -> Included A B.
\end{lstlisting}

\subsection{Preuve de terminaison}
\label{sec:termination}

La construction de \lstinline!inclusion! repose sur la décroissance stricte
d'une mesure $\mu$ pour chaque application de règle.
La fonction de mesure $\mu$ est définie sur les formulations $\Gamma \vdash_{\A,\B} \alpha \Subset \beta$.
%We use the finiteness property of $\Gamma$ that is finite and the size of terms on the right of $\vdash$.
La relation $\Gamma$ est un sous-ensemble de $\mathcal{Q}_A \times \mathcal{Q}_B$ qui est un ensemble fini. Tous les automates ont un nombre fini
d'états. La mesure $\mu({\Gamma \vdash_{A, B} \alpha \unlhd \beta})$ se définit comme le couple $(\mu_1(\Gamma), \mu_2(\alpha)+\mu_2(\beta))$ où:
{\small
\[\left[
  \begin{array}{lcl}
    \mu_1(\Gamma) & = & | \Q_A \times \Q_B | - |\Gamma|\\
    \mu_2(x) &=&
    \left\{\begin{array}{l}
        (m+1-n) \textrm{ si } x = \{c_i\}^m_n\\
        1 \textrm{ si } x = f(q_1, \dots, q_n),\\
        0 \textrm{ sinon}
      \end{array}\right.\\
  \end{array}\right.
\]}

On peut alors définir la relation d'ordre $\ll$ comme la combinaison lexicographique de l'ordre usuel 
$<$ sur les entiers naturels pour $\mu_1$ et $\mu_2$. Ce qui donne:% and $\mu_3$:
\[(x, y) \ll (x', y') \Longleftrightarrow
\bigvee\left\{
\begin{array}{l}
  x < x' \\
  x = x' \land y < y'\\
  %x = x' \land y = y' \land z < z'
\end{array}\right.
\]
Puisque $<$ est une relation bien fondée, il est évident que $\ll$ est aussi une relation bien-fondée.

\begin{theorem}{(Terminaison)}
  A chaque étape de déduction, la mesure décroît strictement:
  \[
  \dfrac{
    \Gamma \vdash_{A, B} \alpha \Subset \beta
  }{
    \Gamma' \vdash_{A, B} \alpha' \Subset \beta'
  }
  \Longrightarrow \mu({\Gamma \vdash_{A, B} \alpha \Subset \beta}) \ll \mu({\Gamma' \vdash_{A, B} \alpha' \Subset \beta'})
  \]
\end{theorem}

\begin{proof}
  Le tableau suivant résume pour chaque règle de dérivation quelle composante du couple
  prouve que $\mu$ décroît entre la conclusion et les prémisses de la règle:
  
  \[\begin{array}[h]{l|c|c|}
    \footnotesize
    & \mu_1 & \mu_2 \\ \hline % & \mu_3\\ \hline
    \textrm{Induction}   & \strut \mu_1(\Gamma) < \mu_1(\Gamma') & - \\ \hline
    \textrm{Split-l} & \mu_1(\Gamma) = \mu_1(\Gamma') & 
    %\begin{array}{l}
      \mu_2(c_i) = 1 < \mu_2(\{c_i\}_1^n)\\
     % \mu_2((\{c_i\}_2^n) < \mu_2(\{c_i\}_1^n)\\
    %\end{array}\\ \hline

    \textrm{Weak-r} & \mu_1(\Gamma) = \mu_1(\Gamma') & \mu_2(c_k) < \mu_2(\{c_i\}_1^n)\\ \hline
    \textrm{Config} & \mu_1(\Gamma) = \mu_1(\Gamma') & 
    \begin{array}{l}
      \mu_2(f(\dots, q_i, \dots)) = \mu_2(f(\dots, q'_i, \dots)) = 1\\
      \mu_2(q_i) = \mu_2(q'_i) = 0 \textrm{ donc } 2 > 0
    \end{array}\\ \hline
    %\textrm{Tuple} & \mu_1(\Gamma) = \mu_1(\Gamma') & 0 = 0 &
    %\begin{array}{l}
    %  \mu_3([q_1, \dots, q_n]) > \mu_3(q1)\\
    %  \mu_3([q_1, \dots, q_n]) > \mu_3([q_2, \dots, q_n])\\
    %\end{array}\\ \hline
\end{array} \]
Pour la règle Split-l (resp. Weak-r), on considère $n > 1$ si l'on a l'ensemble $\alpha = \{c_i\}_1^n$ (resp. $\beta$) qui contient
au moins deux éléments différents. Sinon lorsque $n = 1$, cette règle ne s'applique pas sur ${\Gamma \vdash_{\A, \B} \alpha \Subset \beta}$.\\
\end{proof}


\begin{theorem}
  Lorsque $\mu(\Gamma \vdash_{A, B} \alpha \Subset \beta) = (0, 0)$, on sait que l'on peut directement appliquer la règle Axiom ou la règle Nil:
  la branche courante de la preuve est complète.
\end{theorem}

\begin{proof}
  A partir $\mu(\Gamma \vdash_{A, B} \alpha \Subset \beta) = (0, 0)$ on en déduit immédiatement:
  \begin{enumerate}
  \item $\mu_1(\Gamma) = 0$ on a $\Gamma = \Q_A \times \Q_B$ 
  \item $\mu_2(\alpha) = \mu_2(\beta) = 0$ induit $\alpha$ et $\beta$ sont tous les deux soit un état soit l'ensemble vide.
  \end{enumerate}
  On en déduit que le but que l'on cherche à prouver peut être de la forme:
  \begin{itemize}
  \item $\Gamma \vdash_{A, B} \emptyset \Subset \emptyset$ ce qui correspond au cas de la règle Empty : la dérivation est terminée.
  \item $\Gamma \vdash_{A, B} q \Subset q'$ : on utilise alors le fait que
    $\Gamma = \Q_A \times \Q_B$, et que $(q, q') \in \Q_A\times\Q_B \imp
    (q \Subset q') \in \Gamma$. Ce cas correspond à la règle Axiom ce qui conclut encore la dérivation courante.
  \end{itemize}
\end{proof}

En \coq, on ne définit pas la mesure exactement de cette manière. En réalité, on se passe aisément
de la mesure $\mu_2$ qui se ramène à de la récurrence structurelle. 
En particulier, on utilise le type \lstinline!list! pour représenter les ensembles de configurations,
et les configurations sont des listes d'états (\lstinline!list state!). Comme le parcours des listes 
est structurellement récursif, c'est alors aussi le cas pour les ensembles et les configurations.
Au final, l'implémentation de l'algorithme est basé sur un seul appel récursif de la forme:
\begin{lstlisting}
(* hypothese d'induction *)
(* $\Gamma$ est l'ensemble des hypotheses du but *)
IH: forall $\Gamma'$, $\mu_1\ \Gamma'$ < $\mu_1\ \Gamma$ ->
      forall q q' : state, {I_state $\Delta_1$ $\Delta_2$ $\Gamma'$ q q'} + {~ I_state $\Delta_1$ $\Delta_2$ $\Gamma'$ q q'}.
\end{lstlisting}

% L'algorithme est implémenté en une unique fonction récursive assez importante.
%  Cette fonction ne nécessite donc plus que la mesure $\mu_1$,
% et la preuve de terminaison en \coq\ ce qui réduit la preuve à un cas.
Les ensembles $\Gamma$ et $\Gamma'$ sont modélisés par une structure
qui contient des couples d'états. Initialement implémentée par une \lstinline!list! of \lstinline!state * state!,
cette structure s'est révélée très vite insuffisante pour de gros automates: en effet la recherche d'un élément
dans une liste est peu efficace, puisqu'il faut parcourir toute la liste dont la taille peut aller jusqu'à $|\Q_\A \times \Q_\B|$.
%Afin de gagner notablement, 
Le recours aux tables d'associations s'est imposé pour revenir à une complexité plus raisonnable.
Or la preuve est plus facile à réaliser sur les listes d'association que sur les tables d'association,
notamment pour raisonner sur $\mu_1$.
Pour rendre plus facile la définition d'une mesure pour n'importe quelle implémentation d'ensemble fini
(par exemple les listes ou les tables), on utilise l'interface suivante pour abstraire l'implémentation 
de l'ensemble fini: cette signature de module isole les fonctions nécessaires ainsi que leur propriétés pour construire
les différentes instances $\Gamma$. En fait, la seule opération nécessaire dans cet algorithme est l'ajout d'un élément à $\Gamma$.
Le type $\alpha$ abstrait le types des éléments de l'ensemble, et $\tau$ correspond à la structure qui implémente l'ensemble.
On suppose bien sûr que l'on est en mesure de distinguer les éléments de l'ensemble~(\lstinline!$\alpha$_eq!).
\begin{lstlisting}
Module Type FINITE_SET.
  Variable $\alpha$ : Type.
  Hypothesis $\alpha\_$eq: forall (x y: $\alpha$), {x = y} + {x <> y}.
  
  Variable $\tau$ : Type.
  Variable In : $\alpha \rw \tau \rw$ Prop.
  
  Variable add : $\alpha \rw \tau \rw \tau$.
  Hypothesis add_x_spec : forall E x, In x (add x E).
  Hypothesis add_other_spec :
         forall x y, x <> y -> forall E, (In y (add x E) <-> In y E).

  Hypothesis elements : $\tau \rw$ list $\alpha$.
  Hypothesis elements_spec : forall E a, In a E <-> List.In a (elements E).
End FINITE_SET.
\end{lstlisting}
Cette signature suppose aussi que l'on est en mesure d'établir une équivalence entre le type $\tau$ et une
liste d'éléments $\alpha$. Cette astuce déjà présente dans la librairie standard \coq~\cite{coq-stdlib}, 
se révèle très pratique: elle permet de ramener le raisonnement sur les listes, pour lesquelles les preuves
sont souvent plus simples.
En se basant sur un module \coq\ qui implémente l'interface~\lstinline!FINITE_SET!, on peut construire un nouveau module
en utilisant le foncteur~\lstinline!Make_Wf_Measure!.
\begin{lstlisting}
Module Make_Wf_Measure (Fs : FINITE_SET) <: WF_MEASURE.

Module Type WF_MEASURE.
  Variable $\alpha$: Type.
  Variable $\tau$: Type.
  Variable add: $\alpha$ -> $\tau$ -> $\tau$.
  Variable R: list $\alpha$ -> $\tau$ -> $\tau$ -> Prop.
  Variable In: $\alpha$ -> $\tau$ -> Prop.

  Hypothesis R_add:
    forall top,
      forall a m, List.In a top -> ~In a m -> R top (add a m) m.

  Hypothesis wf_R: forall top, well_founded (R top).
End WF_MEASURE.
\end{lstlisting}
Le module construit contient une relation \lstinline!R top! bien-fondée 
définit comme \lstinline!R top $\Gamma'$ $\Gamma$! équivaut à $\mu_1(\Gamma') < \mu_1(\Gamma)$.
Le premier argument correspond à la totalité de l'ensemble que l'on veut parcourir,
soit \lstinline!top = $\Q_\A \times \Q_\B$!. Au final, le module fournit une preuve que la relation 
\lstinline!(R top)! est bien fondée ainsi qu'une preuve que l'ajout d'un nouvel élément 
est ordonné: $\mu_1(\{q \Subset q'\} \cup \Gamma) < \mu_1(\Gamma)$ si $\{q \Subset q'\} \not\in \Gamma$.

L'implémentation de la fonction~\lstinline!inclusion! repose uniquement sur un module de type
\lstinline!FINITE_SET! et un module de type \lstinline!WF_MEASURE! pour le raisonnement et
la manipulation de $\Gamma$.

%Aussi pour rendre la preuve la moins sensible au changement de structure de données.


\subsection{Preuve de correction}
\label{sec:correction}

Pour la correction, on veut assurer que si il existe un arbre de preuve $\prod$
pour $\emptyset \vdash_{\A, \B} q \Subset q'$ alors on peut en déduire l'inclusion de langages
$\Lang(\Delta_\A, q) \subseteq \Lang(\Delta_\B, q')$

% \begin{proof}
% This can be done by an induction on the size of the term of 
% $\Lang(\Delta_\A, q)$. See~\cite{BoyerGJ-RR08} for details.
% \end{proof}

Pour construire la démonstration de la correction, on a besoin du résultat intermédiaire suivant.

\begin{lemma}{(Coupures dans les arbres de $\Subset$-preuve)}
  \label{lem:cut}
  Soient deux automates d'arbres $\A$ et $\B$, si il existe $\prod$ un arbre de preuve
  pour $\Gamma \vdash_{\A, \B} q \Subset q'$, ainsi qu'un arbre de preuve pour
  $\Gamma \cup \{q \Subset q'\}\vdash_{\A, \B} q_a \Subset q_b$
  alors il est possible dec construire un arbre de preuve pour $\Gamma \vdash_{\A, \B} q_a \Subset q_b$.
\end{lemma}

\begin{proof}
  On procède par induction sur $\mu(\Gamma)$.
  
  \noindent
  si $\mu(\Gamma) = 0$, on sait alors que $\Q_\A \times \Q_\B = \Gamma$.
  Comme $q_a \Subset q_b \in \Gamma$, on peut donc prouver que 
  $\Gamma \vdash_{\A,\B} q_a \Subset q_b$ en utilisant la règle Axiom.
  
  \medskip
  \noindent
  Maintenant, comme hypothèse d'induction, on pose que $\forall \Gamma\  s.t.\ \mu(\Gamma) = n$, $\forall q\ q'$,
  si il existe un arbre de preuve $\prod$ pour $\Gamma \vdash_{\A, \B} q \Subset q'$ et si pour tous les états $q_a, q_b$
  il existe un arbre de preuve pour $\Gamma \cup \{q \Subset q'\}\vdash_{\A, \B} q_a \Subset q_b$ alors on a aussi
  un arbre de preuve pour $\Gamma \vdash_{\A, \B} q_a \Subset q_b$. On va donc montrer que cette propriété est vraie
  pour $\Gamma$ tel que $\mu(\Gamma) = n+1$.
  
  \medskip
  \noindent
  On considère l'arbre de preuve de la seconde hypothèse $\Gamma \cup \{q \Subset
  q'\}\vdash_{\A, \B} q_a \Subset q_b$.  Tout d'abord, si l'arbre de preuve est construit 
  en utilisant la règle Axiom, on sait que $(q_a \Subset q_b) \in \Gamma \cup \{(q \Subset
  q')\}$.  Deux cas sont possibles :
  \begin{itemize}
  \item soit $(q_a \Subset q_b) \in \Gamma$, et alors on construit la preuve de 
    $\Gamma \vdash_{\A,\B} q_a \Subset q_b$ en utilisant la règle Axiom.
    
  \item soit $q = q_a$ et $q' = q_b$, et donc $\Gamma \vdash_{\A,\B} q_a
    \Subset q_b$ est tout simplement équivalent à $\Gamma \vdash_{\A, \B} q \Subset q'$ pour lequel
    l'arbre de preuve est $\prod$.
  \end{itemize}
  
  \noindent
  Ensuite, si l'arbre de preuve pour $\Gamma \cup \{q \Subset q'\}\vdash_{\A, \B} q_a
  \Subset q_b$ est construit en utilisant la règle Induction, alors on a: 
  
  \medskip
  \newcommand{\env}{\Gamma \cup \{q_a \Subset q_b\} \cup \{q \Subset q'\} \vdash_{\A, \B}}
  
  \centerline{
    \begin{minipage}{21cm}
      {\tiny
        \begin{prooftree}
          \AxiomC{\small $\prod_{c_1}$}
          \UnaryInfC{$\env c_1 \Subset c'_{k_1}$}
          \LeftLabel{(Weark-r)}
          \UnaryInfC{$\env 
            c_1 \Subset  \{c_k'| c_k' \rightarrow_\B q_b\}_1^m$}
          % Pointillets du milieu
          \AxiomC{\small \dots\dots}
          \AxiomC{\small $\prod_{c_n}$}
          \UnaryInfC{$\env c_n \Subset c'_{k_n}$}
          \RightLabel{(Weark-r)}
          \UnaryInfC{$\env
            c_n \Subset  \{c_k'| c_k' \rightarrow_\B q_b\}_1^m$}
          \LeftLabel{(Split-l)}
          \TrinaryInfC{$\env
            \{c_i| c_i \rightarrow_\A q_a\}_1^n \Subset  \{c_i'| c_i'\rightarrow_\B q_b\}_1^m$}
          \LeftLabel{(Induction)}
          \UnaryInfC{$\Gamma \cup \{q \Subset q'\} \vdash_{\A, \B} q_a \Subset q_b$}
        \end{prooftree}
      }
    \end{minipage}}
    
  \medskip
  \noindent
  Où chaque $\prod_{c_i}$ est construit de la manière suivante (en supposant que
  $c_i = f(q_{i_1}, \dots, q_{i_n})$ et $c'_{k_i} = f(q'_{i_1}, \dots, q'_{i_n})$ ):
  
  {\tiny
    \begin{prooftree}
      \AxiomC{\small $\prod_{i_1}$}
      \UnaryInfC{$\env q_{i_1} \Subset q'_{i_1}$}
      \AxiomC{\small \dots\dots}
      \AxiomC{\small $\prod_{i_n}$}
      \UnaryInfC{$\env q_{i_n} \Subset q'_{i_n}$}
      \LeftLabel{(Config)}
      \TrinaryInfC{$\env f(q_{i_1}, \dots, q_{i_n}) \Subset f (q'_{i_1}, \dots, q'_{i_n})$}
    \end{prooftree}
  }
    
  Si on essaye de construire la preuve du but $\Gamma \vdash_{\A,\B}
  q_a \Subset q_b$, il commence nécessairement de la même manière,
  excepté que $\{q \Subset q'\}$ n'apparaîtra pas dans la liste des
  hypothèses, \textit{i.e. }dans la partie gauche du but. Chaque branche
  de cet arbre se terminera avec une conclusion de la forme $\Gamma \cup \{q_a
  \Subset q_b\} \vdash_{\A, \B} q_{i_j} \Subset q'_{i_j}$. 

  Pour conclure la preuve, il reste à donner les arbres de preuve 
  $\prod'_{i_j}$ pour chacun des buts correspondants.
  On sait qu'il existe des arbres de preuve 
  $\prod_{i_j}$ pour chaque $\Gamma \cup \{q \Subset q'\} \cup
  \{q_a \Subset q_b\} \vdash_{\A, \B} q_{i_j} \Subset q'_{i_j}$.  On peut
  utiliser l'hypothèse d'induction sur $\prod_{i_j}$ pour obtenir 
  $\prod'_{i_j}$ :
  \begin{itemize}
    
  \item Puisque $\mu(\Gamma) = n + 1$, on a $\mu(\Gamma\cup \{q_a \Subset q_b\}) = n$
      

  \item Comme  $\prod$ est une preuve de $\Gamma \vdash_{\A, \B} q \Subset q'$, c'est aussi une preuve de 
    $\Gamma \cup \{q_a \Subset q_b\}\vdash_{\A, \B} q \Subset q'$.
      
  \item Chaque $\prod_{i_j}$ est une preuve de $\env  q_{i_j} \Subset q'_{i_j}$
  \end{itemize}
    
  \noindent
  En utilisant le principe  d'induction, on en déduit que pour tout $i,j$ il existe
  une arbre de preuve $\prod'_{i_j}$ pour $\Gamma \cup \{q_a \Subset q_b\} \vdash_{\A, \B} q_{i_j}
  \Subset q'_{i_j}$.  Ce qui termine la construction de l'arbre pour le but  $\Gamma \vdash_{\A,
    \B} q_a \Subset q_b$.
\end{proof}
  
\begin{theorem}{(Correction)}
  \label{thm:soundness}
  Soient deux automates d'arbres $\A$ et $\B$, si il existe un arbre $\prod$ de preuve
  pour $\emptyset \vdash_{\A, \B} q \Subset q'$ alors on a $\Lang(\A, q) \subseteq \Lang(\B, q')$
\end{theorem}
  
\begin{proof}
  On montre que la propriété $\forall t$, $t \in \Lang(\A, q) \imp t \in \Lang(\B, q')$ est vérifiée par
  induction sur le terme $t$. On pose $t = f(t_1, \dots, t_n)$. 
  On suppose alors que la propriété est vraie pour chacun sous-terme 
  $t_i$, \textit{i.e.} pour tous les états  $q_i, q'_i$ tels que si il existe un arbre de
  preuve $\prod_i$ pour $\emptyset \vdash_{\A, \B} q_i \Subset q'_i$ alors $t_i
  \rightarrow^*_{\A} q \imp t \rightarrow^*_{\B} q_i'$. Comme $t=f(t_1, \dots,
  t_n) \in \Lang(\A, q)$, alors pour chaque sous-terme $t_i$, on sait qu'il existe $n$ états 
  $q_1, \ldots, q_n$ tels que $t_i \in \Lang(\A, q_i)$ et $f(q_1, \dots, q_n)
  \rightarrow q \in \A$. D'autre part, en dépliant  l'arbre de preuve $\prod$ de
  $\emptyset \vdash_{\A, \B} q \Subset q'$, on peut déduire que pour chaque transition
  de la forme $f(q_1, \dots, q_n) \rightarrow q \in \A$, il existe une transition $f(q'_1,
  \dots, q'_n) \rightarrow q' \in \B$ telle que on ait un arbre de preuve $\prod_i$ pour
  $\{q \Subset q'\} \vdash_{\A, \B} q_i \Subset q'_i$. 
  Comme $f(q_1, \ldots, q_n) \rw q \in \A$, on obtient que $f(q'_1, \dots, q'_n)
  \rightarrow q' \in \B$ et un arbre de preuve $\prod_i$ pour $\{q \Subset q'\} \vdash_{\A,
    \B} q_i \Subset q'_i$. Pour conclure que $f(t_1, \dots, t_n) \in \Lang(\B,
  q')$ il suffit de montrer que  $t_i \in \Lang(\B, q'_i)$. On remarque que l'on a 
  un arbre de preuve $\prod_i$ pour $\{q \Subset q'\} \vdash_{\A,\B} q_i \Subset q'_i$
  et que pour appliquer l'hypothèse d'induction, on a besoin d'un arbre de preuve pour
  $\emptyset \vdash_{\A,\B} q_i \Subset q'_i$. C'est ici que l'on utilise le lemme~\ref{lem:cut} sur $\prod$
  et $\prod_i$, on peut en déduire l'existence de $\prod'_i$, l'arbre de preuve pour
  $\emptyset \vdash_{\A, \B} q_i \Subset q'_i$. Alors en utilisant l'hypothèse d'induction sur
  $t'_i, q_i, q'_i$ et $\prod'_i$,
  on obtient que pour chaque sous-terme $t_i \in \Lang(\A, q_i)$, on a aussi $t_i \in
  \Lang(\B, q'_i)$. Enfin, comme $f(q'_1, \ldots, q'_n)\rw q' \in \B$, on en déduit que
  $t=f(t_1, \dots, t_n) \in \Lang(\B, q')$.
\end{proof}



\subsection{Preuve de complétude}
\label{sec:completness}

Comme expliqué précédemment, l'algorithme présenté n'est pas complet en général.
Cependant, on va montrer qu'il est complet pour les automates d'arbres produits par
la complétion. En effet, si $\A_\R^k$ est obtenu après  $k$
étapes de complétion à partir de l'automate $\A^0$ alors on peut construire une preuve $\prod$ pour 
le but $\emptyset \vdash_{\A^0, \A^k_{\R}} \{\#(q)\ |\ q \in
\mathcal{Q}_{F_0}\} \Subset \{\#(q')\ |\ q'\in
\mathcal{Q}_{F_k}\}$. On rappelle que l'automate produit par la 
$k^\text{ème}$ étape de complétion est noté $\mathcal{A}_k = \langle
\mathcal{F}, \mathcal{Q}_k, \mathcal{Q}_{F_k}, \Delta_k\rangle$.

%\comments{verifier si la version courte ici n'est pas dans la version longue\dots}
Pour rappel, la complétion d'automates d'arbres effectue principalement deux opérations
à chaque étape de calcul: la \emph{normalisation} et la \emph{fusion d'états}.
Dans le cas de la normalisation,  l'inclusion des langages peut être simplement
démontrée en utilisant l'inclusion des états de transitions,la normalisation ne produisant 
que de nouvelles transitions.
Pour l'opération de la fusion d'états, l'inclusion des ensembles de transitions n'est plus suffisante,
puisque la fusion d'états implique aussi la fusion de transitions.
C'est pour cette raison que l'on 
définit une nouvelle relation d'ordre préservée par chaque opération.

% first, let us introduce some new ordered relation on tree automata:

\begin{definition}
  \label{eq:prop0}
  Soient $\A$, $\B$ deux automates. On définit $\sqsubseteq$ la relation réflexive et transitive 
  comme : $\A \sqsubseteq \B$ si il existe une fonction 
  $\varrho$ qui renomme les états de  $\A$ dans les états de $\B$ et telle que 
  toutes les transitions renommées de  $\Delta_\A$ sont contenues dans  $\Delta_\B$:
  
  \begin{equation}
    \A \sqsubseteq \B \Longleftrightarrow
    \exists \varrho : \mathcal{Q_\A} \rightarrow \mathcal{Q_\B},
    \ \varrho(\Delta_\A) \subseteq \Delta_\B\ \wedge \ \varrho(\Q_{F_\A}) \subseteq \Q_{F_\B}
  \end{equation}
\end{definition}


On étend l'application du renommage $\varrho$ pour les ensembles et les configurations de la manière suivante:
\begin{itemize}
\item $\varrho(\{q_i\}_1^n)$ est utilisée pour  $\{\varrho(q_i)\}_1^n$

\item $\varrho(c) = \left\{\begin{array}{ll}
      f (\varrho(c_1), \dots, \varrho(c_n)) &\textrm{ si } c = f(c_1, \dots, c_n)\\
      c &\textrm{ si } c \in \F_{0}\\
      \varrho(q) &\textrm{ si } c = q\in \Q\\
    \end{array}\right.$
  
\item $\varrho(c \rightarrow q)$ est utilisée pour  $\varrho(c) \rightarrow \varrho(q)$


\item $\varrho(\Delta)$ signifie $\{ \varrho(c\rightarrow q)\ |\ c \rightarrow q \in \Delta \}$.
\end{itemize}


Le lemme suivant montre que la \emph{normalisation} et la \emph{fusion
  d'états} préserve la relation $\sqsubseteq$.

\begin{lemma}
  Soit $\A$ un automate d'arbres, et $\Delta_\A$ son ensemble de transitions. 
  On considère $\A'$ tel que:
  
  \[
  \begin{array}{llll}
    1. & \mbox{ si } \Delta_{\A'} = \Delta_\A \cup \norm(r\sigma \rw q) & \mbox{ alors } &\A \sqsubseteq \A' \\
    2. & \mbox{ si } \A' = \merge(\A, q_1, q_2) &\mbox{ alors }& \A \sqsubseteq \A'  \\
  \end{array}
  \]
  où $\merge(\A,q_1,q_2)$ l'automate obtenu par la fusion des états $q_1$ et $q_2$ de $\A$, soit le renommage de  $q_1$
  en $q_2$ dans toutes les transitions de $\A$.
\end{lemma}

\begin{proof}

  \begin{enumerate}
  \item Le premier point est très facile à montrer puisqu'on a évidemment $\Delta_{\A'} \supseteq
    \Delta_\A$ quelque soient $r\sigma$ et $q$ choisis.
    Il suffit de prendre $\varrho = id$, on peut immédiatement conclure $\A
    \sqsubseteq \A'$.

  \item On considère $\Delta_\A$ l'ensemble des transitions de $\A$. Soient $q_1$ et $q_2$ deux états
    que l'on veut fusionner.  On peut alors appliquer à $\Delta_\A$ un fonction de renommage $\varrho$ qui donne le même 
    résultat que la fusion d'états pour $q_1 = q_2$:
    \[
    \varrho(q) = \left\{
      \begin{array}{l}
        \textrm{si }(q\ =\ q_2)\\
        \quad \quad q_1\\
        \textrm{sinon}\\
        \quad \quad q
      \end{array}\right.
    \]
    
    Ainsi la fusion d'états construit $\Delta_{\A'} = \varrho(\Delta_\A)$ et par 
    la définition~\ref{eq:prop0} on en déduit $\Delta_A \sqsubseteq \Delta_{\A'}$.
  \end{enumerate}
\end{proof}


\begin{theorem}
  Pour $\aaexeq^0$ un automate d'arbres donné, $\R$ un ensemble de règles de réécriture
  et $E$ un ensemble d'équations, après $k$ étapes de complétion on obtient 
  $\aaexeq^k$ tel que  $\A^0 \sqsubseteq \aaexeq^k$.
\end{theorem}

\begin{proof}
  Par induction sur $k$:
  
  \begin{itemize}
  \item Comme $\sqsubseteq$ est réflexive, on a évidemment $\aaexeq^0 \sqsubseteq \aaexeq^0$.
  \item Soit $\aaexeq^k$ l'automate d'arbres obtenu après $k$ étapes de complétion tel que $\aaexeq^0 \sqsubseteq \aaexeq^k$.
    Par définition de la complétion $\aaexeq^{k+1}$ est construit à partir de  $\aaexeq^k$ en appliquant successivement la normalisation puis la fusion.
   Alors on a $\aaexeq^k \sqsubseteq \aaexeq^{k+1}$.  
%Let us apply one
%    more step of completion. After normalization of $\A_\R^k$ we have $\A_\R^k
%    \sqsubseteq \A'$; then we merge $\A'$ with set $\mathcal{E}$ to obtain $\A'
%    \sqsubseteq \A_\R^{k+1}$. 
   Par transitivité de $\sqsubseteq$, à partir de$\A^0 \sqsubseteq \A_\R^k$ et
    $\A_\R^k \sqsubseteq \A_\R^{k+1}$ on en déduit immédiatement que $\A^0 \sqsubseteq \A_\R^{k+1}$.
    En effet, la relation $\sqsubseteq$ est transitive, pour cela il suffit de considérer la composition
    des fonctions de renommage, et le fait que l'inclusion d'ensemble est aussi une relation transitive.
  \end{itemize}
\end{proof}


Maintenant, on peut définir la propriété de complétude vis-à-vis de cette relation d'ordre:

\begin{theorem}{(Complétude)}
  Pour deux automates d'arbres donnés $\A$ et $\B$, si on a  $\A \sqsubseteq \B$ alors il existe
  $\prod$ un arbre de preuve pour $\emptyset \vdash_{\A, \B} \{\#(q_f)\ |\  q_f \in \mathcal{Q}_{F_A}\} \Subset  \{\#(q'_f)\ |\  q'_f \in \mathcal{Q}_{F_B}\}$.
\end{theorem}

\begin{proof}
  Par définition de $\A \sqsubseteq \B$, on sait qu'il existe une fonction de renommage $\varrho$.\\
  
  Tout d'abord, on montre par induction sur l'arbre de preuve que pour tout $\Gamma$
  et $q$, $\Gamma \vdash_{\A, \B} q \Subset \varrho(q)$:\\
  L'hypothèse d'induction est alors $\forall \Gamma,\ q,\ q_i,$
  \begin{prooftree}
    \AxiomC{$\prod_i$}
    \UnaryInfC{$\Gamma \cup \{q \Subset \varrho(q)\} \vdash_{\A, \B} q_i \Subset \varrho(q_i)$}
  \end{prooftree}

  On veut construire un arbre de preuve pour $\Gamma \vdash_{\A, \B} q \Subset \varrho (q)$.\\
  Deux cas sont envisageables:
  \begin{itemize}
  \item 
    si $q \Subset \varrho(q) \in \Gamma$ alors on peut conclure immédiatement:
    \begin{prooftree}
      \AxiomC{}
      \LeftLabel{(Axiom)}
      \UnaryInfC{$\Gamma' \cup  \{q \Subset \varrho(q)\} \vdash_{\A, \B} q \Subset \varrho(q)$}
    \end{prooftree}

  \item 
    Sinon,il faut appliquer la règle Induction pour obtenir un arbre de la forme:
  
    \newcommand{\env}{\Gamma \cup \{q \Subset \varrho(q)\} \vdash_{\A, \B}}
    {\tiny
      \begin{prooftree}
        \AxiomC{\small $\prod_{c_1}$}
        \UnaryInfC{$\env 
          c_1 \Subset  \{c_k'| c_k' \rightarrow \varrho(q)\}_1^m$}
        % Pointillets du milieu
        \AxiomC{\small \dots\dots}
        \AxiomC{\small $\prod_{c_n}$}
        \UnaryInfC{$\env
          c_n \Subset  \{c_k'| c_k' \rightarrow \varrho(q)\}_1^m$}
        \LeftLabel{(Split-l)}
        \TrinaryInfC{$\env
          \{c_i| c_i \rightarrow q\}_1^n \Subset  \{c_k'| c_k'\rightarrow \varrho(q)\}_1^m$}
        \LeftLabel{(Induction)}
        \UnaryInfC{$\Gamma \vdash_{\A, \B} q \Subset \varrho(q)$}
      \end{prooftree}
    } A partir de l'hypothèse $\varrho(\Delta_A) \subseteq \Delta_B$ pour chaque transition
    $c \rightarrow q$ de $\Delta_A$, on a $\varrho(c\rightarrow q) \in \Delta_\B$.
    \[\textrm{Ainsi, pour tout } (c\rightarrow q)\in \Delta_\A,\textrm{ on a } \varrho(c) \in \{c_k' | c_k' \rightarrow \varrho(q)\}_1^m \]
    Pour chaque $c_i= f_i(q_{i_1}, \dots, q_{i_n})$ on peut construire l'arbre correspondant
    $\prod_{c_i}$ dont chaque branche est conclue par $\prod_{i_j}$
    une instance de l'hypothèse d'induction pour $q_{i_j}$ l'état correspondant:
    %\comments{Revoir ce passage\dots}
    {\tiny
      \begin{prooftree}
        \AxiomC{\small $\prod_{i_1}$}
        \UnaryInfC{$\env q_{i_1} \Subset \varrho(q_{i_1}) $}
        \AxiomC{\small \dots\dots}
        \AxiomC{\small $\prod_{i_n}$}
        \UnaryInfC{$\env q_{i_n} \Subset \varrho(q_{i_n}) $}
        %%%%%%%%%%%%%%%%%%%%%%% 
        \LeftLabel{(Config)}
        \TrinaryInfC{$\env c_i \Subset  \varrho(c_i) $}
        \LeftLabel{(Weak-r)}
        \UnaryInfC{$\env c_i \Subset  \{c_k'| c_k' \rightarrow \varrho(q)\}_1^m$}
      \end{prooftree}
    }
  \end{itemize}
  
  D'autre part, on sait que  pour tout $\Gamma$ et $q \in \Q_\A$ il existe un arbre de 
  preuve $\prod_q$ pour tout but $\Gamma \vdash_{\A, \B} q \Subset \varrho(q)$.\\
  En particulier, cela est vrai pour $\Gamma = \emptyset$ et pour tout état $q$ de
  $\Q_{F_\A}$.  Comme on a $\A \sqsubseteq \B \imp
  \varrho(\Q_{F_\A}) \subseteq \Q_{F_\B}$, on peut construire un arbre de preuve tel que:

  {\small
    \begin{prooftree}
      \AxiomC{$\prod_{q_{f_1}}$}
      \UnaryInfC{$\emptyset \vdash_{\A, \B} q_{f_1} \Subset \varrho(q_{f_1})$}
      %\AxiomC{$\emptyset \vdash_{\A, \B} [\ ] \Subset [\ ]$}
      \LeftLabel{(Config)}
      \UnaryInfC{$\emptyset \vdash_{\A, \B} \#(q_{f_1}) \Subset \#(\varrho(q_{f_1}))$}
      \LeftLabel{(Weak-r)}
      \UnaryInfC{$\emptyset \vdash_{\A, \B} q_{f_1} \Subset \Q_{F_\B}$}
      \AxiomC{\small \dots\dots}
      \AxiomC{$\prod_{q_{f_n}}$}
      \UnaryInfC{$\emptyset \vdash_{\A, \B} q_{f_1} \Subset \varrho(q_{f_1})$}
      %\AxiomC{$\emptyset \vdash_{\A, \B} [\ ] \Subset [\ ]$}
      \RightLabel{(Config)}
      \UnaryInfC{$\emptyset \vdash_{\A, \B} \#(q_{f_n}) \Subset \#(\varrho(q_{f_n}))$}
      \RightLabel{(Weak-r)}
      \UnaryInfC{$\emptyset \vdash_{\A, \B} \#(q_{f_n}) \Subset \{\#(q)\ |\ \Q_{F_\B}\}$}
      \LeftLabel{(Split-l)}
      \TrinaryInfC{$\emptyset \vdash_{\A, \B} \{\#(q)\ |\ q \in \Q_{F_\A}\} \Subset \{\#(q)\ |\ q \in \Q_{F_\B}\}$}
    \end{prooftree}}
\end{proof}


Ainsi, on peut assurer que pour un automate $\aaexeq^k$ obtenu après $k$
étapes de complétion à partir $\aaexeq^0$, il existe une preuve $\prod$ pour le 
but $\emptyset \vdash_{\A^0, \A^k_{\R}} \{\#(q)\ |\ q \in
\mathcal{Q}_{F_{0}} \Subset \{\#(q')\ |\ q'\in \mathcal{Q}_{F_{k}}\}$.
Il peut être vu comme une conséquence des deux théorèmes précédents.

%\begin{theorem}
%  Given a tree automaton $\mathcal{A}^0$, a TRS $\R$ and an equation set
%  $\mathcal{E}$, after $k$ completion steps we obtain $\A_\R^k$ such that
%\end{theorem}


\paragraph{A propos de l'incomplétude: }
%remarks about restrictions:}
Pour obtenir une procédure de décision pour tout automate d'arbres (\textit{i.e.}
des automates d'arbres obtenus autrement que par la complétion), la règle
Weak-r doit être modifiée. C'est à cause de cette règle que le système est incomplet 
en oubliant de prendre en compte d'autres manière de construire l'union.

\begin{example}
  Soient deux automates d'arbres $\A$ et $\B$ tels que:
  \[\A = \left\{ 
    \begin{array}{rcl}
      a& \rightarrow &q_1\\
      b& \rightarrow &q_2\\
      c& \rightarrow &q_2\\
      f(q_1, q_2)&\rightarrow &{\textbf q}\\
    \end{array}\right\}
  \textrm { and }
  \B = \left\{ 
    \begin{array}{rcl}
      a& \rightarrow &q_1'\\
      b& \rightarrow &q_2'\\
      c& \rightarrow &q_3'\\
      f(q_1', q_2')&\rightarrow &\textbf{q'}\\
      f(q_1', q_3')&\rightarrow &\textbf{q'}\\
    \end{array}\right\}
  \]
  Cet exemple illustre l'incomplétude du système. Les deux automates d'arbres reconnaissent 
  clairement le même langage $\Lang(\A, q) = \Lang(\B, q')$. Si on peut construire une preuve pour le but $\emptyset \vdash_{\B, \A} q' \Subset q$,
  ce n'est en revanche pas le cas pour le but $\emptyset \vdash_{\A, \B} q \Subset q'$. En effet, la règle Weak-r ne peut montrer 
  le but $\Gamma \vdash_{\A, \B} \{f(q_1, q_2)\} \Subset \{f(q_1', q_2'), f(q_1', q_3')\}$ qu'à partir du but $\Gamma \vdash_{\A, \B} f(q_1, q_2) \Subset f(q_1', q_2')$
  ou $\Gamma \vdash_{\A, \B} f(q_1, q_2) \Subset f(q_1', q_3')$ qui sont tous les deux faux.
\end{example}

%Mais c'est aussi grâce à cette limitation que l'on arrive à éviter les problèmes de complexité de l'algorithme d'inclusion standard.


\subsection{Complexité}
Comme expliqué précédemment dans la section~\ref{sec:objectives}, l'algorithme standard pour
vérifier l'inclusion $\Lang(\A) \subseteq \Lang(\B)$ est basé sur le calcul de $\overline{\B}$ l'automate complément.
Cependant, pour les automates d'arbres non-déterministes, la taille~\cite{TATA} de $\overline{\B}$ peut être exponentiellement plus grande
que la taille de $\B$. L'algorithme proposé ici n'a pas cet inconvénient et utilise seulement
une taille mémoire qui est polynomiale par rapport à la taille des automates considérés.

On pose $|\Q|$ le nombre maximum d'états des deux automates d'arbres $\A$ et $\B$.  Les arbres de preuves
construits en utilisant les règles de déduction données sont au plus de hauteur $|\Q|^2$.
On doit cela aux règles Induction et Axiom qui assurent que chaque problème d'inclusion $q \Subset q'$
sera analysé seulement une fois par branche. 
Comme on a $q\in \Q_\A$ et $q' \in \Q_\B$ et que l'on sait que les cardinal de $\Q_\A \times Q_\B$ est borné par $|\Q|^2$,
la longueur de la branche est au plus bornée par $|\Q|^2$.  De plus, l'algorithme \lstinline!inclusion! explorant chaque branche 
de l'arbre de preuve une par une, la mémoire utilisée est donc bornée par $|\Q|^2$,
et donc polynomiale.

Cependant, la complexité d'une exécution de l'algorithme tel qu'il est implémenté est exponentielle.
En effet, même si chaque $q \Subset q'$ n'est considéré qu'une seule fois par branche,
le nombre de branches à considérer est exponentiel par rapport à 
$\Q$ et un même couple $q \Subset q'$ peut être rencontré dans plusieurs branches
et donc le but peut-être reprouvé plusieurs fois.
Une optimisation très simple de cet algorithme consiste à tabuler une partie des résultats
pour certains des couples $q \Subset q'$. Ainsi, on en utilisant le lemme~\ref{lem:cut}
pour chacun des couples $q \Subset q'$ pour lesquels la propriété a été prouvée on peut l'ajouter
à la liste des hypothèses des buts suivants. 
Par exemple, en utilisant cette optimisation pour tous les couples,
chaque couple n'est alors plus considéré qu'une {\em seule fois}, dans tout l'arbre de preuve.
Si les opérations d'accès à la table était réalisé en temps constant, 
cela ramènerait la complexité polynomiale en temps bornée par $|\Q|^2$. 
Néanmoins, on n'a jamais besoin de tabuler la totalité des états car cela augmente par ailleurs
la consommation de mémoire). En fait, on peut voir cette liste de couples, comme une stratégie
pour définir un ordre sur les sous-branches de la preuve que l'on souhaite réutiliser plusieurs fois.
L'approche est correcte car si une branche ne peut être prouvée pour une certaine paire d'états, 
elle ne sera pas ajoutée à l'environnement. On peut donc définir soit à la main ou par l'intermédiaire
d'un oracle la liste d'états pour laquelle on souhaite construire et conserver les preuves.

% \lstinline!inclusion_sound! is more difficult on a tabled version of the \coq\
% \lstinline!inclusion! function. Second, on test cases, it appears that avoiding
% the exponential blow-up of memory was critical but that practical performances
% of the, potentially, exponential time algorithm are sufficient.


%\section{La clôture par réécriture}

\section{Formalisation de la clôture par réécriture}
\label{sec:closure}

Dans cette section, on s'intéresse à définir formellement le prédicat \lstinline!IsClosed!, la fonction 
\lstinline!closure! et on montre que cette fonction est correcte par rapport à la propriété de clôture
\lstinline!IsClosed!. On rappelle que pour vérifier qu'un automate d'arbres $\A = \langle \Q_F,\ \Delta \rangle$ est
clos pour un système de règles de réécriture $\R$, il est suffisant de montrer que pour tout terme $t \in \Lang(\A)$, si
le terme $t'$ est atteignable à partir de $t$ par $\rw_\R^*$ alors $t'$ est aussi reconnu par l'automate. 
En se basant sur les spécifications de la réécriture en \coq\ et des automates d'arbres présentées dans les sections~\ref{sec:rewriting}
et~\ref{sec:automata}, on peut définir plus formellement le prédicat \lstinline!IsClosed! et le théorème \lstinline!closure_sound! à montrer:

\begin{lstlisting}
Definition IsClosed (R : list rule) (A : t_aut) : Prop :=
   forall q t t', IsRec A.delta q t -> Reachable R t t' -> IsRec A.delta q t'.

Theorem closure_sound:
      forall R A', closure R A' = true -> IsClosed R A'.
\end{lstlisting}


L'algorithme utilisé pour vérifier la clôture de  $\A$ par $\R$ calcule pour chaque règle $l \rightarrow r \in \R$
l'ensemble de toutes les substitutions $\sigma$ telles que $l\sigma
\rightarrow_\Delta^* q$ puis vérifie pour chaque $\sigma$ que $r\sigma \rightarrow_\Delta^*q$.
Ensuite, la preuve de correction consiste à prouver que si la fonction \lstinline!closure!
répond vrai, alors le langage $\Lang(\A)$ est clos par $\rw_\R$.

On donne des indications pour définir la fonction \lstinline!closure!.  Tout d'abord, pour
toute règle $l \rw r$ de $\R$, il faut trouver toutes les substitutions
$\sigma:\X \mapsto \Q$ pour chaque état $q \in \Q$ telles que
$l\sigma \rightarrow_\Delta^* q$, ce qui correspond au \emph{problème de filtrage}.
Dans un second temps, il faut vérifier que pour tout couple état-substitution $(q, \sigma)$ trouvé,
la paire critique est bien résolue c'est à dire $r\sigma \rw_{\Delta}^* q$. 
Enfin, dans le théorème de correction, on doit montrer que toutes les substitutions $\sigma:
\X\mapsto \Q$ couvrent l'ensemble des substitutions sur les termes, \textit{i.e.} les substitution de la forme $\sigma':
\X \mapsto \TF$, et ainsi que tous les termes atteignables sont couverts.

On rappelle que le problème de filtrage $l \unlhd q$ consiste à trouver toutes les substitutions
$\sigma:\X \mapsto \Q$ pour chaque état $q \in \Q$ telles que 
$l\sigma \rightarrow_\Delta^* q$. L'algorithme résolvant ce problème est initialement 
défini dans~\cite{Genet-RR97b}. Ici, on présente une version plus algorithmique correspondant 
à la version implémentée en \coq:
{\footnotesize
  % \vspace{-.2cm}
  \[\textrm{(Var) }
  \dfrac{}{x \match q,\ S \vdash_\A \{\sigma \cup \{x \mapsto q\} \sep \sigma \in S \}}
  (x \in \X)
  \]
  % \vspace{-.1cm}
  \[\textrm{(Unfold) }
  \dfrac{
    t_1 \match q_1, \ S_0 \vdash_\A S_1
    \quad \dots \quad
    t_n \match q_n, \ S_{n-1} \vdash_\A S_n
    % f(t_1,\dots, t_n) \match f(q_1,\dots, q_n) \vdash_\A S
  }{f(t_1, \dots, t_n) \match f(q_1, \dots q_n),\ S_0 \vdash_\A S_n}
  \]
  \hspace*{-1cm}
  \[
  \textrm{(Delta) }
   \dfrac{
     f(t_1, \dots, t_n) \match f(q_1^1, \dots, q_n^1)\ S_0 \vdash_\A S_1
     \quad \dots \quad
     f(t_1 ,\dots, t_n) \match f(q_1^m, \dots, q_n^m),\ S_0 \vdash_\A S_m
   }{
     f(t_1, \dots, t_n) \match q,\ S_0 \vdash_\A \bigcup_1^m S_i
   }
   (
   \textrm{si } \{ f(q_1^i, \dots, q_n^i) \rw q \in \Delta\} _1^m 
   )
   \]
 }

On énonce le problème de filtrage pour la règle $l \rw r$ et l'état $q$ de l'automate $\A$,
par $l\match q,\ \{id\} \vdash_\A S$ où $S$ contient l'ensemble des substitutions telles que
$l\sigma \rightarrow_\Delta^* q$. On remarque que l'ensemble initial est le singleton contenant $id$
la substitution identité.

\begin{theorem}
  \label{th:completeness}
  Pour toute substitution $\sigma$ telle que $l\sigma \rightarrow_\Delta^* q$, alors 
  cette substitution est déduite par l'algorithme de filtrage.
\end{theorem}

\begin{proof}
  La preuve est similaire à la preuve de complétude~\ref{prop:matching-complete} de l'algorithme de filtrage
  donnée au chapitre~3.
\end{proof}

L'implémentation de la fonction \lstinline!matching_rec! est \coq\ très proche
des règles de cet algorithme. L'ensemble des substitutions est définie par une 
\lstinline!list! de substitutions, et les substitutions sont définies comme des fonctions.
La signature \coq\ de la fonction de filtrage est alors:

\begin{lstlisting}
Definition substitutions := list substitution.

Definition matching_rec :
       Delta.t -> substitutions -> state -> term -> substitutions :=

   (* ... Body ... *)
.
\end{lstlisting}
Le but $l\match q,\ \{id\} \vdash_\A S$ se pose par l'intermédiaire de 
la définition \lstinline!matching! dont l'évaluation renvoie $S$.
Lorsque le filtrage échoue, c'est à dire si il n'existe aucune substitution
telle que $l\sigma \rightarrow_\Delta^* q$, la fonction \lstinline!matching! retourne
simplement la liste vide \lstinline!nil!, ce qui correspond simplement à l'ensemble vide.
Pour cet algorithme, la terminaison est syntaxique: la fonction \lstinline!matching_rec! est définie par récurrence 
structurelle sur le terme $l$ passé en argument à la fonction.
On énonce la propriété de complétude du filtrage de la manière suivante:
\begin{lstlisting}
(* Le singleton {id} est represente par id_subst::nil *)
Definition matching D q l := matching_rec D (id_subst::nil) l q.

(* L'expression (Fv l) calcule l'ensemble des variables du terme l *)
Theorem matching_spec:
   forall $\Delta$ l $\sigma$ q,
      Qsubst $\sigma$ -> TFX l -> linear l -> IsRed $\Delta$ q (l @ $\sigma$) ->

            exists $\sigma'$, $\sigma'$ \in (matching $\Delta$ q l) /\ (forall x, x \in (Fv l) -> $\sigma'$ x = $\sigma$ x).
\end{lstlisting}

Le prédicat \lstinline!Qsubst: substitution -> Prop! s'assure que la substitution 
$\sigma$ ne substitue les variables que par des états, ce qui se résume à $\sigma: \X \rw \Q$.
Le prédicat \lstinline!linear l! s'assure que le terme \lstinline!l! est le membre d'une règle linéaire à gauche.
Le prédicat \lstinline!IsRed! est une simple extension du prédicat \lstinline!IsRec!
introduit à la section~\ref{sec:automata} pour étendre la reconnaissabilité des termes clos
$\TF$ à l'ensemble des configurations $\TFQ$. 
On peut noter aussi une différence entre l'énoncé du théorème~\ref{th:completeness} et
son équivalent \lstinline!matching_spec!. En effet, dans ce dernier cas, on ne montre pas directement que la substitution 
$\sigma$ appartient à l'ensemble \lstinline!(matching $\Delta$ q l)! mais qu'il existe une substitution $\sigma'$
telle que son image est égale à celle de $\sigma$ pour chaque variable~\lstinline!x! de~\lstinline!l! le terme filtré.
Cette différence s'explique par l'usage implicite de l'extensionnalité dans la preuve papier. Cette propriété n'est pas vraie
en \coq\ et plus généralement en informatique, puisqu'il est toujours possible de donner deux implémentations
différentes pour une même fonction. Par exemple, les deux substitutions suivantes sont équivalentes 
du point de vue de leur résultat mais pas égales du point de vue syntaxique:
\begin{center}
  \begin{tabular}{l|l}
    \begin{minipage}{.45\linewidth}
\begin{lstlisting}
Definition $\sigma$ x :=
  if (id_eq x 0) then
     Some 1
  else if (id_eq x 1) then
     Some 2
  else None.
\end{lstlisting}
    \end{minipage}&
    \hspace{1.5cm}
    \begin{minipage}{.3\linewidth}
\begin{lstlisting}
Definition $\sigma'$ x :=
  if (id_eq x 1) then
     Some 2
  else if (id_eq x 0) then
     Some 1
  else None.
\end{lstlisting}   
    \end{minipage}
  \end{tabular}
\end{center}

En fait, la propriété énoncée telle quelle dans le théorème est suffisante, il n'est pas nécessaire
d'avoir l'égalité entre $\sigma$ et $\sigma'$. Cela étant, le schéma de la preuve \coq\ reste similaire 
à la preuve donnée ci-dessus.

La deuxième partie de la fonction \lstinline!closure! consiste à vérifier que
pour chaque substitution $\sigma$ telle $\sigma \in S$ où $S$ est l'ensemble de
substitutions calculé par le filtrage, on a $r\sigma \rw_\Delta^* q$, où $r$ est le membre droit 
de la règle de réécriture.
Cette vérification est effectuée par la fonction \lstinline!all_reduce_dec!.
Comme dans le cas de l'inclusion, on utilise une spécification avec une signature
riche. Donnée par le théorème suivant, cette spécification nous indique que 
\lstinline!all_reduce_dec $\Delta$ q r S! permet de conclure soit que toutes
les configurations \lstinline!r$\sigma$! sont réduites en l'état \lstinline!q!,
soit qu'il existe une substitution telle que la réduction ne soit pas possible.
\begin{lstlisting}
Definition all_reduce_dec $\Delta$ q (r: term) (S: substitutions): boolopt substitution.

Theorem all_reduce_dec_spec: 
   forall $\Delta$ q r S, all_reduce_dec $\Delta$ q r S = ok -> 
      forall $\sigma$, $\sigma$ $\in$ S -> IsRed $\Delta$ q (r @ s).
\end{lstlisting}

\begin{proof}
  La preuve ici est très simple. Elle se fait par induction sur \lstinline!S! la liste des substitutions. 
  Si la liste est vide la propriété est vérifiée. Sinon on décompose la liste en $\sigma::l$.
  On se base alors sur un théorème qui énonce la décidabilité du prédicat \lstinline!IsRed!.
  Ainsi~\lstinline!IsRed $\Delta$ q t! est équivalent à la relation de reconnaissabilité~$t \rw^*_\delta q$
  pour $t \in \TFQ$. Or la reconnaissabilité est bien sûr décidable, ce qui s'énonce en \coq\ par le théorème suivant

  \begin{lstlisting}
Lemma IsRed_dec:
   forall $\Delta$ q t, {IsRed $\Delta$ q t} + {~ IsRed $\Delta$ q t}.
 \end{lstlisting}

 On regarde alors le résultat de~\lstinline!IsRed_dec $\Delta$ q (r @ $\sigma$)!:
 \begin{itemize}
 \item Si la réponse est négative, alors il n'est pas nécessaire d'aller plus loin, la propriété est violée.
   En fait, la fonction \lstinline!all_reduce_dec! ne renvoie pas simplement~\lstinline!ok! ou~\lstinline!ko!,
   mais renvoie un contre-exemple. Donc la fonction va renvoyer~\lstinline!ko $\sigma$!. 
 \item Lorsque la réponse est positive, on a alors une preuve~\lstinline!IsRed $\Delta$ q (r @ $\sigma$)!.
   En utilisant l'hypothèse d'induction sur le reste de la liste $l$, on regarde le résultat. 
   En cas d'échec, on se contente de faire suivre le contre-exemple rencontré dans la liste, puisque si 
   il existe $\sigma' \in l$ qui viole le prédicat~\lstinline!IsRed!, on a bien $\sigma' \in \sigma::l$
   qui est un contre-exemple pour~\lstinline!all_reduce_dec $\Delta$ q r ($\sigma::l$)!.
   Sinon, on sait que toutes les substitutions de $l$ respectent la propriété. Comme c'est aussi le cas
   pour $\sigma$, on en déduit que pour toutes les substitutions \lstinline!s! de la liste  $\sigma::l$, 
   on a bien~\lstinline!IsRed $\Delta$ q (r @ s)!.
 \end{itemize}
\end{proof}

Il est important de remarquer que la signature de la fonction~\lstinline!all_reduce_dec! est basée
sur le type de données \lstinline!boolopt! qui correspond à construire un type de données
isomorphe au booléen paramétré par un type. En cas d'échec, 
on aimerait récupérer la substitution la substitution pour laquelle la clôture n'est pas vérifiée, celle-ci
constituant simplement un contre-exemple. Cela permet de donner un caractère plus informatif à la procédure
de vérification, en précisant la nature de l'échec: la règle de réécriture $l \rw r$, l'état $q$, et la substitution $\sigma$. 
Pour cela, on a donc introduit le type suivant:
\begin{lstlisting}
Inductive boolopt (A: Type): Type :=
  | ok : sumdec A
  | ko : A -> sumdec A.
\end{lstlisting}
Avec une expertise un peu plus avancée, on peut montrer que  le type~\lstinline!boolopt! est
équivalent au type~\lstinline!option!. Si d'un point de vue logique les types sont effectivement équivalents,
il en est tout autrement d'un point de vue syntaxique. En effet, il est possible de les utiliser tous les deux
avec la construction~\lstinline!if $\dots$ then $\dots$ else $\dots$! qui se réduit pour tout type isomorphe
à~\lstinline!bool! mais qui suppose toujours que la branche gauche correspond au cas "positif", ce qui n'est pas le cas
dans l'utilisation du type~\lstinline!option! lorsqu'on l'utilise pour renvoyer un contre-exemple: dans ce cas, on suppose
que le constructeur \lstinline!Some: A -> option A! correspond à la branche négative puisqu'il est sensé contenir un
contre-exemple. Or la définition du type option introduit dans l'ordre le constructeur~\lstinline!Some! suivi 
de~\lstinline!None! ce qui correspond implicitement à l'ordre branche "positive" puis branche "négative".
C'est pour cette raison que l'on introduit le type~\lstinline!boolopt! pour être cohérent par rapport à cette notion
de branchement. Privilégier le type~\lstinline!boolopt! par rapport au type~\lstinline!option! évite d'avoir à renommer
systématiquement par la négative toutes les fonctions dans le cas où l'on veut renvoyer un contre-exemple, 
ce qui peut-être une source d'incompréhension et d'erreur. 

En combinant les fonctions~\lstinline!matching! et~\lstinline!all_red!, 
on obtient l'algorithme pour vérifier toutes les paires critiques 
trouvées à l'état $q$ avec la règle $l \rightarrow r$:

\begin{lstlisting}
Definition closure_at_state $\Delta$ q l r := 
   catch
          all_red $\Delta$ q r (matching $\Delta$ q l)
   (fun $\sigma$ => Closure_ko q l r $\sigma$).

Theorem closure_at_state_spec : 
   forall $\Delta$ q l r, closure_at_state $\Delta$ q l r = ok -> 
      (forall $\sigma$, IsRed $\Delta$ q (l @ $\sigma$) -> IsRed $\Delta$ q (r @ $\sigma$)).
\end{lstlisting}

Pour une règle donnée $l \rightarrow r$ et un état donné $q$, cette fonction répond
\lstinline!ok! si pour toute substitution $\sigma: \X \mapsto \Q$ telle que $l
\sigma \rw^*_\Delta q$ alors on sait que $r\sigma \rw^*_{\Delta} q$. Sinon, on peut 
voir l'emploi de la fonction \lstinline!catch! qui en cas d'échec de la vérification
par la fonction~\lstinline!all_dec! capture le contre-exemple (une substitution)
pour construire et retourner l'ensemble des éléments qui forment le contexte
à l'origine l'échec: la valeur de retour \lstinline!ko(Closure_ko q l r $\sigma$)!
nous indique que la clôture a échoué pour $r\sigma \not\rw^*_\Delta q$ alors que l'on a bien $l\sigma \rw^*_{\Delta} q$,
la paire critique semble ne pas être résolue. La preuve du théorème~\lstinline!closure_at_state_spec!
est une conséquence immédiate du théorème~\lstinline!matching_spec! et du théorème~\lstinline!all_reduce_dec_spec!.
  

\paragraph{Des configurations aux termes.}

Pour l'instant la vérification est faite seulement pour les substitutions
calculées par le filtrage qui sont de la forme $\sigma:\X \mapsto \Q$ (\lstinline!Qsubst $\sigma$!).
Symboliquement, la réécriture de la configuration $l\sigma$ en $r\sigma$ à l'état $q$ consiste en la réécriture
de n'importe quel terme clos de la forme $l\sigma' \rw_\Delta^* q$ en un terme clos $r\sigma' \rw_\Delta^* q$.
On va donc transférer cette propriété de clôture aux substitutions $\sigma':\X \mapsto \TF$. On pourra alors en déduire
que si la vérification par \lstinline!closure_at_state $\Delta$ $q$ $l$ $r$! est un succès à l'état $q$ pour la règle
$l \rw r$ alors pour tout terme clos $u \in \Lang(\Delta, q)$ si $u$ se réécrit en $v$ à la racine par $l\rw r$  alors $v \in \Lang(\Delta, q)$.

En fait, on doit être capable de passer des substitutions de $\X \rw \Q$ aux substitutions de $\X \to \TF$, mais on a aussi
besoin de montrer que l'inverse est possible. Pour cela on a besoin des deux théorèmes suivants:


\begin{lstlisting}
Theorem IsRed_subst: forall $\Delta$ c $\sigma$ q, TFX c -> 

   (forall $x_i$, $x_i$ \in Fv c -> exists $q_i$, s x = Some (State $q_i$)) -> 
      IsRed $\Delta$ q (c @ $\sigma$) ->
  
   forall $\sigma'$, 
      (forall $x_i$ $q_i$, $x_i$ \in Fv c -> $\sigma$ $x_i$ = Some (State $q_i$) -> IsRec $\Delta$ $q_i$ ((Var $x_i$) @ $\sigma'$)) ->
      IsRec $\Delta$ q (c @ $\sigma'$).
\end{lstlisting}


Ce lemme permet de composer avec les prédicats de reconnaissabilité: on considère tous les termes de la forme $c\sigma' \in \TF$ avec $\sigma' :\X \mapsto \TF$ 
qui se réduisent $c\sigma' \rw^*_\Delta c\sigma$ avec $\sigma :\X \mapsto \Q$. On peut alors en déduire que si $c\sigma \rw^*_\Delta q$
alors on a $c\sigma' \rw^*_\Delta q$. La "petite" difficulté dans ce théorème est la formalisation $c\sigma' \rw^*_\Delta c\sigma$ qui ne 
s'exprime pas immédiatement avec les prédicats~\lstinline!IsRec! et~\lstinline!IsRed!. On est obligé d'exprimer la propriété plus localement,
c'est à dire d'exprimer comment est reconnu le sous-terme $\sigma'(x_i)$ associé à chaque variable $x_i$ de $c$.
Ce qui revient à décomposer la relation $c\sigma' \rw^*_\Delta c\sigma$ en une suite de réductions de la forme $\sigma'(x_i) \rw^*_\Delta \sigma(x_i)$.

% Si le membre gauche $c$ d'une règle de réécriture filtre une configuration $u\sigma \in \TFQ$ telle que $u\sigma \rw^*_{\Delta} q$
% alors tous les termes clos $u\sigma' \in \TF$, que l'automate peut réécrire dans la configuration $u\sigma' \rw_\Delta^* u\sigma$
% sont bien des termes réductibles dans l'état $q$, soit $u\sigma' \rw^*_{\Delta} q$.

\begin{lstlisting}
Theorem IsRec_subst: forall c s q D, IsRec D q (c @ s) ->
   exists s', 
         (forall x, ~ x \in Fv c -> s' x = None)
      /\
         (forall x, x \in Fv c -> exists q', s' x = Some (State q') /\ IsRec D q' ((Var x) @ s))
      /\
         IsRed D q (c @ s').
\end{lstlisting}


Dualement, le théorème montre que tout terme clos de la forme $c\sigma \in \TF$ réduit par l'automate 
en l'état $q$ par est forcément réécrit en une configuration intermédiaire de la forme $c\sigma'$ avec
$\sigma': \X \mapsto \Q$. Ce qui correspond à :

\[\forall \sigma,\ \exists \sigma' \text{ t.q. } c\sigma \rw^*_\Delta c\sigma' \text{ et } c\sigma' \rw^*_\Delta q \]
  
Le théorème~\lstinline!IsRec_subst! utilise la même astuce pour formaliser la réduction de $c\sigma$ en $c\sigma'$.
Pour ces deux théorèmes la preuve se fait simplement par induction sur le terme $c \in \TFX$.


En utilisant ces deux théorèmes, on peut reformuler le théorème~\lstinline!closure_at_state_spec!
en transférant la propriété sur les termes clos, c'est à dire la relation de réécriture à la racine(\lstinline!Trew u v!)
entre deux termes clos:

\begin{lstlisting}
Lemma closure_at_state_spec : 
   forall $\Delta$ lr, 
      forall l r Hsub Hl Hr, lr = LRule l r Hsub Hl Hr ->
         closure_at_state $\Delta$ q l r = ok ->
            forall u v, IsRec D q u -> TRew lr u v -> IsRec D q v.
\end{lstlisting}


\paragraph{Lorsque le membre droit est une variable...}
\label{par:si-le-membre}
Dans le cas où le membre droit de la règle est une variable c'est à dire que l'on a une règle de la forme 
$f(x) \rw x$. Supposons qu'il existe une substitution $\sigma : \X \rw \Q$ telle l'on ait la relation 
$f(q') \rw_\Delta^* q$ à l'état $q \not= q'$. Comme on considère des automates d'arbres sans $\varepsilon$-transition,
on ne peut pas avoir $q \rw^*_\Delta q'$, la réflexivité de la relation ne pouvant s'appliquer à $q$ et $q'$.
Comme le prédicat~\lstinline!IsRed $\Delta$ q (Var x @ $\sigma$)! est faux, la procédure~\lstinline!closure_at_state $\Delta$ q l r!
va échouer et rendre la substitution $\sigma$ comme contre-exemple. 
Dans ce type de cas, la complétion (sans $\varepsilon$-transition) duplique chaque transition de la forme $f(q_1,\dots,q_n) \rw q'$ en une $f(q_1,\dots,q_n) \rw q$,
ce qui est évidemment suffisant pour avoir la relation l'inclusion $\Lang(\Delta, q') \subseteq \Lang(\Delta, q)$.
On ajoute donc la fonction~\lstinline!closure_at_q'! qui vérifie ce critère syntaxique:

\begin{lstlisting}

Definition closure_at_q' :
        Delta.t -> state -> term -> ident -> boolopt closure_ko :=
   (*   ... Body ... *)
.

Lemma closure_at_q'_spec : forall $\Delta$ q l x,
      linear l -> In x (Fv l) ->
         closure_at_q' D q l x = ok ->
             forall s, IsRec D q (l @ s) -> IsRec D q (Var x @ s).
\end{lstlisting}


Il ne reste plus qu'à vérifier la propriété de clôture pour toutes les règles de réécriture 
à tous les états de l'automate. C'est le rôle jouée par la fonction~\lstinline!closure!,
qui itère sur l'ensemble des couples règle-état de \lstinline!$\R$ $\times$ (Delta.dom $\Delta$)!.

\begin{lstlisting}
(* L'ensemble [$\R$: list lrule] est lineaire gauche *)

Lemma closure_spec_0 : 
   forall $\Delta$ $\R$, closure $\Delta$ $\R$ = ok ->
      forall q lr, lr \in $\R$ ->
         forall u v, IsRec D q u -> TRew lr u v -> IsRec D q v.
\end{lstlisting}
Enfin il est facile d'en déduire que la relation est suffisante 
pour en déduire la clôture de la relation de réécriture
par contexte avec le prédicat~\lstinline!LRew! (la formulation inductive de la réécriture pour toute position $p$)
puis par réflexivité et transitivité~\lstinline!ReachableL!. Ce qui permet de conclure
à la reconnaissabilité de tout terme atteignable: 

\begin{lstlisting}
Theorem closure_spec:
   forall $\Delta$ $\R$, closure $\Delta$ $\R$ = ok ->
         forall q u v, IsRec $\Delta$ q u -> Reachable $\R$ u v -> IsRec $\Delta$ q v.
\end{lstlisting}

%\section{Performance, Exemples}
\subsection{Extraction et benchmarks}
\label{sec:benchmarks}

A partir de la spécification formelle en \coq, il est possible d'extraire une implémentation
du $\kw{checker}$ en \ocaml. 
Le code extrait est interfacé avec un analyseur syntaxique qui accepte les fichiers générés
par \timbuk. Si cette opération se déroule globalement bien, il subsiste quelques subtilités
dues au mécanisme d'extraction dont il faut tenir compte pour que le $\kw{checker}$ en \ocaml\ soit conforme 
à sa spécification. En effet, pour produire des termes \ocaml\ à partir des termes \coq, l'extraction
nettoie les termes \coq\ de tout ce qui n'a pas trait au calcul, en particulier tout ce qui est du type
\lstinline!Prop! soit toutes les propriétés. Par exemple, si l'on reprend le cas de la définition 
d'une règle de réécriture linéaire à gauche, on a :
\begin{lstlisting}
Inductive l_rule: Set :=
| Lrule (l r: term)(Hsub: subseteq (Fv r) (Fv l)) ... (Hl: linear l): l_rule.
\end{lstlisting}
L'extraction produit le type \ocaml\ suivant:
\switchlstcaml
\begin{lstlisting}
type l_rule = Lrule of term * term
\end{lstlisting}
L'extraction supprime dans le type \ocaml, la propriété \lstinline!H! qui permet de s'assurer que la règle est bien
formée et \lstinline!Hl! qui s'assure que le membre gauche est bien linéaire. L'absence de ces propriétés dans le type 
\ocaml\ rend le type de données plus permissif. Cela peut poser des problèmes pour les fonctions extraites puisque
leur comportement peut être non spécifié dans le cas où les propriétés \lstinline!H! et \lstinline!Hl! ne sont pas
assurées. Or, le type \ocaml\ nous permet de construire des règles de réécriture qui en fait ne sont ni bien formées ni
linéaires à gauche: $f(x,x) \rw g(y)$. Si l'on ne tient pas compte de cette remarque, on peut par construire un $\kw{checker}$
\ocaml\ qui n'est plus conforme à la spécification \coq. La solution la plus simple pour contourner le problème consiste 
à écrire une fonction qui vérifie que la règle \lstinline!l_rule! est bien formée et linéaire à gauche en \ocaml.
Cependant, on peut toujours se demander si cette fonction est correcte vis-à-vis de la spécification de propriétés
~\lstinline!H! et~\lstinline!Hl! qui est donnée en \coq\dots\ Pour cette raison, on préférera définir une fonction qui 
construira une règle conforme à la spécification à partir d'un simple couple de termes \lstinline!(l, r)!.
\switchlstcoq
\begin{lstlisting}
Definition build_l_rule: term -> term -> option l_rule :=
        (* ... Body ... *)
\end{lstlisting}
On peut alors légitimement se poser la même question pour l'extraction du type \lstinline!term!. 
Mais dans ce cas, la définition \coq\ ne contient pas de propriété supplémentaire, ce qui implique
que la totalité de la définition est extraite vers \ocaml.
Ainsi, on en déduit un critère raisonnablement simple pour interfacer correctement en \ocaml, du code extrait
à du code externe: les types de données mis en commun, doivent être le résultat d'une extraction totale.


Le tableau~\ref{tab:benchs} résume quelques tests
d'efficacité. Pour chaque test, le tableau donne la taille des automates d'arbres 
($\A^0$ initial et $\aaexeq^*$ pour le complété) en nombre de transitions et nombre d'états.
Pour chaque système de règles de réécriture $\R$, on donne le nombre de règles.
La colonne 'CS' indique le nombre d'étapes de complétion nécessaires pour compléter
$\A^0$ en $\aaexeq^*$ et la colonne 'CT' donne le temps de calcul pour effectuer ces étapes.
Enfin, on donne le temps (colonne 'CKT') et la mémoire (colonne 'CKM') consommés par le $\kw{checker}$
pour vérifier que l'automate $\aaexeq^*$ est bien un point fixe.
Le point important à relever ici est que le temps de complétion est souvent très long 
(parfois plus d'une journée), mais que globalement le temps nécessaire à la vérification
est très rapide puisque se limitant à une poignée de secondes. Ces résultats plus que
satisfaisants soulignent notablement l'intérêt d'une vérification systématique des automates
complétés.
Les quatre tests sont issus de programmes Java compilés par \textsc{Copster}~\cite{Copster}
vers un système de réécriture en utilisant la technique détaillée dans~\cite{BoichutGJL-RTA07}.
Pour tous les exemples excepté \texttt{List2.java}, l'automate complété a été produit par \timbuk.
Concernant l'exemple \texttt{List2.java}, le calcul a été complété en utilisant 
une implémentation en Tom~\cite{TOM} réalisée par Yohan Boichut et Emilie Balland~\cite{BallandBGM-AMAST08}.
Cette version offre des performances accrues en temps d'exécution par rapport à \timbuk. Celle-ci repose
sur une représentation interne qui n'est plus basée sur des automates d'arbres, mais qui transforme 
à posteriori le résultat en un automate clos par réécriture tout en respectant les propriétés de l'algorithme
initial. Ce qui implique que ces résultats lorsqu'ils sont corrects sont acceptés par le $\kw{checker}$.
Les exemples~\texttt{List1.java} et~\texttt{List2.java} correspondent à deux programmes Java 
manipulant des listes chaînées d'entiers. Dans un cas ils sont sensés être tous positifs,
alors que dans l'autre ils sont négatifs. Les entiers sont lus sur l'entrée standard définie
comme une séquence infinie d'entiers. L'ensemble d'équations utilisé permet d'abstraire les entiers 
en deux classes d'équivalence les entiers positifs ou nuls, et les entiers strictement négatifs. On abstrait aussi
la représentation mémoire des listes, en capturant dans la même classe d'équivalence toutes les listes quelque 
soit leur longueur.
L'exemple~\texttt{Ex\_poly.java} est l'exemple utilisé dans~\cite{BoichutGJL-RTA07} 
pour illustrer comment définir une abstraction pour réaliser 
une analyse $k$-CFA sur des programmes Java grâce à la complétion d'automates d'arbres.
Enfin l'exemple~\texttt{Bad\_Fixp.java} est le même problème que~\texttt{Ex\_poly.java} dont l'automate
complété $\aaexeq^*$ a été volontairement corrompu: l'automate non clos est alors rejeté
par le $\kw{checker}$.

\begin{figure}[ht!]
  \centering
  \begin{tabular}{l|c|c|c|c|c|c|c|c}
    Name & $\aaexeq^0$ & $\aaexeq^*$ & $\R$ & CS & CT & CKT & CKM \\ \hline
    
    \texttt{ List1.java} & 118/82 & 422/219 & 228 & 180 & $\approx$ 3 j. & 0,9s & 2,3 Mo \\ \hline
    % Yoh
    \texttt{ List2.java} & 1/1 & 954/364 & 308 & 473 & 1h30 & 2,2s & 3,1 Mo \\ \hline
    % Res_
    \texttt{ Ex\_poly.java} & 88/45 & 951/352 & 264 & 161 & $\approx$ 1 j. & 2,5s & 3,3 Mo \\ \hline
    % Res_ modifié
    \texttt{ Bad\_Fixp} & 88/45 & 949/352 & 264 & 161 & $\approx$ 1 j. & 1,6s & 3,2 Mo \\ \hline
    % NSPK & 26tr. 12 states & 243 tr. 12st. & 13 & 6 & 8'/21.4s & 5 Mo \\ \hline
  \end{tabular}
  \caption{Vérification de quelques résultats}
  \label{tab:benchs}
\end{figure}


\section{Conclusion}
\label{sec:conclusion}
Ce chapitre résume l'approche adoptée pour définir un $\kw{checker}$ certifié en \coq\ 
pour la complétion d'automates d'arbres. Le point important ici est le transfert
de la certification non pas sur une implémentation spécifique de l'algorithme
mais plutôt sur le résultat. De ce fait, le $\kw{checker}$ reste valide même si l'implémentation
de la complétion change, ou est améliorée. En fait, il existe à ce jour d'autres implémentations
que celle proposée par \timbuk\ pour la complétion d'automates d'arbres: on peut citer parmi les versions
récentes \textsf{Tomed Timbuk}~\cite{BallandBGM-AMAST08} implémenté en Tom~\cite{TOM} un langage de manipulation
d'arbres, ainsi qu'une implémentation en~\textsf{SICStus Prolog} reposant sur une spécification des systèmes
de réécriture par des clauses de Horn~\cite{GallagherR-LPAR08}. Le $\kw{checker}$ est utilisé par les auteurs
de ces différents outils afin de vérifier la correction des résultats obtenus. 
Le deuxième aspect fondamental est qu'un $\kw{checker}$ en \ocaml\ est extrait directement de la preuve
de correction de la spécification \coq\ grâce au mécanisme d'extraction de \coq.
A la vue des performances, on peut conclure qu'il est raisonnable d'exécuter le $\kw{checker}$ sur chaque résultat pour en vérifier
correction, puisque le coût de ressources pour la vérification reste négligeable devant le coût nécessaire
à la construction du point-fixe.
Le troisième point important est l'attention portée lors de la formalisation pour obtenir un
outil efficace. En particulier, le recours à un algorithme d'inclusion
particulier permet d'éviter le facteur exponentiel de l'algorithme d'inclusion standard et de justifier en 
grande partie des performances mentionnées ci-dessus. Enfin, l'approche se base principalement sur
des arbres dès lors que des structures de données nécessitent des accès aléatoires. Bien que leur utilisation puisse rendre les preuves plus complexes, on peut établir
une projection des arbres dans les listes. L'astuce permet ainsi de transférer les propriétés 
des arbres vers leur projection dans les listes ce qui simplifie les preuves, puisqu'il est généralement
plus facile de raisonner sur les listes que sur les arbres.
Enfin, la principale évolution de la complétion est l'utilisation des $\varepsilon$-transitions 
pour résoudre les paires critiques. Cette approche qui est utilisée comme base dans les
techniques d'analyses des chapitres précédents peut être couverte en nettoyant l'automate
point-fixe de ses $\varepsilon$-transitions. %, mais cela représente un coût supplémentaire.
D'autre part, le théorème~\ref{sec:completness} sur la complétude de l'algorithme
d'inclusion reste valide pour les automates avec des $\varepsilon$-transitions. La seule partie
qu'il est nécessaire d'adapter, est la vérification de la clôture par le système de réécriture,
ce qui peut être réalisé très facilement.
% à partir du moment où le formalisme des automates 
% est étendu aux automates à $\varepsilon$-transitions.
Ensuite, on peut se poser la question d'étendre la clôture pour les systèmes de règles
de réécriture non linéaires à gauche qui sont requis 
pour formaliser certains protocoles cryptographiques. Le problème est un peu plus délicat,
et pose déjà un challenge du point de vue de la complétion elle-même. La plupart du temps
la prise en compte d'un système de règles non linéaires à gauche effondre les performances.
Il existe alors plusieurs astuces qui peuvent être mises en oeuvre, mais
demandent pour chacune une méthode de vérification particulière.

Dans les pistes à explorer, on peut imaginer exécuter le
$\kw{checker}$ directement en \coq, et grâce à la réflexivité en
déduire des preuves de non-atteignabilité pour systèmes de transitions
en \coq. La preuve est alors d'une part constituée du point-fixe
calculé au préalable par un outil externe et d'autre part par la
vérification par le $\kw{checker}$ directement en \coq.
%dont la conclusion de correction permet facilement de construire des preuves de non-atteignabilité. 
Enfin, à la vue des performances du
$\kw{checker}$, il est envisageable d'imaginer l'adaptation des
analyses basées sur la complétion d'automates d'arbres vers un modèle
à la Proof-Carrying Code.  Dans un tel modèle, le $\kw{checker}$ est
embarqué dans un environnement où l'on ne fait pas confiance aux
programmes étrangers au système. L'installation d'un programme
nécessite alors une preuve de sa sûreté (programme fiable,
non-malicieux, préservant la politique de confidentialité\dots), et
cette preuve doit être vérifiable facilement. Dans ce cas, l'automate
produit par la complétion peut constituer cette preuve, dont la
correction est facile à vérifier.

%\putbib

%%% Local Variables: 
%%% coding: utf-8
%%% mode: latex
%%% TeX-master: "../main"
%%% TeX-PDF-mode: t
%%% End: 
