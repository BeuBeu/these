\section{Proof about semantics}
\setcounter{savetheorem}{\thetheorem}
\setcounter{theorem}{4}

\begin{theorem}
  \[\forall t\in\TFQ,\; q \in \Q,\; t \xrw{\alpha}_\A q \equ t \rw_\A q \]
\end{theorem}

\begin{proof}
  The proof is easily done by induction by arguing that it is enough to forget the formulas
  manipulated by the definition~\ref{def:xrw_alpha} to have the equivalent step with $\rw_\A$.
\end{proof}
\setcounter{theorem}{\thesavetheorem}

\section{Proofs about $\compl$}


\setcounter{savetheorem}{\thetheorem}

\begin{property}
  $\aaexeq^0$ is well-defined.
\end{property}

\begin{proof}
   $\aaexeq^0 = \la \F, \Q^0, \Q_f, \Delta^0\ra$ fits the definition \ref{def:well-defined}, only if the two items
   of the definition \ref{def:well-defined} are verified.

   We know that $\aaex^0$ has no $\varepsilon$-transitions. All
   terms are recognized using transitions of $\Delta^0$. It means that for all state $q$
   the set of terms is defined as $\{ t \in \TF | t \rwne{\Delta^0} q\}$ which is
   equal to $Rep(q)$, the set of representatives for $q$.
   We also remark that for all term $t$, $t \rwne{\Delta_0} q$ implies $t \xrw{\top}_{\aaexeq^0} q$:
   the second and third point of the definition \ref{def:xrw_alpha}, are not used, since $\Drw^0$ and $\Deq^0$ are empty.
   The first item of definition \ref{def:well-defined} is ensured: for all state $q$, and all term $t \xrw{\top} q$,
   we have $t \in Rep(q)$, and $t \rwR^* t$ by reflexivity.

   The second item of \ref{def:well-defined} holds, since $\Drw^0$ is empty. 
\end{proof}




\begin{lemma}[Solving one critical pair preserves well-definition]
\label{lemma:C-well-defined}
  Let $\A$ and $\A'$ be two $\RE-$automaton such that 
  $\A'$ is obtained from $\A$ by solving 
  a critical pair $\la r\sigma,\alpha,q \ra$ of $\A$. 
  If $\A$ is well-defined then so is $\A'$. 
\end{lemma}
\begin{proof}
  Assume that $\A=\la \F, \Q, \Q_f, \Delta\cup \Drw\cup \Deq\ra$ and
  $\A'=\la \F, \Q', \Q_f, \Delta'\cup \Drw'\cup \Deq' \ra$.  According
  to Definition \ref{def:resolution_cp}, $\Delta'=\Delta \cup
  \norm(r\sigma, \Delta\setminus \Delta^0)$, $\Drw'= \{
  q'\xrw{\alpha} q\}\cup \Drw$ and $\Deq'=\Deq$.
  Following Definition \ref{def:well-defined}, we first show in (\ref{one}) that 
  for all state $q''$ of $\A'$, and all term $v$ such that
  $v \xrw{\top}_{\A'} q''$, there exists $u$ a term representative
  of $q''$ such that $u \rw^*_\R v$. Then, in (\ref{two}) we show that
  if $q_1 \xrw{\phi} q_2$ is a transition of $\Drw'$, then there exist terms
    $s,t\in \TF$ such that $s\rwtag{\phi}_{\A'} q_1$, $t\rwtag{\top}_{\A'} q_2$
    and $t \rw_\R s$.

\medskip
  
\begin{enumerate}
\item \label{one} 

% Let $q_{\A'}$ be a state of $\Q'$.  Since $\A$ is
%   sintactically included in $\A'$, one can deduce that the property
%   still holds for any term $v\in\TF$ sucht $v \xrw{\top}_{\A}
%   q_{\A'}$.  Suppose now a term $v\in\TF$ such that
%   $v\in\Lang(\A',q_{\A'})$, $v\notin\Lang(\A,q_{\A'})$ and
%   $v\rwtag{\top}q_{\A'}$.  Then, there are two possibilities: either
%   $q_{\A'}$ is a new state, i.e. $q_{\A'}\notin \Q$, or $q_{\A'}$ is a
%   state of $\A$, $\alpha=\top$ and the transition $q'\rwtag{\top} q$
%   added to $\Drw$ is used at least once during the reduction of $v$ on
%   $q_{\A'}$.  Let us study the two possibilies.


\newcommand{\xrwa}{\xrw{\top}_\A}
\newcommand{\xrwap}{\xrw{\top}_{\A'}}

We prove the property by induction on the height of $t$. Let us assume that for all term $t'$ of height
lesser than the height of $t$ and for all $q\in \Q_{\A'}$, we have $t' \xrwap q
\spf \exists u\in Rep(q): u \rwR^* t'$. Now let us prove that this is true for
$t$. We prove it by case on $q\in\Q_\A$ and $t\xrwa q$:

\begin{itemize}
\item If $q\in\Q_\A$ and $t\xrwa q$ then since $\A$ is well defined, we get the
  representative $u\in Rep(q)$ such that $u\rwR^* t$ from well-definition of $\A$.
\item If $q\in\Q_\A$, $t \not\xrwa q$ and $t \xrwap q$. We prove the property by
induction on the height of $t$. Now let us consider the term $t$. Since $t$ is recognized
in $\A'$ and not in $\A$, this means that the run $t' \xrwap q$ needs the
transitions added by the resolution of a critical pair. Hence there exists a
rewrite rule $l \rw r$, a substitution $\sigma:\X \mapsto \Q_\A$, a formula
$\alpha$ and a state $q_c$ such that $l\sigma \xrw{\alpha}_\A q_c$ and $\la
r\sigma, \alpha, q_c\ra$ is the critical pair. Moreover, the resolution of this
critical pair creates $\Delta_{\A'}=\norm(r\sigma,\Delta_\A\setminus \Delta_0)$
and $\Drw^{\A'}=\Drw^{\A} \cup \{q'_c \rw q_c\}$ such that $r\sigma
\rw^!_{\Delta'\setminus\Delta_0} q'_c$. Recall that $t' \xrwap q$ needs transitions not occurring
in $\A$. However, all the {\em new} transitions produced by
$\norm(r\sigma,\Delta_\A\setminus \Delta_0)$ necessarily range on {\em new}
states, i.e. states not occurring in $\Q_\A$. As a result, those transitions
cannot be used to get $t' \xrwap q$ with $q\in\Q_\A$. This means that the run $t
\xrwap q$ uses at least once $q'_c \rw q_c$ and $\alpha=\top$ since the whole
run is labelled by $\top$. To sum up, we know that there exists a ground context
$C[\,]$ such that $t=C[t'] \xrwap C[q'_c] \xrwap C[q_c] \xrwa q$. Note that if
$q'_c \rw q_c$ the same reasonning can be applied. We start to reason on the 
occurrence of $q'_c \rw q_c$ that is the closest to $q$. Now, our
objective is to show that there exists $u\in Rep(q_c')$ such that $u \rwR^*
t'$. If $t'\xrwa q'_c$ then since $\A$ is well defined we directly have the
result using definition~\ref{def:well-defined}. Otherwise this means that $q'_c$
is new for $\A$ (i.e. $q'_c\not\in\Q_\A$) and has been added by the resolution
of the critical pair, i.e. $r\sigma \rw^!_{\Delta'} q'_c$. Because of
Theorem~\ref{lem:norm_determinism}, we get that there exists a substitution
$\sigma':\X \mapsto \TF$ such that $t'=r\sigma'$. Using the same theorem, from
$t'=r\sigma' \xrwap q'_c$ and $r\sigma \xrwap q'_c$, we get that for all
variable $x$ of $r$: $\sigma'(x) \xrwap \sigma(x)$. Note that
$\sigma(x)\in\Q_\A$ and that $\sigma'(x)$ are necessarily terms of height lesser
to the height of $t$. Using the induction hypothesis, we get that for all state
$\sigma(x)$ there exists a representative $u_x$ such that $u_x \rwR^*
\sigma'(x)$. Let $\sigma_{Rep}$ be the substitution mapping every variable $x$
to $u_x$. We have $r\sigma_{Rep} \in Rep(q'_c)$. Moreover, $r\sigma_{Rep} \rwR^*
r\sigma'=t'$. Now, our objective is to show that $l\sigma_{Rep} \rwR
r\sigma_{Rep}$. This is not straightforward since $\var(l)\supseteq \var(r)$.
However, it is possible to extend $\sigma_{Rep}$ into $\sigma'_{Rep}$ where every
variable $y$ of $\var(l)$ not occurring in $\sigma_{Rep}$ is mapped to a
representative of $\sigma(y)$. Hence, $l \sigma'_{Rep} \rwR r\sigma'_{Rep}
\rwR^* t'$. From the critical pair we know that $l\sigma \xrw{\alpha}_\A q_c$ and
we found that $\alpha=\top$. Hence $l\sigma'_{Rep} \xrwa q_c$. Since $\A$ is
well-defined, we get that there is a representative $v \in Rep(q_c)$ such that
$v \rwR^* l\sigma'_{Rep}$. By transitivity of $\rwR$, we get that $v \rwR^* t'$.
Above, we found that $t=C[t'] \xrwap C[q'_c] \xrwap C[q_c] \xrwa q$. From this
and $v\in Rep(q_c)$, we get that $C[v] \xrwa q$. Since $\A$ is well defined, we
know that there exists a representative $w\in Rep(q)$ such that $w\rwR^* C[v]$.
To conclude, we found $w\in Rep(q)$ and $w \rwR^* C[v] \rw C[t']=t$.

\item If $q \not \in \Q_A$ ($q\in\Q'_\A\setminus \Q_\A$), $t \not\xrwa q$ and $t
  \xrwap q$. Since $q\in \Q'_\A\setminus \Q_\A$, we know that $q$ has been added
  by the resolution of a critical pair. As above, we can deduce that there
  exists a rewrite rule $l \rw r$, a substitution $\sigma:\X \mapsto \Q_\A$, a
  formula $\alpha$ and a state $q_c$ such that $l\sigma \xrw{\alpha}_\A q_c$ and
  $\la r\sigma, \alpha, q_c\ra$ is the critical pair. Moreover, the resolution
  of this critical pair creates $\Delta_{\A'}=\norm(r\sigma,\Delta_\A\setminus
  \Delta_0)$ and $\Drw^{\A'}=\Drw^{\A} \cup \{q'_c \rw q_c\}$ such that $r\sigma
  \rw^!_{\Delta'\setminus\Delta_0} q'_c$. Since $q$ is a new state of $\A'$, $q$
  has been necessarily used for the normalization of a subterm of
  $r\sigma$. More precisely, we know that there exists a term $s\in\TFX$ and a
  context $C[\,]$ (possibly empty) such that $r\sigma=C[s]$, $C[s]\sigma \rw_{\Delta'}^* q'_c$
  and $s\sigma \rw^*_{\Delta'} q$. Similarly, we know that there exists a
  substitution $\sigma':\X \mapsto \TF$ such that $s\sigma'=t$.
  We get that for every variable $x$ of $r$: $\sigma'(x) \xrwap \sigma(x)$.
  Note that $\sigma(x)\in\Q_\A$ and that $\sigma'(x)$ are necessarily terms of
  height lesser to the height of $t$. Using the induction hypothesis, we get
  that for every state $\sigma(x)$ there exists a representative $u_x$ such that
  $u_x \rwR^* \sigma'(x)$. Let $\sigma_{Rep}$ be the substitution mapping every
  variable $x$ to $u_x$. We have $s\sigma_{Rep} \in Rep(q)$ and $s\sigma_{Rep}
  \rwR^* s\sigma'=t$.
\end{itemize}


    
\item \label{two} Easily, for any transitions $q_1\stackrel{\phi}{\f}
  q_2\in \Drw$, the property still holds.  Let us focus now on the
  transition $q'\stackrel{\alpha}{\f}q$ resulting from the resolving
  of the critical pair $\la r\sigma,\alpha,q\ra$.  By definition, the
  critical pair $\la r\sigma,\alpha,q\ra$ results from the application
  of the matching algorithm of Definition
  \ref{def:matching-algorithm}.  So there exists a rule $l\f r\in \R$
  such that $(\alpha,\sigma)\in S$ with $l\match q \vdash_\A S
  $. Moreover, since the critical pair has to be solved:
  $l\sigma\stackrel{\alpha}{\f}q$ and there is no formula $\alpha'$
  such that $r\sigma \xrw{\alpha'}_{\A} q$.  Since $\R$ is
  left-linear, for each variable $x\in\var(l)$, one can define the
  substitution $\sigma':\X\f \TF$ as follows: Assuming $q_s$ being the
  state of $\A$ such that $\sigma(x)=q_s$, let $\sigma'(x)=Rep(q_s)$.
  By definition of $Rep$, $Rep(q_s)\stackrel{\top}{\f}q_s$.  So, there
  exists a derivation such that $l\sigma'\stackrel{\top}{\f}l\sigma$
  and $l\sigma\stackrel{\alpha}{\f}q$.  One can deduce that
  $r\sigma'\stackrel{\top}{\f}{r\sigma}$. According to Lemma
  \ref{lem:norm_determinism}, one can deduce that there exists a
  unique $q'$ such that
  $r\sigma\f^*_{\norm(r\sigma,\Delta\setminus\Delta^0)}q'$. If
  $\norm(r\sigma,\Delta\setminus\Delta^0)\not=\emptyset$ then each
  transition composing it is of the form $f(q'_1,\ldots,q'_n)\f
  q'_{n+1}$. Consequently, $r\sigma\stackrel{\top}{\f} q'$.
  Considering the transition $q'\stackrel{\alpha}{\f}{q}$, one has
  $r\sigma'\stackrel{\top}{\f}r\sigma\stackrel{\top}{\f}q'\stackrel{\alpha}{\f}{q}
  $.  Finally, assuming $s=l\sigma'$ and $t=r\sigma'$, there exists
  $s,t\in\TF$ such that one has $s\stackrel{\alpha}{\f}q $,
  $t\stackrel{\alpha}{\f}q' $ and $s\rw_\R t $.

\end{enumerate}
To conclude, $\A'$ is also well-defined.

\end{proof}



\setcounter{theorem}{9}


\begin{theorem}
  \label{thm:C}
  If $\aaexeq^i$ is well-defined then so is $\compl(\aaexeq^i)$, 
  and $\forall q \in \Q^i$, $\forall t \in
  \Lang(\aaexeq^i, q)$, $\forall t \in\TF$, $t \rw_\R t' \imp t' \in
  \Lang(\compl(\aaexeq^{i}), q)$.
\end{theorem}

\begin{proof}

  Let $CP$ be the finite set of critical pairs computed from
  $\aaexeq^i$ to solve.  Assume that $CP=\{\la
  r1\sigma_1,\alpha_1,q_1\ra,\ldots, \la r1\sigma_m,\alpha_1,q_m\ra
  \}$. By definition, considering $\A_0=\aaexeq^i$ there exists a
  sequence of $\RE-$automata $\A_1,\ldots,\A_m$ where $\A_j$ is obtained
  from $\A_{j-1}$ by solving the critical pair $\la r_j\sigma_j,
  \alpha_j,q_j\ra$ according Definition \ref{def:resolution_cp}. Thus, 
  $\compl(\aaexeq^i)=\A_n$. For a 
  question of readability and in order to prevent any confusion between notations, 
  each $\RE-$automaton $\A_j$ is defined as follows:  $\A_j=\la \F,\Q^{n+1},\Q_{f},\Delta'^{j}\cup \Drw'^{j}\cup\Deq'^{j} \ra$.

  First, let us show that $\compl(\aaexeq^i)$ is well-defined if $\aaexeq^i$ is well-defined.

  \paragraph{ $\compl(\aaexeq^i)$ is well-defined: }
    
  Let $P_n$ be the following proposition: $\A_n$ is well-defined. 

  \begin{itemize}
  \item $P_0$: Trivial since $\A_0=\aaexeq^i$ and $\aaexeq^O$ is well-defined by hypothesis.
  \item $ P_n\Rightarrow P_{n+1}$: By hypothesis, $\A_{n+1}$ is
    obtained from $\A_n$ by solving the critical pair $\la
    r_{n+1}\sigma_{n+1},\alpha_{n+1},$ $q_{n+1}\ra$. Applying Lemma \ref{lemma:C-well-defined}, 
    one obtains automatically that $\A_{n+1}$ is well-defined. 
  \end{itemize}
  So, one can deduce that $\compl(\aaexeq^i)$ is well-defined.

  \paragraph{$\compl(\aaexeq^i)$  covers terms accessible in one rewrite step from terms of  $\aaexeq^i$:}
  Let $q$ be a state of $\aaexeq^i$ and $t$ be a term of
  $\Lang(\aaexeq^i,q)$.  Suppose there exist a position $p\in
  \pos(t)$, a rule $l\f r\in\R$ and a substitution $\sigma':\X\f \TF$
  such that $t|_p=l\sigma'$.  Let $t'$ be the term such that
  $t'=t[r\sigma']_p$.  Since $t\in \Lang(\aaexeq^i,q)$, there exists a
  state $q'$ of $\aaexeq^i$ such that $t|_p=l\sigma'\f^*_{\aaexeq^i}
  q'$ and $t[q']_p\f^*_{\aaexeq^i}q$.  Following Property
  \ref{prop:matching-complete}, there exists $(\alpha,\sigma)\in S$
  with $l\match q'\vdash_{\aaexeq^i}S$ such that
  $l\sigma'\f^*_{\aaexeq^i}l\sigma$ and $l\sigma\f^*_{\aaexeq^i} q'$.
  If $\la r\sigma,\alpha,q' \ra$ is already solved then
  $r\sigma\f^*_{\aaexeq^i}q'$. Consequently, $r\sigma'$ can also be
  reduced to $q'$ in $\aaexeq^i$. Since
  $t'=t[r\sigma']_p\f^*_{\aaexeq^i}q$, $t'\in
  \Lang(\compl(\aaexeq^i),q)$. Suppose now that
  $r\sigma\not\f^*_{\aaexeq^i}q'$.  So, there exists $\la r_i\sigma_
  i,\alpha_i,q_i\ra\in CP$ such that $\la r_i\sigma_
  i,\alpha_i,q_i\ra=\la r\sigma,\alpha,q'\ra$.  By construction,
  $r\sigma\f^*_{\A_i}q'$. Consequently, $r\sigma'$ can also be
  reduced to $q'$ in $\A_i$. Since $\A_i$ is syntactically included in $\compl(\aaexeq^i)$, one can deduce that 
  $t'=t[r\sigma']_p\f^*_{C(\aaexeq^i)}q$. 
  Concluding the proof.
\end{proof}

\medskip

\setcounter{theorem}{\thesavetheorem}


% \begin{definition}[$\Delta-$coherence]
%   \label{prop:determinism}

% Let $\Delta$ be a set of normalized transitions. 
% Thus, $\aaex^i$ is said to be $\Delta-$coherent iff
% \begin{enumerate}
%   \item $\Delta\subseteq \Delta^i$,
%   \item if $\Delta^i\setminus\Delta$ contains two transitions of the form $f(q_1, \dots, q_n) \rw q$ and
%   $f(q_1, \dots, q_n) \rw q'$, then $q = q'$ and 
% \item if $\Delta^i\setminus\Delta$ contains two transitions of the form $t \rw q$ and
%   $t' \rw q$ with $t,t'$ two normalized configurations, then $t = t'$.
% \end{enumerate}
% \end{definition}
% \comments{Yohan : A virer ? }





%%% Local Variables: 
%%% mode: latex
%%% TeX-master: "main"
%%% TeX-PDF-mode: t
%%% End: 


\section{Proofs about $\widen$}

\begin{lemma}[$\widen$ preserves well-definition]
  Let $\A$ be a $\RE$-automaton. If $\A$ is well-defined, then so is $\widen(\A)$.
\end{lemma}

\begin{proof}
  Assume that $\A = \la \F, \Q, Q_f, \Delta \cup \Drw \cup \Deq\ra$ is well-defined.
  We have $\widen(\A) = \la \F, \Q, \Q_f, \Delta \cup \Drw \cup \Deq'\ra$, where $\Drw \supseteq \Drw'$,
  since $\widen$ only adds transitions to the $\Drw$. We have to prove the two items of definition~\ref{def:well-defined}.
  \begin{itemize}
  \item 
    The transitions of $\Deq'$ are never used for a run $\xrw{\alpha}$ where $\alpha = \top$,
    thanks to the second point of the definition~\ref{def:xrw_alpha}.
    It means that for all term $t$ and all state $q$, $t \xrw{\top}_{W(\A)} q$ is equivalent to $t \xrw{\top}_\A q$.
    Since $\A$ is well-defined, we know that there exists $u \in Rep(q)$ such that $u \rw^*_\R t$.
    $u$ is also a representative of $\widen(\A)$, and we deduce that first point of the definition~\ref{def:well-defined} 
    holds for $\widen(\A)$.
  \item
    Function $\widen$ only adds transitions to $\Deq'$ and do not remove transitions of $\A$.
    For all transitions $q \xrw{\alpha} q' \in \Drw'$ we have $q \xrw{\alpha} q' \in Drw$.
    Since $\A$ is well-defined, we know that there exist terms $s,t\in \TF$ such that
    $s\xrw{\phi}_\A q$, $t\xrw{\top}_\A q'$ and $t \rw_\R s$.
    We also have $s\xrw{\phi}_{\widen(\A)} q$, $t\xrw{\top}_{\widen(\A)} q'$ and $t \rw_\R s$.
  \end{itemize}
\end{proof}

\begin{lemma}
  For all $\RE$-automaton $\A$, $\Lang(\widen(\A)) \supseteq \Lang(\A)$.
\end{lemma}
\begin{proof}
This is easy to see since widening only adds transitions, and thus, does
not restrict the recognized language.
\end{proof}


% \begin{lemma}
% \label{lem:wf}
% The widening relation $\simp_E$ is well founded. 
% \end{lemma}

% \begin{proof} Widening can add, at most, transitions between all states
%   of the tree automaton $\A$ which has a finite number of states.
% \end{proof}


% Couvert par l'autre section

% \section{Proofs about intersection}

% \begin{lemma}[Emptiness decision of the product of a $\RE$-automaton and a
%   tree automaton]
%   Let $\A$ be a $\RE$-automaton and $\B$ a tree automaton. Let $S$ be the set of
%   reachable states of $\A\times \B$ defined according to
%   definition~\ref{def:reachable-states}. For all final state $q$ of $\A$, all
%   final state $q'$ of $\B$, all formulas $\phi_S\neq\bot$, $\phi\neq \bot$ and
%   all term $t\in\TF$, we have $t\xrw{\phi}^*_\A q$ and $t \rw_\B^* q'$
%   (i.e. $\Lang(\A)\cap \Lang(\B) \neq \emptyset$) if and only if there exists a
%   triple $(q,q',\phi_S)\in S$ such that $\phi \models \phi_S$. 
% \end{lemma}

% \begin{proof}
%   Let $\A =\langle \F, \Q^\A,\Q^\A_f,\Delta^\A, \Drw, \Deq \rangle$
%   be the $\RE$-automaton and $\B=\langle \F, \Q^\B,\Q^\B_f,\Delta^\B \rangle$ be
%   the tree automaton.  
%   % We use definition~\ref{def:xrw_alpha} for the recognized language
%   % $\Lang(\A,q)=\{t\in \TF \sep t \xrw{\phi}_\A^* q \mbox{ and $\phi$ 
%   %   satisfiable}\}$. 
%   % We prove that $(q,q',\phi_S)\in S$ if 
%   % and only if %$\Lang(\A,q) \cap \Lang(\B,q') \neq \emptyset$.
%   % there exists a term $t\in\TF$ such that $t\xrw{\phi}^*_\A q$, $t
%   % \rw_\B^* q'$ and $\phi\models \phi_S$.
%   We prove a stronger property on all states $q$ of $\A$ and $q'$ of $\B$ (and not
%   only for final states).
%   First, we prove the 'only if' part. Let us assume that 
%   % $\Lang(\A,q) \cap \Lang(\B,q') \neq \emptyset$, hence that 
%   there exists a term $t\in\TF$ such that $t\xrw{\phi}^*_\A q$, $t \rw_\B^* q'$.
%   % and $\phi$ satisfiable. 
%   By induction on the height of $t$ we have:
  
%   \begin{itemize}
%   \item If $t$ is a constant, since $\B$ is an epsilon-free tree automaton, the
%     only way to have $t \rw^*_\B q'$ is to have $t \rw q' \in \B$. With regards to
%     $\A$, by definition~\ref{def:xrw_alpha}, $t\xrw{\phi}^*_\A q$ means that
%     there exists states $q_0,q_1, \ldots, q_n$ and formulas $\phi_1,\ldots,
%     \phi_n$ such that $t \rw_{\Delta_\A} q_0 \xrw{\phi_1} q_1 \xrw{\phi_2}
%     \ldots q_n$ with $q=q_n$ and $\phi=\phi_1\wedge \ldots \wedge
%     \phi_n$.% is satisfiable.
%     Transitions $q_i \xrw{\phi_i} q_{i+1}$ are either transitions of $\Drw$ or
%     transitions of $\Deq$ with $\phi_i=\top$.  Because of transitions $t \rw q_0
%     \in \Delta_\A$ and $t \rw q' \in \Delta_\B$, using the first case of
%     definition~\ref{def:reachable-states}, we get that $(q_0,q',\top) \in
%     S$. Similarly, using the second case of the definition, we obtain that there
%     exists formulas $\phi'_i$ with $i=1\ldots n$ such that $(q_1,q', \phi_1\vee
%     \phi'_1), (q_2,q',(\phi_1\wedge\phi_2)\vee \phi'_2),\ldots (q_n,q', (\phi_1
%     \wedge \ldots \wedge \phi_n)\vee \phi'_n)$ belong to $S$. Finally, since
%     $q_n=q$ and $\phi=\phi_1\wedge \ldots \wedge \phi_n$, we that
%     $(q,q',\phi\vee\phi_n') \in S$. Furthermore, we trivially have that $\phi_S=
%     \phi\vee \phi_n'$ and $\phi \models \phi_S$. % since $\phi$ is
%     % satisfiable, so is $\phi \vee \phi'$.
    
%   \item Assume that for all term of height lesser or equal to $n\in\NN$, the
%     property is true. Let us prove that it is also true for a term $f(t_1, \ldots,
%     t_n)$ with $t_1, \ldots, t_n$ of height lesser or equal to $n$. Since $f(t_1,
%     \ldots, t_n) \rw^*_\B q'$ and $\B$ is an epsilon free tree automaton, we
%     obtain that $\exists q'_1,\ldots,q'_n\in\Q^\B$ such that $\forall i=1\ldots n:
%     t_i \rw^*_\B q'_i$ and $f(q'_1,\ldots,q'_n) \rw q' \in \Delta_\B$. With
%     regards to $\A$,
%     by definition~\ref{def:xrw_alpha}, $f(t_1, \ldots, t_n)
%     \xrw{\phi}^*_\A q$ means that there exists states $q_0,q_1, \ldots,
%     q_m,q''_1,\ldots,q''_n$ and formulas $\phi_1,\ldots, \phi_m, \phi'_1, \ldots,
%     \phi'_n$ such that $\forall i=1\ldots n: t_i \xrw{\phi'_i}^*_\A q''_i$,
%     $f(q''_1,\ldots, q''_n) \rw_{\Delta_\A} q_0$ and $q_0 \xrw{\phi_1} q_1
%     \xrw{\phi_2} \ldots q_n$, $q=q_n$. Furthermore, we obtain that $\phi=
%     \bigwedge_{i=1}^{n} \phi'_i \wedge \bigwedge_{i=1}^{m} \phi_i$. % is
%     % satisfiable. 
%     Since terms $t_i$ are of height lesser or equal to $n$, $\forall
%     i=1\ldots n: t_i \rw^*_\B q_i$ and $\forall i=1 \ldots n: t_i \xrw{\phi'_i}_\A^*
%     q''_i$, we can apply the induction hypothesis and obtain that $\forall
%     i=1\ldots n: (q_i, q''_i, \phi''_i) \in S$ with $\phi'_i \models \phi''_i$. % and $\phi'_i$ satisfiable.  
%     Besides to this, using case~1 of definition~\ref{def:xrw_alpha} on
%     $f(q_1,\ldots,q_n) \rw q' \in \Delta_\B$, $f(q''_1,\ldots,q''_n) \rw q_0 \in
%     \Delta_\A$, and $\forall i=1\ldots n: (q_i, q''_i, \phi''_i) \in S$, we obtain
%     that there exists a formula $\phi'$ such that $(q_0,q', (\bigwedge_{i=1}^n
%     \phi''_i) \vee \phi')\in S$. Then, like in the base case, since $q_0
%     \xrw{\phi_1} q_1 \xrw{\phi_2} \ldots q_n$, $q=q_n$, we can deduce that
%     there exists a formula $\phi''$ such that $(q,q',(\bigwedge_{i=1}^n \phi''_i
%     \wedge \bigwedge_{i=1}^{m} \phi_i) \vee \phi'')\in S$. Let $\phi_S= (\bigwedge_{i=1}^n \phi''_i
%     \wedge \bigwedge_{i=1}^{m} \phi_i) \vee \phi''$. Since we know from above that
%     $\phi=\bigwedge_{i=1}^n \phi'_i \wedge \bigwedge_{i=1}^{m} \phi_i$ and
%     $\forall i=1\ldots n: \phi'_i \models \phi''_i$, we obtain that % is satisfiable, so is 
%     $\phi \models \phi_S$.
%   \end{itemize}
  
%   \medskip
%   Second, we prove the 'if' part: if $(q,q',\phi_S)\in S$ and $\phi_S \neq \bot$
%   then there exists a term $t$ and a formula $\phi \neq \bot$ such that $\phi \models \phi_S$,
%   % and $S$ satisfiable,
%   $t\xrw{\phi}^*_\A q$ and $t \rw_\B^* q'$. We make a proof by induction
%   on the number of applications of the two rules of
%   definition~\ref{def:reachable-states}, necessary to prove that $(q,q',\phi_S)$ in $S$.

%   \begin{itemize}
%   \item If the number of steps is $0$ then, since the computation of $S$ starts
%     from the set $\Q^\A \times \Q^\B \times \bot$, then all $(q,q',\phi_S)$
%     are such that $\phi_S=\bot$, which is a contradiction.

%   \item We assume that the property is true for any triple $(q,q',\phi_S)$ which can
%     be deducted by $n$ or less applications of the rules of
%     definition~\ref{def:reachable-states}. Now, we consider the case of a triple
%     $(q,q',\phi_S)$ that is deduced at the $n+1$-th step of application of the
%     deduction rules.
%     \begin{itemize}
%     \item If the first rule is concerned, this means that there exists triples
%       $(q_1,q'_1, \phi_1),\ldots,(q_n,q'_n,\phi_n)$ and $(q,q',\phi)$ in $S$
%       deduced before $n+1$-th step, as well as transitions $f(q_1,\ldots,q_n)\rw q
%       \in \Delta_\A$ and $f(q'_1,\ldots,q'_n)\rw q' \in \Delta_\B$. Furthermore,
%       we know that $\phi_S=\phi \vee \bigwedge_{i=1}^n \phi_i$.
%       % and by hypothesis $\rho$ is satisfiable. 
%       If $\phi\neq \bot$ then, since $(q,q',\phi)$ was shown to belong to $S$
%       before $n+1$-th step, we can apply the induction hypothesis and directly
%       obtain that there exists a term $t$ and a formula $\phi'$ such that $\phi'
%       \models \phi$, $t \xrw{\phi'}^*_\A q$ and $t \rwB^* q'$. Note that $\phi'
%       \models \phi$ implies $\phi' \models \phi_S$.
%       Otherwise, if $\phi=\bot$, then % $\bigwedge_{i=1}^n \phi_i$ is then 
%       we can apply the induction hypothesis on triples $(q_i,q'_i,\phi_i)$,
%       $i=1\ldots n$ and obtain that $\forall i=1\ldots n:\exists \phi_i':\exists
%       t_i \in\TF: \phi'_i \models \phi_i$, $t_i \xrw{\phi_i'}_\A^* q_i$ and $t_i
%       \rwB^* q'_i$. Finally, 
%       because of the two transitions $f(q_1,\ldots,q_n)\rw q \in \Delta_\A$ and
%       $f(q'_1,\ldots,q'_n)\rw q' \in \Delta_\B$, we get that $f(t_1,\ldots,t_n)
%       \xrw{\phi'}_\A^* f(q_1,\ldots,q_n) \rwA^* q$ with
%       $\phi'=\bigwedge_{i=1}^n \phi'_i$ on one side and $f(t_1,\ldots,t_n) 
%       \rwB f(q'_1,\ldots,q'_n) \rwB^* q$ on the other side. Furthermore,
%       since $\forall i=1\ldots n: \phi'_i \models \phi_i$, we have
%       $\bigwedge_{i=1}^n \phi'_i \models \bigwedge_{i=1}^n \phi_i$. Recall that
%       $\phi'= \bigwedge_{i=1}^n \phi'_i$ and $\phi_S= \phi\vee \bigwedge_{i=1}^n
%       \phi_i$. Hence, $\phi' \models \phi_S$.
%     \item If the second rule is concerned, this means that there exists triples
%       $(q_1,q',\phi_1)$ and $(q,q',\phi_2)$ in $S$ deduced before the $n+1$-th
%       step. Furthermore, we know that $\phi_S=(\phi_1 \wedge \phi) \vee \phi_2$.
%       Like above, if $\phi_2 \neq \bot$ then we can apply induction hypothesis on
%       $(q,q',\phi_2)$ and trivially get the result. Otherwise, if $\phi_2=\bot$
%       then we can use induction hypothesis on the triple $(q_1,q',\phi_1)$ and
%       obtain that there exists a formula $\phi_1'$ and a term $t_1$ such that 
%       $t_1\xrw{\phi_1'}_\A^* q_1$, $t_1 \rwB^* q'$ and $\phi_1' \models \phi_1$.
%       Then, by case on the epsilon transition used for the deduction on $S$, we
%       prove that $t_1 \xrw{\phi_1'\wedge \phi}_\A^* q$:
%       \begin{itemize}
%       \item Assume that $q_1 \xrw{\phi} q \in \Drw$. Then, by
%         definition~\ref{def:xrw_alpha}, we obtain that $t_1
%         \xrw{\phi'_1 \wedge \phi}_\A^* q$. Furthermore, since $\phi'_1
%         \models \phi_1$, we have that $\phi'_1 \wedge \phi \models
%         \phi_1 \wedge \phi$ and, finally, that $\phi'_1 \wedge \phi
%         \models \phi_S$.
%       \item Assume that $q_1 \rw q \in \Deq$. Then, by
%         definition~\ref{def:xrw_alpha}, we obtain that $t \xrw{\phi_1
%           \wedge Eq(q_1,q)}_\A^* q$. Finally, like above, we can
%         deduce that $\phi'_1 \wedge Eq(q_1,q) \models \phi_1 \wedge
%         Eq(q_1,q)$ and thus $\phi'_1 \wedge Eq(q_1,q) \models \phi_S$.
%       \end{itemize}
%     \end{itemize}
%   \end{itemize}
% \end{proof}



%%% Local Variables: 
%%% mode: latex
%%% TeX-master: "main"
%%% TeX-PDF-mode: t
%%% End: 
