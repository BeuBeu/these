Since the $\kw{merge}$ procedure does not allow to differentiate the
exact term from the approximation. Instead of renaming states, a way
to make the difference is to introduce an odered relation between
those states. This relation denoted by ($\Deq$) a set of labelled
$\varepsilon$-transitions of the form $q' \rwtag{eq} q$ has to be understood as : the language
of $q'$ is considered as the approximation added by the equation $eq$ to
the language of $q$. That prompts us to extend the definition of 
an $\R$-Automaton.

\begin{Definition}{$\RE$-Automaton}
  \label{def:RE-automaton}
  A $\RE$-Automaton is a tree automton defined as quadruple $\A= \langle \F, \Q, \Q_F,\Delta \cup \Deps \cup \Deq\rangle$,
  where $\Q_F \subseteq \Q$ and transitions are now splited in three sets~:
  \begin{description}
  \item [$\Delta$:] the normalized transitions
  \item [$\Deps$:] $\varepsilon$-transitions labelled by constraints.
  \item [$\Deq$:] $\varepsilon$-transitions labelled by an equation as detailled bellow.
  \end{decription}
  
  The equivalence relation induced by $(Q, \rw^!)$ simply associates to each state sets of terms.
  Since the relation $\f^!$ is deterministic, thoses sets are disjoined.

  We introduce $\f^=$ which denotes the transitive closure of transitions of $\Delta \cup \Deq$.
  The couple $(Q, \f^=)$ induces the equivalence relation such that if terms $u$ and $v$
  are in the same equivalence classe, then $u =_E v$.
  
  
\end{Definition}
%%% Local Variables: 
%%% mode: latex
%%% TeX-master: "main"
%%% End: 
