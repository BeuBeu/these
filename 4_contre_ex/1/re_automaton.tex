%%% Motiver l'introduction du $\RE$-Automaton
\section{$\RE$-Automaton for refining}
\label{sec:re-automaton}


In this section, we propose to extend the completion technique with a
counter-example based procedure. Contrary to existing approaches that
have to perform a backward propagation from the bad term to the set of
initial state, we propose to extend the transition relation of tree
automata with information about the rewriting rules and equations that
have been applied to the initial automaton.

More precisely, we use the set $\Drw$ to distinguish a term
from its successors that has been obtained by applying one or several
rewriting rules. Instead of merging states according to the set of
equations, our model link them with transitions that belongs to the
set $\Deq$.

% $\Deq$ denotes a subset
% of the equivalence relation induced by the set of equations.
% Moreover, we will be able to know that a term recognized using a
% transition of $\Deq$ is the result of a widening step.  The
% example~\ref{ex:semantics} illustrates the principle of a
% $\RE$-Automaton.


%We now give the formal definition for $\RE$-automata.

\begin{definition}[$\RE$-automaton]
  \label{def:re-automaton}
  Given a TRS $\R$ and a set $E$ of equations, a $\RE$-automaton $\A$
  is a tuple $\langle \F, \Q, \Q_F, % \Delta_0 \cup
  \Delta \cup \Drw \cup \Deq\rangle$.  $\Delta$ is a set of normalized
  transitions. $\Deq$ is a set of $\varepsilon$-transitions. $\Drw$ is
  a set of $\varepsilon$-transitions labeled by $\top$ or conjunctions 
  over predicates of the form $Eq(q, q')$ where $q, q' \in \Q$, and $q \rw
  q' \in \Deq$. %$Eq(q, q')$ holds iff $q \rw q' \in \Deq$.
\end{definition}


\begin{example}
  \label{ex:semantics}
  Let $\A$ be the $\RE$-automaton recognizing the program
  where $I = f(a)$, $\R = \{f(c) \rw g(c),\ a \rw b\}$ and $E = \{ b = c\}$.

  \centering
  \medskip
  \begin{tabular}{lc}
  \hspace{-.3cm}
  \begin{minipage}{.75\linewidth}
      In $A$, the equality $b = c$ is denoted by two transitions $q_c
      \rw q_b$ and $q_b \rw q_c$ of $\Deq$, assuming that $b$, $c$ are
      recognized into $q_b$, $q_c$, respectively. For the state $q_c$,
      the transition $q_b \rw q_c$ indicates that the term $b$ is
      obtained from the term $c$ by equality.
      % The term $g(b)$ is recognized into $q_g$, thanks to $q_b \rw q_c$ too.
      % Conversely, we have that $c$ is the approximation of $d$ from state $q_d$.
      Transitions $q_g \rw q_f$ and $q_b \rw q_a$ denote rewriting steps.
      Those transitions allow us to deduce $f(a) \rw^*_\RE g(c)$ and, $a \rw^*_\RE b$.
      To have $f(a) \rw_\R f(b) =_E f(c) \rw_\R g(c)$, which is indeed
      $f(a) \rw^*_\RE g(c)$ unfolded,
      we use the equality $b = c$ to obtain $c$ from $b$, relation denoted by the 
      by the transition $q_c \rw q_b$. Thus, we label the transition  $q_g \rw q_f$
      with the formula $Eq(q_c, q_b)$ to save this information, whereas 
      the transition $q_b \rw q_a$ is labeled with $\top$ which means $a \rw^*_\R b$.
    \end{minipage}&%\quad
    \begin{minipage}{0.25\linewidth}
      \tikz[thick, scale=.8]{
        \node (qf) at (0, 0)   {$\mathbf{q_f}$};
        \node (qg) at (4, 0)   {$q_g$};
        \node (f)  at (0,-1.5) {$f(\;q_a\;)$};
        \node (g)  at (4,-1.5) {$g(\;q_c\;)$};
        \node (qa) at (0,-2.8) {$q_a$};
        \node (qb) at (2,-2.8) {$q_b$};
        \node (qc) at (4,-2.8) {$q_c$};
        \node (a)  at (0,-4.3) {$a$};
        \node (b)  at (2,-4.3) {$b$};
        \node (c)  at (4,-4.3) {$c$};
        % \Delta
        \draw [->] (f) edge (qf);
        \draw [->] (g) edge (qg);
        \draw [->] (a) edge (qa);
        \draw [->] (b) edge (qb);
        \draw [->] (c) edge (qc);
        % liens de descendance
        \draw [dashed] (f) edge (qa);
        \draw [dashed] (g) edge (qc);
        % \Drw
        \draw [->, bend right=15] (qg) to node[auto, swap] {\footnotesize$Eq(q_c, q_b)$} (qf);
        \draw [->, bend right=15] (qb) to node[auto, swap] {\footnotesize$\top$} (qa);
        % \Deq
        \draw [->, bend right=15] (qb) to node[auto, swap] {\footnotesize$=$} (qc);
        \draw [->, bend right=15] (qc) to node[auto, swap] {\footnotesize$=$} (qb);
      }
    \end{minipage}
  \end{tabular}
%  \caption{Example of $\RE$-automaton}
\end{example}

%%% Local Variables: 
%%% mode: latex
%%% TeX-master: "main"
%%% TeX-PDF-mode: t
%%% End:


%$\RE$-automata make it possible to check wether a term is {\em
%  really} reachable, i.e. a counter example. 


% A $\RE$-automaton is an extension of a $\R$-automaton introduced in
% \cite{BoyerG-RULE09}. Intuitively, terms are recognized using
% transitions of $\Delta$, the transitions of $\Drw$ denote the
% rewriting relation between those terms, and the
% $\varepsilon$-transitions of $\Deq$ denote the approximations. Each
% transition of $\Drw$ is labeled by a logical formula denoting the
% approximations needed so as to have the rewriting step.  Intuitively,
% if the formula is $\top$, no approximation is necessary and the term
% is reachable by rewriting only.
%Otherwise, if the formula is not equivalent to $\top$, th

%\comments{On s'en sert?: In the following, we assume that logical
%formulas are always transformed into an equivalent \emph{Disjunctive Normal
%  Form} using standard logic equivalences.}
\noindent 
%%Normalized transitions of $\Delta$ in a $\RE$-automaton recognize terms, called
%%{\em representative}, whereas $\varepsilon$-transitions represent rewriting and
%%equivalence relations between them. 

%%\comments{Axel: shall we keep the definition of sets of representative?}

In what follows, we use $\rwned$ to denote the transitive and
reflexive closure of $\Delta$.  Given a set $\Delta$ of normalized
transitions, the set of representatives of a state is defined by
$Rep(q) = \{ t \in \TF | t \rwned q\}$. We now formally describe the
runs of a $\RE$-automaton.

%This $\rwned$ is a particular case of the new rewriting relation $\xrw{\alpha}_\A$
%of $\RE$-automata. 

\begin{definition}[Run of a $\RE$-automaton $\A$]
  \label{def:xrw_alpha}
  \begin{itemize}
  \item $t|_p = f(q_1, \dots, q_n)$ and $f(q_1, \dots, q_n) \rw q \in \Delta$
    then $t \xrw{\top}_\A t[q]_p$
  \item $t|_p = q$ and $q \rw q' \in \Deq$ then $t \xrw{Eq(q, q')}_\A t[q']_p$
  \item $t|_p = q$ and $q \xrw{\alpha} q' \in \Drw$ then $t \xrw{\alpha}_\A t[q']_p$ 
%     where $\alpha = \left\{
%       \begin{array}{ll}
%         \alpha_k &\textrm{if } 1 \le k \le n \land \phi = \bigvee_1^n \alpha_i\\
%         \top &  \textrm{if } \phi = \top
%       \end{array}\right.
%     $
  \item $u \xrw{\alpha}_\A v$ and $v \xrw{\alpha'}_\A w$ then $u \xrw{\alpha \land \alpha'}_\A w$
  \end{itemize}
\end{definition}

\noindent
A run $\xrw{\alpha}$ abstracts a rewriting path of $\rwmod$. If $t
\xrw{\alpha} q$, then there exists a term $s\in Rep(q)$ such that
$s\rwmod^* t$. The formula $\alpha$ denotes the subset of transitions
of $\Deq$ needed to recognize $t$ into $q$.
%%i.e. the equivalence
%%steps, induced by $E$, needed to rewrite $s$ into $t$ using
%%$\rwmod$.
\begin{example}
  Consider the $\RE$-automaton $A$ of Example \ref{ex:semantics} and
  let $g(b) \xrw{Eq(q_b, q_c) \land Eq (q_c, q_b)} q$, we know that
  there exists a rewriting path of $\rwmod$ from $f(a)$, the unique
  term of $Rep(q)$ to $g(b)$. The formula indicates that this
  rewriting path uses the equivalence relation induced by $b = c$ in
  both directions (transitions $q_b \rw q_c$ and $q_c \rw q_b$).
\end{example}


The relation $\xrw{\alpha}$ corresponds to the standard rewriting
relation (see Section~\ref{sec:definitions}) of a tree-automaton
instrumented with logical formulas.

\begin{theorem}\label{th:equiv}{\quad\quad
  $\forall t\in\TFQ,\; q \in \Q,\; t \xrw{\alpha}_\A q \equ t \rw_\A^* q$}
\end{theorem}

We now introduce {\em well-defined} $\RE$-automata. The well-defined
property will be used in the refinement procedure to
distinguish between counter-examples and false positives. 


\begin{definition}[A well-defined $\RE$-automaton]
  \label{def:well-defined}
  $\A$ is a \emph{well-defined} $\RE$-automaton, if :
% The second point of the definition is used to refine the $\RE$-automaton: a rewriting step of $\rwmod$
% denoted by $q \xrw{\phi} q'$ holds thanks to the subset of transitions of $\Deq$
% occurring in $\phi$. If we remove the transitions of $\Deq$ such that $\phi$
% does not hold, then the transition $q \xrw{\phi} q'$ disappears and the term is 
% no longer recognized.
  \begin{itemize}
  \item For all state $q$ of $\A$, and all term $v$ such that
    $v \xrw{\top}_\A q$, there exists $u$ a term representative
    of $q$ such that $u \rw^*_\R v$
  \item If $q \xrw{\phi} q'$ is a transition of $\Drw$, then there exist terms
    $s,t\in \TF$ such that $s\rwtag{\phi}_\A q$, $t\rwtag{\top}_\A q'$
    and $t \rw_\R s$.
  \end{itemize}
\end{definition}

\noindent
The first item in the definition~\ref{def:well-defined} guarantees
that every term recognized by using transitions labeled with the
formula $\top$ is indeed reachable from the initial set.  The second
item is used to refine the automaton. A rewriting step of $\rwmod$
denoted by $q \xrw{\phi} q'$ holds thanks to some transitions of
$\Deq$ that occurs in $\phi$. If we remove transitions in $\Deq$ in
such a way that $\phi$ does not hold, then the transition $q
\xrw{\phi} q'$ should also be removed.

According to the above construction, a term $t$ that is recognized by
using at least a transition labeled with a formula different from $\top$
can be removed from the $\RE$-automaton by removing some transitions
in $\Deq$. This ``pruning'' operation is illustrated hereafter.

\begin{example}
  \label{ex:pruning}
  We consider the $\RE$-automaton $\A$ of Example~\ref{ex:semantics}.
  This automaton recognizes the term $g(c)$. Indeed, by
  Definition\ref{def:xrw_alpha}, we have $g(c) \xrw{Eq(q_c, q_b)}
  q_f$. Consider now the rewriting path $f(a) \rw_\R f(b) =_E f(c)
  \rw_\R g(c)$. We can see that if the step $f(b) =_E f(c)$ denoted by
  the transition $q_c \rw q_b$ is removed, then $g(c)$ becomes
  unreachable. The first step in pruning $\A$ consists thus in
  removing this transition. In a second step, we propagate the
  information by removing all transition of $\Drw$ labeled by a
  formula formed with $Eq(q_c, q_b)$. This is done to remove all terms
  obtained by rewriting with the equivalence $b =_E c$. After having
  pruned all the transitions, we observe that the terms recognized by
  $\A$ are given by the following set $\{f(a), f(b)\}$.
\end{example}

%%As we shall see, the prunning technique sketched in the above example
%%will serve as a basis for the refinement technique presented in
%%section~\ref{sec:refinement}.






%%% Local Variables: 
%%% mode: latex
%%% TeX-master: "main"
%%% TeX-PDF-mode: t
%%% End: 
