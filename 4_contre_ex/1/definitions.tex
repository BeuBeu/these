\section{Definitions}
\label{sec:definitions}
% \comments{
%   Toutes les definitions à propos de Réécriture, substitutions ($\Q$-subst inclues) et 
%   Automates d'arbres.
%   Attention : \\
%   pas de def independante de normalized transitions et epsilon-transition
%   uniquement une def globale pour les automates 
%   suivie du langage d'un automate d'arbres sans format def.
% }

In this section, we introduce some definitions and concepts that will
be used through the rest of the paper (see
also\,\cite{BaaderN-book98,TATA,GilleronTison-FI95}). Let $\F$ be a
finite set of symbols, each associated with an arity function, and let
$\X$ be a countable set of {\em variables}. $\TFX$ denotes the set of
{\em terms} and $\TF$ denotes the set of {\em ground terms} (terms
without variables). The set of variables of a term $t$ is denoted by
$\var(t)$. A {\em substitution} is a function $\sigma$ from $\X$ into
$\TFX$, which can be uniquely extended to an endomorphism of $\TFX$. A
{\em position} $p$ for a term $t$ is a word over $\NN$. The empty
sequence $\lambda$ denotes the top-most position. The set $\pos(t)$ of
positions of a term $t$ is inductively defined by $\pos(t)= \{
\lambda\} $ if $t \in \X$ and $\pos(f(t_1,\dots,t_n)) = \{ \lambda \}
\cup \{i.p \mid 1 \leq i \leq n \et p \in \pos(t_i) \}$ otherwise.  If
$p \in \pos(t)$, then $t|_p$ denotes the subterm of $t$ at position
$p$ and $t[s]_p$ denotes the term obtained by replacement of the
subterm $t|_p$ at position $p$ by the term $s$.


A {\em term rewriting system} (TRS) $\R$ is a set of {\em rewrite
  rules} $l \rw r$, where $l, r \in \TFX$, $l \not \in \X$, and
$\var(l) \supseteq \var(r)$.  A rewrite rule $l \rw r$ is {\em
  left-linear} if each variable of $l$ occurs only once in $l$.  A TRS
$\R$ is left-linear if every rewrite rule $l \rw r$ of $\R$ is
left-linear.  The TRS $\R$ induces a rewriting relation $\rw_{\R}$ on
terms as follows. Let $s, t\in \TFX$ and $l \rw r \in \R$, $s \rw_{\R}
t$ denotes that there exists a position $p\in\pos(t)$ and a
substitution $\sigma$ such that $s|_p= l\sigma$ and $r=s[r\sigma]_p$.
%  $s \rw^p_{\R} t$ denotes that there exists a
% position $p\in\pos(t)$ and a substitution $\sigma$ such that $s|_p= l\sigma$ and
% $r=s[r\sigma]_p$. Note that the rewriting position $p$ can generally be omitted,
% i.e. we generally write $s \rw_{\R} t$. 
The reflexive transitive closure of $\rw_{\R}$ is denoted by
$\rw^*_{\R}$ and $s \rw^!_\R t$ denotes that $s\rw^*_\R t$ and $t$ is
irreducible by $\R$. The set of $\R$-descendants of a set of ground
terms $I$ is $\desc(I) = \{t \in \TF \sep \exists s \in I \st s
\rw^*_{\R} t \}$.  An {\em equation set} $E$ is a set of {\em
  equations} $l = r$, where $l, r \in \TFX$.  For all equation $l = r
\in \R$ and all substitution $\sigma$ we have $l\sigma =_E r\sigma$.
The relation $=_E$ is the smallest congruence such that for all
substitution $\sigma$ we have $l\sigma = r\sigma$. Given a TRS $\R$
and a set of equations $E$, a term $s\in\TF$ is rewritten modulo $E$
into $t\in\TF$, denoted $s \rwmod t$, if there exist $s\in\TF'$ and
$t'\in\TF$ such that $s=_Es'\rw_\R t'=_E t$. Thus, the set of
$\R$-descendants modulo $E$ of a set of ground terms $I$ is $\descE(I)
= \{t \in \TF \sep \exists s \in I \st s \rwmod^* t \}.$

We now define tree automata that are used to recognize possibly
infinite sets of terms. Let $\Q$ be a finite set of symbols with arity
$0$, called {\em states}, such that $\Q \cap \F= \emptyset$.  $\TFQ$
is called the set of {\em configurations}. A {\em transition} is a
rewrite rule $c \f q$, where $c$ is a configuration and $q$ is state.
A transition is {\em normalized} when $c = f(q_1, \ldots, q_n)$, $f
\in \F$ whose arity is $n$, and $q_1, \ldots, q_n \in \Q$. A {\em
  $\varepsilon$-transition} is a transition of the form $q \f q'$
where $q$ and $q'$ are states.

\begin{definition}[Bottom-up nondeterministic finite tree automaton]
\label{def:automata}
A bottom-up nondeterministic finite tree automaton (tree automaton for
short) over the alphabet $\F$ is a tuple $\A= \langle \F, \Q,
\Q_F,\Delta \rangle$, where $\Q_F \subseteq \Q$, $\Delta$ is a set of
normalized transitions and $\varepsilon$-transitions.
\end{definition}

\noindent
The transitive and reflexive {\em rewriting relation} on $\TFQ$
induced by all the transitions of $\A$ is denoted by $\f_{\A}^*$. The
tree language recognized by $\A$ in a state $q$ is $\Lang{}(\A,q) =
\{t \in \TF \sep t \rw^*_\A q \}$. The language recognized by $\A$ is
$\Lang{}(\A) = \bigcup_{q \in \Q_F} \Lang{}(\A, q)$. 

% Let $\TFX$ be a
% set of terms represented by an automaton $A$, we use $\R(A)$ to denote
% the automaton that represents the terms that can be obtained from
% $\TFX$ by applying $\R$.



  %A tree language is regular if and only if it can be recognized by a tree automaton.
  %We say that $t$ is a {\em canonical term} of the state $q$, if $t \f^!_\A q$.

% \begin{example}
%    Let $\A$ be the tree automaton $\langle \F, \Q, \Q_F, \Delta \rangle$ such that $\F=\{f,g,a,b\}$, $\Q= \{q_0, q_1, q_2\}$, $\Q_F=\{q_0\}$,
%    $\Delta= \{f(q_0)
%    \rw q_0, g(q_1) \rw q_0, a \rw q_1, b \rw q_2 \}$ and $\Delta_{\epsilon}=\{q_2 \rw q_1 \}$. In $\Delta$, transitions are
%    {\em normalized}. A transition of the form $f(g(q_1)) \f q_0$ is not normalized. The term $g(a)$
%    is a term of $\TFQ$ (and of $\TF$) and can be rewritten by $\Delta$ in the following way:
%    $g(a) \rwne_\A g(q_1)
%    \rwne_\A q_0$. Hence $g(a)$ is a canonical term of $q_1$. Note also that $b \rw_\A q_2 \rw_\A q_1$.
%    Hence, $\Lang{}(\A, q_1)=
%    \{a, b\}$ and $\Lang{}(\A)=\Lang{}(\A, 
%    q_0) = \{g(a), g(b),f(g(a)), f(f(g(b))),\ldots\}=\{f^*(g([a|b]))\}$.
%   \end{example}




%%% C'est peut �tre correct sans cette def inutile ??? !!!
%%% => compacite + determinisme de la normalisation...

% \begin{property}[$\f^!$ deterministic]
%   \label{prop:deterministic}
%   If $\Delta$ contains two normalized transitions of the form 
%   $f(q_1, \dots, q_n) \rw q$ and $f(q_1, \dots, q_n) \rw q'$, it means $q = q'$. 
%   This ensures that the rewriting relation $\f^!$ is deterministic.
% \end{property}



%%%%% FIN R-AUTOMATON : LA SUITE AILLEURS





%%% Local Variables: 
%%% mode: latex
%%% TeX-master: "main"
%%% End: 
