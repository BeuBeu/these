\documentclass{fsttcs}
% Note that \begin{document} comes before title information
\usepackage[utf8]{inputenc}
\usepackage[all, curve]{xy}
\usepackage{amsmath, amssymb,graphicx,upgreek}
\usepackage{url}
\usepackage{tikz}

\usepackage{wrapfig}
%%%%%%%%%%%%%%%%%%%%%%%%%%%%%%%%%%%%%%%%%%%%%%%%%%%%%%%%%%%%%
%%Ces macros sont toutes utilisees par le document
%%%%%%%%%%%%%%%%%%%%%%%%%%%%%%%%%%%%%%%%%%%%%%%%%%%%%%%%%%%%%



% merging of states 
% la commande est deja definie dans le package xypic 
% utilise pour la figure de la paire critique...
% je l'ai remplace de maniere temporaire par \kw{merge}

% \newcommand{\merge}{Merge}
% suggestion de remplacement ?
% \newcommand{\merges}{Merge}


%Insertion de commentaires dans le PDF
%\newcommand{\comments}[1]{\paragraph{\bf Commentaire~: }{\color{blue}#1\newline}}

%%Mots clefs :
\newcommand{\kw}[1]{\mathtt{#1}}
\newcommand{\parts}[1]{\mathcal{P}(#1)}

\newcommand{\bydef}{\stackrel{\triangle}{=}}
\newcommand{\mapswith}[1]{\stackrel{#1}{\longrightarrow}}

\newcommand{\coq}{\textsf{Coq}}
\newcommand{\timbuk}{\textsf{Timbuk}}
\newcommand{\ocaml}{\textsf{OCaml}}
\newcommand{\tom}{\textsf{Tom}}
\newcommand{\scala}{\textsf{Scala}}


%%% A VOIR SI JE GARDE OU PAS
\def \N {\mathbb{N}}
%\def \norm {Norm}
\newcommand{\merge}{Merge}
\newcommand{\E}{\mathcal{E}}
\newcommand{\nr}{N}
\newcommand{\ddom}{\mathcal{D}om}
\newcommand{\dom}{\mathcal{D}om}

%Model Checking

\newcommand{\K}{\mathit K}
\newcommand{\M}{\mathit M}
\newcommand{\trans}{\Delta_\varepsilon^{\leftarrow}}
\newcommand{\Ap}{\mathcal A_p}
\newcommand{\validates}{\models}
\newcommand{\nxt}{\mathbf X}
\newcommand{\fut}{\mathbf F}
\newcommand{\gbl}{\mathbf G}
\newcommand{\unt}{\mathbf U}
\newcommand{\rel}{\mathbf R}


%Maths
\newcommand{\suchthat}{\textrm{ s.t. }}
\newcommand{\Nat}{\mathbb N}
\newcommand{\Natmod}{\Nat[\ ]}
\newfont{\amstoto}{msbm10}
%\newcommand{\NN}{\mbox{\amstoto\char'116}}
\newcommand{\NN}{\N^*}
\newcommand{\ZZ}{\mbox{\amstoto\char'132}}

%Logic
\newcommand{\equ}{\; \Longleftrightarrow \;}
%\newcommand{\validates}{\models}

%Shortcuts
\newcommand{\tn}{T}
\newcommand{\la}{\langle}
\newcommand{\ra}{\rangle}
\newcommand{\eps}{\varepsilon}

\newcommand{\iem}{\text{ième}}
\newcommand{\er}{\text{er}}
\newcommand{\et}{\mbox{ et }}
\newcommand{\st}{\mbox{ t.q. }}
\newcommand{\oute}{\mbox{ ou }}
\newcommand{\ou}{\; \vee \;}

\newcommand{\spf}{\;\; \Longrightarrow \;\;}
\newcommand{\dbf}{\;\; \Longleftrightarrow \;\;}
\newcommand{\imp}{\; \Longrightarrow \;}
\newcommand{\rimp}{\; \Longleftarrow \;}
\newcommand{\notimp}{\; \mathrel{\not\hspace*{-1mm}\Longrightarrow} \;}
\newcommand{\sep}{\; | \;}
\newcommand{\dsep}{\; || \;}


%Term Rewriting Systems
%
\newcommand{\Q}{\mathit Q}
\newcommand{\Qf}{\Q_F}
\newcommand{\arity}{ar}
%\newcommand{\Sub}{{\cal S}}
\newcommand{\F}{\mathcal{F}}
%\newcommand{\Y}{{\cal Y}}
%\newcommand{\C}{{\cal C}}
\newcommand{\D}{{\cal D}}
\newcommand{\TF}{\mathcal{T(F)}}
\newcommand{\TFX}{\mathcal{T(F, X)}}
\newcommand{\TFQ}{\mathcal{ T(F \cup \Q)}}
\newcommand{\TFQp}{\mathcal{ T(F \cup \Q')}}
\newcommand{\TFQX}{\mathcal{ T(F \cup \Q, X)}}
\newcommand{\TFXQ}{\mathcal{ T(F, X \times \Q)}}
\newcommand{\TC}{\mathcal{ T(C)}}
\newcommand{\aut}{\langle \F, \Q, \Qf, \Delta \rangle} 
%\newcommand{\B}{\mathcal{ B}}
\newcommand{\ordlexico}{\prec}
\newcommand{\bottom}{\perp}
%\newcommand{\match}{\unlhd}

\newcommand{\var}{\mathcal{ V}ar}
\newcommand{\pos}{\mathcal{ P}os}
\newcommand{\R}{\mathcal R}
\newcommand{\RE}{{\R_{/E}}}
\newcommand{\Rep}{\R ep}
\newcommand{\desc}{\R^*}
\newcommand{\descE}{\RE^*}
\newcommand{\T}{\mathcal T}
\newcommand{\X}{\mathcal X}
\newcommand{\vars}{\mathcal V}
\newcommand{\rw}{\rightarrow}
\newcommand{\trw}{\rightarrow^{\lambda}}
\newcommand{\lrw}{\longrightarrow}
\newcommand{\xrw}{\xrightarrow}
\newcommand{\rwegal}{\rw^=}

\newcommand{\nrw}{\nrightarrow}
\newcommand{\arw}{\dashrightarrow}
%\newcommand{\rwne}[1]{\rw^*_{#1}}
\newcommand{\rwne}{\xrw{\not\varepsilon}}
%\newcommand{\rwned}{\rwne{\Delta}}
\newcommand{\rweq}{\rw^=}
%\newcommand{\rwtag}[1]{\stackrel{#1}{\rw}}
\newcommand{\rwc}{\twoheadrightarrow}
%\newcommand{\rwneq}{\rw^{\not =}}
%\newcommand{\rwe}{\f^\varepsilon}
\newcommand{\rwmod}{\rw_{\R/E}}


%substitutions
\newcommand{\plus}{\sqcup}
\newcommand{\bigplus}{\bigsqcup}
\newcommand{\id}{id}

%completion
\newcommand{\match}{\unlhd}
\newcommand{\matchi}{\lhd}
\newcommand{\matchb}{\dot{\unlhd}}
\newcommand{\matchbi}{\dot{\lhd}}
\newcommand{\Deps}{\upvarepsilon}
\newcommand{\Drw}{\upvarepsilon_{\R}}
\newcommand{\Deq}{\upvarepsilon_{=}}
%\newcommand{\Deps}{\upvarepsilon}
%\newcommand{\norm}{\downarrow}
\newcommand{\norm}{\kw{Norm}}
\newcommand{\slice}{\kw{Slice}}
\newcommand{\compl}{\kw{C}}
\newcommand{\widen}{\kw{W}}
\newcommand{\prune}{\kw{P}}

\newcommand{\simp}{\leadsto}
                          
%Tree Automata
\newcommand{\A}{\mathit A}
\newcommand{\B}{\mathit B}
\newcommand{\C}{\mathit C}
%\newcommand{\F}{\mathcal F}
\newcommand{\Pred}{\mathcal P}
\newcommand{\sub}{\Subset}
\newcommand{\Lang}{\mathcal{L}}
\newcommand{\f}{\rw}
\newcommand{\aaex}{{\mathit A}_{\R}}
\newcommand{\aaexeq}{{\mathit A}_{\R,E}}
\newcommand{\aapprox}{\A^*_{\R,E}}
\newcommand{\automaton}[3]{\la #1, #2, #3 \ra}


\newcommand{\rwA}{\rw_\A}
\newcommand{\rwR}{\rw_\R}
\newcommand{\rwB}{\rw_\B}


%\newcommand{\none}{\kw{none}}
\newcommand{\true}{\mathit{true}}
\newcommand{\false}{\mathit{false}}
\newcommand{\ifte}{\mathit{if}}

% \newcommand{\completion}{\kw{next}}
% \newcommand{\refinedCompletion}{\kw{completion}}
% \newcommand{\equations}{\kw{applyEquations }}
% \newcommand{\refinement}{\kw{refinement}}
% \newcommand{\update}{\kw{updateAndClean}}
% \newcommand{\automatonCleaning}{\kw{ automatonCleaning}}

% \newcommand{\badApproximations}{\kw{badApprox }}
% \newcommand{\badTerms}{\kw{Bad }}
% \newcommand{\moduloTerms}{\kw{moduloTerms }}
% \newcommand{\transToDelete}{\kw{transToDelete }}
% \newcommand{\reachableTerms}{\kw{reachableTerms }}
% \newcommand{\reachableStates}{\kw{reachableStates }}
% \newcommand{\states}{\kw{statesIn }}




%\newcommand{\sun}{\textsc{Sun Microsystem\ }}
%\newcommand{\java}{\textsc{Java\ }}
%\newcommand{\midp}{\textsc{Java MIDP\ }}
%\newcommand{\lande}{\textsc{Lande\ }}
%\newcommand{\danger}{\textsc{\textbf{Danger !}}\normalsize}
%\newcommand{\coq}{{\textit Coq}}

%%%%%%%%%%%%%%%%%%%%%%%%%%%%%%%%%%%%%%%%%%%%%%%%%%%%%%%%%%%%%%%%%%%%
\newcommand{\reff}[1]{[\textsc{Fig}. \ref{#1}]}
\newcommand{\refs}[1]{[sec. \ref{#1}]}
\newcommand{\refa}[1]{[cf. \ref{#1} p. \pageref{#1}]}

%Macros pour construire une regle d'inference small-step semantics
\newcommand{\instr}[1]{instruction_P(m, pc) = \kw{#1}}
\newcommand{\hs}[1]{\langle #1 \rangle}
\newcommand{\infer}[3]
{\dfrac{\instr{#1}}{\hs{#2}::sf \to_{#1}\hs{#3}::sf}}

% #2 = side-conditions : 
\newcommand{\inferi}[5]
{\dfrac{\instr{#1} \quad\ #2 }{#3::sf, \rho \to_{#1} #4::sf, #5}}
%%%%%%%%%%%%%%%%%%%%%%%%%%%%%%%%%%%%%%%%%%%%%%%%%%%%%%%%%%%%%%%%%%%%%

%\newtheorem{property}{Property}
%\newenvironment{property}{\theoremlike{Property}}{\par\medskip}
%\newtheorem{algorithm}[subsection]{Algorithm}
%\newtheorem{example}{Example}
%\newenvironment{example}{\theoremlike{Example}}{\par\medskip}

\newcounter{savetheorem}

%%% Local Variables: 
%%% coding: utf-8
%%% mode: latex
%%% TeX-master: "main"
%%% TeX-PDF-mode: t
%%% ispell-local-dictionary: "french"
%%% End: 


\begin{document}

\title{Fast Equational Abstraction Refinement \\ for Regular Tree Model Checking\\}

% Equational Abstractions and Refinement for Regular Tree Model-Checking

% Automatic Refinement for Equational Abstractions

% Efficient Automatic Refinement for Tree Automata Completion

% Tree Automata Completion with Refinement

% Yet another tree regular model checking approach

% Equations for precise and fast abstraction refinement

% Precise and fast abstraction refinement for tree regular model checking


% These are required for headers and for copyright on title page
\runningtitle{Fast Equational Abstraction Refinement for Regular Tree Model Checking}
\runningauthors{Boichut, Boyer, Genet, Legay}

% Multiple authors, sharing an affiliation, use  \affiliation{...}

%\author[lab2]{B. Boyer}{Benoît Boyer}
%\address[lab2]{IRISA - Université Rennes 1, France}
%\email{Benoit.Boyer@irisa.fr}  %optional

\author{
  Y. Boichut\inst{1}%\thanks{Funded by DOT MOT ROT},
  ,
  B. Boyer\inst{2}%\thanks{Funded by DOT MOT ROT},
  ,
  T. Genet\inst{2}%\thanks{Funded by DOT MOT ROT},
  and
  A. Legay\inst{3}%\thanks{Funded by DOT MOT ROT}
}
  
%   S. Mello, R. Shello\thanks{\ldots but no thanks!}}

\affiliation{
  LIFO - Université Orléans,
  France
}
\affiliation{
  IRISA - 
  Université Rennes 1,
  France%\\
  % \email{\{Benoit.Boyer, Thomas.Genet\}@irisa.fr}
}
\affiliation{
  INRIA - Rennes,
  France
}

% You can also use 
% 
% \institute{}{
%   Indian Institute of Science\\
%   Bangalore\\
%   \email{\{tweedledum,tweedledee,hohum\}@iisc.ernet.in}
% }  
% 
% which will leave more space between author and affiliation

% No \maketitle required

\begin{abstract} {\it Tree Regular model checking} is the name of a
  family of techniques for analyzing infinite-state systems in which
  states are represented by trees and sets of states by tree
  automata. From the verification point of view, the central problem
  is to compute the set of reachable states providing a given
  transition relation. A main obstacle is that this set is in general
  not computable in a finite time. In this paper, we propose a new
  CounterExample Guided Abstraction Refinement technique that can be
  used to check whether a set of state can be reached from the initial
  set. Contrary to existing techniques, our approach relies on
  equational abstraction to ease the definition of approximations and
  on a specific model of tree automata to avoid heavy backward
  refinement steps.
  

\end{abstract}
%DECOMMENTER POUR COMPILER l'ABSTRACT SEUL
% \end{document}

\section{Introduction}
\label{sec:introduction}

At the heart of all the techniques that have been proposed for
exploring infinite state spaces, is a symbolic representation that can
finitely represent infinite sets of states.

In this paper, we assume that states of the system are represented by
trees (terms) and set of states by tree automata. In this context, the
transition relation of the system is naturally represented by a set of
rewriting rules. It is well-known that this {\em Tree Regular Model
  Checking framework} is expressive enough to describe communication
protocols\,\cite{ALRd05} as well as a wide range of cryptographic
protocols~\cite{GenetK-CADE00,GenetTTVTT-wits03,avispa-site} and JAVA
applications\,\cite{BoichutGJL-RTA07}. 

In {\em Tree Regular Model Checking}, the main objective is to compute
an automaton representing the set of states of the system. As we are
dealing with infinite-state systems, the problem remains undecidable
and only partial solutions can be proposed. Among theses solutions, we
find the {\em predicate abstraction methodology} that was promoted by
Bouajjani et al.\,\cite{BHRV06a,BHRV06b}. The idea behind abstract
Tree Regular Model Checking consists in computing the automata
obtained after successive applications of the rewriting relation and
then use techniques coming from the predicate abstraction area in
order to over-approximate the set of reachable states. If the property
holds on the abstraction, then it also holds on the concrete
system. If a counter-example is found on the abstraction, then one has
to check if it is indeed a counter-example to the real system. If not,
this spurious counter-example must be used to refine the
abstraction. Bouajjani's algorithm, which may not terminate, proceeds
by successive abstraction/refinement until a decision can be taken.

Independently, Genet et
al.~\cite{Genet-RTA98,FeuilladeGVTT-JAR04,GenetR-JSC10} proposed {\em
  completion} that is another technique to compute an
over-approximation of the set of reachable states. The main difference
with the work in~\cite{BHRV06a} is that completion techniques use
equations to compute the abstraction~\cite{GenetR-JSC10}. Equations
gives a simple and formal semantics to abstractions on
trees~\cite{MeseguerPM-TCS08}.  Contrary to the work in
\cite{BHRV06a,BHRV06b}, completion techniques have been applied to
very complex case studies such as the verification of (industrial)
cryptographic
protocols~\cite{GenetK-CADE00,GenetTTVTT-wits03,avispa-site} and Java
bytecode applications~\cite{BoichutGJL-RTA07}. Unfortunately, these
completion techniques do not embed any notion of counter-example based
refinement.

The objective of this paper is to overcome the above mentioned problem
and propose a {\em CounterExample Guided Abstraction Refinement
  procedure} for completion algorithms. Our contribution is in two
steps. First, we propose {\em $\RE$-automaton}, that is a new
extension of tree automata. A $\RE$-automaton keeps trace of the
equations and rewriting rules applied to the initial automaton. The
automaton can be used to decide whether a term $t$ (or a set of terms)
is reachable from the set of initial states. If the procedure
concludes positively, then the term is indeed reachable. If the
procedure concludes negatively, then one has to refine the
$\RE$-automaton and restart the process. Our second contribution is to
propose a procedure that refines a $\RE$-automaton in an efficient
manner.
%%Our approach is naturally more efficient than the one of
%%Bouajjani et al. as it does not require to store the intermediary
%%computation steps and apply a reverse of the transition relation to
%%conclude whether the term is reachable or not.

\vspace{-.6cm}
\paragraph{Related work.} Regular model checking was first apply to
compute the set of reachable states of systems whose configurations
are represented by words\,\cite{BJNT00,BLW03}. The approach was then
extended to trees and first applied to very simple case
studies\,\cite{ALRd05}. In~\cite{BHV04}, Bouajjani et al. introduced
the first Counterexample abstraction techniques for regular model
checking and extended it to trees in \cite{BHRV06a}.  One of the main
difference with our work is that they have to record all the
intermediary automata to decide whether the counter-example is
spurious, while we avoid this enumeration by synthetizing the
information in a single and hopefully more compact $\RE$-automaton.
Their technique also requires to apply the reverse of the transition
relation in a successive manner. This operation, which involves
determinization and inclusion checks, may be computationally
expensive. Moreover, in their work, they represent the relation with a
tree transducer. This automaton, which encodes the full transition
relation in a whole, may be huge, while rewriting systems are
generally compact. Finally, their abstractions are defined using
automata-based predicates which are less declarative than
equations. In~\cite{BCHK08}, the authors use rewriting rules instead
of transducers, but intermediary steps are still recorded. Moreover,
we shall see that our approach is potentially more general than the
one in~\cite{BCHK08}. Finally, our work extends equational
abstractions~\cite{MeseguerPM-TCS08,Takai-RTA04} with counterexample
detection and refinement.



% The verification of infinite state systems often relies on finite state automata
% to finitely represent the set of states.  When states of a system are more
% accurately represented by trees rather than words, {\em tree automata} can be
% used to finitely represent the infinite set of tree structured states. In this
% case the transition relation between states is usually defined as a tree
% transducer. The regular tree model-checking problem has been intensively studied
% like in~\cite{AbdullaJMd-CAV02,AbdullaLDR-JLAP06}. It consists in showing that a
% set of states can (or cannot) be reached by applying the transition relation on
% a set of initial states.  To deal with systems having a non regular behavior,
% this framework has been extended in~\cite{BouajjaniHRV-ENTCS06}, so as to deal
% with regular abstractions of non regular infinite systems. In this case, since
% an over-approximation of the system behavior is considered, only unreachability
% of states can be proven.

% In parallel, the tree automata completion
% technique~\cite{genet-RTA98,FeuilladeGVTT-JAR04} has been proposed. It also
% tackles the verification of infinite state systems using tree automata. However,
% the transition relation is defined using term rewriting systems. Term rewriting
% systems offer a more concise way to define transitions between states and, as a
% result, scales up more easily than tree transducer based techniques. For
% instance, tree automata completion has been used for the verification of
% (industrial) cryptographic
% protocols~\cite{GenetK-CADE00,GenetTTVTT-wits03,avispa-site} and Java bytecode
% applications~\cite{BoichutGJL-RTA07}. Another interesting point of this
% technique is that abstractions are defined using equations~\cite{GenetR-JSC10},
% %whose expressivity is stronger than predicate abstractions
% %of~\cite{BouajjaniHRV-ENTCS06}. Equations used in completion 
% relying on the equational abstraction framework of~\cite{MeseguerPM-TCS08}.
% %with fewer restrictions though.
% Using equations yields abstractions whose expressivity is very different from
% those produced by predicate abstractions of~\cite{BouajjaniHRV-ENTCS06}.


% On the opposite, with regards to transducer-based techniques, tree automata
% completion has severals weaknesses. When dealing with abstractions, if the
% completion claims that a term is reachable, it is not possible to know if
% the term is actually reachable or if it has been found because of a too coarse
% abstraction. This problem is solved by most of the abstract model-checking
% techniques, like~\cite{BouajjaniHRV-ENTCS06} in the case of tree transducers.

% The first objective of this paper is to equip the tree automata completion
% algorithm with counter-example generation. The aim is here to discriminate, in
% the tree automata, between reachable terms and terms of the abstraction while
% preserving the efficiency of the completion algorithm. When completion stops
% because of a too coarse abstraction, the second objective is to automatically
% refine the abstraction. Here, we aim at taking advantage of the expressivity of
% equations so as to have terminating refinements where predicate abstraction
% of~\cite{BouajjaniHRV-ENTCS06} diverges. 

% Finally, though it will not be discussed in this paper, 
% the objective is to automatically verify Java programs re-using the framework 
% of~\cite{BoichutGJL-RTA07} and the Copster~\cite{copster} tool which compiles
% Java bytecode programs into term rewriting systems.

%%% Local Variables: 
%%% mode: latex
%%% TeX-master: "main"
%%% TeX-PDF-mode: t
%%% End: 


\section{Definitions}
\label{sec:definitions}
% \comments{
%   Toutes les definitions à propos de Réécriture, substitutions ($\Q$-subst inclues) et 
%   Automates d'arbres.
%   Attention : \\
%   pas de def independante de normalized transitions et epsilon-transition
%   uniquement une def globale pour les automates 
%   suivie du langage d'un automate d'arbres sans format def.
% }

In this section, we introduce some definitions and concepts that will
be used through the rest of the paper (see
also\,\cite{BaaderN-book98,TATA,GilleronTison-FI95}). Let $\F$ be a
finite set of symbols, each associated with an arity function, and let
$\X$ be a countable set of {\em variables}. $\TFX$ denotes the set of
{\em terms} and $\TF$ denotes the set of {\em ground terms} (terms
without variables). The set of variables of a term $t$ is denoted by
$\var(t)$. A {\em substitution} is a function $\sigma$ from $\X$ into
$\TFX$, which can be uniquely extended to an endomorphism of $\TFX$. A
{\em position} $p$ for a term $t$ is a word over $\NN$. The empty
sequence $\lambda$ denotes the top-most position. The set $\pos(t)$ of
positions of a term $t$ is inductively defined by $\pos(t)= \{
\lambda\} $ if $t \in \X$ and $\pos(f(t_1,\dots,t_n)) = \{ \lambda \}
\cup \{i.p \mid 1 \leq i \leq n \et p \in \pos(t_i) \}$ otherwise.  If
$p \in \pos(t)$, then $t|_p$ denotes the subterm of $t$ at position
$p$ and $t[s]_p$ denotes the term obtained by replacement of the
subterm $t|_p$ at position $p$ by the term $s$.


A {\em term rewriting system} (TRS) $\R$ is a set of {\em rewrite
  rules} $l \rw r$, where $l, r \in \TFX$, $l \not \in \X$, and
$\var(l) \supseteq \var(r)$.  A rewrite rule $l \rw r$ is {\em
  left-linear} if each variable of $l$ occurs only once in $l$.  A TRS
$\R$ is left-linear if every rewrite rule $l \rw r$ of $\R$ is
left-linear.  The TRS $\R$ induces a rewriting relation $\rw_{\R}$ on
terms as follows. Let $s, t\in \TFX$ and $l \rw r \in \R$, $s \rw_{\R}
t$ denotes that there exists a position $p\in\pos(t)$ and a
substitution $\sigma$ such that $s|_p= l\sigma$ and $r=s[r\sigma]_p$.
%  $s \rw^p_{\R} t$ denotes that there exists a
% position $p\in\pos(t)$ and a substitution $\sigma$ such that $s|_p= l\sigma$ and
% $r=s[r\sigma]_p$. Note that the rewriting position $p$ can generally be omitted,
% i.e. we generally write $s \rw_{\R} t$. 
The reflexive transitive closure of $\rw_{\R}$ is denoted by
$\rw^*_{\R}$ and $s \rw^!_\R t$ denotes that $s\rw^*_\R t$ and $t$ is
irreducible by $\R$. The set of $\R$-descendants of a set of ground
terms $I$ is $\desc(I) = \{t \in \TF \sep \exists s \in I \st s
\rw^*_{\R} t \}$.  An {\em equation set} $E$ is a set of {\em
  equations} $l = r$, where $l, r \in \TFX$.  For all equation $l = r
\in \R$ and all substitution $\sigma$ we have $l\sigma =_E r\sigma$.
The relation $=_E$ is the smallest congruence such that for all
substitution $\sigma$ we have $l\sigma = r\sigma$. Given a TRS $\R$
and a set of equations $E$, a term $s\in\TF$ is rewritten modulo $E$
into $t\in\TF$, denoted $s \rwmod t$, if there exist $s\in\TF'$ and
$t'\in\TF$ such that $s=_Es'\rw_\R t'=_E t$. Thus, the set of
$\R$-descendants modulo $E$ of a set of ground terms $I$ is $\descE(I)
= \{t \in \TF \sep \exists s \in I \st s \rwmod^* t \}.$

We now define tree automata that are used to recognize possibly
infinite sets of terms. Let $\Q$ be a finite set of symbols with arity
$0$, called {\em states}, such that $\Q \cap \F= \emptyset$.  $\TFQ$
is called the set of {\em configurations}. A {\em transition} is a
rewrite rule $c \f q$, where $c$ is a configuration and $q$ is state.
A transition is {\em normalized} when $c = f(q_1, \ldots, q_n)$, $f
\in \F$ whose arity is $n$, and $q_1, \ldots, q_n \in \Q$. A {\em
  $\varepsilon$-transition} is a transition of the form $q \f q'$
where $q$ and $q'$ are states.

\begin{definition}[Bottom-up nondeterministic finite tree automaton]
\label{def:automata}
A bottom-up nondeterministic finite tree automaton (tree automaton for
short) over the alphabet $\F$ is a tuple $\A= \langle \F, \Q,
\Q_F,\Delta \rangle$, where $\Q_F \subseteq \Q$, $\Delta$ is a set of
normalized transitions and $\varepsilon$-transitions.
\end{definition}

\noindent
The transitive and reflexive {\em rewriting relation} on $\TFQ$
induced by all the transitions of $\A$ is denoted by $\f_{\A}^*$. The
tree language recognized by $\A$ in a state $q$ is $\Lang{}(\A,q) =
\{t \in \TF \sep t \rw^*_\A q \}$. The language recognized by $\A$ is
$\Lang{}(\A) = \bigcup_{q \in \Q_F} \Lang{}(\A, q)$. 

% Let $\TFX$ be a
% set of terms represented by an automaton $A$, we use $\R(A)$ to denote
% the automaton that represents the terms that can be obtained from
% $\TFX$ by applying $\R$.



  %A tree language is regular if and only if it can be recognized by a tree automaton.
  %We say that $t$ is a {\em canonical term} of the state $q$, if $t \f^!_\A q$.

% \begin{example}
%    Let $\A$ be the tree automaton $\langle \F, \Q, \Q_F, \Delta \rangle$ such that $\F=\{f,g,a,b\}$, $\Q= \{q_0, q_1, q_2\}$, $\Q_F=\{q_0\}$,
%    $\Delta= \{f(q_0)
%    \rw q_0, g(q_1) \rw q_0, a \rw q_1, b \rw q_2 \}$ and $\Delta_{\epsilon}=\{q_2 \rw q_1 \}$. In $\Delta$, transitions are
%    {\em normalized}. A transition of the form $f(g(q_1)) \f q_0$ is not normalized. The term $g(a)$
%    is a term of $\TFQ$ (and of $\TF$) and can be rewritten by $\Delta$ in the following way:
%    $g(a) \rwne_\A g(q_1)
%    \rwne_\A q_0$. Hence $g(a)$ is a canonical term of $q_1$. Note also that $b \rw_\A q_2 \rw_\A q_1$.
%    Hence, $\Lang{}(\A, q_1)=
%    \{a, b\}$ and $\Lang{}(\A)=\Lang{}(\A, 
%    q_0) = \{g(a), g(b),f(g(a)), f(f(g(b))),\ldots\}=\{f^*(g([a|b]))\}$.
%   \end{example}




%%% C'est peut �tre correct sans cette def inutile ??? !!!
%%% => compacite + determinisme de la normalisation...

% \begin{property}[$\f^!$ deterministic]
%   \label{prop:deterministic}
%   If $\Delta$ contains two normalized transitions of the form 
%   $f(q_1, \dots, q_n) \rw q$ and $f(q_1, \dots, q_n) \rw q'$, it means $q = q'$. 
%   This ensures that the rewriting relation $\f^!$ is deterministic.
% \end{property}



%%%%% FIN R-AUTOMATON : LA SUITE AILLEURS





%%% Local Variables: 
%%% mode: latex
%%% TeX-master: "main"
%%% End: 


% Version light du papier de TACC

% \section{Tree Automata Completion}
%\label{section:completion}

Given a tree automaton $\A$ and a TRS $\R$, the tree automata completion
algorithm, proposed in~\cite{Genet-RTA98,FeuilladeGVTT-JAR04}, computes a \emph{tree complete
automaton} $\aaex^*$ such that $\Lang{}(\aaex^*)=\desc(\Lang{}(\A))$ when it is
possible (for some of the classes of TRSs where an exact computation is
possible, see~\cite{FeuilladeGVTT-JAR04}), and such that $\Lang{}(\aaex^*)
\supseteq \desc(\Lang{}(\A))$ otherwise. 
In this paper, we only consider the exact case.

The tree automata completion with $\varepsilon$-transtions works as follow.
From $\A=\aaex^0$ completion builds a sequence $\aaex^0.\aaex^1\ldots\aaex^k$ of automata such that if
$s\in\Lang{}(\aaex^i)$ and $s\f_{\R} t$ then $t\in\Lang{}(\aaex^{i+1})$. Transitions of $\aaex^i$ are denoted by the set
$\Delta^i \cup \Deps^i$. Since for every tree automaton, there exists a
deterministic tree automaton recognizing the same language, we can assume
that initially $A$ has the following properties:

\begin{property}[$\rwne$ deterministic]
  \label{prop:deterministic}
  If $\Delta$ contains two normalized transitions of the form 
  $f(q_1, \dots, q_n) \rw q$ and $f(q_1, \dots, q_n) \rw q'$, it means $q = q'$. 
  This ensures that the rewriting relation $\rwne$ is deterministic.
\end{property}

\begin{property}
  \label{prop:wellinitial}
  For all state $q$ there is at most one normalized transition $f(q_1, \dots, q_n) \rw q$
  in $\Delta$. This ensures that if we have $t \rwne q$ and $t' \rwne q$ then $t = t'$.
\end{property}

If we find a fixpoint automaton $\aaex^k$ such that $\desc(\Lang{}(\aaex^k)) =
\Lang{}(\aaex^k)$, then we note $\aaex^*=\aaex^k$ 
and we have $\Lang{}(\aaex^*) \supseteq \desc(\Lang{}(\aaex^0))$~\cite{FeuilladeGVTT-JAR04}.
% , or $\Lang{}(\aaex^*)\supseteq
%\desc(\Lang{}(\A))$ if $\R$ is not in one class of~\cite{FeuilladeGVTT-JAR04}.
To build $\aaex^{i+1}$ from $\aaex^{i}$, we achieve a \textit{completion step}
which consists of finding \textit{critical pairs} between $\f_{\R}$ and
$\f_{\aaex^i}$. To define the notion of critical pair, we extend the definition
of substitutions to the terms of $\TFQ$. For a substitution $\sigma:\X\mapsto\Q$ and
a rule $l\f r \in \R$, a critical pair is an instance $l\sigma $ of $l$ such
that there exists $q\in\Q$ satisfying $l\sigma \f^*_{\aaex^i}q$ and $l\sigma
\f_{\R} r\sigma$. Note that since
$\R$, $\aaex^i$ and the set $\Q$ of states of $\aaex^i$ are finite, there is only a finite
number of critical pairs. For every critical pair detected between $\R$ and
$\aaex^i$ such that we do not have a state $q$' for which $r\sigma \rwne_{\aaex^i}q'$ and $q' \rw q \in \Deps^i$, the
tree automaton $\aaex^{i+1}$ is constructed by adding new transitions $r\sigma \rwne q'$ to $\Delta^i$
and $q' \rw q$ to $\Deps^i$ such that $\aaex^{i+1}$ recognizes $r\sigma$ in $q$, i.e. $r\sigma \f^*_{\aaex^{i+1}} q$, see
Figure~\ref{fig:cp}.
%\vspace*{-5mm}
\begin{figure}[!ht]
  {\small
    \[
    \xymatrix{
      l\sigma \ar[r]_-{\R}\ar[d]^-{*}_-{\aaex^i} & r\sigma \ar[d]_-{\not\varepsilon}^{\aaex^{i+1}}\\
      q & q' \ar[l]^-{\aaex^{i+1}}
    }
    \]}
  \vspace*{-7mm}
  \caption{\footnotesize A critical pair solved \label{fig:cp}
  }
\end{figure}
%\vspace*{-3mm}
%%%%%%%
It is important to note that we consider the critical pair only if the
last step of the reduction $l\sigma \f^*_{\aaex^i}q$, is the last step of rewriting is not a $\varepsilon$-transition.
Without this condition, the completion computes the transitive closure of the
expected relation $\Deps$, and thus looses precision. %waste of information ?
%%%%%%
The transition $r \sigma \f q'$ is not necessarily a normalized
transition of the form $f(q_1, \ldots, q_n) \f q'$ and so it has to be normalized
first. Instead of adding $r\sigma \rw q'$ we add $\norm(r\sigma \rw q')$ to
transitions of $\Delta^i$.
Here is the $\norm$ function used to normalize transitions. Note that, in 
this function, transitions are normalized using new states of $\Q_{new}$.
%As we will see in Lemma~\ref{lemma:approx}, this has no effect on the
%safety of the normalization but only on its precision.

\begin{definition} [$\norm$] Let $\A=\la \F, \Q, \Qf, \Delta\cup\Deps\ra$ be a tree automaton, $\Q_{new}$ a
  set of {\em new} states such that $\Q\cap \Q_{new} = \emptyset$, $s \in \TFQ$ and $q'\in
  \Q$.
  The normalization of the transition $s \rw q'$ is done in two mutually inductive steps.
  The first step denoted by $\norm(s \rw q'\sep\Delta)$, we rewrite $s$ by $\Delta$ until rewriting 
  is impossible: we obtain a unique configuration $t$ if $\Delta$ respects the property~\ref{prop:deterministic}.
  The second step $\norm'$ is inductively defined by:
  % TODO : A REVOIR 
  \begin{itemize}
%  \item 
%    $\norm'(t \rw q')= \emptyset$ if $t\in\Q$,
  \item
    $\norm'(f(t_1, \ldots, t_n) \rw q\sep\Delta)= \Delta \cup \{f(t_1, \ldots,
    t_n) \rw q\}$ if $\forall i = 1\ldots n:\ t_i \in \Q$
  \item 
    $\norm'(f(t_1, \ldots, t_n) \rw q \sep \Delta)= \norm(f(t_1, \ldots, q_i,\ldots, t_n) \rw q\sep \norm'(t_i\rw q_i\sep \Delta)\ )$
    where $t_i$ is subterm s.t. $t_i \in \TFQ\setminus \Q$ and $q_i \in \Q_{new}$.
  \end{itemize}
\end{definition}

 \begin{lemma}
   \label{lem:welldefined}
   If the property \ref{prop:deterministic} holds for $\aaex^i$ then it holds also for $\aaex^{i+1}$.
 \end{lemma}

 \begin{proof}[Intuition]
   The determinism of $\rwne$ is preserved by $\Delta$, since when a new set of transitions
   is added to $\Delta$ for a subterm $t_i$, we rewrite all other subterms $t_j$ with the new $\Delta$ until rewriting is impossible 
   before resuming the normalization. Then, if we try to add to $\Delta$ a transition $f(q_1, \dots, q_n) \rw q$
   though there exists a transition $f(q_1, \dots, q_n) \rw q'\in \Delta$, it means that the configuration $f(q_1, \dots, q_n)$ 
   can be rewritten by $\Delta$. This is a contradiction : when we resume the normalization all subterms $t_i$ can not be rewritten 
   by the current $\Delta$. So, we never add a such transition to $\Delta$. The normalization produces a new set of transitions $\Delta$
   that preserves the property \ref{prop:deterministic}.
 \end{proof}

It is very important to remark that the transition $q'\rw q$ in Figure~\ref{fig:cp}
creates an order between the language recognized by $q$ and the one recognized by
$q'$.  Intuitively, we know that for all substitution $\sigma' : \X \rw \TF$ such that $l\sigma'$ is
a term recognized by $q$, it is rewritten by $\R$ into a canonical term ($r\sigma'$) of $q'$.
By duality, the term $r\sigma'$ has a parent ($l\sigma'$) in the state $q$.
Extending this reasoning, $\Deps$ defines a relation between canonical
terms. This relation follows rewriting steps at the top position and forgets
rewriting in the subterms.

\begin{definition}[$\arw$]
  Let $\R$ be a TRS. For all terms $u$ $v$, we have $u \arw_{\R} v$ iff there exists
  $w$ such that $u \rw_\R^* w$, $w \trw_{\R} v$ and there is not
  rewriting on top position $\lambda$ on the sequence denoted by $u
  \rw_\R^* w$.
  
\end{definition}
%A d�tailler

In the following, we show that the completion builds a tree automaton where
the set $\Deps$ is an \emph{abstraction} $\arw_{\R_i}$ of the rewriting relation $\rw_\R$, for
any relevant set $\R_i$.


\begin{theorem}[Correctness]
  \label{thm:correct}
  Let be $\aaex^*$ a complete tree automaton %obtained from $\R$ and $\A_0$,
  such that $q'\rw q$ is a $\varepsilon$-transition of $\aaex^*$. Then, for all canonical
  terms $u$ $v$ of states $q$ and $q'$ respectively s.t. $q'\rw q$, we have :
  \vspace*{-5mm}
  
  \[\xymatrix{ 
    u \ar[d]^-{\not\varepsilon}_-{\aaex^*} \ar@{-->}[r]_-{\R}
                &v \ar[d]^-{\not\varepsilon}_-{\aaex^*}\\ 
    q &q' \ar[l] }
  \]
\end{theorem}

First, we have to prove that the property \ref{prop:deterministic} is preserved by completion.
To prove theorem \ref{thm:correct}, we need a stronger lemma.

\begin{lemma}[]
  \label{lem:correct}
  Let be $\aaex^*$ a complete tree automaton, $q$ a state of $\aaex^*$ and $v\in\Lang{}(\aaex^*,q)$.
  Then, for all canonical term $u$ of $q$, we have $u \rw_\R^* v$. 
\end{lemma}

\begin{proof}[Proof sketch]
  
  The proof is done by induction on the number of  completion steps
  to reach the post-fixpoint $\aaex^*$ : we are going to show that
  if $\aaex^i$ respects the property of lemma~\ref{lem:correct},
  then $\aaex^{i+1}$ also does.
  
  The initial $\aaex^0$ respects the expected property~: we consider
  any state $q$ and a canonical term $t$ of $q$: since no completion
  step was done, $\aaex^0$ has no $\varepsilon$-transitions. It means
  that for all term $t'\rwne q$. Thanks to the property
  \ref{prop:wellinitial}, we have $t = t'$ and obviously $t \rw^*_\R
  t'$.

  Now, we consider the normalization of a transition of the form $r\sigma \rwne q'$
  such that $l\sigma \rw^*_{\aaex^i} q$ with $\Delta$ the ground transition set and $\Deps$ the $\varepsilon$-transition set of $\aaex^i$.
  We show that the property is true for all new states (including $q'$). 
  Then, in a second time, we will show that it is true for state $q$,
  if we add the second transition of completion: $q'\rw q$. 
  
  %
  Let us focus on the normalization of $\norm'(r\sigma \rw q'\sep \Delta)$ where for
  any existing state $q$ and for all $u\ v \in \TF$ such that $v \rw_{\Delta\cup\Deps} q$ and $u \rw_{\Delta} q$, we have $u \rw_\R^* v$.
  We show that for all $t \in \TFQ$, if we have $\Delta' = \norm'(t \rw q'\sep \Delta)$, for all $u\ v \in \TF$ such that $v \rw_{\Delta'\cup\Deps} q'$ and $u \rw_{\Delta'} q$, we have $u \rw_\R^* v$. 
  The induction is done on the %decreasing
  number of symbols of $\F$ used to build $t$.

  First case $\norm'(t \rw q \sep \Delta)$ where $t = f(q_1,\dots, q_n)$ : we define $\Delta'$ by adding the transition $f(q_1, \dots, q_n) \rw q$
  to $\Delta$, where $q$ is a new state. Then, for all substitutions $\sigma' : \Q \mapsto \TF$ such that $t\sigma' \rw_{\Delta\cup\Deps} q$, and all 
  substitutions $\sigma'' : \Q \mapsto \TF$ such that $t\sigma'' \rw_{\Delta'} q$ we aim at proving that $t\sigma''
  \rw_\R^* t\sigma'$. Since each state $q_i$
  is already defined, using the hypothesis on $\Delta$ we deduce that $\sigma''(q_i) \rw^*_\R \sigma'(q_i)$. This implies that $t\sigma'' \rw_\R^* t\sigma'$, the property 
  also holds for $\Delta'$.

  Second case $\norm'(t \rw q \sep \Delta)$ where $t = f(t_1,\ldots,t_n)$: we select $t_i$ a subterm of $t$, obviously the number
  of symbols is strictly lower to the number of symbols of $t$.
  By induction, for the normalization of $\norm'(t_i \rw q_i\sep \Delta)$ we have a new 
  set $\Delta'$ that respects the expected property. Then, we normalize $t$ into $t' = f(t'_1, \dots, q_i, \dots, t'_n)$, 
  the term obtained after rewriting with $\Delta'$ thanks to $\norm$. Since $t_i \not\in \Q$, the number of
  symbols of $\F$ in $t' = f(t_1, \dots, q_i, \dots, t_n)$ is strictly smaller than the number of symbols of $ \F$ in $t$. Note 
  that rewriting $t'$ with $\Delta'$ can only decrease the number of symbols of $\F$ in $t'$.
  Since $t'$ has a decreasing number of symbols and $\Delta'$ respects the property we can deduce by induction
  that we have $\Delta'' = \norm'(t'\rw q\sep \Delta')$ such that for all $v \rw_{\Delta''\cup\Deps} q'$ and $u \rw_{\Delta''} q$, $u \rw_\R^* v$.
  
  So, we conclude that the normalization $\norm'(r\sigma \rw q'\sep \Delta)$ computes $\Delta'$ the set of ground transitions for $\aaex^{i+1}$.
  For all terms $u$ $v$ such that $u \rw_{\Delta'\cup\Deps} q'$ and $u \rw_{\Delta'} q'$ we have $u \rw_\R^* v$. 

  Now, let us consider the second added transition $q' \rw q$ to $\Deps$, all canonical terms
  $r\sigma''$ of $q'$, and all terms $l\sigma''' \in
  \Lang{}(\aaex^i, q)$ such that $l\sigma''' \rw_\R r\sigma'''$ and
  $r\sigma''' = r\sigma''$.  By hypothesis on $\aaex^i$, we know that every canonical term $u$ of $q$
  we have $u \rw_\R^*
  l\sigma'''$. By transitivity, we have $u \rw_\R^* r\sigma''$.  The
  last step consists in proving that for all terms of all states of
  $\aaex^{i+1}$, the property holds: this can be done by induction on
  the depth of the recognized terms.
\end{proof}

The theorem \ref{thm:correct} is shown by considering the introduction of the
transition $q' \rw q$. By construction, there exists a substitution $\sigma : \X \mapsto \Q$ and a rule
$l \rw r \in \R$ such that we have $l\sigma \rw^*_{\aaex^*} q$ and $r\sigma \rwne_{\aaex^*} q'$. We consider all substitution  
$\sigma' : \X \mapsto \TF$ such that for each variable $x \in \vars(l)$, $\sigma'(x)$ is a canonical term
of the state $\sigma(x)$. Obviously, using the result of the lemma \ref{lem:correct},
for all canonical term $u$ of $q$ we have $u \rw^*_\R l\sigma'$. Since the last step of rewriting 
in the reduction $l\sigma \rw^*_{\aaex^*} q$ is not a $\varepsilon$-transition, we also deduce that $l\sigma'$ is not produced
by a rewriting at the top position of $u$ whereas it is the case for $r\sigma'$ and we have $u \arw_{\R} r\sigma'$.

 
\begin{theorem}[Completeness]
  Let $\aaex^*$ be a complete tree automaton, %obtained from $\R$ and $\A_0$,
  $q,q'$ states of $\aaex^*$ and $u,v \in \TF$ such that $u$ is a canonical term of $q$
  and $v$ is a canonical term of $q'$. If $u \arw_\R v$ then there exists a $\varepsilon$-transition $q' \rw q$ in $\aaex^*$.
\end{theorem}
\begin{proof}[Proof sketch]
  By definition of $u \arw_\R v$ there exists a term $w$ such that $u \rw_\R^* w$ and
  and there exists a rule $l \rw r \in \R$ and a substitution $\sigma : \X \mapsto \TF$ such that 
  $w = l\sigma$ and $v = r\sigma$.
  Since $\aaex^*$ is a complete tree automaton, it is closed by rewriting. This means 
  that any term obtained by rewriting any term of $\Lang{}(\aaex^*, q)$ is also in $\Lang{}(\aaex^*, q)$. This
  property is true in particular for the terms $u$ and $w$. 
  Since $w$ is rewritten in $q$ by transitions of $\aaex^*$, we can define
  a second substitution $\sigma' : \X \mapsto \Q$ such that $l\sigma \rw^*_{\aaex^*} l\sigma' \rw^*_{\aaex^*} q$.
  Using again the closure property of $\aaex^*$, we know that the critical pair $l\sigma' \rw_\R r\sigma'$
  and $l\sigma' \rw^*_{\aaex^*} q$ is solved by adding the transitions $r\sigma' \rwne_{\aaex^*} q''$ and $q'' \rw q$. Since the property \ref{prop:deterministic}
  is preserved by completion steps, we can deduce that $q'' = q'$ which means $q' \rw q$.
\end{proof}
%Bon il faut mettre du baratin ici, la preuve ne marche pas...


%When using only new states to normalize all the new transitions occurring in all
%the completion steps, completion is as precise as possible.

% However, doing so, completion is likely not to terminate (because of general undecidability
% results~\cite{GilleronTison-FI95}).  Enforcing termination of completion can be
% easily done by bounding the set of new states to be used with $\norm$ during the
% whole completion. We then obtain a finite tree automaton over-approximating the
% set of reachable states. The fact that normalizing with any set of states (new
% or not) is {\em safe} is guaranteed by the following simple lemma. For the
% general safety theorem of completion see~\cite{FeuilladeGVTT-JAR04}.

% \begin{lemma}
% \label{lemma:approx}
% For all tree automaton $\A=\aut$, $t\in \TFQ\setminus \Q$ and $q\in\Q$, if $\Pi=\norm(t \rw
% q)$ whatever the states chosen in $\norm(t \rw q)$ we have $t \rw^*_{\Pi} q$.
% \end{lemma}
% \begin{proof}
% This can be done by a simple induction on transitions~\cite{FeuilladeGVTT-JAR04}.
% %to normalize, see~\cite{FeuilladeGVTT-JAR04}.
% \end{proof}

% To let the user of completion guide the approximation, we use two different
% tools: a set $\nr$ of {\em normalization rules} (see~\cite{FeuilladeGVTT-JAR04})
% and a set $\E$ of {\em approximation equations}. Rules and equations can be
% either defined by hand so as to prove a complex
% property~\cite{GenetTTVTT-wits03}, or generated automatically when the property
% is more standard~\cite{BoichutHKO-AVIS04}. Normalization rules can be seen as a
% specific strategy for normalizing new transitions using the $\norm$
% function. We have seen that Lemma~\ref{lemma:approx} is enough to
% guarantee that the chosen normalization strategy has no impact on the safety of
% completion. Similarly, for our checker, we will see in Section~\ref{sec:closure}
% that the related \coq\ safety proof can be carried out independently of the
% normalization strategy (i.e. set $N$ of normalization rules).  On the opposite, the
% effect of approximation equations is more complex and has to be studied more
%%carefully.


% On the one side, normalization
% rules define which states are to be used to normalize a transition with
% $\norm$. When using $\nr$ to guide the normalization, we note $\norm_{\nr}$ the
% normalization function.  On the other side, approximation equations define some
% approximated equivalence classes. Equations of $\E$ are applied directly on
% $\A^{i+1}_{\R}$ to merge together the states whose recognized terms are in the
% same equivalence class w.r.t. $\E$.
%
% For all $s,l_1, \ldots, l_n \in\TFQX$ and for all
% $x,x_1, \ldots, x_n \in \Q\cup\X$, the general form for a normalization rule
% is:
% \[[s \rw x] \rw [l_1 \rw x_1, \ldots, l_n \rw x_n]\] where the expression $[s
% \rw x]$ is a pattern to be matched with the new transitions $t \rw q'$ obtained
% by completion. The expression $ [l_1 \rw x_1, \ldots, l_n \rw x_n]$ is a set of
% rules used to normalize $t$. To normalize a transition of the form $t \rw q'$,
% we match $s$ with $t$ and $x$ with $q'$, obtain a substitution $\sigma$ from the
% matching and then we normalize $t$ with the rewrite system $\{l_1\sigma \rw
% x_1\sigma, \ldots, l_n \sigma \rw x_n\sigma\}$. Furthermore, if $\forall
% i=1\ldots n: x_i\in \Q$ or $x_i \in \var(l_i) \cup \var(s) \cup \{x\}$ then
% since $\sigma: \X \mapsto \Q$, $x_1 \sigma, \ldots, x_n\sigma$ are necessarily
% states. 
% If a transition cannot be fully normalized using approximation rules
% $\nr$, normalization is finished using some new states, see Example
% \ref{example:approx}. 
% %We denote by $\norm_\nr(t \rw q')$ the set of transitions
% %obtained by the normalization of $t \rw q'$ by normalization rules $\nr$.
% An approximation equation is of the form $u=v$ where $u,v\in\TFX$.  Let $\sigma:
% \X \mapsto \Q$ be a substitution such that $u\sigma \rw_{\A_{\R}^{i+1}} q$,
% $v\sigma \rw_{\A_{\R}^{i+1}} q'$ and $q\neq q'$, see
% Figure~\ref{fig:merge}. Then, we know that there exists some terms recognized by
% $q$ and some recognized by $q'$ which are equivalent modulo $\E$. A correct
% over-approximation of $\aaex^{i+1}$ consists in applying the $\merge$ function to
% it, i.e. replace $\aaex^{i+1}$ by $\merge(\aaex^{i+1},q, q')$, as long as an
% approximation equation of $\E$ applies. The $\merge$ function, defined below,
% merges states in a tree automaton.  See~\cite{BoyerGJ-RR08} for examples of
% completion and approximation.
% \begin{definition}[$\merge$]
%   Let $\A= \langle \F, \Q, \Q_F, \Delta \rangle$ be a tree automaton and
%   $q_1,q_2$ be two states of $\A$. We denote by $\merge(\A,q_1, q_2)$ the tree
%   automaton where every occurrence of $q_2$ is replaced by $q_1$ in $\Q$, $\Q_F$
%   and in every left-hand side and right-hand side of every transition of
%   $\Delta$.
% \end{definition}

% The following examples illustrate completion and how to carry out an
% approximation, using equations, when the language $\desc(\Lang(\A)) $ is not
% regular.

% \label{example:merge}
% Let $\R=\{g(x,y) \rw g(f(x),f(y))\} $ and let $\A$ be the tree automaton such
% that $\Q_F=\{q_f\}$ and $\Delta=\{a \rw q_a, g(q_a,q_a)\rw q_f\}$. Hence
% $\Lang(\A)= \{g(a,a)\}$ and $\desc(\Lang(\A))=\{g(f^n(a),f^n(a))~|~n\geq
% 0\}$. Let $\E=\{f(x)=x\}$ be the set of approximation equations. During the
% first completion step on $\aaex^0=\A$,  we find $\sigma=\{x \mapsto q_a\}$ and
% the following critical pair

% {\small
% $$
% \xymatrix{
%   g(q_a,q_a) \ar[r]_{\R}\ar[d]^{*}_{\aaex^0} & g(f(q_a),f(q_a)) \ar@/^1.2pc/[ld]_{*}^{\aaex^{1}}\\
%   q_f & %\ar[l]^{\A_{i+1}} q'
% }
% $$}

% Hence, we have to add the transition $g(f(q_a),f(q_a)) \rw q_f$ to $\aaex^0$ to
% obtain $\aaex^1$. This transitions can be normalized in the following way:
% $\norm(g(f(q_a),f(q_a)) \rw q_f)=\{g(q_1, q_2) \rw q_f, f(q_a)\rw q_1,f(q_a)\rw
% q_2 \}$ where $q_1$ and $q_2$ are new states. Those new states and transitions
% are added to $\aaex^0$ to obtain $\aaex^1$. On this tree automaton, we can apply
% the equation $f(x)=x$ of $\E$ with the substitution $\sigma=\{x\mapsto q_a\}$:

% {\small
% $$\xymatrix{
% f(q_a) \ar@{=}[r]_{\E}\ar[d]_{\aaex^{1}}^{*} & q_a \ar[d]_{*}^{\aaex^{1}}\\
% q_1 & q_a
% }
% $$}

% Hence, we can replace $\aaex^1$ by
% $\merge(\aaex^1,q_1,q_a)$ where $\Delta$ is $\{ a \rw q_1, g(q_1,q_1)\rw q_f,
% g(q_1, q_2) \rw q_f, f(q_1) \rw q_1, f(q_1) \rw q_2\}$. Similarly, in this last
% tree automaton, we have 

% {\small $$\xymatrix{
% f(q_1) \ar@{=}[r]_{\E}\ar[d]_{\aaex^{1}}^{*} & q_1 \ar[d]_{*}^{\aaex^{1}}\\
% q_2 & q_1
% }
% $$}

% and we can thus apply $\merge(\aaex^1, q_2, q_1)$. Finally, the value of
% $\Delta$ for $\aaex^1$ approximated by $\E$ is $\{a \rw q_2, g(q_2,q_2)\rw q_f,
% f(q_2) \rw q_2 \}$. Now, the only critical pair that can be found on $\aaex^1$ is
% joinable:

% {\small
% $$
% \xymatrix{
%   g(q_2,q_2) \ar[r]_{\R}\ar[d]^{*}_{\aaex^1} & g(f(q_2),f(q_2)) \ar@/^1.2pc/[ld]_{*}^{\aaex^{1}}\\
%   q_f & %\ar[l]^{\A_{i+1}} q'
% }
% $$}

% Hence, we have $\aaex^*=\aaex^1$ and
% $\Lang(\aaex^*)=\{g(f^n(a),f^m(a))~|~n,m\geq 0\}$ which is an over-approximation
% of $\desc(\Lang(\A))$.
% \end{example}

% The tree automata completion algorithm and the approximation mechanism are
% implemented in the \timbuk~\cite{timbuk-site} tool. On the previous example, once
% the fixpoint automaton $\aaex^*$ has been computed, it is possible to check
% whether some terms are reachable, i.e. recognized by $\aaex^*$ or not. This
% can be done using tree automata 
% intersections~\cite{FeuilladeGVTT-JAR04}. 



%%% Local Variables: 
%%% mode: latex
%%% TeX-master: "main"
%%% End: 


%%% Motiver l'introduction du $\RE$-Automaton
\section{$\RE$-Automaton for refining}
\label{sec:re-automaton}


In this section, we propose to extend the completion technique with a
counter-example based procedure. Contrary to existing approaches that
have to perform a backward propagation from the bad term to the set of
initial state, we propose to extend the transition relation of tree
automata with information about the rewriting rules and equations that
have been applied to the initial automaton.

More precisely, we use the set $\Drw$ to distinguish a term
from its successors that has been obtained by applying one or several
rewriting rules. Instead of merging states according to the set of
equations, our model link them with transitions that belongs to the
set $\Deq$.

% $\Deq$ denotes a subset
% of the equivalence relation induced by the set of equations.
% Moreover, we will be able to know that a term recognized using a
% transition of $\Deq$ is the result of a widening step.  The
% example~\ref{ex:semantics} illustrates the principle of a
% $\RE$-Automaton.


%We now give the formal definition for $\RE$-automata.

\begin{definition}[$\RE$-automaton]
  \label{def:re-automaton}
  Given a TRS $\R$ and a set $E$ of equations, a $\RE$-automaton $\A$
  is a tuple $\langle \F, \Q, \Q_F, % \Delta_0 \cup
  \Delta \cup \Drw \cup \Deq\rangle$.  $\Delta$ is a set of normalized
  transitions. $\Deq$ is a set of $\varepsilon$-transitions. $\Drw$ is
  a set of $\varepsilon$-transitions labeled by $\top$ or conjunctions 
  over predicates of the form $Eq(q, q')$ where $q, q' \in \Q$, and $q \rw
  q' \in \Deq$. %$Eq(q, q')$ holds iff $q \rw q' \in \Deq$.
\end{definition}


\begin{example}
  \label{ex:semantics}
  Let $\A$ be the $\RE$-automaton recognizing the program
  where $I = f(a)$, $\R = \{f(c) \rw g(c),\ a \rw b\}$ and $E = \{ b = c\}$.

  \centering
  \medskip
  \begin{tabular}{lc}
  \hspace{-.3cm}
  \begin{minipage}{.75\linewidth}
      In $A$, the equality $b = c$ is denoted by two transitions $q_c
      \rw q_b$ and $q_b \rw q_c$ of $\Deq$, assuming that $b$, $c$ are
      recognized into $q_b$, $q_c$, respectively. For the state $q_c$,
      the transition $q_b \rw q_c$ indicates that the term $b$ is
      obtained from the term $c$ by equality.
      % The term $g(b)$ is recognized into $q_g$, thanks to $q_b \rw q_c$ too.
      % Conversely, we have that $c$ is the approximation of $d$ from state $q_d$.
      Transitions $q_g \rw q_f$ and $q_b \rw q_a$ denote rewriting steps.
      Those transitions allow us to deduce $f(a) \rw^*_\RE g(c)$ and, $a \rw^*_\RE b$.
      To have $f(a) \rw_\R f(b) =_E f(c) \rw_\R g(c)$, which is indeed
      $f(a) \rw^*_\RE g(c)$ unfolded,
      we use the equality $b = c$ to obtain $c$ from $b$, relation denoted by the 
      by the transition $q_c \rw q_b$. Thus, we label the transition  $q_g \rw q_f$
      with the formula $Eq(q_c, q_b)$ to save this information, whereas 
      the transition $q_b \rw q_a$ is labeled with $\top$ which means $a \rw^*_\R b$.
    \end{minipage}&%\quad
    \begin{minipage}{0.25\linewidth}
      \tikz[thick, scale=.8]{
        \node (qf) at (0, 0)   {$\mathbf{q_f}$};
        \node (qg) at (4, 0)   {$q_g$};
        \node (f)  at (0,-1.5) {$f(\;q_a\;)$};
        \node (g)  at (4,-1.5) {$g(\;q_c\;)$};
        \node (qa) at (0,-2.8) {$q_a$};
        \node (qb) at (2,-2.8) {$q_b$};
        \node (qc) at (4,-2.8) {$q_c$};
        \node (a)  at (0,-4.3) {$a$};
        \node (b)  at (2,-4.3) {$b$};
        \node (c)  at (4,-4.3) {$c$};
        % \Delta
        \draw [->] (f) edge (qf);
        \draw [->] (g) edge (qg);
        \draw [->] (a) edge (qa);
        \draw [->] (b) edge (qb);
        \draw [->] (c) edge (qc);
        % liens de descendance
        \draw [dashed] (f) edge (qa);
        \draw [dashed] (g) edge (qc);
        % \Drw
        \draw [->, bend right=15] (qg) to node[auto, swap] {\footnotesize$Eq(q_c, q_b)$} (qf);
        \draw [->, bend right=15] (qb) to node[auto, swap] {\footnotesize$\top$} (qa);
        % \Deq
        \draw [->, bend right=15] (qb) to node[auto, swap] {\footnotesize$=$} (qc);
        \draw [->, bend right=15] (qc) to node[auto, swap] {\footnotesize$=$} (qb);
      }
    \end{minipage}
  \end{tabular}
%  \caption{Example of $\RE$-automaton}
\end{example}

%%% Local Variables: 
%%% mode: latex
%%% TeX-master: "main"
%%% TeX-PDF-mode: t
%%% End:


%$\RE$-automata make it possible to check wether a term is {\em
%  really} reachable, i.e. a counter example. 


% A $\RE$-automaton is an extension of a $\R$-automaton introduced in
% \cite{BoyerG-RULE09}. Intuitively, terms are recognized using
% transitions of $\Delta$, the transitions of $\Drw$ denote the
% rewriting relation between those terms, and the
% $\varepsilon$-transitions of $\Deq$ denote the approximations. Each
% transition of $\Drw$ is labeled by a logical formula denoting the
% approximations needed so as to have the rewriting step.  Intuitively,
% if the formula is $\top$, no approximation is necessary and the term
% is reachable by rewriting only.
%Otherwise, if the formula is not equivalent to $\top$, th

%\comments{On s'en sert?: In the following, we assume that logical
%formulas are always transformed into an equivalent \emph{Disjunctive Normal
%  Form} using standard logic equivalences.}
\noindent 
%%Normalized transitions of $\Delta$ in a $\RE$-automaton recognize terms, called
%%{\em representative}, whereas $\varepsilon$-transitions represent rewriting and
%%equivalence relations between them. 

%%\comments{Axel: shall we keep the definition of sets of representative?}

In what follows, we use $\rwned$ to denote the transitive and
reflexive closure of $\Delta$.  Given a set $\Delta$ of normalized
transitions, the set of representatives of a state is defined by
$Rep(q) = \{ t \in \TF | t \rwned q\}$. We now formally describe the
runs of a $\RE$-automaton.

%This $\rwned$ is a particular case of the new rewriting relation $\xrw{\alpha}_\A$
%of $\RE$-automata. 

\begin{definition}[Run of a $\RE$-automaton $\A$]
  \label{def:xrw_alpha}
  \begin{itemize}
  \item $t|_p = f(q_1, \dots, q_n)$ and $f(q_1, \dots, q_n) \rw q \in \Delta$
    then $t \xrw{\top}_\A t[q]_p$
  \item $t|_p = q$ and $q \rw q' \in \Deq$ then $t \xrw{Eq(q, q')}_\A t[q']_p$
  \item $t|_p = q$ and $q \xrw{\alpha} q' \in \Drw$ then $t \xrw{\alpha}_\A t[q']_p$ 
%     where $\alpha = \left\{
%       \begin{array}{ll}
%         \alpha_k &\textrm{if } 1 \le k \le n \land \phi = \bigvee_1^n \alpha_i\\
%         \top &  \textrm{if } \phi = \top
%       \end{array}\right.
%     $
  \item $u \xrw{\alpha}_\A v$ and $v \xrw{\alpha'}_\A w$ then $u \xrw{\alpha \land \alpha'}_\A w$
  \end{itemize}
\end{definition}

\noindent
A run $\xrw{\alpha}$ abstracts a rewriting path of $\rwmod$. If $t
\xrw{\alpha} q$, then there exists a term $s\in Rep(q)$ such that
$s\rwmod^* t$. The formula $\alpha$ denotes the subset of transitions
of $\Deq$ needed to recognize $t$ into $q$.
%%i.e. the equivalence
%%steps, induced by $E$, needed to rewrite $s$ into $t$ using
%%$\rwmod$.
\begin{example}
  Consider the $\RE$-automaton $A$ of Example \ref{ex:semantics} and
  let $g(b) \xrw{Eq(q_b, q_c) \land Eq (q_c, q_b)} q$, we know that
  there exists a rewriting path of $\rwmod$ from $f(a)$, the unique
  term of $Rep(q)$ to $g(b)$. The formula indicates that this
  rewriting path uses the equivalence relation induced by $b = c$ in
  both directions (transitions $q_b \rw q_c$ and $q_c \rw q_b$).
\end{example}


The relation $\xrw{\alpha}$ corresponds to the standard rewriting
relation (see Section~\ref{sec:definitions}) of a tree-automaton
instrumented with logical formulas.

\begin{theorem}\label{th:equiv}{\quad\quad
  $\forall t\in\TFQ,\; q \in \Q,\; t \xrw{\alpha}_\A q \equ t \rw_\A^* q$}
\end{theorem}

We now introduce {\em well-defined} $\RE$-automata. The well-defined
property will be used in the refinement procedure to
distinguish between counter-examples and false positives. 


\begin{definition}[A well-defined $\RE$-automaton]
  \label{def:well-defined}
  $\A$ is a \emph{well-defined} $\RE$-automaton, if :
% The second point of the definition is used to refine the $\RE$-automaton: a rewriting step of $\rwmod$
% denoted by $q \xrw{\phi} q'$ holds thanks to the subset of transitions of $\Deq$
% occurring in $\phi$. If we remove the transitions of $\Deq$ such that $\phi$
% does not hold, then the transition $q \xrw{\phi} q'$ disappears and the term is 
% no longer recognized.
  \begin{itemize}
  \item For all state $q$ of $\A$, and all term $v$ such that
    $v \xrw{\top}_\A q$, there exists $u$ a term representative
    of $q$ such that $u \rw^*_\R v$
  \item If $q \xrw{\phi} q'$ is a transition of $\Drw$, then there exist terms
    $s,t\in \TF$ such that $s\rwtag{\phi}_\A q$, $t\rwtag{\top}_\A q'$
    and $t \rw_\R s$.
  \end{itemize}
\end{definition}

\noindent
The first item in the definition~\ref{def:well-defined} guarantees
that every term recognized by using transitions labeled with the
formula $\top$ is indeed reachable from the initial set.  The second
item is used to refine the automaton. A rewriting step of $\rwmod$
denoted by $q \xrw{\phi} q'$ holds thanks to some transitions of
$\Deq$ that occurs in $\phi$. If we remove transitions in $\Deq$ in
such a way that $\phi$ does not hold, then the transition $q
\xrw{\phi} q'$ should also be removed.

According to the above construction, a term $t$ that is recognized by
using at least a transition labeled with a formula different from $\top$
can be removed from the $\RE$-automaton by removing some transitions
in $\Deq$. This ``pruning'' operation is illustrated hereafter.

\begin{example}
  \label{ex:pruning}
  We consider the $\RE$-automaton $\A$ of Example~\ref{ex:semantics}.
  This automaton recognizes the term $g(c)$. Indeed, by
  Definition\ref{def:xrw_alpha}, we have $g(c) \xrw{Eq(q_c, q_b)}
  q_f$. Consider now the rewriting path $f(a) \rw_\R f(b) =_E f(c)
  \rw_\R g(c)$. We can see that if the step $f(b) =_E f(c)$ denoted by
  the transition $q_c \rw q_b$ is removed, then $g(c)$ becomes
  unreachable. The first step in pruning $\A$ consists thus in
  removing this transition. In a second step, we propagate the
  information by removing all transition of $\Drw$ labeled by a
  formula formed with $Eq(q_c, q_b)$. This is done to remove all terms
  obtained by rewriting with the equivalence $b =_E c$. After having
  pruned all the transitions, we observe that the terms recognized by
  $\A$ are given by the following set $\{f(a), f(b)\}$.
\end{example}

%%As we shall see, the prunning technique sketched in the above example
%%will serve as a basis for the refinement technique presented in
%%section~\ref{sec:refinement}.






%%% Local Variables: 
%%% mode: latex
%%% TeX-master: "main"
%%% TeX-PDF-mode: t
%%% End: 


\section{On solving the reachability using $\RE$-automaton}
\label{sec:recompletion}

%%We assume that we have a reachability problem composed a set of
%%initial terms $I$, a set of rules $\R$, a set of undesirable terms
%%$Bad$. We consider the approximation defined gby a set of equations
%%$E$. To build a $\RE$-automatxon, we instantiate the framework of the
%%completion presented in section \ref{sec:completion} by defining functions $C$
%%for the completion step and $W$ for the widening step.


In this section, we extend the completion and widening principles
introduced in Section~\ref{sec:completion} to $\RE-$automata. We
consider an initial set $I$ that can be represented by a tree
automaton, and transition relation represented by a set of rewriting
rules $\R$. We compute successive approximations $\aaexeq^i$ from
$\aaexeq^0$ using $\aaexeq^{i+1}=W(C(\aaexeq^i))$. We define
$\aaexeq^0 = \la \F, \Q^0, \Q_F, \Delta^0\ra$ that the language of
$\aaexeq^0$ is the terms in $I$. Observe that $\aaexeq^0$ is
well-defined as the sets $\Drw^0$ or $\Deq^0$ are empty. We now detail
the completion and widening steps i.e. $C$ and $W$.

% In this section, we extend completion and widening steps of Section
% \ref{sec:completion} to $\RE-$automata. Given an automaton $A$ that
% represents an initial set $I$, and transition relation represented by
% a set of rewriting rules $\R$, we compute successive approximations 
% $\aaexeq^i$ from $\aaexeq^0$ the $\RE$-automata version of $A$, i.e.,
% $\aaexeq^{i+1}=\widen(\compl(\aaexeq^i))$. We define $\aaexeq^0 = \la \F, \Q^0,
% \Q_F, \Delta^0\ra$ such that the language of $\aaexeq^0$ is the terms
% in $I$. Observe that $\aaexeq^0$ is well-defined as the sets $\Drw^0$
% or $\Deq^0$ are empty. We now detail the completion and widening
% steps i.e. $\compl$ and $\widen$.

\vspace{-.6cm}
\paragraph{The completion step $\compl$.}
Consider a $\RE$-automaton $\aaexeq^i = \la \F, \Q^i, \Q_f, \Delta^i \cup \Drw^i
\cup \Deq^i \ra$, the completion steps consists in computing an automaton
$\compl(\aaexeq^i)$ that is obtained from $\aaexeq^i$ by applying $\R$. As
already explained in Section \ref{sec:completion}, this is done by finding and
resolving all critical pairs.  A {\em critical pair} for a $\RE$-automaton is a
triple $\la r\sigma, \alpha, q\ra$ such that $l\sigma \rw r\sigma$, $l\sigma
\xrw{\alpha}_{\aaexeq^i} q$ and there is no formula $\alpha'$ such that $r\sigma
\xrw{\alpha'}_{\aaexeq^i} q$. Resolution of such a critical pair consists of
adding to $\compl(\aaexeq^i)$ the transitions to obtain $r\sigma
\xrw{\alpha}_{\compl(\aaexeq^{i})} q$. This is followed by a normalization step
where $\Q_{new}$ is a set of new states s.t. $\Q_{new} \cap =\emptyset$.

% More precisely, we add transitions
% $\norm(r\sigma,\Delta^i\setminus\Delta^0)$ (see Definition
% \ref{def:normalization}) to $\Delta^i$. Actually, the set of
% transitions $\Delta^i\setminus\Delta^0$ is deterministic and thus,
% after the normalization of $r\sigma$, there exists a state $q'$ with
% $r\sigma \xrw{\top}_{\compl(\aaexeq^{i})} q'$. Consequently, it remains to
% add the transition $q'\xrw{\alpha} q$ to $\Drw^{i}$. The whole process
% leads to the $\RE-$automaton $\compl(\aaexeq^i)$. This is formally defined as
% follows.



%%The transtion $r\sigma
%%\xrw{\top}_{\aaexeq^{i+1}} q'$ is actually build by adding normalized
%%transitions to $\Delta^{i+1}$ thanks to the {\em normalisation} of the
%%definition~\ref{def:normalization}.

\begin{definition}[Normalization]
  \label{def:normalization}
  The normalization is done in two mutually inductive steps
  parametrized by the configuration $c$ to recognize, and by the set
  of transitions $\Delta$ to extend.
  \[
  \left\{
    \begin{array}{lr}
      \norm(c , \Delta) = \slice(d, \Delta) & c \rw^!_\Delta d\textrm{, and }c,\ d \in \TFQ\\
      \slice(q,\Delta) = \Delta & q \in Q\\
      \slice(f(q_1,\dots, q_n), \Delta) = \Delta \cup \{f(q_1,\dots, q_n) \rw q\} & q_i \in \Q\textrm{ and }q \in \Q_{new}\\
      \slice(f(t_1,\dots, t_n), \Delta) = \norm(f(t_1,\dots, t_n), \slice(t_i, \Delta)) & t_i \in \TFQ \setminus \Q\\
    \end{array}
  \right.
  \]
\end{definition}

\noindent
We thus add transitions such that there exists a state $q'$ with
$r\sigma \xrw{\top}_{\aaexeq^{i+1}} q'$ and $q'\xrw{\alpha} q \in
\Drw^{i+1}$. 

We are now ready to define the resolution of a critical pair $p=\la
r\sigma, \alpha, q\ra$. 

\begin{definition}[Resolution of a critical pair]
\label{def:resolution_cp}
Given a $\RE$-automaton $\A=\la \F, \Q, \Q_f, \Delta\cup \Drw\cup \Deq\ra$ and a critical
pair $p=\la r\sigma, \alpha, q \ra$, the {\em resolution} of $p$ on $\A$ is the
$\RE$-automaton $\A'=\la \F, \Q', \Q_f, \Delta'\cup \Drw'\cup \Deq \ra$ where
\begin{itemize}
\item $\Delta'= \Delta \cup \norm(r\sigma,\Delta\setminus \Delta^0)$
\item $\Drw'= \Drw \cup \{q' \xrw{\alpha} q\}$ where $q'$ is the state such that $r\sigma
  \rw^!_{\Delta'\setminus \Delta_0} q'$
\item $\Q'$ is the union of $\Q$ and the set of states occurring in $\Delta'$
\end{itemize}
\end{definition}

Note that $\Delta_0$, the set of transitions of $\aaex^0$, is never
used for normalization of all $r\sigma$. This is needed to preserve
the well-defidness of $\A'$.  The $\RE$-automaton $\compl(\aaexeq^i)$
is obtained by recursively applying the above resolution principle to
all critical pairs $p$ of the set of critical pairs between $\R$ and
$\aaexeq^i$. The set of all critical pairs is obtained by solving {\em
  matching problems} $l \match q$ for all rewrite rule $l \rw r\in \R$
and all state $q\in\aaexeq^i$.  Solving the matching problem $l \match
q$ consists of computing $S$ that is the set of all couples $(\alpha,
\sigma)$ such that $\alpha$ is a formula, $\sigma$ is a substitution
of $\X \mapsto \Q^i$, and $l\sigma \xrw{\alpha} q$. Each configuration
$l\sigma$ corresponds to a subset of terms of $\Lang(\aaexeq^i, q)$
that can be rewritten by $l \rw r$. The terms characterized by
$l\sigma$ are defined as $l\sigma\sigma'$ where $\sigma': \Q \mapsto
\TF$ is a substitution, which maps each state $q$ to a term
$\sigma'(q) \in \Lang(\aaexeq^i,
q)$. Definition~\ref{def:matching-algorithm} introduces the matching
algorithm to compute the set $S$, which is denoted by the statement $l
\match q \vdash_{\aaexeq^i} S$. Note that when $S$ is empty, there is
no term to rewrite by $l \rw r$.

\begin{definition}[Matching Algorithm]\\
  \label{def:matching-algorithm}
  Assuming the matching problem $l \match q$ for a $\RE$-automaton $\aaexeq^i$.
  $S$ is the solution of the matching problem, if there exists a derivation
  of the statement $l \match q \vdash_{\aaexeq^i} S$ using the rules:
  {\footnotesize
    \vspace{-.2cm}
    \[\textrm{(Var) }
    \dfrac{}
    {x \match q \vdash_\A \{(\alpha_k, \{x \mapsto q_k\}) \sep q_k \xrw{\alpha_k}_\A q\}}
    (x \in \X)
    \]
    \vspace{-.1cm}
    \[\textrm{(Delta) }
    \dfrac{
      t_1 \match q_1 \vdash_\A S_1
      \quad \dots \quad
      t_n \match q_n \vdash_\A S_n
      % f(t_1,\dots, t_n) \match f(q_1,\dots, q_n) \vdash_\A S
    }{f(t_1, \dots, t_n) \matchi q \vdash_\A \bigotimes_1^n S_k
    }\left(
      \begin{array}{l}
        f(q_1, \dots, q_n) \rw q \in \Delta\\
      \end{array}
    \right )
    \footnote{
      {\footnotesize using $\bigotimes_1^n S_j = \{ (\top, id) \oplus (\phi_1,
        \sigma_1) \oplus \dots \oplus (\phi_n, \sigma_n) \sep
        (\phi_j,\sigma_j) \in S_j\}$}, and $(\phi, \sigma) \oplus (\phi',
      \sigma') = (\phi \land \phi',\sigma \cup \sigma')$.}
    \]
    \vspace{-.2cm}
    \[\textrm{(Epsilon) }
    \dfrac{ 
      t \matchi q    \vdash_\A S_0   \quad
      t \matchi q'_1 \vdash_\A S_1 \quad \dots \quad
      t \matchi q'_n \vdash_\A S_n
    }{
      t \match q \vdash_\A S_0 \cup
      \bigcup_{k=1}^n \{(\phi \land \alpha_k, \sigma) \sep (\phi, \sigma) \in S_k\}
    }\left(
      \begin{array}{l}
        \{(q_k, \alpha_k) \sep q_k \xrw{\alpha_k} q\}_1^n\\% \land q \not= q_k\}_1^n\\
        t \notin \X\\
      \end{array}
    \right)
    \]
  }
\end{definition}
\vspace{-.4cm}

Observe that, by definition, the matching problem considers possibly
infinite runs of the form $l\sigma \xrw{\alpha} q$. Indeed,
transitions in $\Drw^i \cup \Deq^i$ can produce loops.  In the
matching algorithm, we exclude such runs. This is done
to keep a finite set of rewriting path, which is computable in a
finite amount of time. It is worth mentioning that removing loops does
not remove any important information. Consider the automaton $\A$ of
Example~\ref{ex:semantics}. We observe that $f(b) \xrw{Eq(q_b, q_c)
  \land Eq(q_c, q_b)}_\A q_f$ uses the loop $f(b) =_E f(c) =_E
f(b)$. This loop can be removed as $f(a) \rw_\R^* f(b)$ can be
obtained by $f(b) \xrw{\top}_\A q_f$, which does not contain any
loops.

%%We now state the important property of our construction.
 
\begin{property}%[Matching Algorithm is complete]
  \label{prop:matching-complete}
  Let $\A$ be a $\RE-$automaton, $q$ one of its states, $l \in \TFX$
  the linear left member of a rewriting rule and a substitution $\sigma : \X
  \mapsto \Q$ whose domain is range-restricted to $\vars(l)$. Assuming that $S$
  is the solution of the matching 
  problem $l\sigma \match q$, for all $(\alpha, \sigma)$ such that
  $l\sigma \xrw{\alpha}_\A q$, a loop free run, then we have $(\alpha,
  \sigma) \in S$.
\end{property}


After computing $S$ for $l \match q$, we identify its elements that
correspond to critical pairs. By definition of $S$, we know that there
exists a transition $l\sigma \xrw{\alpha}_{\aaexeq^i} q$ for $(\alpha,
\sigma) \in S$. If there exists a transition $r\sigma
\xrw{\alpha'}_{\aaexeq^i} q$, then $r\sigma$ has already been added to
$\aaexeq^i$.  It is thus sufficient to deduce that all terms
$l\sigma\sigma'$ of the set represented by the configuration $l\sigma$
are rewritten into terms $r\sigma\sigma'$ represented by the
configuration $r\sigma$.  In the case where there exists no transition
$r\sigma \xrw{\alpha'}_{\aaexeq^i} q$, then $\la r\sigma, \alpha',
q\ra$ is a critical pair to solve on $\aaexeq^i$.  The following
theorem shows that our methodology is complete.

\begin{theorem}
  \label{thm:C}
  If $\aaexeq^i$ is well-defined then so is $\compl(\aaexeq^i)$, 
  and $\forall q \in \Q^i$, $\forall t \in
  \Lang(\aaexeq^i, q)$, $\forall t \in\TF$, $t \rw_\R t' \imp t' \in
  \Lang(\compl(\aaexeq^{i}), q)$.
\end{theorem}

\begin{example}
  \label{ex:C}
  Let $\R=\{f(x) \rw f(s(s(x)))\}$ and $\aaexeq^0=\langle \F, \Q,
  \Q_F, \Delta^0\rangle$ be a tree automaton such that $\Q_F=\{q_0\}$
  and $\Delta^0=\{ a \rw q_1, f(q_1) \rw q_0\}$.  Following Definition
  \ref{def:matching-algorithm}, the solution $S$ of the matching
  problem $f(x)\match q_0 $ is $S=\{(\sigma,\phi)\}$ with
  $\sigma=\{x\rw q_1\}$ and $\phi=\top$.  Hence, $\la f(s(s(q_1))),
  \top, q_0\ra$ is the only critical pair to solve, since
  $f(s(s(q_1)))\not\xrw{\top}_{\aaexeq^0} q_0$.  So,
  $\compl(\aaexeq^0) $ is a $\RE-$automaton such that
  $\compl(\aaexeq^0)= \la\F, \Q^1, \Q_F, \Delta^1\cup \Drw^1 \cup
  \Deq^0 \ra$ with:
  \begin{description}
  \item $\Delta^1=\norm(f(s(s(q_1))),\emptyset) \cup \Delta^0= \{s(q_1)\rw q_2,s(q_2)\rw q_3, f(q_3)\rw q_4\}\cup \Delta^0$, 
  \item $\Drw^1=\{q_4 \xrw{\top} q_0\}$, since $f(s(s(q_1)))\rw^!_{\Delta^1\setminus\Delta^0} q_4$, 
  \item $\Deq^0=\emptyset$ and $\Q^1=\{q_0,q_1,q_2,q_3,q_4\}$.
  \end{description}
\end{example}
Observe that if $\compl(\aaexeq^i)=\aaexeq^i$, then we have reached a fixpoint.
\vspace{-.6cm}
\paragraph{The Widening  $\widen$.}
Consider a $\RE$-automaton $A = \la \F, \Q, \Q_f, \Delta \cup \Drw
\cup \Deq \ra$, the widening operation consists in computing a
$\RE$-automaton $\widen(A)$ that is obtained from $A$ by using the set
of equations $E$.

\begin{tabular}{lc}
  \hspace{-1.1cm}
  \begin{minipage}{.75\linewidth}
    For each equation $l = r$ in $E$, we consider all pair
    $(q, q')$ of distinct states of $\Q^i$ such that there exists a
    substitution $\sigma$ to have the following diagram.
    Observe that $\rw^=_{A}$ is the transitive and reflexive
    rewriting relation induced by $\Delta \cup \Deq$.  
  \end{minipage}&
  \begin{minipage}{.25\linewidth}
    $\xymatrix{
      l\sigma \ar@{=}[r]_{E}\ar[d]^{=}_{A} & r\sigma \ar[d]^{A}_{=}\\
      q & q'
    }$
  \end{minipage}
\end{tabular}

Intuitively, if we have $u \rw^=_{A} q$, then we know that there
exists a term $t$ of $Rep(q)$ such that $t =_E u$. The automaton
$\widen(A)$ is $\la \F, \Q, \Q_f, \Delta \cup \Drw \cup \Deq' \ra$,
where $\Deq'$ is obtained by adding the transitions $q \rw q'$ and $q'
\rw q$ to $\Deq$, for each pair $(q, q')$. We also keep $\Deq'$ {\em
  closed by transitivity}, but only for pair of distinct states.
Roughly, the transitive closure of $\Deq'$ corresponds to propagate
explicitly terms that are equivalent by $=_E$.  As show in
section~\ref{sec:refinement}, this aspect is important to refine with
accuracy. Note that $\widen$ terminates since the number of states of
$\A$ is finite and the number of transitions to be added to $\Deq'$ is
finite.

\begin{theorem}
  \label{thm:W}
  Assuming that $A$ is well-defined, we have $A $ syntactically included in 
  $ \widen(A)$, and $\widen(A)$ is well-defined.
\end{theorem}

\begin{example}
  \label{ex:W}
  Consider the $\RE$-automaton $\compl(\aaexeq^0)$ in Example~\ref{ex:C}.\\
  \begin{tabular}{lc}
    \hspace{-.3cm}
    \begin{minipage}{.75\linewidth}
      We compute $\aaexeq^1 = \widen(\compl(\aaexeq^0))$ using the equation
      $s(s(x))=s(x)$.  We have $\sigma=\{x \mapsto q_1\}$ and the
      following diagram.  Then, we obtain $\aaexeq^1 = \la \F, \Q^1,
      \Q_f, \Delta^1 \cup \Drw^1 \cup \Deq^1\ra$, where $\Deq^1= \Deq^0 \cup \{
      q_3 \rw q_2,\ q_2 \rw q_3\}$ and $\Deq^0=\emptyset$.  Observe that $\aaexeq^1$ is a
      fixpoint: $\compl(\aaexeq^1) = \aaexeq^1$.
    \end{minipage}&
    \begin{minipage}{.25\linewidth}
      $\xymatrix{
        s(s(q_1)) \ar@{=}[r]_{E}\ar[d]^{=}_{\compl(\aaexeq^0)} & s(q_1) \ar[d]^{\compl(\aaexeq^0)}_{=}\\
        q_3 & q_2
      }$
    \end{minipage}
  \end{tabular}
\end{example}



% \begin{example}
%   Let $\A$ be a $\RE$-automaton such that $\Delta=\{f(q_1,q_2) \rw q_0, a \rw
%   q_1, a \rw q_2, g(q_1) \rw q_3\}$, $\Drw=\emptyset$ and $\Deq=\emptyset$.
% \begin{itemize}
%   \item If $E=\{g(x)=a\}$, we have $\sigma=\{x \mapsto q_1\}$ and 
% \[\xymatrix{
% g(q_1) \ar@{=}[r]_{E}\ar[d]_{=} & a \ar[d]^{=}\\
% q_3 & q_2
% }
% \]
% Hence, $\A \simp_E \A'$ where $\A'$ is similar to $\A$ except that $\Deq$
% becomes $\{q_2 \rw q_3, q_3 \rw q_2\}$. 
% % Note that a
% % similar situation could have been found between $g(q_1) \rwAnestar q_3$ and $a
% % \rwAnestar q_1$. We will see in the following that the order used to achieve
% % widening steps does not matter.
%   \item If $E=\{f(x,x)=g(x)\}$ then  there is no substitution $\sigma$ such that $f(x,x)\sigma=f(q_1,q_2)$:  the automaton is unchanged.
% \end{itemize}
% \end{example}


% Consider a $\RE$-automaton $\aaexeq^i = \la \F, \Q^i, \Q_f, \Delta^i
% \cup \Drw^i \cup \Deq^i \ra$, the widening consists in computing a
% $\RE$-automaton $\widen(\aaexeq^i)$ that is obtained from $\aaexeq^i$ by
% using $E$. 
% \begin{tabular}{lc}
%   \hspace{-.3cm}
%   \begin{minipage}{.75\linewidth}
%     For each equation $l = r$ in $E$, we consider all pair
%     $(q, q')$ of distinct states of $\Q^i$ such that there exists a
%     substitution $\sigma$ to have the following diagram.
%     Observe that $\rw^=_{\aaexeq^i}$ is the transitive and reflexive
%     rewriting relation induced by $\Delta^i \cup \Deq^i$.  
%   \end{minipage}&
%   \begin{minipage}{.25\linewidth}
%     $\xymatrix{
%       l\sigma \ar@{=}[r]_{E}\ar[d]^{=}_{\aaexeq^i} & r\sigma \ar[d]^{\aaexeq^i}_{=}\\
%       q & q'
%     }$
%   \end{minipage}
% \end{tabular}

% Intuitively, if we have $u \rw^=_{\aaexeq^i} q$, then we know that there exists a term
% $t$ of $Rep(q)$ such that $t =_E u$.
% We obtain $\Deq^{i+1}$ by adding the transitions $q \rw q'$ and $q'
% \rw q$ to $\Deq^i$, for each pair $(q, q')$. We also keep $\Deq^{i+1}$
% {\em closed by transitivity}, but only for pair of distinct states.
% Roughly, the transitive closure of $\Deq^{i+1}$ corresponds to
% propagate explicitly terms that are equivalent by $=_E$.  As show in
% section~\ref{sec:refinement}, this aspect is important to refine with
% accuracy.  Finally, we have $\widen(\aaexeq^i) = \la \F, \Q^i, \Q_f,
% \Delta^i \cup \Drw^i \cup \Deq^{i+1} \ra$.

% \begin{theorem}
%   \label{thm:W}
%   Assuming that $\aaexeq^i$ is well-defined, we have $\aaexeq^{i} \subseteq \widen(\aaexeq^i)$, and $\widen(\aaexeq^i)$ is well-defined.
% \end{theorem}

% \begin{example}
%   \label{ex:W}
%   Consider the $\RE$-automaton $\aaexeq^1$ in Example~\ref{ex:C}.\\
%   \begin{tabular}{lc}
%     \hspace{-.3cm}
%     \begin{minipage}{.75\linewidth}
%       We compute $\aaexeq^2 = \widen(\aaexeq^1)$ using the equation
%       $s(s(x))=s(x)$.  We have $\sigma=\{x \mapsto q_1\}$ and the following diagram.
%       Then, we obtain $\aaexeq^2 = \la \F, \Q^1, \Q_f, \Delta^1 \cup \Drw^1 \cup \Deq^2\ra$, where $\Deq^2=\{ q_3 \rw q_2,\ q_2 \rw q_3\}$.
%       Observe that $\aaexeq^2$ is a fixpoint: $\compl(\aaexeq^2) = \aaexeq^2$.
%     \end{minipage}&
%     \begin{minipage}{.25\linewidth}
%       $\xymatrix{
%         s(s(q_1)) \ar@{=}[r]_{E}\ar[d]^{=}_{\aaexeq^1} & s(q_1) \ar[d]^{\aaexeq^1}_{=}\\
%         q_3 & q_2
%       }$
%     \end{minipage}
%   \end{tabular}
% \end{example}



% \begin{example}
%   Let $\A$ be a $\RE$-automaton such that $\Delta=\{f(q_1,q_2) \rw q_0, a \rw
%   q_1, a \rw q_2, g(q_1) \rw q_3\}$, $\Drw=\emptyset$ and $\Deq=\emptyset$.
% \begin{itemize}
%   \item If $E=\{g(x)=a\}$, we have $\sigma=\{x \mapsto q_1\}$ and 
% \[\xymatrix{
% g(q_1) \ar@{=}[r]_{E}\ar[d]_{=} & a \ar[d]^{=}\\
% q_3 & q_2
% }
% \]
% Hence, $\A \simp_E \A'$ where $\A'$ is similar to $\A$ except that $\Deq$
% becomes $\{q_2 \rw q_3, q_3 \rw q_2\}$. 
% % Note that a
% % similar situation could have been found between $g(q_1) \rwAnestar q_3$ and $a
% % \rwAnestar q_1$. We will see in the following that the order used to achieve
% % widening steps does not matter.
%   \item If $E=\{f(x,x)=g(x)\}$ then  there is no substitution $\sigma$ such that $f(x,x)\sigma=f(q_1,q_2)$:  the automaton is unchanged.
% \end{itemize}
% \end{example}


%%% Local Variables: 
%%% mode: latex
%%% TeX-PDF-mode: t
%%% TeX-master: "main"
%%% End: 


\section{Approximation Refinement}
\label{sec:refinement}

Let $I$ be a set of initial terms characterised by the $\RE-$automaton
$\aaexeq^0$, $\R$ be a TRS, and $Bad$ the set of forbidden terms
represented by $A_{Bad}$ a tree automaton.  The reachability problem
boils down to check $\desc(I) \cap Bad
\stackrel{?}{=}\emptyset$. There are classes of systems for which
$\desc(I)$ is regular and can be computed in a finite amount of time
(see appendix~\ref{sec:exact}) but, in general, the computation does
not terminate. For such cases, our only hope is to work with a
\textit{Counterexample-Guided Abstraction
  Refinement\,\cite{DBLP:conf/time/Clarke03}} procedure that computes
successive abstractions and successively refine them until the
property can be prove correct or not. In the first part of the paper,
we have focused on computing the abstraction. We now propose a
technique that checks whether a term is indeed reachable from the
initial set of terms. If the term is a spurious counterexample, then
it has to be eliminated from the approximation. We then generalize the
operation to (possibly infinite) sets of terms.

%Finally, we develop a complete example of our new approach.

% show how the procedure $P$ handles a term of ment procedure. 
%  In
% Section \ref{subsec:refinementstep}, which exploits the
% well-definition introduced in Section \ref{sec:re-automaton} in order
% to refine a $\RE$-automaton by removing a (possibly infinite) set of
% terms that is characterized by a formula of equations on the states of
% the automaton.


%\Input{ref-step}

% Instead, most automata based techniques compute the
% intersection between $\aaexeq^i$ and $Bad$ and decide its
% emptiness. This cannot be straightforwardly applied here. First,
% designing an intersection algorithm is difficult because our two
% automata do not have a common structure. Second, the emptiness
% algorithm is very specific since emptiness is constrained by formulas
% on $\Drw$ transitions of $\aaexeq^i$.

%%we propose an
%%algorithm that given an approximation $\aaex^*$ and a set $Bad$
%%computes such a formula. If all the terms in $\aaex^*\cap Bad$ are
%%indeed reachable, then this formula reduces to $\top$ , else the
%%formula characterize a set of terms (accepted from the states
%%characterized by the equations) that may or may not be reachable and
%%that must be refined.


% %From a verification point of view, given a
% %reachability property $\phi$, $Bad$ represents the set of
% %configurations satisfying $\neg\phi$.  
% For some classes of $\R$,
% $\desc(\Lang(\A))$ is regular. Thus, the reachability problem,
% i.e. $\desc(\Lang(\A))\cap Bad \stackrel{?}{=}\emptyset$, can be
% decided. For most of those classes, reachability can be decided using
% $\RE$-automata completion since it stops without approximation
% ($E=\emptyset$) on a $\RE$-automaton $\aaex^*$ such that
% $\Lang(\aaex^*)=\desc(\Lang(\A))$ (see Theorem~\ref{thm:regular}).
% When the TRS is outside of those (restricted) classes, approximation
% and refinement are necessary. The principle of our approach is based on CEGAR
% approach. In~\cite{DBLP:conf/time/Clarke03}, the CEGAR
% (\textit{Counterexample-Guided Abstraction Refinement}) paradigm has
% been summarised and a general algorithm consisting in refining an
% abstraction function by analysing a spurious counterexample has been
% given. In section \ref{sec:recompletion}, it has been shown that the
% property of {\it well-definition of $\RE-$automaton}, see Definition
% \ref{def:well-defined}, is preserved by our approach. So, our approach
% fits perfectly with the CEGAR paradigm, since the previous property
% allows the discrimination between terms actually reachable and
% artifacts of the approximations. Moreover, an artifact of the
% approximation is characterized by a logic formula pointing out 
% which approximations done have produced the problematic term. 

% In section \ref{subsec:refinementstep}, we first describe how
% refinement is performed on a $\RE-$automaton $\aaexeq^i$ obtained by
% completion.  Moreover, we observe that $Bad \cap \Lang(\aaexeq^i)$ may
% be infinite. So, we can not iterate on each term in the intersection
% to prune $\aaexeq^i$. Thus, section \ref{subsec:intersectionformula}
% proposes to characterize by a logical formula the set of approximations performed 
% leading to a non-empty intersection. Obtaining the formula $\top$ means that there exists 
% actually a reachable term in the intersection. 


Let $\aaexeq^k = \la \F, \Q^k, \Q_f, \Delta^k \cup \Drw^k \cup \Deq^k
\ra$ be a $\RE$-automaton obtained after $k$ steps of completion and
widening from $\aaexeq^0$ and assume
that %the verification of the property fails for this $\RE$-automaton:
$\Lang(\aaexeq^k) \cap Bad \not=\emptyset$.  Let $t$ be a term of
$\Lang(\aaexeq^k) \cap Bad$.  Then, we know that there exists a run $t
\xrw{\phi}_{\aaexeq^k} q_f$ with $q_f \in \Q_f$.  We know that
$\aaexeq^k$ is well-defined by construction.  It implies that if $\phi
= \top$, we deduce $t \in \desc(I)$.  It means that $t$ is a
counter-example, so from a verification point of view, the property is
broken as formulated in Section~\ref{sec:completion}.  Otherwise, we
have that $\phi = \bigwedge_1^n Eq(q_j, q'_j)$, and $t$ is possibly a
spurious counterexample. We thus decide to remove it from the
approximation. For doing so, we use the {\em pruning
  methodology} that was informally introduced in
Example\,\ref{ex:pruning}.
\vspace{-.6cm}
\paragraph{The pruning step $\prune$.}
\label{subsec:refinementstep}
As we have seen in Example~\ref{ex:pruning}, we compute the pruned
$\RE$-automaton $\prune(\aaexeq^k, \phi)$ in two steps. The first step
consists of removing some transitions of $\Deq^k$ until $\phi$ does
not not hold anymore {\em i.e.} $\phi = \bottom$.  Consider the
formula $\phi$ containing the predicate $Eq(q,q')$: we replace this
predicate by $\bottom$ if we decide to remove the transition $q \rw
q'$ from $\Deq^k$.  The second step consists in propagating the
information. Indeed, we also have to remove all transitions $q
\xrw{\alpha} q' \in \Drw^k$, where the conjunction $\alpha$ contains
some atoms transitions removed from $\Deq^k$.  The procedure is
iterated for each possible reduction of $t$ in $\aaexeq^k$ and thus,
we finally get $\aaexeq^{k+1}$.  It is easy to see that, for any
$\phi$, there exists no run $t \xrw{\phi}_{\aaexeq^{k+1}} q_f$.

Observe that, when removing $t$ from the abstraction, our procedure
may also remove other terms that are generated by $\phi$. We now
briefly show (see the appendix for details) that the {\em possibly
  infinite} set of terms that belong to both the abstraction and the
set of bad states can be removed by exploiting the structure of the
formula. More precisely, we build a set $S$ containing triples of the
form $(q,q',\phi)$ where $q$ is a state of $\aaexeq^k$, $q'$ is a
state of $A_{Bad}$ and $\phi$ is a formula on $\Deq$ transitions of
$\aaexeq^k$.  For all triple $(q,q',\phi)$, the formula $\phi$ holds
if and only if $\Lang(\aaexeq^k,q)\cap \Lang(A_{Bad},q')\not=
\emptyset$. For all triple $(q,q',\phi)$, where $q$ is final in
$\aaexeq^k$, $q'$ is final in $A_{Bad}$ and $\phi = \top$, then some
of the terms recognized by $q'$ in $A_{Bad}$ are reachable.
Otherwise, $\phi$ is the formula to invalidate, i.e. negate some of
its atom so that it becomes $\bot$. Starting from $\phi$, the
refinement process is performed using the technique that presented
above.

\vspace{-.6cm}
%\input{ref-example}
\paragraph{Example.}
Consider the $\RE-$automaton $\aaexeq^1$ given in Example~\ref{ex:W}.
We define $A_{Bad}$ to be the tree automaton whose final state is
$q'_0$ and whose transitions are $a\f q'_1, s(q'_1)\f q'_2, s(q'_2)\f
q'_1$ and $s(q'_2)\f q_0$.  The forbidden terms that belong to the
language of $\Lang(A_{Bad})$ are of the form $f(s^{2k + 1}(a))$.
% Let us recall that the set of transitions of $\aaexeq^1$ is $\Delta^1\cup\Drw^1\cup\Deq^1$ with
% $\Delta^1= \Delta^0 \cup \{s(q_1)\f q_2,s(q_2)\f q_3,f(q_3)\f q_4\}$,
% $\Drw^1=\{q_4\stackrel{\top}{\f}q_0\}$ and $\Deq^1=\{q_2\f q_3, q_3\f
% q_2\}$. 
We observe that $\Lang(\aaexeq^1) \cap \Lang(A_{Bad}) \not=
\emptyset$.  According to the intersection algorithm sketched above,
one can construct a set $S$ of triples $(q_0,q'_0,\phi)$, where $\phi$
is the formula used to prune $\aaexeq^1$ in order to remove those
terms that belong to $\Lang(\aaexeq^1) \cap \Lang(A_{Bad})$.  Here,
$S=\{(q_0,q'_0,Eq(q_2,q_3)\land Eq(q_3,q_2)), (q_0,q'_0,Eq(q_2,q_3)),(q_0,q'_0,Eq(q_3,q_2))\}$.  We apply the pruning step for each
formula $\phi$ in the set.  We compute $\prune(\aaexeq^1,
Eq(q_2,q_3)\land Eq(q_3,q_2))$.  Removing the transition $q_2 \f q_3$
from $\Deq^1$ is sufficient to invalidate $Eq(q_2,q_3)\land
Eq(q_3,q_2)$. Moreover, this removing invalidates
$(q_0,q'_0,Eq(q_2,q_3))$ too. It remains to prune with
$(q_0,q'_0,Eq(q_3,q_2))$. This is done by removing the transition
$q_3\f q_2$ from $\Deq^1$.  At this step, no transition of $\Drw^1$
can be removed anymore. Indeed, all these transitions are all labeled
by $\top$. We thus define
$\aaexeq^2=\prune(\prune(\prune(\aaexeq^1,Eq(q_2,q_3)\land
Eq(q_3,q_2)),Eq(q_2,q_3)),$ $Eq(q_3,q_2))$ with $\Delta^2=\Delta^1$,
$\Drw^2= \Drw^1$ and $\Deq^2=\emptyset$. We observe that $\aaexeq^2$
is not $\R-$closed and should be completed. We thus define
$\aaexeq^3=\widen(\compl(\aaexeq^2))$. We found a new critical pair
for $f(x) \rw f(s(s(x)))$ and we obtain $\Delta^3 = \Delta^2 \cup
\{(q_3)\f q_5, s(q_5)\f q_6, f(q_6) \f q_7$, and $\Drw^3 = \Drw^2 \cup
\{q_7\xrw{\top}q_4\}$.

The interesting point concerns the application of $\widen$.  Observe
that the transitions of $\Deq^3$ directly results from the application
of the equation $s(x) = s(s(x))$ {\em i.e.} transitions $q_2\f q_3,
q_3\f q_2, q_3\f q_5,q_5\f q_3, q_5\f q_6,$ and $q_6\f q_5$.  After
the transitive closure of $\Deq^3$ has been computed, some new
transitions are inferred and added to $\Deq^3$, i.e., $q_2\f q_5,q_5\f
q_2, q_2\f q_6, q_6\f q_2, q_3\f q_6$ and $q_6\f q_3$.  We again check
the emptiness of $\Lang(\aaexeq^3) \cap \Lang(A_{Bad})$.  This
intersection is still not empty and most of the transitions in
$\Deq^3$ can be removed by the pruning operation.  Let $\aaexeq^4$ be
the $\RE-$automaton obtained by applying $\prune$ on $\aaexeq^3$, it
only remains $ q_3\f q_6$ and $q_6\f q_3$ in $\Deq^4$. We also observe
that no transition of $\Drw^3$ are removed: $\Drw^4=\Drw^3$. In fact,
all the transitions in $\Drw^4$ are labeled by $\top$. Then, we
restart the completion process and we observe that
$\aaexeq^4=\compl(\aaexeq^4)$.  We have thus reached a
fix-point. There, we observe that $\Lang(\aaexeq^4)\cap
\Lang(A_{Bad})=\emptyset$ and conclude that $\R^*(I) \cap Bad
=\emptyset$. Observe that our refinement is accurate in this
case. Indeed $\Lang(\aaexeq^4)=f(s^{2*k}(a))$, that is the exact set
of reachable states.

\begin{remark}
The above example cannot be handled with
the approach of~\cite{BCHK08} that also proposes a counterexample
guided abstraction technique. Indeed this technique would prove that a
property where the bad terms are bounded: $Bad_k = \{ f(s^{2*i +1}(a))
\sep i < k\}$ Else the procedure loops, if we consider all the set
$Bad$.
\end{remark}
% compute $\prune(\widen(\compl(\aaexeq^1)))$. According to Section
% \ref{sec:recompletion}, the set of transitions of $\widen(\compl(\aaexeq^1))$ is
% $\Delta^2\cup\Drw^2\cup\Deq^2$ with $\Delta^2=\Delta^1\cup\{s(q_3)\f
% q_5,s(q_5)\f q_6,f(q_6)\f q_7 \}$,
% $\Drw^2=\Drw^1\cup\{q_7\stackrel{\top}{q_4}\}$ and $\Deq^2=\{q_2\f
% q_3, q_3\f q_2, q_3\f q_5,q_5\f q_3, q_5\f q_6,q_6\f q_5, q_2\f
% q_5,q_5\f q_2, q_2\f q_6, q_6\f q_2, q_3\f q_6,q_6\f q_3\}$. Note that
% the transitions $q_2\f q_5,q_5\f q_2, q_2\f q_6, q_6\f q_2, q_3\f
% q_6,q_6\f q_3$ have been inferred from the transitions resulting from
% equation applications i.e. $\{q_2\f q_3, q_3\f q_2, q_3\f q_5,q_5\f
% q_3, q_5\f q_6,q_6\f q_5\}$. Applying the same process as previously,
% one concludes that the transitions $q_2\f q_3, q_3\f q_2, q_3\f
% q_5,q_5\f q_3, q_5\f q_6,q_6\f q_5, q_2\f q_6, q_6\f q_2, q_2\f q_5$
% and $q_5\f q_2$ have to be removed from $\Drw^2$.
 %  Consequently, one obtains $\aaexeq^4$ whose set of transitions is
%   $\Delta^4\cup\Drw^4\cup\Deq^4$ with $\Deq^4=\{q_3\f q_6, q_6\f
%   q_3\}$. 


% all $(\sigma,\alpha)\in S$ where $t\match q \vdash_{\aaexeq^k}S$ and
% $q\in\Q_f$, one can deduce that $\sigma=\emptyset$ and $\alpha
% \not=\top$.  Composing the formula $\alpha_t$ such that
% $\alpha_t=\bigvee_{(\sigma,\alpha)\in S}\alpha$, $\alpha_t$
% characterizes the set of approximations required to make $t$ be a term
% of $\Lang(\aaexeq^k)$.
%
% We propose a refinement operator $P$ such that t
% $\Lang(\prune(W(C(\aaexeq^{k}))))\cap Bad = \emptyset$
%
% We extend our new completion algorithm with 
% a refinement step denoted by $P$. Thus, $\aaexeq^{i+1}$ is defined as 
% $\prune(W(C(\aaexeq^i)))$. 
% Let $\aaexeq^k$ be the current automaton obtained by completion from
% $\aaex^0$ and such that
% $W(C(\aaexeq^k))=\la\F,\Q^k,\Q_f,\Delta^k\cup\Drw^k\cup\Deq^k\ra$ and 
% $\Lang(\aaexeq^{k})\cap Bad =\emptyset$. 
%
% If $\Lang(W(C(\aaexeq^{k})))\cap Bad =\emptyset$ then
% $\prune(W(C(\aaexeq^{k}))) = W(C(\aaexeq^{k}))$.  Assume now that
% $\Lang(W(C(\aaexeq^{k})))\cap Bad \not=\emptyset$ as illustrated in
% Figure \ref{fig:completion_rafinement} (considering that
% $A'=W(C(\aaexeq^{k}))$) and for all $i<k$,
% $\Lang(W(C(\aaexeq^{i})))\cap Bad=\emptyset$.
%
%
% Consequently, if one removes the transitions of $\Deq^k$ involved in
% $\psi$, then the term $t$ not recognized anymore.  This step is called
% {\it approximation refinement}.  The $\RE-$automaton
% $\aaexeq^{k+1}=\prune(W(C(\aaexeq^i)))$ is then obtained by constructing
% the smallest formula $\psi$ as a disjunction of $Eq(q,q')$ atoms with
% $q\f q'\in \Deq^k $ such that $\alpha_t\wedge\neg(\psi)$ can be
% simplified into $\bottom$.  Thus, the transitions of $\Deq^k$ involved
% in $\psi$ are removed from $\Deq^k$. Moreover, for all transitions
% $q''\stackrel{\alpha}{\f}q'\in\Drw^k$, let $\alpha'$ be the formula
% built from $\alpha$ in such a way that all $Eq(q,q')$ occuring in
% $\psi$ are replaced by $\bottom$ in $\alpha$. Consequently, one can
% simplified the formula by applying the rules $x \vee \bottom \f x$ and
% $x\wedge \bottom \f \bottom$ ($\vee$ and $\wedge$ are
% commutative). Let $\beta$ be the normal form after having applying
% these rules on $\alpha'$. If $\beta=\bottom$ then the transition
% $q''\stackrel{\alpha}{\f}q'''$ is removed from $\Drw^k$. Otherwise,
% $q''\stackrel{\alpha}{\f}q'''$ is replaced by
% $q''\stackrel{\beta}{\f}q'''$ in $\Drw^k$. Let $\Drw^{k+1}$ and
% $\Deq^{k+1}$ be respectively the updated sets $\Drw^k$ and
% $\Deq^k$. One can construct $\Delta^{k+1}$ by cleaning $\Delta^k $ of
% useless transitions.
%
%
%
%
%
%%Finally, one obtains
%%$\prune(W(C(\aaexeq^k)=\la\F,\Q^k,\Q_f,\Delta^{k+1}\cup\Drw^{k+1}\cup\Deq^{k+1}\ra
%%$ after having iterating this process on each term $t\in
%%\Lang(\aaexeq^{k})\cap Bad \not=\emptyset$. Note that
%%$\Lang(\aaexeq^{k+1})\cap Bad=\emptyset$.  Completion goes on until
%%finding a fixpoint proving the property $\phi$ or a counterexample of
%%$\phi$.
%
%
% $P$, which can be done at any time during the completion process,
% consists of prunning the current $\RE$-automaton to remove from it some spurious counterexamples.
% Assuming the reachability problem composed by $I$, $\R$, $Bad$, and $E$ a set of equations.
% If $\aaexeq^i$ is the $\RE$-automaton obtained after $i$ steps of completion and
% approximation, we know that $\aaexeq^i$ is well-defined.
% For all term $t$ of $Bad$, if we have $t \xrw{\top} q$, then $t$ is reachable, 
% and $t$ is a counterexample of the expected property. But, if we have $t \xrw{\alpha} q$
% with $\alpha \not\models \top$, then $t$ is a spurious counterexample, and we prune $\aaexeq^i$
% to remove $t$ before resuming the completion. $\alpha$ denotes the transitions of $\Deq$ corresponding
% to the approximation steps used to obtain $t$ by rewriting. Thus, we use $\alpha$ to find the transitions
% of $\aaexeq^i$ to prune.
%
% We observe that $Bad \cap \Lang(\aaexeq^i)$ may be infinite, we can not iterate on each term in the intersection
% to prune $\aaexeq^i$. We propose to a compute a formula that why the $\RE$-automaton 
%
%
%
%
%\begin{figure}
  \centering
  \begin{tikzpicture}[scale=1]
    \pgfdeclareimage[width=1.2cm]{A0}{4_contre_ex/img/A0}
    \pgfdeclareimage[width=2cm]{A1}{4_contre_ex/img/A1}
    \pgfdeclareimage[width=2.2cm]{A2}{4_contre_ex/img/A2}
    \pgfdeclareimage[width=2.2cm]{A3}{4_contre_ex/img/A3}
    \pgfdeclareimage[width=2.4cm]{A4}{4_contre_ex/img/A4}
    % 
    \node (a0) at (0, 0) {\pgfuseimage{A0}};
    \node (a1) at (4, 0) {\pgfuseimage{A1}};
    \node (a2) at (7.5, -.5) {\pgfuseimage{A2}};
    \node (a3) at (11, 0) {\pgfuseimage{A3}};
    \node (a4) at (15, 0) {\pgfuseimage{A4}};
    % virtual nodes
    % \node [circle, minimum size=0.8cm] (v0) at (3, 0) {};\\
    \node [circle, minimum size=2cm] (v1) at (7.6, 0) {};
    % 
    \node (n0) at (0, 0)       {\footnotesize $\aaex^0$};
    \node (n1) at ( 3.9, -.27) {\footnotesize $\aaexeq^{k}$};
    \node (n2) at ( 7.5, -.27) {\footnotesize $\aaexeq^{k+1}$};
    \node (n3) at (11, -.27) {\footnotesize $\aaexeq^{k+2}$};
    \node (n4) at (15  , -.27) {\footnotesize $\aaexeq^*$};
    \node (bad)at ( 7.3, -1.9) {\footnotesize $Bad$};
    % 
    \draw [->, dashed] (a0) to node[above] {\footnotesize $k-1$} node[below] {\footnotesize iter.} (a1);
    \draw [->] (a1) to node[above] {C + W } (v1);
    \draw [->] (v1) to node[above] {P} (a3);
    %\draw [->, dashed] (a3) to node[above] {\footnotesize point-fixe} (a4);
    \draw [->, dashed] (a3) to (a4);
  \end{tikzpicture}
  \caption{Étape de raffinement durant la complétion}
  \label{fig:completion_refinement}
\end{figure}


%%% Local Variables: 
%%% mode: latex
%%% TeX-master: "../../main"
%%% TeX-PDF-mode: t
%%% End: 

%
%
%
%\begin{figure}
%\begin{center}
%\includegraphics[scale=0.7]{img/refine.pdf}
%\end{center}
%\end{figure}
%
% Schéma "à la Bouajjani" qui explique le principe de l'approche à la CEGAR:
% \begin{itemize}
% \item Completion + Etape eventuelle de raffinement ...
%
% \item Dans quel cas l'étape de raffinement est-elle déclenchée???
%
%   \begin{itemize}
%   \item On a calculé $\aaex^{i+1}$
%   \item Un terme de "bad" est reconnu par l'automate : on calule en utilisant le filtrage
%     toutes les formules $\phi$:
%     soit c'est un contre-exemple car il existe $\phi = \top$ (preuve de correction pour le matching alors...)
%     donc le système viole la propriété sinon c'est un contre-exemple 
%     sinon on raffine
%   \item Expliquer que 3 étapes sont nécessaires pour raffiner :
%     \begin{itemize}
%     \item 
%       supprimer toutes les transitions de $\Drw$ telles que
%       toutes les formules deviennent fausses puisque $Eq(q, q') \equiv q \rw q' \in \Drw$
%       si toutes les formules $\phi$ sont fausses alors le terme n'est plus reconnus
%     \item 
%       exliquer le mécanisme pour garder l'information que l'on pouvait inférer
%       par clôture!!! Reprendre l'exemple $f(s(\dots s(a)\dots ))$
%      
%     \item
%       il faut supprimer tous les termes obtenus par réécriture de termes contenus
%       par dans la partie de l'approx que l'on vient de raffiner... 
%       on ne sait plus si ils sont atteignables pour l'instant, 
%       tant que l'on a pas tenter de compléter l'automate obtenu par l'élagage.
%     \end{itemize}
%   \end{itemize}
% \end{itemize}
%
%
%%\subsection{Intersection and emptyness decision}
%%\input{intersection.tex}



%%% Local Variables: 
%%% mode: latex
%%% TeX-PDF-mode: t
%%% TeX-master: "main"
%%% End: 


\section{Conclusion}
\label{sec:conclusion}
  We have presented a new CounterExample Guided Abstraction Refinement
  procedure for the verification of infinite-state systems whose
  states are represented by trees. Transitions are defined using
  rewriting rules and abstractions using equations. This work equip
  the equational abstraction framework~\cite{MeseguerPM-TCS08} with
  counterexample extraction and automatic refinement. Our next
  objective is to implement and evaluate our approach. Our technique
  should be applied to systems that are out of the scope
  of~\cite{BCHK08,BHRV06a} and should be more efficient as we do avoid
  intermediary potentially computation intensive determinization and
  inclusion checks.

A challenge in our implementation will be to develop efficient
strategies to refine the abstraction (see the discussion in Section
\ref{subsec:refinementstep}). In a second step, we are definitively
interested in extending our theory to more complex properties. As an
example, one could consider special classes of liveness properties
that were introduced in \cite{BLW04b} for parametric systems.
  

% \comments{avantage : on ne change pas la classe de complexité de la représentation,
%   on est toujours dans le domaine des réguliers\dots
%   (implique de rester dans la même classe de complexité des automates
%   meilleures que les automates à contraintes, TAGGED automatas \dots)
% }

\small{
\bibliographystyle{abbrv}
%\bibliography{biblio/sabbrev,biblio/eureca,biblio/genet,biblio/mc}
\bibliography{sabbrev,eureca,genet,additional,thesis}%,mc}}
\appendix

\newpage
\section{Proof about semantics}
\setcounter{savetheorem}{\thetheorem}
\setcounter{theorem}{4}

\begin{theorem}
  \[\forall t\in\TFQ,\; q \in \Q,\; t \xrw{\alpha}_\A q \equ t \rw_\A q \]
\end{theorem}

\begin{proof}
  The proof is easily done by induction by arguing that it is enough to forget the formulas
  manipulated by the definition~\ref{def:xrw_alpha} to have the equivalent step with $\rw_\A$.
\end{proof}
\setcounter{theorem}{\thesavetheorem}

\section{Proofs about $\compl$}


\setcounter{savetheorem}{\thetheorem}

\begin{property}
  $\aaexeq^0$ is well-defined.
\end{property}

\begin{proof}
   $\aaexeq^0 = \la \F, \Q^0, \Q_f, \Delta^0\ra$ fits the definition \ref{def:well-defined}, only if the two items
   of the definition \ref{def:well-defined} are verified.

   We know that $\aaex^0$ has no $\varepsilon$-transitions. All
   terms are recognized using transitions of $\Delta^0$. It means that for all state $q$
   the set of terms is defined as $\{ t \in \TF | t \rwne{\Delta^0} q\}$ which is
   equal to $Rep(q)$, the set of representatives for $q$.
   We also remark that for all term $t$, $t \rwne{\Delta_0} q$ implies $t \xrw{\top}_{\aaexeq^0} q$:
   the second and third point of the definition \ref{def:xrw_alpha}, are not used, since $\Drw^0$ and $\Deq^0$ are empty.
   The first item of definition \ref{def:well-defined} is ensured: for all state $q$, and all term $t \xrw{\top} q$,
   we have $t \in Rep(q)$, and $t \rwR^* t$ by reflexivity.

   The second item of \ref{def:well-defined} holds, since $\Drw^0$ is empty. 
\end{proof}




\begin{lemma}[Solving one critical pair preserves well-definition]
\label{lemma:C-well-defined}
  Let $\A$ and $\A'$ be two $\RE-$automaton such that 
  $\A'$ is obtained from $\A$ by solving 
  a critical pair $\la r\sigma,\alpha,q \ra$ of $\A$. 
  If $\A$ is well-defined then so is $\A'$. 
\end{lemma}
\begin{proof}
  Assume that $\A=\la \F, \Q, \Q_f, \Delta\cup \Drw\cup \Deq\ra$ and
  $\A'=\la \F, \Q', \Q_f, \Delta'\cup \Drw'\cup \Deq' \ra$.  According
  to Definition \ref{def:resolution_cp}, $\Delta'=\Delta \cup
  \norm(r\sigma, \Delta\setminus \Delta^0)$, $\Drw'= \{
  q'\xrw{\alpha} q\}\cup \Drw$ and $\Deq'=\Deq$.
  Following Definition \ref{def:well-defined}, we first show in (\ref{one}) that 
  for all state $q''$ of $\A'$, and all term $v$ such that
  $v \xrw{\top}_{\A'} q''$, there exists $u$ a term representative
  of $q''$ such that $u \rw^*_\R v$. Then, in (\ref{two}) we show that
  if $q_1 \xrw{\phi} q_2$ is a transition of $\Drw'$, then there exist terms
    $s,t\in \TF$ such that $s\rwtag{\phi}_{\A'} q_1$, $t\rwtag{\top}_{\A'} q_2$
    and $t \rw_\R s$.

\medskip
  
\begin{enumerate}
\item \label{one} 

% Let $q_{\A'}$ be a state of $\Q'$.  Since $\A$ is
%   sintactically included in $\A'$, one can deduce that the property
%   still holds for any term $v\in\TF$ sucht $v \xrw{\top}_{\A}
%   q_{\A'}$.  Suppose now a term $v\in\TF$ such that
%   $v\in\Lang(\A',q_{\A'})$, $v\notin\Lang(\A,q_{\A'})$ and
%   $v\rwtag{\top}q_{\A'}$.  Then, there are two possibilities: either
%   $q_{\A'}$ is a new state, i.e. $q_{\A'}\notin \Q$, or $q_{\A'}$ is a
%   state of $\A$, $\alpha=\top$ and the transition $q'\rwtag{\top} q$
%   added to $\Drw$ is used at least once during the reduction of $v$ on
%   $q_{\A'}$.  Let us study the two possibilies.


\newcommand{\xrwa}{\xrw{\top}_\A}
\newcommand{\xrwap}{\xrw{\top}_{\A'}}

We prove the property by induction on the height of $t$. Let us assume that for all term $t'$ of height
lesser than the height of $t$ and for all $q\in \Q_{\A'}$, we have $t' \xrwap q
\spf \exists u\in Rep(q): u \rwR^* t'$. Now let us prove that this is true for
$t$. We prove it by case on $q\in\Q_\A$ and $t\xrwa q$:

\begin{itemize}
\item If $q\in\Q_\A$ and $t\xrwa q$ then since $\A$ is well defined, we get the
  representative $u\in Rep(q)$ such that $u\rwR^* t$ from well-definition of $\A$.
\item If $q\in\Q_\A$, $t \not\xrwa q$ and $t \xrwap q$. We prove the property by
induction on the height of $t$. Now let us consider the term $t$. Since $t$ is recognized
in $\A'$ and not in $\A$, this means that the run $t' \xrwap q$ needs the
transitions added by the resolution of a critical pair. Hence there exists a
rewrite rule $l \rw r$, a substitution $\sigma:\X \mapsto \Q_\A$, a formula
$\alpha$ and a state $q_c$ such that $l\sigma \xrw{\alpha}_\A q_c$ and $\la
r\sigma, \alpha, q_c\ra$ is the critical pair. Moreover, the resolution of this
critical pair creates $\Delta_{\A'}=\norm(r\sigma,\Delta_\A\setminus \Delta_0)$
and $\Drw^{\A'}=\Drw^{\A} \cup \{q'_c \rw q_c\}$ such that $r\sigma
\rw^!_{\Delta'\setminus\Delta_0} q'_c$. Recall that $t' \xrwap q$ needs transitions not occurring
in $\A$. However, all the {\em new} transitions produced by
$\norm(r\sigma,\Delta_\A\setminus \Delta_0)$ necessarily range on {\em new}
states, i.e. states not occurring in $\Q_\A$. As a result, those transitions
cannot be used to get $t' \xrwap q$ with $q\in\Q_\A$. This means that the run $t
\xrwap q$ uses at least once $q'_c \rw q_c$ and $\alpha=\top$ since the whole
run is labelled by $\top$. To sum up, we know that there exists a ground context
$C[\,]$ such that $t=C[t'] \xrwap C[q'_c] \xrwap C[q_c] \xrwa q$. Note that if
$q'_c \rw q_c$ the same reasonning can be applied. We start to reason on the 
occurrence of $q'_c \rw q_c$ that is the closest to $q$. Now, our
objective is to show that there exists $u\in Rep(q_c')$ such that $u \rwR^*
t'$. If $t'\xrwa q'_c$ then since $\A$ is well defined we directly have the
result using definition~\ref{def:well-defined}. Otherwise this means that $q'_c$
is new for $\A$ (i.e. $q'_c\not\in\Q_\A$) and has been added by the resolution
of the critical pair, i.e. $r\sigma \rw^!_{\Delta'} q'_c$. Because of
Theorem~\ref{lem:norm_determinism}, we get that there exists a substitution
$\sigma':\X \mapsto \TF$ such that $t'=r\sigma'$. Using the same theorem, from
$t'=r\sigma' \xrwap q'_c$ and $r\sigma \xrwap q'_c$, we get that for all
variable $x$ of $r$: $\sigma'(x) \xrwap \sigma(x)$. Note that
$\sigma(x)\in\Q_\A$ and that $\sigma'(x)$ are necessarily terms of height lesser
to the height of $t$. Using the induction hypothesis, we get that for all state
$\sigma(x)$ there exists a representative $u_x$ such that $u_x \rwR^*
\sigma'(x)$. Let $\sigma_{Rep}$ be the substitution mapping every variable $x$
to $u_x$. We have $r\sigma_{Rep} \in Rep(q'_c)$. Moreover, $r\sigma_{Rep} \rwR^*
r\sigma'=t'$. Now, our objective is to show that $l\sigma_{Rep} \rwR
r\sigma_{Rep}$. This is not straightforward since $\var(l)\supseteq \var(r)$.
However, it is possible to extend $\sigma_{Rep}$ into $\sigma'_{Rep}$ where every
variable $y$ of $\var(l)$ not occurring in $\sigma_{Rep}$ is mapped to a
representative of $\sigma(y)$. Hence, $l \sigma'_{Rep} \rwR r\sigma'_{Rep}
\rwR^* t'$. From the critical pair we know that $l\sigma \xrw{\alpha}_\A q_c$ and
we found that $\alpha=\top$. Hence $l\sigma'_{Rep} \xrwa q_c$. Since $\A$ is
well-defined, we get that there is a representative $v \in Rep(q_c)$ such that
$v \rwR^* l\sigma'_{Rep}$. By transitivity of $\rwR$, we get that $v \rwR^* t'$.
Above, we found that $t=C[t'] \xrwap C[q'_c] \xrwap C[q_c] \xrwa q$. From this
and $v\in Rep(q_c)$, we get that $C[v] \xrwa q$. Since $\A$ is well defined, we
know that there exists a representative $w\in Rep(q)$ such that $w\rwR^* C[v]$.
To conclude, we found $w\in Rep(q)$ and $w \rwR^* C[v] \rw C[t']=t$.

\item If $q \not \in \Q_A$ ($q\in\Q'_\A\setminus \Q_\A$), $t \not\xrwa q$ and $t
  \xrwap q$. Since $q\in \Q'_\A\setminus \Q_\A$, we know that $q$ has been added
  by the resolution of a critical pair. As above, we can deduce that there
  exists a rewrite rule $l \rw r$, a substitution $\sigma:\X \mapsto \Q_\A$, a
  formula $\alpha$ and a state $q_c$ such that $l\sigma \xrw{\alpha}_\A q_c$ and
  $\la r\sigma, \alpha, q_c\ra$ is the critical pair. Moreover, the resolution
  of this critical pair creates $\Delta_{\A'}=\norm(r\sigma,\Delta_\A\setminus
  \Delta_0)$ and $\Drw^{\A'}=\Drw^{\A} \cup \{q'_c \rw q_c\}$ such that $r\sigma
  \rw^!_{\Delta'\setminus\Delta_0} q'_c$. Since $q$ is a new state of $\A'$, $q$
  has been necessarily used for the normalization of a subterm of
  $r\sigma$. More precisely, we know that there exists a term $s\in\TFX$ and a
  context $C[\,]$ (possibly empty) such that $r\sigma=C[s]$, $C[s]\sigma \rw_{\Delta'}^* q'_c$
  and $s\sigma \rw^*_{\Delta'} q$. Similarly, we know that there exists a
  substitution $\sigma':\X \mapsto \TF$ such that $s\sigma'=t$.
  We get that for every variable $x$ of $r$: $\sigma'(x) \xrwap \sigma(x)$.
  Note that $\sigma(x)\in\Q_\A$ and that $\sigma'(x)$ are necessarily terms of
  height lesser to the height of $t$. Using the induction hypothesis, we get
  that for every state $\sigma(x)$ there exists a representative $u_x$ such that
  $u_x \rwR^* \sigma'(x)$. Let $\sigma_{Rep}$ be the substitution mapping every
  variable $x$ to $u_x$. We have $s\sigma_{Rep} \in Rep(q)$ and $s\sigma_{Rep}
  \rwR^* s\sigma'=t$.
\end{itemize}


    
\item \label{two} Easily, for any transitions $q_1\stackrel{\phi}{\f}
  q_2\in \Drw$, the property still holds.  Let us focus now on the
  transition $q'\stackrel{\alpha}{\f}q$ resulting from the resolving
  of the critical pair $\la r\sigma,\alpha,q\ra$.  By definition, the
  critical pair $\la r\sigma,\alpha,q\ra$ results from the application
  of the matching algorithm of Definition
  \ref{def:matching-algorithm}.  So there exists a rule $l\f r\in \R$
  such that $(\alpha,\sigma)\in S$ with $l\match q \vdash_\A S
  $. Moreover, since the critical pair has to be solved:
  $l\sigma\stackrel{\alpha}{\f}q$ and there is no formula $\alpha'$
  such that $r\sigma \xrw{\alpha'}_{\A} q$.  Since $\R$ is
  left-linear, for each variable $x\in\var(l)$, one can define the
  substitution $\sigma':\X\f \TF$ as follows: Assuming $q_s$ being the
  state of $\A$ such that $\sigma(x)=q_s$, let $\sigma'(x)=Rep(q_s)$.
  By definition of $Rep$, $Rep(q_s)\stackrel{\top}{\f}q_s$.  So, there
  exists a derivation such that $l\sigma'\stackrel{\top}{\f}l\sigma$
  and $l\sigma\stackrel{\alpha}{\f}q$.  One can deduce that
  $r\sigma'\stackrel{\top}{\f}{r\sigma}$. According to Lemma
  \ref{lem:norm_determinism}, one can deduce that there exists a
  unique $q'$ such that
  $r\sigma\f^*_{\norm(r\sigma,\Delta\setminus\Delta^0)}q'$. If
  $\norm(r\sigma,\Delta\setminus\Delta^0)\not=\emptyset$ then each
  transition composing it is of the form $f(q'_1,\ldots,q'_n)\f
  q'_{n+1}$. Consequently, $r\sigma\stackrel{\top}{\f} q'$.
  Considering the transition $q'\stackrel{\alpha}{\f}{q}$, one has
  $r\sigma'\stackrel{\top}{\f}r\sigma\stackrel{\top}{\f}q'\stackrel{\alpha}{\f}{q}
  $.  Finally, assuming $s=l\sigma'$ and $t=r\sigma'$, there exists
  $s,t\in\TF$ such that one has $s\stackrel{\alpha}{\f}q $,
  $t\stackrel{\alpha}{\f}q' $ and $s\rw_\R t $.

\end{enumerate}
To conclude, $\A'$ is also well-defined.

\end{proof}



\setcounter{theorem}{9}


\begin{theorem}
  \label{thm:C}
  If $\aaexeq^i$ is well-defined then so is $\compl(\aaexeq^i)$, 
  and $\forall q \in \Q^i$, $\forall t \in
  \Lang(\aaexeq^i, q)$, $\forall t \in\TF$, $t \rw_\R t' \imp t' \in
  \Lang(\compl(\aaexeq^{i}), q)$.
\end{theorem}

\begin{proof}

  Let $CP$ be the finite set of critical pairs computed from
  $\aaexeq^i$ to solve.  Assume that $CP=\{\la
  r1\sigma_1,\alpha_1,q_1\ra,\ldots, \la r1\sigma_m,\alpha_1,q_m\ra
  \}$. By definition, considering $\A_0=\aaexeq^i$ there exists a
  sequence of $\RE-$automata $\A_1,\ldots,\A_m$ where $\A_j$ is obtained
  from $\A_{j-1}$ by solving the critical pair $\la r_j\sigma_j,
  \alpha_j,q_j\ra$ according Definition \ref{def:resolution_cp}. Thus, 
  $\compl(\aaexeq^i)=\A_n$. For a 
  question of readability and in order to prevent any confusion between notations, 
  each $\RE-$automaton $\A_j$ is defined as follows:  $\A_j=\la \F,\Q^{n+1},\Q_{f},\Delta'^{j}\cup \Drw'^{j}\cup\Deq'^{j} \ra$.

  First, let us show that $\compl(\aaexeq^i)$ is well-defined if $\aaexeq^i$ is well-defined.

  \paragraph{ $\compl(\aaexeq^i)$ is well-defined: }
    
  Let $P_n$ be the following proposition: $\A_n$ is well-defined. 

  \begin{itemize}
  \item $P_0$: Trivial since $\A_0=\aaexeq^i$ and $\aaexeq^O$ is well-defined by hypothesis.
  \item $ P_n\Rightarrow P_{n+1}$: By hypothesis, $\A_{n+1}$ is
    obtained from $\A_n$ by solving the critical pair $\la
    r_{n+1}\sigma_{n+1},\alpha_{n+1},$ $q_{n+1}\ra$. Applying Lemma \ref{lemma:C-well-defined}, 
    one obtains automatically that $\A_{n+1}$ is well-defined. 
  \end{itemize}
  So, one can deduce that $\compl(\aaexeq^i)$ is well-defined.

  \paragraph{$\compl(\aaexeq^i)$  covers terms accessible in one rewrite step from terms of  $\aaexeq^i$:}
  Let $q$ be a state of $\aaexeq^i$ and $t$ be a term of
  $\Lang(\aaexeq^i,q)$.  Suppose there exist a position $p\in
  \pos(t)$, a rule $l\f r\in\R$ and a substitution $\sigma':\X\f \TF$
  such that $t|_p=l\sigma'$.  Let $t'$ be the term such that
  $t'=t[r\sigma']_p$.  Since $t\in \Lang(\aaexeq^i,q)$, there exists a
  state $q'$ of $\aaexeq^i$ such that $t|_p=l\sigma'\f^*_{\aaexeq^i}
  q'$ and $t[q']_p\f^*_{\aaexeq^i}q$.  Following Property
  \ref{prop:matching-complete}, there exists $(\alpha,\sigma)\in S$
  with $l\match q'\vdash_{\aaexeq^i}S$ such that
  $l\sigma'\f^*_{\aaexeq^i}l\sigma$ and $l\sigma\f^*_{\aaexeq^i} q'$.
  If $\la r\sigma,\alpha,q' \ra$ is already solved then
  $r\sigma\f^*_{\aaexeq^i}q'$. Consequently, $r\sigma'$ can also be
  reduced to $q'$ in $\aaexeq^i$. Since
  $t'=t[r\sigma']_p\f^*_{\aaexeq^i}q$, $t'\in
  \Lang(\compl(\aaexeq^i),q)$. Suppose now that
  $r\sigma\not\f^*_{\aaexeq^i}q'$.  So, there exists $\la r_i\sigma_
  i,\alpha_i,q_i\ra\in CP$ such that $\la r_i\sigma_
  i,\alpha_i,q_i\ra=\la r\sigma,\alpha,q'\ra$.  By construction,
  $r\sigma\f^*_{\A_i}q'$. Consequently, $r\sigma'$ can also be
  reduced to $q'$ in $\A_i$. Since $\A_i$ is syntactically included in $\compl(\aaexeq^i)$, one can deduce that 
  $t'=t[r\sigma']_p\f^*_{C(\aaexeq^i)}q$. 
  Concluding the proof.
\end{proof}

\medskip

\setcounter{theorem}{\thesavetheorem}


% \begin{definition}[$\Delta-$coherence]
%   \label{prop:determinism}

% Let $\Delta$ be a set of normalized transitions. 
% Thus, $\aaex^i$ is said to be $\Delta-$coherent iff
% \begin{enumerate}
%   \item $\Delta\subseteq \Delta^i$,
%   \item if $\Delta^i\setminus\Delta$ contains two transitions of the form $f(q_1, \dots, q_n) \rw q$ and
%   $f(q_1, \dots, q_n) \rw q'$, then $q = q'$ and 
% \item if $\Delta^i\setminus\Delta$ contains two transitions of the form $t \rw q$ and
%   $t' \rw q$ with $t,t'$ two normalized configurations, then $t = t'$.
% \end{enumerate}
% \end{definition}
% \comments{Yohan : A virer ? }





%%% Local Variables: 
%%% mode: latex
%%% TeX-master: "main"
%%% TeX-PDF-mode: t
%%% End: 


\section{Proofs about $\widen$}

\begin{lemma}[$\widen$ preserves well-definition]
  Let $\A$ be a $\RE$-automaton. If $\A$ is well-defined, then so is $\widen(\A)$.
\end{lemma}

\begin{proof}
  Assume that $\A = \la \F, \Q, Q_f, \Delta \cup \Drw \cup \Deq\ra$ is well-defined.
  We have $\widen(\A) = \la \F, \Q, \Q_f, \Delta \cup \Drw \cup \Deq'\ra$, where $\Drw \supseteq \Drw'$,
  since $\widen$ only adds transitions to the $\Drw$. We have to prove the two items of definition~\ref{def:well-defined}.
  \begin{itemize}
  \item 
    The transitions of $\Deq'$ are never used for a run $\xrw{\alpha}$ where $\alpha = \top$,
    thanks to the second point of the definition~\ref{def:xrw_alpha}.
    It means that for all term $t$ and all state $q$, $t \xrw{\top}_{W(\A)} q$ is equivalent to $t \xrw{\top}_\A q$.
    Since $\A$ is well-defined, we know that there exists $u \in Rep(q)$ such that $u \rw^*_\R t$.
    $u$ is also a representative of $\widen(\A)$, and we deduce that first point of the definition~\ref{def:well-defined} 
    holds for $\widen(\A)$.
  \item
    Function $\widen$ only adds transitions to $\Deq'$ and do not remove transitions of $\A$.
    For all transitions $q \xrw{\alpha} q' \in \Drw'$ we have $q \xrw{\alpha} q' \in Drw$.
    Since $\A$ is well-defined, we know that there exist terms $s,t\in \TF$ such that
    $s\xrw{\phi}_\A q$, $t\xrw{\top}_\A q'$ and $t \rw_\R s$.
    We also have $s\xrw{\phi}_{\widen(\A)} q$, $t\xrw{\top}_{\widen(\A)} q'$ and $t \rw_\R s$.
  \end{itemize}
\end{proof}

\begin{lemma}
  For all $\RE$-automaton $\A$, $\Lang(\widen(\A)) \supseteq \Lang(\A)$.
\end{lemma}
\begin{proof}
This is easy to see since widening only adds transitions, and thus, does
not restrict the recognized language.
\end{proof}


% \begin{lemma}
% \label{lem:wf}
% The widening relation $\simp_E$ is well founded. 
% \end{lemma}

% \begin{proof} Widening can add, at most, transitions between all states
%   of the tree automaton $\A$ which has a finite number of states.
% \end{proof}


% Couvert par l'autre section

% \section{Proofs about intersection}

% \begin{lemma}[Emptiness decision of the product of a $\RE$-automaton and a
%   tree automaton]
%   Let $\A$ be a $\RE$-automaton and $\B$ a tree automaton. Let $S$ be the set of
%   reachable states of $\A\times \B$ defined according to
%   definition~\ref{def:reachable-states}. For all final state $q$ of $\A$, all
%   final state $q'$ of $\B$, all formulas $\phi_S\neq\bot$, $\phi\neq \bot$ and
%   all term $t\in\TF$, we have $t\xrw{\phi}^*_\A q$ and $t \rw_\B^* q'$
%   (i.e. $\Lang(\A)\cap \Lang(\B) \neq \emptyset$) if and only if there exists a
%   triple $(q,q',\phi_S)\in S$ such that $\phi \models \phi_S$. 
% \end{lemma}

% \begin{proof}
%   Let $\A =\langle \F, \Q^\A,\Q^\A_f,\Delta^\A, \Drw, \Deq \rangle$
%   be the $\RE$-automaton and $\B=\langle \F, \Q^\B,\Q^\B_f,\Delta^\B \rangle$ be
%   the tree automaton.  
%   % We use definition~\ref{def:xrw_alpha} for the recognized language
%   % $\Lang(\A,q)=\{t\in \TF \sep t \xrw{\phi}_\A^* q \mbox{ and $\phi$ 
%   %   satisfiable}\}$. 
%   % We prove that $(q,q',\phi_S)\in S$ if 
%   % and only if %$\Lang(\A,q) \cap \Lang(\B,q') \neq \emptyset$.
%   % there exists a term $t\in\TF$ such that $t\xrw{\phi}^*_\A q$, $t
%   % \rw_\B^* q'$ and $\phi\models \phi_S$.
%   We prove a stronger property on all states $q$ of $\A$ and $q'$ of $\B$ (and not
%   only for final states).
%   First, we prove the 'only if' part. Let us assume that 
%   % $\Lang(\A,q) \cap \Lang(\B,q') \neq \emptyset$, hence that 
%   there exists a term $t\in\TF$ such that $t\xrw{\phi}^*_\A q$, $t \rw_\B^* q'$.
%   % and $\phi$ satisfiable. 
%   By induction on the height of $t$ we have:
  
%   \begin{itemize}
%   \item If $t$ is a constant, since $\B$ is an epsilon-free tree automaton, the
%     only way to have $t \rw^*_\B q'$ is to have $t \rw q' \in \B$. With regards to
%     $\A$, by definition~\ref{def:xrw_alpha}, $t\xrw{\phi}^*_\A q$ means that
%     there exists states $q_0,q_1, \ldots, q_n$ and formulas $\phi_1,\ldots,
%     \phi_n$ such that $t \rw_{\Delta_\A} q_0 \xrw{\phi_1} q_1 \xrw{\phi_2}
%     \ldots q_n$ with $q=q_n$ and $\phi=\phi_1\wedge \ldots \wedge
%     \phi_n$.% is satisfiable.
%     Transitions $q_i \xrw{\phi_i} q_{i+1}$ are either transitions of $\Drw$ or
%     transitions of $\Deq$ with $\phi_i=\top$.  Because of transitions $t \rw q_0
%     \in \Delta_\A$ and $t \rw q' \in \Delta_\B$, using the first case of
%     definition~\ref{def:reachable-states}, we get that $(q_0,q',\top) \in
%     S$. Similarly, using the second case of the definition, we obtain that there
%     exists formulas $\phi'_i$ with $i=1\ldots n$ such that $(q_1,q', \phi_1\vee
%     \phi'_1), (q_2,q',(\phi_1\wedge\phi_2)\vee \phi'_2),\ldots (q_n,q', (\phi_1
%     \wedge \ldots \wedge \phi_n)\vee \phi'_n)$ belong to $S$. Finally, since
%     $q_n=q$ and $\phi=\phi_1\wedge \ldots \wedge \phi_n$, we that
%     $(q,q',\phi\vee\phi_n') \in S$. Furthermore, we trivially have that $\phi_S=
%     \phi\vee \phi_n'$ and $\phi \models \phi_S$. % since $\phi$ is
%     % satisfiable, so is $\phi \vee \phi'$.
    
%   \item Assume that for all term of height lesser or equal to $n\in\NN$, the
%     property is true. Let us prove that it is also true for a term $f(t_1, \ldots,
%     t_n)$ with $t_1, \ldots, t_n$ of height lesser or equal to $n$. Since $f(t_1,
%     \ldots, t_n) \rw^*_\B q'$ and $\B$ is an epsilon free tree automaton, we
%     obtain that $\exists q'_1,\ldots,q'_n\in\Q^\B$ such that $\forall i=1\ldots n:
%     t_i \rw^*_\B q'_i$ and $f(q'_1,\ldots,q'_n) \rw q' \in \Delta_\B$. With
%     regards to $\A$,
%     by definition~\ref{def:xrw_alpha}, $f(t_1, \ldots, t_n)
%     \xrw{\phi}^*_\A q$ means that there exists states $q_0,q_1, \ldots,
%     q_m,q''_1,\ldots,q''_n$ and formulas $\phi_1,\ldots, \phi_m, \phi'_1, \ldots,
%     \phi'_n$ such that $\forall i=1\ldots n: t_i \xrw{\phi'_i}^*_\A q''_i$,
%     $f(q''_1,\ldots, q''_n) \rw_{\Delta_\A} q_0$ and $q_0 \xrw{\phi_1} q_1
%     \xrw{\phi_2} \ldots q_n$, $q=q_n$. Furthermore, we obtain that $\phi=
%     \bigwedge_{i=1}^{n} \phi'_i \wedge \bigwedge_{i=1}^{m} \phi_i$. % is
%     % satisfiable. 
%     Since terms $t_i$ are of height lesser or equal to $n$, $\forall
%     i=1\ldots n: t_i \rw^*_\B q_i$ and $\forall i=1 \ldots n: t_i \xrw{\phi'_i}_\A^*
%     q''_i$, we can apply the induction hypothesis and obtain that $\forall
%     i=1\ldots n: (q_i, q''_i, \phi''_i) \in S$ with $\phi'_i \models \phi''_i$. % and $\phi'_i$ satisfiable.  
%     Besides to this, using case~1 of definition~\ref{def:xrw_alpha} on
%     $f(q_1,\ldots,q_n) \rw q' \in \Delta_\B$, $f(q''_1,\ldots,q''_n) \rw q_0 \in
%     \Delta_\A$, and $\forall i=1\ldots n: (q_i, q''_i, \phi''_i) \in S$, we obtain
%     that there exists a formula $\phi'$ such that $(q_0,q', (\bigwedge_{i=1}^n
%     \phi''_i) \vee \phi')\in S$. Then, like in the base case, since $q_0
%     \xrw{\phi_1} q_1 \xrw{\phi_2} \ldots q_n$, $q=q_n$, we can deduce that
%     there exists a formula $\phi''$ such that $(q,q',(\bigwedge_{i=1}^n \phi''_i
%     \wedge \bigwedge_{i=1}^{m} \phi_i) \vee \phi'')\in S$. Let $\phi_S= (\bigwedge_{i=1}^n \phi''_i
%     \wedge \bigwedge_{i=1}^{m} \phi_i) \vee \phi''$. Since we know from above that
%     $\phi=\bigwedge_{i=1}^n \phi'_i \wedge \bigwedge_{i=1}^{m} \phi_i$ and
%     $\forall i=1\ldots n: \phi'_i \models \phi''_i$, we obtain that % is satisfiable, so is 
%     $\phi \models \phi_S$.
%   \end{itemize}
  
%   \medskip
%   Second, we prove the 'if' part: if $(q,q',\phi_S)\in S$ and $\phi_S \neq \bot$
%   then there exists a term $t$ and a formula $\phi \neq \bot$ such that $\phi \models \phi_S$,
%   % and $S$ satisfiable,
%   $t\xrw{\phi}^*_\A q$ and $t \rw_\B^* q'$. We make a proof by induction
%   on the number of applications of the two rules of
%   definition~\ref{def:reachable-states}, necessary to prove that $(q,q',\phi_S)$ in $S$.

%   \begin{itemize}
%   \item If the number of steps is $0$ then, since the computation of $S$ starts
%     from the set $\Q^\A \times \Q^\B \times \bot$, then all $(q,q',\phi_S)$
%     are such that $\phi_S=\bot$, which is a contradiction.

%   \item We assume that the property is true for any triple $(q,q',\phi_S)$ which can
%     be deducted by $n$ or less applications of the rules of
%     definition~\ref{def:reachable-states}. Now, we consider the case of a triple
%     $(q,q',\phi_S)$ that is deduced at the $n+1$-th step of application of the
%     deduction rules.
%     \begin{itemize}
%     \item If the first rule is concerned, this means that there exists triples
%       $(q_1,q'_1, \phi_1),\ldots,(q_n,q'_n,\phi_n)$ and $(q,q',\phi)$ in $S$
%       deduced before $n+1$-th step, as well as transitions $f(q_1,\ldots,q_n)\rw q
%       \in \Delta_\A$ and $f(q'_1,\ldots,q'_n)\rw q' \in \Delta_\B$. Furthermore,
%       we know that $\phi_S=\phi \vee \bigwedge_{i=1}^n \phi_i$.
%       % and by hypothesis $\rho$ is satisfiable. 
%       If $\phi\neq \bot$ then, since $(q,q',\phi)$ was shown to belong to $S$
%       before $n+1$-th step, we can apply the induction hypothesis and directly
%       obtain that there exists a term $t$ and a formula $\phi'$ such that $\phi'
%       \models \phi$, $t \xrw{\phi'}^*_\A q$ and $t \rwB^* q'$. Note that $\phi'
%       \models \phi$ implies $\phi' \models \phi_S$.
%       Otherwise, if $\phi=\bot$, then % $\bigwedge_{i=1}^n \phi_i$ is then 
%       we can apply the induction hypothesis on triples $(q_i,q'_i,\phi_i)$,
%       $i=1\ldots n$ and obtain that $\forall i=1\ldots n:\exists \phi_i':\exists
%       t_i \in\TF: \phi'_i \models \phi_i$, $t_i \xrw{\phi_i'}_\A^* q_i$ and $t_i
%       \rwB^* q'_i$. Finally, 
%       because of the two transitions $f(q_1,\ldots,q_n)\rw q \in \Delta_\A$ and
%       $f(q'_1,\ldots,q'_n)\rw q' \in \Delta_\B$, we get that $f(t_1,\ldots,t_n)
%       \xrw{\phi'}_\A^* f(q_1,\ldots,q_n) \rwA^* q$ with
%       $\phi'=\bigwedge_{i=1}^n \phi'_i$ on one side and $f(t_1,\ldots,t_n) 
%       \rwB f(q'_1,\ldots,q'_n) \rwB^* q$ on the other side. Furthermore,
%       since $\forall i=1\ldots n: \phi'_i \models \phi_i$, we have
%       $\bigwedge_{i=1}^n \phi'_i \models \bigwedge_{i=1}^n \phi_i$. Recall that
%       $\phi'= \bigwedge_{i=1}^n \phi'_i$ and $\phi_S= \phi\vee \bigwedge_{i=1}^n
%       \phi_i$. Hence, $\phi' \models \phi_S$.
%     \item If the second rule is concerned, this means that there exists triples
%       $(q_1,q',\phi_1)$ and $(q,q',\phi_2)$ in $S$ deduced before the $n+1$-th
%       step. Furthermore, we know that $\phi_S=(\phi_1 \wedge \phi) \vee \phi_2$.
%       Like above, if $\phi_2 \neq \bot$ then we can apply induction hypothesis on
%       $(q,q',\phi_2)$ and trivially get the result. Otherwise, if $\phi_2=\bot$
%       then we can use induction hypothesis on the triple $(q_1,q',\phi_1)$ and
%       obtain that there exists a formula $\phi_1'$ and a term $t_1$ such that 
%       $t_1\xrw{\phi_1'}_\A^* q_1$, $t_1 \rwB^* q'$ and $\phi_1' \models \phi_1$.
%       Then, by case on the epsilon transition used for the deduction on $S$, we
%       prove that $t_1 \xrw{\phi_1'\wedge \phi}_\A^* q$:
%       \begin{itemize}
%       \item Assume that $q_1 \xrw{\phi} q \in \Drw$. Then, by
%         definition~\ref{def:xrw_alpha}, we obtain that $t_1
%         \xrw{\phi'_1 \wedge \phi}_\A^* q$. Furthermore, since $\phi'_1
%         \models \phi_1$, we have that $\phi'_1 \wedge \phi \models
%         \phi_1 \wedge \phi$ and, finally, that $\phi'_1 \wedge \phi
%         \models \phi_S$.
%       \item Assume that $q_1 \rw q \in \Deq$. Then, by
%         definition~\ref{def:xrw_alpha}, we obtain that $t \xrw{\phi_1
%           \wedge Eq(q_1,q)}_\A^* q$. Finally, like above, we can
%         deduce that $\phi'_1 \wedge Eq(q_1,q) \models \phi_1 \wedge
%         Eq(q_1,q)$ and thus $\phi'_1 \wedge Eq(q_1,q) \models \phi_S$.
%       \end{itemize}
%     \end{itemize}
%   \end{itemize}
% \end{proof}



%%% Local Variables: 
%%% mode: latex
%%% TeX-master: "main"
%%% TeX-PDF-mode: t
%%% End: 

\section{Proofs about matching}
\begin{theorem}[Matching Algorithm is complete]
  \label{thm:matching-complete}
  Let $A$ be a $\RE-$automaton, $q$ one of its states, $l \in \TFX$ the
  linear left member of a rewriting rule and $\sigma$ a $\Q$-substitution with 
  a domain range-restricted to $\vars(l)$. If the set $S$ is solution of the matching problem
  $l\sigma \match q$, then we have $\forall (\alpha, \sigma),\ l\sigma \xrw{\alpha}_\A q \Longleftrightarrow (\alpha, \sigma) \in S$
\end{theorem}

\begin{proof}
  Assuming $\F$ a set of symbols, $\X$ a set of variable and
  $\Q$ a set of states. We define $A = \la \F, \Q, \Q_f, \Delta \cup \Drw \cup \Deq\ra$;
  $l \in \TFX$ and $q \in \Q$; $\sigma : \var(l) \rw \Q$ and $\alpha = \bigwedge_1^n Eq(q_k,q'_k)$ such that $l\sigma \xrw{\alpha}_\A q$.

  \medskip
  \noindent
  The proof is done by induction on the term $l$.

  \medskip
  \noindent
  {\bf Base case}: $l$ is a variable.

  In this case, $\sigma$ must be a $\Q$-substitution of the form $\sigma =\{ l \mapsto q' \}$.
  Using this observation and the hypothesis, we have $q' \xrw{\alpha}_\A q$.
  The matching problem $l \match q$ is solved using Rule (Var).
  This means that $S = \{(\alpha_k, \{l \mapsto q_k\}) \sep q_k \xrw{\alpha_k}_\A q\}$.
  By definition of $S$ we see that $S$ contains ($\alpha, \sigma$).

  \medskip
  \noindent
  {\bf Induction} :
  Assume now $l$ is a linear term of the form $f(t_1,\dots, t_n)$.

  We are going to decompose $f(t_1,\dots, t_n)\sigma \xrw{\alpha}_\A q$ into sequences of transitions.
  First observe that, by splitting $\sigma$ into $\sigma_1$ \dots $\sigma_n$, we have that $f(t_1,\dots, t_n)\sigma$ 
  is equal to $f(t_1\sigma_1, \dots,t_n\sigma_n)$.
  Assume $\sigma = \sigma_1 \sqcup \dots \sqcup \sigma_n$ with $dom(\sigma_i) = \vars(t_i)$ 
  and $\forall x \in dom(\sigma_i),\ \sigma_i(x) = \sigma(x)$.
  Since $l$ is linear, each variable in $X$ occurs at most one time in $l$.
  This means that the sets $\vars(t_i)$ are disjoints and so are the domains of the $\sigma_i$.
  This ensures that $\sigma$ is well-defined.


  %%%%%%%%%%%%% PAS DU TOUT CLAIR ICI LA DECOMPOSITION
  Now, we study the decomposition of $f(t_1\sigma_1,\dots, t_n\sigma_n) \xrw{\alpha}_\A q$ to show that transitions of $\A$ used
  to recognized the term $f(t_1\sigma_1,\dots, t_n\sigma_n)$ are considered by the corresponding steps of the matching algorithm.

  We observe that the term $f(t_1\sigma_1,\dots, t_n\sigma_n)$ is recognized in state $q$. Indeed, we have $f(q_1,\dots, q_n) \rw q' \in \Delta$, and each subterm
  $t_i\sigma_i$ is recognized in state $q_i$ such that $t_i\sigma_i \xrw{\alpha_i} q_i$. 
  %%%% ICI %%%%%%%%%%%% Correction pas lisible
  Composing recognizing of each subterm, we obtain the following sequence:
  \[
  f(t_1,\dots, t_n) \xrw{\alpha_1} f(q_1, t_2, \dots, t_n) \xrw{\bigwedge_1^2 \alpha_i} f(q_1, q_2, t_3, \dots, t_n) \xrw{\bigwedge_1^3 \alpha_i} 
  \dots \xrw{\bigwedge_1^n\alpha_i} f(q_1,\dots, q_n) \xrw{\bigwedge_1^n \alpha_i \land \top} q'\]
  There are two cases to consider : (1) $q=q'$ and (2) $q \not = q'$.
  %
  (1) If $q=q'$, the decomposition is complete and $f(t_1\sigma_1,\dots, t_n\sigma_n) \xrw{\alpha}_\A q$ with $\alpha = \bigwedge_1^n \alpha_i$.
  \[f(t_1\sigma_1,\dots, t_n\sigma_n) \xrw{\bigwedge_{i = 1}^n \alpha_i} f(q_1,\dots,q_n) \xrw{\bigwedge_{i = 1}^n \alpha_i} q\]
  %
  (2) $q \not = q'$: $f(t_1\sigma_1,\dots, t_n\sigma_n) \xrw{\alpha}_\A q$ holds only if we have a transition $q' \xrw{\alpha'} q$
  such that $\alpha = \bigwedge_1^n \alpha_i \land \alpha'$.
  \[f(t_1\sigma_1,\dots, t_n\sigma_n) \xrw{\bigwedge_{i = 1}^n \alpha_i} f(q_1,\dots,q_n) \xrw{\top} q' \xrw{\alpha'} q\]
  % \suchthat \alpha \equ \bigwedge_{i = 1}^n \alpha_i \land \alpha'\]
  % Now, we reorganize transitions used for $f(t_1\sigma_1,\dots, t_n\sigma_n) \xrw{\alpha}_\A q$ to obtain the sequences $t_i\sigma_i \xrw{\alpha_i}_\A q_i$.
%   followed by the sequence $f(q_1,\dots,q_n) \xrw{\alpha'}_\A q$ such that $\alpha = \bigwedge_1^n\alpha_i \land \alpha'$. 
%   This last sequence is composed by a transition $f(q_1,\dots,q_n) \rw q \in \Delta$ eventually followed by the sequence $q' \xrw{\alpha'} q$
%   if $q \not = q'$. We have the decompostion:
%   % $ $ f(t_1, \dots, t_n)\sigma = 
%   $$f(t_1\sigma_1,\dots, t_n\sigma_n) \xrw{\bigwedge_{i = 1}^n \alpha_i} f(q_1,\dots,q_n) \xrw{\top} q' \xrw{\alpha'} q
%   \suchthat \alpha \equ \bigwedge_{i = 1}^n \alpha_i \land \alpha'$$
  %%%%%%%%%%%%%%%%%%%%%%%%%%%%%%%%%%%%%%%%%%%%%%%%%%%%%%%%%%%%%%%%%%%%%%
  By induction, we know that for each sequence $t_i\sigma_i
  \xrw{\alpha_i} q_i$, the matching problem is solved {\it i.e.}
  $t_i \match q_i \vdash S_i$ with $S_i$ contains $(\alpha_i, \sigma_i)$.
  Rule (Delta) is applied to all premises $t_i \match q_i \vdash_\A S_i$ for the transition $f(q_1,\dots,q_n) \rw q' \in \Delta$.  From
  this, we obtain a set $S' = \bigotimes^n_1 S_i$.  By unfolding the definition of $\bigotimes$, we have $S = \{(\top, id) \oplus (a^1,
  s^1) \oplus \dots (a^n, s^n) \sep (a^i, s^i) \in S_i \}$. Since each $S_i$ contains $(\alpha_i, \sigma_i)$, $S'$ contains $(\top, id)
  \oplus (\alpha_1, \sigma_1) \oplus \dots (\alpha_n, \sigma_n)$ which is, by definition of $\oplus$ equal to $(\bigwedge_1^n \alpha_i,\ \sigma)$.
  % $ which is equal to
  % $(\top \land \alpha \land \dots \land \alpha_n, id \sqcup \sigma_1 \sqcup \dots \sqcup \sigma_n)$ or simply $(\bigwedge_1^n \alpha_i,\ \sigma)$. 
  Thus, we obtain a intermediate statement $f(t_1, \dots, t_n) \matchi q' \vdash_\A S'$ such that $f(t_1, \dots, t_n)\sigma \xrw{\bigwedge_1^n \alpha_i} q'$,
  where $(\bigwedge_1^n \alpha_i, \sigma) \in S'$.

  This statement must correspond to one of the premises of Rule (Epsilon) to produce the expected statement $f(t_1,\dots, t_n) \match q \vdash_\A S$.
  There are two cases to consider :$q=q'$ and $q \not = q'$.

  If $f(q_1,\dots,q_n) \rw q' \in \Delta$ is the last transition used to have $f(t_1, \dots, t_n)\sigma \xrw{\alpha}_\A q$ then
  we have $\alpha = \bigwedge_1^n \alpha_i$ and we are in the case $q = q'$: this case corresponds to the premiss $0$ of
  Rule (Epsilon) and $S' = S_0$. By definition of Rule (Epsilon), $S'$ is included in $S$. This means that $(\alpha, \sigma) \in S$.

  If we have $q \not= q'$, then it remains a sequence of transitions $q' \xrw{\alpha'} q$ to have $f(t_1, \dots, t_n)\sigma \xrw{\alpha}_\A q$.
  The couple $(\alpha', q')$ is in the set $\{ (q_k, \alpha_k) \sep q_k \xrw{\alpha_k} q\}$.%_1^m$.
  This means that the statement $f(t_1,\dots, t_n) \match q \vdash_\A S'$ is one the remaining premisses. % $1$ to $m$.
  By definition of Rule (Epsilon), $S$ contains all couple $(a \land \alpha', s)$ where $(a, s) \in S'$. 
  In particular, $S$ contains $(\bigwedge_1^n \alpha_i \land \alpha', \sigma)$ which concludes the proof.
\end{proof}

\section{Proofs about normalization}

To normalize, we assume that $\Delta$, the second argument of $\norm$, is determinist. 
It means that if $\Delta$ contains two normalized transitions of the form 
$f(q_1, \dots, q_n) \rw q$ and $f(q_1, \dots, q_n) \rw q'$, then we have $q = q'$. 
It ensures that there exists a normal form for any term which is rewritten by $\Delta$.
It is required by the first step of $\norm$.

Note that all proofs about $\norm$ are done using the measure $\mu : \TFQ \rw \Nat$ that counts the number of occurences of
symbols in $\F$ of a configuration. Example : $\mu(f(q_1, g(q_2), a)) = 3$.
We define it inductively by $\mu(q) = 0$ if $q \in \Q$, and $\mu(f(t_1,\dots, t_n)) = 1 + \sum_1^n \mu(t_i)$.
   


\begin{lemma}[Existence of a representative]
\label{lem:norm_determinism}
Assume that $\aaexeq$ is a $\RE$-automaton obtained after $k$ steps of completion
from $\aaexeq^0$. Let $c$ be a configuration. 
If $\Delta' = \norm(c, \Delta \setminus \Delta^0)$, then there exists a state $q$ such that $c \rw^!_{\Delta'} q$.
\end{lemma}

\begin{proof}\\
  Assuming $\F$ a set of symbols, and $\Q$ a set of states. 
  We define $\aaexeq = \la \F, \Q, \Q_f, \Delta \cup \Drw \cup \Deq\ra$; $c \in \TFQ$.
  Assume that $\Delta^1 = \Delta \setminus \Delta^0$ is determinist.

  The first step $\norm(t, \Delta^1)$ consists in rewriting $c$ by
  $\Delta^1$ in its normal form $d$
   
  The second step $\slice(d, \Delta^1)$ returns $\Delta^2$ such that
  there exists a unique state $q$ such that $d \rw^!_{\Delta^2} q$.

  The proof is one by induction on the decreasing of $\mu(d)$.
  We consider the 3 cases of $\slice(d, \Delta^1)$
   
  \begin{enumerate}

   \item $\slice(q, \Delta^1) = \Delta^1$. It means that $d$ is the state $q$. There exists a unique state ,which is $q$, such that $d \rw^!_{\Delta^1} q$.

   \item $\slice(f(q_1, \dots, q_n), \Delta^1) = \Delta^1 \cup \{ f(q_1, \dots, q_n) \rw q \sep q \in \Q_{new}\}$. 
     Each $q_i$ is a state. The configuration $f(q_1, \dots, q_n)$ can be used as the left-member of a normalised
     ground transition. We build the new transition  $f(q_1, \dots, q_n) \rw q$ using a new state $q$. Adding a such
     transition to $\Delta^1$ preserves determinism. We know that it is impossible to rewrite more $d = f(q_1, \dots, q_n)$ 
     using transitions of $\Delta^1$ : the new transition $f(q_1, \dots, q_n) \rw q$ is the unique way to rewrite $d$.
     We deduce that $\Delta^2 = \Delta^1 \cup \{ f(q_1, \dots, q_n) \rw q \sep q \in \Q_{new}\}$ is deterministic, and $d \rw^!_{\Delta^2} q$.

   \item $\slice(f(t_1, \dots, t_n), \Delta^1) = \norm(f(t_1,\dots, t_n), \slice(t_i, \Delta^1)\ )$, $t_i \in \TFQ \setminus \Q$.
     Here, we have the direct subterm $t_i$ of $d$ which is not a state. We deduce $\mu(t_i) < \mu(d)$ from the definition of $\mu$.
     By induction, $\Delta$ is extended by $\slice(t_i, \Delta^1)$ to obtain $\Delta^2$ for which there exists a state $q$ 
     such that $t_i \rw^!_{\Delta^2} q$.  Using this new set $\Delta^2$, we unfold $\norm(f(t_1,\dots, t_n), \Delta^2)$ which consists in rewriting 
     $f(t_1, \dots, t_n)$ using $\Delta^2$. We obtain a new configuration $f(t'_1, \dots, t'_n)$ where we know at less 
     $t_i'$ is equal to $q$ since the direct subterm $t_i$ can be rewritten in $q$ using $\Delta^2$. Note that if some subterms of $t_i$
     are also subterms of some other $t_j$, it will also be rewritten by $\Delta^2$ in $t'_j$ until we reach the normal form.
     Each step of rewriting by $\Delta^2$ necessarly replaces a symbol of $\F$ by a state of $\Q$ by definition of a normalised transition.
     This remark allows to prove that $\mu(f(t_1,\dots,t_n) > \mu(f(t'_1,\dots,t'_n)$.
     For the direct subterm $t_i$, we know $\mu(t_i) > 0$ ($t_i$ is not a state), and $\mu(t'_i) = 0$ ($t'_i$ is the state $q$).
     For all other direct subterm $t_j$ with $j <> i$ we deduce $\mu(t_j) \ge \mu(t'_j)$ from $t_j \rw^!_{\Delta^2} t'_j$ using $\Delta^2$.
     We have $\mu(f(t_1,\dots,t_n) > \mu(f(t'_1,\dots,t'_n)$ by definition of $\mu$, and $f(t'_1,\dots,t'_n)$ is rewritten as most
     as possible by the deterministic $\Delta^2$. Then, we use again the induction hypothesis to deduce that $\Delta' = \slice(f(t'_1,\dots,t'_n), \Delta^2)$
     extends $\Delta^2$ in order to have a unique state $q$ such that $f(t'_1,\dots,t'_n) \rw^!_{\Delta^3} q$.
     By transivity, we have $d \rw^!_{\Delta'} q$ using the deterministic set $\Delta'$ for $d$ which is equal to $f(t_1, \dots, t_n)$.

   \end{enumerate}
   Finally, we proved that $\Delta' = \slice(d, \Delta^1)$ extends $\Delta^1$ preserving its determinism such that there exists a state
   $q$ for which $d \rw^!_{\Delta'} q$. We also know that $c \rw^!_{\Delta'} d$. We can conclude that $\Delta' = \norm(c, \Delta^1)$
   is determinist, and there exists a state $q$ such that $c \rw^!_{\Delta'} q$.
 \end{proof}

 Let $\A = \la \F, \Q, \Q_f, \Delta \cup \Drw \cup \Deq \ra$ be a $\RE$-automaton, and $c \in \TFQ$ a
 configuration of $\A$.
 We prove that the $\Delta^1 = \Delta \setminus \Delta^0$ is injective {\em i.e.} for all $c,\ d \in \TFQ$,
 if we have $c \rw^*_{\Delta^1} q$ and $c \rw^*_{\Delta^1} q$, then $c = d$. From this property, we deduce
 that if we have $\Delta^2 = \norm(r\sigma, \Delta^1)$ all term $t$ such that $t \rw^*_{\Delta^2 \cup \Delta} q$,
 then we deduce that $t = r\sigma'$ where $\sigma' : \X \rw \TF$. It is important to ensure that there is no more
 term added than terms defined as $t$, when a critical pair is solved.
 added 
 
 \begin{lemma}[Normalisation and injectivity]
   If $\Delta^1$ is injective, then so is $\norm(c \sep \Delta^1)$.
 \end{lemma}
 
 \begin{proof}
   Assuming $\F$ a set of symbols, and $\Q$ a set of states. We define: $A = \la \F, \Q, \Q_f, \Delta \cup \Drw \cup \Deq\ra$;
   $c \in \TFQ$, and $\Delta^1 = \Delta \setminus \Delta^0$.
   
   The injectivity of $\Delta^2 = \norm(c, \Delta^1)$ holds, if there there is only one transition per state in $\Delta^1$.
   When function $\slice$ creates new transition, it uses a new state.
   Assuming that $\Delta^1$ has only one transition per state, so has $\Delta^2$.
   This is enougth to ensure the injectivity of $\Delta^2$.
   Note that all initial $\RE$-automaton $\aaexeq^0$ ensures this property, since the set of transitions used 
   by function $\norm$ is empty : $\Delta^0 \setminus \Delta^0$.
   
\end{proof}


\section{Using $\RE$-automata completion for exact computation of reachable terms}
\label{sec:exact}
The $\RE$-completion can be used for regular model-checking for most of the
known classes of $\R$ for which $\desc(\Lang(\A))$ is regular. On those classes,
completion always stops on a $\RE$-automaton $\aaex^*$ with an empty set $\Deq$. As
a result, all the terms recognized by $\aaex^*$ are reachable and all triples
$(q,q',\phi)\in S$ are such that $\phi=\top$.

\begin{theorem}[Exact computation with completion]
\label{thm:regular}
  If $E=\emptyset$ and $\R$ is
  ground~\cite{pDauchetTison-LICS90,pBrainerd-IC69}, right-linear and
  monadic~\cite{pSalomaa88}, linear and
  semi-monadic~\cite{pCoquideDauchetGV-FCT89}, linear and inversely
  growing~\cite{pJacquemard-RTA96}, constructor~\cite{pRety-LPAR99} or linear generalized finite path
  overlapping~\cite{Takai-RTA04}, then completion of a tree automaton $\A$
  terminates on $\A_{\R,\emptyset}^*$ and $\Lang(\A_{\R,\emptyset}^*)=\desc(\Lang(\A))$.
\end{theorem}

\begin{proof}
  When $E=\emptyset$, the completion algorithm does not produce any transitions
  for the $\Deq$ set and, thus, every transition of $\Drw$ is
  labelled by $\top$. In other words, in this case, $\RE$-completion produces a usual
  tree automaton instead of a $\RE$-automaton. As a result, when $E=\emptyset$,
  the algorithm proposed in this paper totally coincides with the one
  of~\cite{GenetR-JSC10} dealing with tree automata. In~\cite{Genet-Habil}, it
  has been shown that the algorithm of~\cite{GenetR-JSC10} terminates with
  $E=\emptyset$ for the above classes (Theorem~114). Furthermore, Theorem~45 and
  Theorem~49 of~\cite{GenetR-JSC10} guarantee that, in this case,
  $A_{\R,\emptyset}$ is such that $\Lang(\A_{\R,\emptyset}^*)=\desc(\Lang(\A))$.
\end{proof}


\renewcommand{\bibname}{Additional References}
\begin{thebibliography}{10}

\bibitem[25]{pBrainerd-IC69}
W.~S. Brainerd.
\newblock Tree generating regular systems.
\newblock {\em Information and Control}, 14:217--231, 1969.

\bibitem[26]{pCoquideDauchetGV-FCT89}
J.~Coquid\'e, M.~Dauchet, R.~Gilleron, and S.~V\'agv\"olgyi.
\newblock Bottom-up tree pushdown automata and rewrite systems.
\newblock In R.~V. Book, editor, {\em Proc.\ 4th RTA Conf., Como (Italy)},
  volume 488 of {\em LNCS}, pages 287--298. Springer-Verlag, 1991.

\bibitem[27]{pDauchetTison-LICS90}
M.~Dauchet and S.~Tison.
\newblock The theory of ground rewrite systems is decidable.
\newblock In {\em Proc.\ 5th LICS Symp., Philadelphia (Pa., USA)}, pages
  242--248, June 1990.

\bibitem[28]{pGilleronTison-FI95}
R.~Gilleron and S.~Tison.
\newblock Regular tree languages and rewrite systems.
\newblock {\em Fundamenta Informaticae}, 24:157--175, 1995.

\bibitem[29]{pJacquemard-RTA96}
F.~Jacquemard.
\newblock Decidable approximations of term rewriting systems.
\newblock In H.~Ganzinger, editor, {\em Proc.\ 7th RTA Conf., New Brunswick
  (New Jersey, USA)}, pages 362--376. Springer-Verlag, 1996.

\bibitem[30]{pRety-LPAR99}
P.~R\'ety.
\newblock {R}egular {S}ets of {D}escendants for {C}onstructor-based {R}ewrite
  {S}ystems.
\newblock In {\em Proc.\ 6th LPAR Conf., Tbilisi (Georgia)}, volume 1705 of
  {\em LNAI}. Springer-Verlag, 1999.

\bibitem[31]{pSalomaa88}
K.~Salomaa.
\newblock Deterministic {T}ree {P}ushdown {A}utomata and {M}onadic {T}ree
  {R}ewriting {S}ystems.
\newblock {\em J. of Computer and System Sciences}, 37:367--394, 1988.
\end{thebibliography}


\section{Algorithms and proofs for emptiness decision of intersection}

\label{sec:intersection}
We define a specific algorithm building the set $S$ of reachable states for the
product automaton for $\RE$-automaton $\A$ and automaton $\B$ where each product
state is labelled by a formula on states of $\A$. We first define an order $>$
on formulas.

\begin{definition}
Given $\phi_1$ and $\phi_2$ two formulas, $\phi_1 > \phi_2$ iff $\phi_2 \models
\phi_1$ and $\phi_1 \not \models \phi_2$.
\end{definition}


\begin{definition}[Reachable states of the product of a $\RE$-automaton and a
  tree automaton]
  \label{def:reachable-states}
  Let $\A =\langle \F, \Q^\A,\Q^\A_f,\Delta^\A, \Drw, \Deq \rangle$ be a
  $\RE$-automaton and $\B=\langle \F, \Q^\B,\Q^\B_f,\Delta^\B \rangle$ be an
  epsilon-free tree automaton. The set $S$ of reachable states of $\A\times \B$ is the set of triples
  $(q,q',\phi)$ where $q\in\Q^A,q'\in\Q^\B$ and $\phi$ is a formula. Starting from the set $\Q^A
  \times \Q^\B \times \{ \bot\}$, the value of $S$ can be computed using the following two
  deduction rules : 

\medskip
\noindent
{\footnotesize
\centerline{\begin{tabular}{c|c}
  $\displaystyle\frac
  {\{(q_1,q'_1, \phi_1), \ldots, (q_n,q'_n,\phi_n)\} \cup \{(q,q',\phi)\} \cup P}
  {\{(q_1,q'_1,\phi_1), \ldots, (q_n,q'_n,\phi_n)\} \cup \{(q,q',\phi \vee \bigwedge_{i=1}^n \phi_i)\}
    \cup P}$ &  
  $\displaystyle\frac   
  {\{(q_1,q,\phi_1), (q_2,q,\phi_2)\} \cup P}
  {\{(q_1,q,\phi_1), (q_2,q,(\phi_1 \wedge \phi) \vee \phi_2)\} \cup P}$\\
  & \\
  \begin{tabular}{c}
    if $f(q_1, \ldots, q_n) \rw q \in \Delta^\A$  \\ 
    and $f(q'_1, \ldots, q'_n) \rw q' \in \Delta^\B$  \\ 
    and $(\phi \vee \bigwedge_{i=1}^n \phi_i) > \phi$ 
  \end{tabular}
  &
  \begin{tabular}{lc|l}
    if $q_1 \xrw{\phi} q_2 \in \Drw$ &or& if $q_1 \rw q_2 \in \Deq$ \\
    and $((\phi_1 \wedge \phi) \vee \phi_2) > \phi_2$ && and $\phi=Eq(q_1,q_2)$ \\
     && and $((\phi_1 \wedge \phi) \vee \phi_2) > \phi_2$ 
\end{tabular}
\end{tabular}}}

\end{definition}

With regards to the reachability problem, this definition, provides a
way to distinguish between real counterexamples and terms which can be rejected
using abstraction refinement. 
Indeed, for all triple $(q,q',\phi)\in S$ with $q$ final in $\A$ and $q'$ final
in $\B$, if $\phi \models \top$
then some of the terms recognized by $q'$ in $\B$ are reachable. 
Otherwise, $\phi$ is the formula to invalidate, i.e. negate some of
its atom so that it becomes $\bot$.

\begin{lemma}[Emptiness decision of the product of a $\RE$-automaton and a
  tree automaton]
  Let $\A$ be a $\RE$-automaton and $\B$ a tree automaton. Let $S$ be the set of
  reachable states of $\A\times \B$ defined according to
  definition~\ref{def:reachable-states}. For all final state $q$ of $\A$, all
  final state $q'$ of $\B$, all formulas $\phi_S\neq\bot$, $\phi\neq \bot$ and
  all term $t\in\TF$, we have $t\xrw{\phi}^*_\A q$ and $t \rw_\B^* q'$
  (i.e. $\Lang(\A)\cap \Lang(\B) \neq \emptyset$) if and only if there exists a
  triple $(q,q',\phi_S)\in S$ such that $\phi \models \phi_S$. 
\end{lemma}

\begin{proof}
  Let $\A =\langle \F, \Q^\A,\Q^\A_f,\Delta^\A, \Drw, \Deq \rangle$
  be the $\RE$-automaton and $\B=\langle \F, \Q^\B,\Q^\B_f,\Delta^\B \rangle$ be
  the tree automaton.  
%We use definition~\ref{def:xrw_alpha} for the recognized language
%$\Lang(\A,q)=\{t\in \TF \sep t \xrw{\phi}_\A^* q \mbox{ and $\phi$ 
%    satisfiable}\}$. 
%  We prove that $(q,q',\phi_S)\in S$ if 
%  and only if %$\Lang(\A,q) \cap \Lang(\B,q') \neq \emptyset$.
%  there exists a term $t\in\TF$ such that $t\xrw{\phi}^*_\A q$, $t
%  \rw_\B^* q'$ and $\phi\models \phi_S$.
We prove a stronger property on all states $q$ of $\A$ and $q'$ of $\B$ (and not
only for final states).
First, we prove the 'only if' part. Let us assume that 
%$\Lang(\A,q) \cap \Lang(\B,q') \neq \emptyset$, hence that 
there exists a term $t\in\TF$ such that $t\xrw{\phi}^*_\A q$, $t \rw_\B^* q'$.
%and $\phi$ satisfiable. 
By induction on the height of $t$ we have:


\begin{itemize}
\item If $t$ is a constant, since $\B$ is an epsilon-free tree automaton, the
  only way to have $t \rw^*_\B q'$ is to have $t \rw q' \in \B$. With regards to
  $\A$, by definition~\ref{def:xrw_alpha}, $t\xrw{\phi}^*_\A q$ means that
  there exists states $q_0,q_1, \ldots, q_n$ and formulas $\phi_1,\ldots,
  \phi_n$ such that $t \rw_{\Delta_\A} q_0 \xrw{\phi_1} q_1 \xrw{\phi_2}
  \ldots q_n$ with $q=q_n$ and $\phi=\phi_1\wedge \ldots \wedge
  \phi_n$.% is satisfiable.
  Transitions $q_i \xrw{\phi_i} q_{i+1}$ are either transitions of $\Drw$ or
  transitions of $\Deq$ with $\phi_i=\top$.  Because of transitions $t \rw q_0
  \in \Delta_\A$ and $t \rw q' \in \Delta_\B$, using the first case of
  definition~\ref{def:reachable-states}, we get that $(q_0,q',\top) \in
  S$. Similarly, using the second case of the definition, we obtain that there
  exists formulas $\phi'_i$ with $i=1\ldots n$ such that $(q_1,q', \phi_1\vee
  \phi'_1), (q_2,q',(\phi_1\wedge\phi_2)\vee \phi'_2),\ldots (q_n,q', (\phi_1
  \wedge \ldots \wedge \phi_n)\vee \phi'_n)$ belong to $S$. Finally, since
  $q_n=q$ and $\phi=\phi_1\wedge \ldots \wedge \phi_n$, we that
  $(q,q',\phi\vee\phi_n') \in S$. Furthermore, we trivially have that $\phi_S=
  \phi\vee \phi_n'$ and $\phi \models \phi_S$. % since $\phi$ is
                                % satisfiable, so is $\phi \vee \phi'$.

\item Assume that for all term of height lesser or equal to $n\in\NN$, the
  property is true. Let us prove that it is also true for a term $f(t_1, \ldots,
  t_n)$ with $t_1, \ldots, t_n$ of height lesser or equal to $n$. Since $f(t_1,
  \ldots, t_n) \rw^*_\B q'$ and $\B$ is an epsilon free tree automaton, we
  obtain that $\exists q'_1,\ldots,q'_n\in\Q^\B$ such that $\forall i=1\ldots n:
  t_i \rw^*_\B q'_i$ and $f(q'_1,\ldots,q'_n) \rw q' \in \Delta_\B$. With
  regards to $\A$,
  by definition~\ref{def:xrw_alpha}, $f(t_1, \ldots, t_n)
  \xrw{\phi}^*_\A q$ means that there exists states $q_0,q_1, \ldots,
  q_m,q''_1,\ldots,q''_n$ and formulas $\phi_1,\ldots, \phi_m, \phi'_1, \ldots,
  \phi'_n$ such that $\forall i=1\ldots n: t_i \xrw{\phi'_i}^*_\A q''_i$,
  $f(q''_1,\ldots, q''_n) \rw_{\Delta_\A} q_0$ and $q_0 \xrw{\phi_1} q_1
  \xrw{\phi_2} \ldots q_n$, $q=q_n$. Furthermore, we obtain that $\phi=
  \bigwedge_{i=1}^{n} \phi'_i \wedge \bigwedge_{i=1}^{m} \phi_i$. % is
                                % satisfiable. 
  Since terms $t_i$ are of height lesser or equal to $n$, $\forall
  i=1\ldots n: t_i \rw^*_\B q_i$ and $\forall i=1 \ldots n: t_i \xrw{\phi'_i}_\A^*
  q''_i$, we can apply the induction hypothesis and obtain that $\forall
  i=1\ldots n: (q_i, q''_i, \phi''_i) \in S$ with $\phi'_i \models \phi''_i$. % and $\phi'_i$ satisfiable.  
  Besides to this, using case~1 of definition~\ref{def:xrw_alpha} on
  $f(q_1,\ldots,q_n) \rw q' \in \Delta_\B$, $f(q''_1,\ldots,q''_n) \rw q_0 \in
  \Delta_\A$, and $\forall i=1\ldots n: (q_i, q''_i, \phi''_i) \in S$, we obtain
  that there exists a formula $\phi'$ such that $(q_0,q', (\bigwedge_{i=1}^n
  \phi''_i) \vee \phi')\in S$. Then, like in the base case, since $q_0
  \xrw{\phi_1} q_1 \xrw{\phi_2} \ldots q_n$, $q=q_n$, we can deduce that
  there exists a formula $\phi''$ such that $(q,q',(\bigwedge_{i=1}^n \phi''_i
  \wedge \bigwedge_{i=1}^{m} \phi_i) \vee \phi'')\in S$. Let $\phi_S= (\bigwedge_{i=1}^n \phi''_i
  \wedge \bigwedge_{i=1}^{m} \phi_i) \vee \phi''$. Since we know from above that
  $\phi=\bigwedge_{i=1}^n \phi'_i \wedge \bigwedge_{i=1}^{m} \phi_i$ and
  $\forall i=1\ldots n: \phi'_i \models \phi''_i$, we obtain that % is satisfiable, so is 
  $\phi \models \phi_S$.
\end{itemize}


\medskip
Second, we prove the 'if' part: if $(q,q',\phi_S)\in S$ and $\phi_S \neq \bot$
then there exists a term $t$ and a formula $\phi \neq \bot$ such that $\phi \models \phi_S$,
% and $S$ satisfiable,
$t\xrw{\phi}^*_\A q$ and $t \rw_\B^* q'$. We make a proof by induction
on the number of applications of the two rules of
definition~\ref{def:reachable-states}, necessary to prove that $(q,q',\phi_S)$ in $S$.

\begin{itemize}
\item If the number of steps is $0$ then, since the computation of $S$ starts
  from the set $\Q^\A \times \Q^\B \times \bot$, then all $(q,q',\phi_S)$
  are such that $\phi_S=\bot$, which is a contradiction.


\item We assume that the property is true for any triple $(q,q',\phi_S)$ which can
  be deducted by $n$ or less applications of the rules of
  definition~\ref{def:reachable-states}. Now, we consider the case of a triple
  $(q,q',\phi_S)$ that is deduced at the $n+1$-th step of application of the
  deduction rules.
  \begin{itemize}
  \item If the first rule is concerned, this means that there exists triples
    $(q_1,q'_1, \phi_1),\ldots,(q_n,q'_n,\phi_n)$ and $(q,q',\phi)$ in $S$
    deduced before $n+1$-th step, as well as transitions $f(q_1,\ldots,q_n)\rw q
    \in \Delta_\A$ and $f(q'_1,\ldots,q'_n)\rw q' \in \Delta_\B$. Furthermore,
    we know that $\phi_S=\phi \vee \bigwedge_{i=1}^n \phi_i$.
    % and by hypothesis $\rho$ is satisfiable. 
    If $\phi\neq \bot$ then, since $(q,q',\phi)$ was shown to belong to $S$
    before $n+1$-th step, we can apply the induction hypothesis and directly
    obtain that there exists a term $t$ and a formula $\phi'$ such that $\phi'
    \models \phi$, $t \xrw{\phi'}^*_\A q$ and $t \rwB^* q'$. Note that $\phi'
    \models \phi$ implies $\phi' \models \phi_S$.
    Otherwise, if $\phi=\bot$, then % $\bigwedge_{i=1}^n \phi_i$ is then 
    we can apply the induction hypothesis on triples $(q_i,q'_i,\phi_i)$,
    $i=1\ldots n$ and obtain that $\forall i=1\ldots n:\exists \phi_i':\exists
    t_i \in\TF: \phi'_i \models \phi_i$, $t_i \xrw{\phi_i'}_\A^* q_i$ and $t_i
    \rwB^* q'_i$. Finally, 
    because of the two transitions $f(q_1,\ldots,q_n)\rw q \in \Delta_\A$ and
    $f(q'_1,\ldots,q'_n)\rw q' \in \Delta_\B$, we get that $f(t_1,\ldots,t_n)
    \xrw{\phi'}_\A^* f(q_1,\ldots,q_n) \rwA^* q$ with
    $\phi'=\bigwedge_{i=1}^n \phi'_i$ on one side and $f(t_1,\ldots,t_n) 
    \rwB f(q'_1,\ldots,q'_n) \rwB^* q$ on the other side. Furthermore,
    since $\forall i=1\ldots n: \phi'_i \models \phi_i$, we have
    $\bigwedge_{i=1}^n \phi'_i \models \bigwedge_{i=1}^n \phi_i$. Recall that
    $\phi'= \bigwedge_{i=1}^n \phi'_i$ and $\phi_S= \phi\vee \bigwedge_{i=1}^n
    \phi_i$. Hence, $\phi' \models \phi_S$.
  \item If the second rule is concerned, this means that there exists triples
    $(q_1,q',\phi_1)$ and $(q,q',\phi_2)$ in $S$ deduced before the $n+1$-th
    step. Furthermore, we know that $\phi_S=(\phi_1 \wedge \phi) \vee \phi_2$.
    Like above, if $\phi_2 \neq \bot$ then we can apply induction hypothesis on
    $(q,q',\phi_2)$ and trivially get the result. Otherwise, if $\phi_2=\bot$
    then we can use induction hypothesis on the triple $(q_1,q',\phi_1)$ and
    obtain that there exists a formula $\phi_1'$ and a term $t_1$ such that 
    $t_1\xrw{\phi_1'}_\A^* q_1$, $t_1 \rwB^* q'$ and $\phi_1' \models \phi_1$.
    Then, by case on the epsilon transition used for the deduction on $S$, we
    prove that $t_1 \xrw{\phi_1'\wedge \phi}_\A^* q$:

    \begin{itemize}
    \item Assume that $q_1 \xrw{\phi} q \in \Drw$. Then, by
      definition~\ref{def:xrw_alpha}, we obtain that $t_1 \xrw{\phi'_1
        \wedge \phi}_\A^* q$. Furthermore, since $\phi'_1 \models \phi_1$, we have
      that $\phi'_1 \wedge \phi \models \phi_1 \wedge \phi$ and, finally, that 
      $\phi'_1 \wedge \phi \models \phi_S$.

    \item Assume that $q_1 \rw q \in \Deq$. By
      definition~\ref{def:xrw_alpha}, we obtain that $t \xrw{\phi_1 \lor Eq(q_1, q)}_\A^* q$.
      Finally, like above, we can deduce that
      $\phi'_1 \wedge Eq(q_1,q) \models \phi_1 \wedge Eq(q_1,q)$ and thus $\phi'_1 \wedge
      Eq(q_1,q) \models \phi_S$.
    \end{itemize}
  \end{itemize}
\end{itemize}
\end{proof}
\end{document}



%%% Local Variables: 
%%% mode: latex
%%% TeX-master: "main"
%%% TeX-PDF-mode: t
%%% End: 

\section{Using $\RE$-completion for verification}





%%% Local Variables: 
%%% mode: latex
%%% TeX-master: "main"
%%% End: 

\end{document}
%%% Local Variables: 
%%% mode: latex
%%% TeX-master: t
%%% TeX-PDF-mode: t
%%% End: 
