
\section{Introduction}
\label{sec:introduction}

At the heart of all the techniques that have been proposed for
exploring infinite state spaces, is a symbolic representation that can
finitely represent infinite sets of states. In early work on the
subject, this representation was domain specific, for example linear
constraints for sets of real vectors. For several years now, the idea
that a generic finite-automaton based representation could be used in
many settings has gained ground, starting with systems manipulating
queues and integers (see \cite{WB98} for a survey), then moving to
parametric systems~\cite{KMMPS97}, and, more recently, reaching systems
using real variables~\cite{BJW01}.

Beyond the necessary symbolic representation, there is also a need to
``accelerate'' the search through the state space in order to reach,
in a finite amount of time, states at unbounded depths. In
acceleration techniques, the move has again been from the specific to
the generic, the latter approach being often referred to as regular
model checking.  The advantages of using a generic technique are of
course that there is only one method to implement independently of the
domain considered, that multidomain situations can potentially be
handled transparently, and that the scope of the technique can include
cases not handled by specific approaches. Beyond these concrete
arguments, one should not forget the elegance of the generic approach,
which can be viewed as an indication of its potential, thus justifying
a thorough investigation.

Most of existing results on regular model checking assume that states
are represented by words and set of states by finite-state
automata\,\cite{BLW03,BJNT00}. There are many interesting
infinite-state systems whose sets of reachable states cannot be
encoded by regular sets of words. This is either because the behavior
of the system cannot be captured by a regular representation
\cite{FP01}, or because the topology of the system is not suited to be
expressed with a word-based representation. A solution to the latter
problem is to extend the applicability of the regular model checking
approach beyond words.

In this paper, we assume that states of the system are represented by
trees and set of states by tree automata. In this context, the
transition relation of the system is naturally represented by a set of
rewriting rules. It is well-known that this {\em tree regular model
  checking framework} is expressive enough to describe communication
protocols\,\cite{} as well as a wide range of cryptographic protocols
and JAVA applications\,\cite{}.

In {\em tree regular model checking}, the main objective is to compute
an automaton representing the set of states of the system. As we are
dealing with infinite-state systems, the problem remains undecidable
and only partial solutions can be proposed. Among theses solutions, we
find the predicate abstraction methodology which was promoted by
Bouajjani et al.\,\cite{BHRV06a,BHRV06b}. The idea behind abstract
tree regular model checking consists in computing the successive
automata that obtained after successive applications of the rewriting
relation and then use techniques coming from the predicate abstraction
area in order to over-approximate the set of reachable states. If the
property holds on the abstraction, then it also holds on the concrete
system. If a counter-example is found on the abstraction, then one can
check if it is also a counter-example to the real system. If not, this
spurious counter-example must be used to refine the
abstraction. Bouajjani's algorithm, which may not terminate, proceeds
by successive abstraction/refinement until a decision can be
taken. 

Independently, Genet et al. proposed {\em completion} that is another
technique to compute an over-approximation of the set of reachable
states. The main difference with the work of Bouajjani is that
completion techniques uses equations to compute the
abstraction. Equations can be more expressive than predicate
abstraction and hence produce a more accurate abstraction. Contrary to
the work of \cite{BHRV06a,BHRV06b}, completion techniques have been
applied to very complex case studies such as the verification of
(industrial) cryptographic
protocols~\cite{GenetK-CADE00,GenetTTVTT-wits03,avispa} and Java
bytecode applications~\cite{BoichutGJL-RTA07}. Unfortunately,
completion techniques do not embed any notion of counter-example based
refinement.

The objective of this paper is to overcome the above mentioned problem
and propose a CounterExample Guided Abstraction Refinement procedure
for completion algorithms.  \comments{ca n'est pas le premier travail
  du genre: boichut,courbis etc... mais c'est vachement mieux parce
  que? reprendre argumentaire (court) de RTA, Yohan?}  Our
contribution is in two steps. First, we propose {\em $\RE$-automaton},
that is a new extension model of tree automata. More precisely, a
$\RE$-automaton keeps trace of the equations and rewriting rules
applied to the initial automaton. The automaton can be used to decide
whether a term $t$ (or a set of terms) is reachable from the set of
initial states. If the procedure concludes positively, then the term
is indeed reachable. If the procedure concludes negatively, then one
has to refine the $\RE$-automaton and restart the process. Our second
contribution is to propose a procedure that refines a $\RE$-automaton
in an efficient manner. Our approach is naturally more efficient than
the one of Bouajjani et al. as it does not require to store the
intermediary computation steps and apply a reverse of the transition
relation to conclude whether the term is reachable or not.



% The verification of infinite state systems often relies on finite state automata
% to finitely represent the set of states.  When states of a system are more
% accurately represented by trees rather than words, {\em tree automata} can be
% used to finitely represent the infinite set of tree structured states. In this
% case the transition relation between states is usually defined as a tree
% transducer. The regular tree model-checking problem has been intensively studied
% like in~\cite{AbdullaJMd-CAV02,AbdullaLDR-JLAP06}. It consists in showing that a
% set of states can (or cannot) be reached by applying the transition relation on
% a set of initial states.  To deal with systems having a non regular behavior,
% this framework has been extended in~\cite{BouajjaniHRV-ENTCS06}, so as to deal
% with regular abstractions of non regular infinite systems. In this case, since
% an over-approximation of the system behavior is considered, only unreachability
% of states can be proven.

% In parallel, the tree automata completion
% technique~\cite{genet-RTA98,FeuilladeGVTT-JAR04} has been proposed. It also
% tackles the verification of infinite state systems using tree automata. However,
% the transition relation is defined using term rewriting systems. Term rewriting
% systems offer a more concise way to define transitions between states and, as a
% result, scales up more easily than tree transducer based techniques. For
% instance, tree automata completion has been used for the verification of
% (industrial) cryptographic
% protocols~\cite{GenetK-CADE00,GenetTTVTT-wits03,avispa} and Java bytecode
% applications~\cite{BoichutGJL-RTA07}. Another interesting point of this
% technique is that abstractions are defined using equations~\cite{GenetR-JSC10},
% %whose expressivity is stronger than predicate abstractions
% %of~\cite{BouajjaniHRV-ENTCS06}. Equations used in completion 
% relying on the equational abstraction framework of~\cite{MeseguerPM-TCS08}.
% %with fewer restrictions though.
% Using equations yields abstractions whose expressivity is very different from
% those produced by predicate abstractions of~\cite{BouajjaniHRV-ENTCS06}.


% On the opposite, with regards to transducer-based techniques, tree automata
% completion has severals weaknesses. When dealing with abstractions, if the
% completion claims that a term is reachable, it is not possible to know if
% the term is actually reachable or if it has been found because of a too coarse
% abstraction. This problem is solved by most of the abstract model-checking
% techniques, like~\cite{BouajjaniHRV-ENTCS06} in the case of tree transducers.

% The first objective of this paper is to equip the tree automata completion
% algorithm with counter-example generation. The aim is here to discriminate, in
% the tree automata, between reachable terms and terms of the abstraction while
% preserving the efficiency of the completion algorithm. When completion stops
% because of a too coarse abstraction, the second objective is to automatically
% refine the abstraction. Here, we aim at taking advantage of the expressivity of
% equations so as to have terminating refinements where predicate abstraction
% of~\cite{BouajjaniHRV-ENTCS06} diverges. 

% Finally, though it will not be discussed in this paper, 
% the objective is to automatically verify Java programs re-using the framework 
% of~\cite{BoichutGJL-RTA07} and the Copster~\cite{copster} tool which compiles
% Java bytecode programs into term rewriting systems.