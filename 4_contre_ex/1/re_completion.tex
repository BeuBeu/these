\section{On solving the reachability using $\RE$-automaton}
\label{sec:recompletion}

%%We assume that we have a reachability problem composed a set of
%%initial terms $I$, a set of rules $\R$, a set of undesirable terms
%%$Bad$. We consider the approximation defined gby a set of equations
%%$E$. To build a $\RE$-automatxon, we instantiate the framework of the
%%completion presented in section \ref{sec:completion} by defining functions $C$
%%for the completion step and $W$ for the widening step.


In this section, we extend the completion and widening principles
introduced in Section~\ref{sec:completion} to $\RE-$automata. We
consider an initial set $I$ that can be represented by a tree
automaton, and transition relation represented by a set of rewriting
rules $\R$. We compute successive approximations $\aaexeq^i$ from
$\aaexeq^0$ using $\aaexeq^{i+1}=W(C(\aaexeq^i))$. We define
$\aaexeq^0 = \la \F, \Q^0, \Q_F, \Delta^0\ra$ that the language of
$\aaexeq^0$ is the terms in $I$. Observe that $\aaexeq^0$ is
well-defined as the sets $\Drw^0$ or $\Deq^0$ are empty. We now detail
the completion and widening steps i.e. $C$ and $W$.

% In this section, we extend completion and widening steps of Section
% \ref{sec:completion} to $\RE-$automata. Given an automaton $A$ that
% represents an initial set $I$, and transition relation represented by
% a set of rewriting rules $\R$, we compute successive approximations 
% $\aaexeq^i$ from $\aaexeq^0$ the $\RE$-automata version of $A$, i.e.,
% $\aaexeq^{i+1}=\widen(\compl(\aaexeq^i))$. We define $\aaexeq^0 = \la \F, \Q^0,
% \Q_F, \Delta^0\ra$ such that the language of $\aaexeq^0$ is the terms
% in $I$. Observe that $\aaexeq^0$ is well-defined as the sets $\Drw^0$
% or $\Deq^0$ are empty. We now detail the completion and widening
% steps i.e. $\compl$ and $\widen$.

\vspace{-.6cm}
\paragraph{The completion step $\compl$.}
Consider a $\RE$-automaton $\aaexeq^i = \la \F, \Q^i, \Q_f, \Delta^i \cup \Drw^i
\cup \Deq^i \ra$, the completion steps consists in computing an automaton
$\compl(\aaexeq^i)$ that is obtained from $\aaexeq^i$ by applying $\R$. As
already explained in Section \ref{sec:completion}, this is done by finding and
resolving all critical pairs.  A {\em critical pair} for a $\RE$-automaton is a
triple $\la r\sigma, \alpha, q\ra$ such that $l\sigma \rw r\sigma$, $l\sigma
\xrw{\alpha}_{\aaexeq^i} q$ and there is no formula $\alpha'$ such that $r\sigma
\xrw{\alpha'}_{\aaexeq^i} q$. Resolution of such a critical pair consists of
adding to $\compl(\aaexeq^i)$ the transitions to obtain $r\sigma
\xrw{\alpha}_{\compl(\aaexeq^{i})} q$. This is followed by a normalization step
where $\Q_{new}$ is a set of new states s.t. $\Q_{new} \cap =\emptyset$.

% More precisely, we add transitions
% $\norm(r\sigma,\Delta^i\setminus\Delta^0)$ (see Definition
% \ref{def:normalization}) to $\Delta^i$. Actually, the set of
% transitions $\Delta^i\setminus\Delta^0$ is deterministic and thus,
% after the normalization of $r\sigma$, there exists a state $q'$ with
% $r\sigma \xrw{\top}_{\compl(\aaexeq^{i})} q'$. Consequently, it remains to
% add the transition $q'\xrw{\alpha} q$ to $\Drw^{i}$. The whole process
% leads to the $\RE-$automaton $\compl(\aaexeq^i)$. This is formally defined as
% follows.



%%The transtion $r\sigma
%%\xrw{\top}_{\aaexeq^{i+1}} q'$ is actually build by adding normalized
%%transitions to $\Delta^{i+1}$ thanks to the {\em normalisation} of the
%%definition~\ref{def:normalization}.

\begin{definition}[Normalization]
  \label{def:normalization}
  The normalization is done in two mutually inductive steps
  parametrized by the configuration $c$ to recognize, and by the set
  of transitions $\Delta$ to extend.
  \[
  \left\{
    \begin{array}{lr}
      \norm(c , \Delta) = \slice(d, \Delta) & c \rw^!_\Delta d\textrm{, and }c,\ d \in \TFQ\\
      \slice(q,\Delta) = \Delta & q \in Q\\
      \slice(f(q_1,\dots, q_n), \Delta) = \Delta \cup \{f(q_1,\dots, q_n) \rw q\} & q_i \in \Q\textrm{ and }q \in \Q_{new}\\
      \slice(f(t_1,\dots, t_n), \Delta) = \norm(f(t_1,\dots, t_n), \slice(t_i, \Delta)) & t_i \in \TFQ \setminus \Q\\
    \end{array}
  \right.
  \]
\end{definition}

\noindent
We thus add transitions such that there exists a state $q'$ with
$r\sigma \xrw{\top}_{\aaexeq^{i+1}} q'$ and $q'\xrw{\alpha} q \in
\Drw^{i+1}$. 

We are now ready to define the resolution of a critical pair $p=\la
r\sigma, \alpha, q\ra$. 

\begin{definition}[Resolution of a critical pair]
\label{def:resolution_cp}
Given a $\RE$-automaton $\A=\la \F, \Q, \Q_f, \Delta\cup \Drw\cup \Deq\ra$ and a critical
pair $p=\la r\sigma, \alpha, q \ra$, the {\em resolution} of $p$ on $\A$ is the
$\RE$-automaton $\A'=\la \F, \Q', \Q_f, \Delta'\cup \Drw'\cup \Deq \ra$ where
\begin{itemize}
\item $\Delta'= \Delta \cup \norm(r\sigma,\Delta\setminus \Delta^0)$
\item $\Drw'= \Drw \cup \{q' \xrw{\alpha} q\}$ where $q'$ is the state such that $r\sigma
  \rw^!_{\Delta'\setminus \Delta_0} q'$
\item $\Q'$ is the union of $\Q$ and the set of states occurring in $\Delta'$
\end{itemize}
\end{definition}

Note that $\Delta_0$, the set of transitions of $\aaex^0$, is never
used for normalization of all $r\sigma$. This is needed to preserve
the well-defidness of $\A'$.  The $\RE$-automaton $\compl(\aaexeq^i)$
is obtained by recursively applying the above resolution principle to
all critical pairs $p$ of the set of critical pairs between $\R$ and
$\aaexeq^i$. The set of all critical pairs is obtained by solving {\em
  matching problems} $l \match q$ for all rewrite rule $l \rw r\in \R$
and all state $q\in\aaexeq^i$.  Solving the matching problem $l \match
q$ consists of computing $S$ that is the set of all couples $(\alpha,
\sigma)$ such that $\alpha$ is a formula, $\sigma$ is a substitution
of $\X \mapsto \Q^i$, and $l\sigma \xrw{\alpha} q$. Each configuration
$l\sigma$ corresponds to a subset of terms of $\Lang(\aaexeq^i, q)$
that can be rewritten by $l \rw r$. The terms characterized by
$l\sigma$ are defined as $l\sigma\sigma'$ where $\sigma': \Q \mapsto
\TF$ is a substitution, which maps each state $q$ to a term
$\sigma'(q) \in \Lang(\aaexeq^i,
q)$. Definition~\ref{def:matching-algorithm} introduces the matching
algorithm to compute the set $S$, which is denoted by the statement $l
\match q \vdash_{\aaexeq^i} S$. Note that when $S$ is empty, there is
no term to rewrite by $l \rw r$.

\begin{definition}[Matching Algorithm]\\
  \label{def:matching-algorithm}
  Assuming the matching problem $l \match q$ for a $\RE$-automaton $\aaexeq^i$.
  $S$ is the solution of the matching problem, if there exists a derivation
  of the statement $l \match q \vdash_{\aaexeq^i} S$ using the rules:
  {\footnotesize
    \vspace{-.2cm}
    \[\textrm{(Var) }
    \dfrac{}
    {x \match q \vdash_\A \{(\alpha_k, \{x \mapsto q_k\}) \sep q_k \xrw{\alpha_k}_\A q\}}
    (x \in \X)
    \]
    \vspace{-.1cm}
    \[\textrm{(Delta) }
    \dfrac{
      t_1 \match q_1 \vdash_\A S_1
      \quad \dots \quad
      t_n \match q_n \vdash_\A S_n
      % f(t_1,\dots, t_n) \match f(q_1,\dots, q_n) \vdash_\A S
    }{f(t_1, \dots, t_n) \matchi q \vdash_\A \bigotimes_1^n S_k
    }\left(
      \begin{array}{l}
        f(q_1, \dots, q_n) \rw q \in \Delta\\
      \end{array}
    \right )
    \footnote{
      {\footnotesize using $\bigotimes_1^n S_j = \{ (\top, id) \oplus (\phi_1,
        \sigma_1) \oplus \dots \oplus (\phi_n, \sigma_n) \sep
        (\phi_j,\sigma_j) \in S_j\}$}, and $(\phi, \sigma) \oplus (\phi',
      \sigma') = (\phi \land \phi',\sigma \cup \sigma')$.}
    \]
    \vspace{-.2cm}
    \[\textrm{(Epsilon) }
    \dfrac{ 
      t \matchi q    \vdash_\A S_0   \quad
      t \matchi q'_1 \vdash_\A S_1 \quad \dots \quad
      t \matchi q'_n \vdash_\A S_n
    }{
      t \match q \vdash_\A S_0 \cup
      \bigcup_{k=1}^n \{(\phi \land \alpha_k, \sigma) \sep (\phi, \sigma) \in S_k\}
    }\left(
      \begin{array}{l}
        \{(q_k, \alpha_k) \sep q_k \xrw{\alpha_k} q\}_1^n\\% \land q \not= q_k\}_1^n\\
        t \notin \X\\
      \end{array}
    \right)
    \]
  }
\end{definition}
\vspace{-.4cm}

Observe that, by definition, the matching problem considers possibly
infinite runs of the form $l\sigma \xrw{\alpha} q$. Indeed,
transitions in $\Drw^i \cup \Deq^i$ can produce loops.  In the
matching algorithm, we exclude such runs. This is done
to keep a finite set of rewriting path, which is computable in a
finite amount of time. It is worth mentioning that removing loops does
not remove any important information. Consider the automaton $\A$ of
Example~\ref{ex:semantics}. We observe that $f(b) \xrw{Eq(q_b, q_c)
  \land Eq(q_c, q_b)}_\A q_f$ uses the loop $f(b) =_E f(c) =_E
f(b)$. This loop can be removed as $f(a) \rw_\R^* f(b)$ can be
obtained by $f(b) \xrw{\top}_\A q_f$, which does not contain any
loops.

%%We now state the important property of our construction.
 
\begin{property}%[Matching Algorithm is complete]
  \label{prop:matching-complete}
  Let $\A$ be a $\RE-$automaton, $q$ one of its states, $l \in \TFX$
  the linear left member of a rewriting rule and a substitution $\sigma : \X
  \mapsto \Q$ whose domain is range-restricted to $\vars(l)$. Assuming that $S$
  is the solution of the matching 
  problem $l\sigma \match q$, for all $(\alpha, \sigma)$ such that
  $l\sigma \xrw{\alpha}_\A q$, a loop free run, then we have $(\alpha,
  \sigma) \in S$.
\end{property}


After computing $S$ for $l \match q$, we identify its elements that
correspond to critical pairs. By definition of $S$, we know that there
exists a transition $l\sigma \xrw{\alpha}_{\aaexeq^i} q$ for $(\alpha,
\sigma) \in S$. If there exists a transition $r\sigma
\xrw{\alpha'}_{\aaexeq^i} q$, then $r\sigma$ has already been added to
$\aaexeq^i$.  It is thus sufficient to deduce that all terms
$l\sigma\sigma'$ of the set represented by the configuration $l\sigma$
are rewritten into terms $r\sigma\sigma'$ represented by the
configuration $r\sigma$.  In the case where there exists no transition
$r\sigma \xrw{\alpha'}_{\aaexeq^i} q$, then $\la r\sigma, \alpha',
q\ra$ is a critical pair to solve on $\aaexeq^i$.  The following
theorem shows that our methodology is complete.

\begin{theorem}
  \label{thm:C}
  If $\aaexeq^i$ is well-defined then so is $\compl(\aaexeq^i)$, 
  and $\forall q \in \Q^i$, $\forall t \in
  \Lang(\aaexeq^i, q)$, $\forall t \in\TF$, $t \rw_\R t' \imp t' \in
  \Lang(\compl(\aaexeq^{i}), q)$.
\end{theorem}

\begin{example}
  \label{ex:C}
  Let $\R=\{f(x) \rw f(s(s(x)))\}$ and $\aaexeq^0=\langle \F, \Q,
  \Q_F, \Delta^0\rangle$ be a tree automaton such that $\Q_F=\{q_0\}$
  and $\Delta^0=\{ a \rw q_1, f(q_1) \rw q_0\}$.  Following Definition
  \ref{def:matching-algorithm}, the solution $S$ of the matching
  problem $f(x)\match q_0 $ is $S=\{(\sigma,\phi)\}$ with
  $\sigma=\{x\rw q_1\}$ and $\phi=\top$.  Hence, $\la f(s(s(q_1))),
  \top, q_0\ra$ is the only critical pair to solve, since
  $f(s(s(q_1)))\not\xrw{\top}_{\aaexeq^0} q_0$.  So,
  $\compl(\aaexeq^0) $ is a $\RE-$automaton such that
  $\compl(\aaexeq^0)= \la\F, \Q^1, \Q_F, \Delta^1\cup \Drw^1 \cup
  \Deq^0 \ra$ with:
  \begin{description}
  \item $\Delta^1=\norm(f(s(s(q_1))),\emptyset) \cup \Delta^0= \{s(q_1)\rw q_2,s(q_2)\rw q_3, f(q_3)\rw q_4\}\cup \Delta^0$, 
  \item $\Drw^1=\{q_4 \xrw{\top} q_0\}$, since $f(s(s(q_1)))\rw^!_{\Delta^1\setminus\Delta^0} q_4$, 
  \item $\Deq^0=\emptyset$ and $\Q^1=\{q_0,q_1,q_2,q_3,q_4\}$.
  \end{description}
\end{example}
Observe that if $\compl(\aaexeq^i)=\aaexeq^i$, then we have reached a fixpoint.
\vspace{-.6cm}
\paragraph{The Widening  $\widen$.}
Consider a $\RE$-automaton $A = \la \F, \Q, \Q_f, \Delta \cup \Drw
\cup \Deq \ra$, the widening operation consists in computing a
$\RE$-automaton $\widen(A)$ that is obtained from $A$ by using the set
of equations $E$.

\begin{tabular}{lc}
  \hspace{-1.1cm}
  \begin{minipage}{.75\linewidth}
    For each equation $l = r$ in $E$, we consider all pair
    $(q, q')$ of distinct states of $\Q^i$ such that there exists a
    substitution $\sigma$ to have the following diagram.
    Observe that $\rw^=_{A}$ is the transitive and reflexive
    rewriting relation induced by $\Delta \cup \Deq$.  
  \end{minipage}&
  \begin{minipage}{.25\linewidth}
    $\xymatrix{
      l\sigma \ar@{=}[r]_{E}\ar[d]^{=}_{A} & r\sigma \ar[d]^{A}_{=}\\
      q & q'
    }$
  \end{minipage}
\end{tabular}

Intuitively, if we have $u \rw^=_{A} q$, then we know that there
exists a term $t$ of $Rep(q)$ such that $t =_E u$. The automaton
$\widen(A)$ is $\la \F, \Q, \Q_f, \Delta \cup \Drw \cup \Deq' \ra$,
where $\Deq'$ is obtained by adding the transitions $q \rw q'$ and $q'
\rw q$ to $\Deq$, for each pair $(q, q')$. We also keep $\Deq'$ {\em
  closed by transitivity}, but only for pair of distinct states.
Roughly, the transitive closure of $\Deq'$ corresponds to propagate
explicitly terms that are equivalent by $=_E$.  As show in
section~\ref{sec:refinement}, this aspect is important to refine with
accuracy. Note that $\widen$ terminates since the number of states of
$\A$ is finite and the number of transitions to be added to $\Deq'$ is
finite.

\begin{theorem}
  \label{thm:W}
  Assuming that $A$ is well-defined, we have $A $ syntactically included in 
  $ \widen(A)$, and $\widen(A)$ is well-defined.
\end{theorem}

\begin{example}
  \label{ex:W}
  Consider the $\RE$-automaton $\compl(\aaexeq^0)$ in Example~\ref{ex:C}.\\
  \begin{tabular}{lc}
    \hspace{-.3cm}
    \begin{minipage}{.75\linewidth}
      We compute $\aaexeq^1 = \widen(\compl(\aaexeq^0))$ using the equation
      $s(s(x))=s(x)$.  We have $\sigma=\{x \mapsto q_1\}$ and the
      following diagram.  Then, we obtain $\aaexeq^1 = \la \F, \Q^1,
      \Q_f, \Delta^1 \cup \Drw^1 \cup \Deq^1\ra$, where $\Deq^1= \Deq^0 \cup \{
      q_3 \rw q_2,\ q_2 \rw q_3\}$ and $\Deq^0=\emptyset$.  Observe that $\aaexeq^1$ is a
      fixpoint: $\compl(\aaexeq^1) = \aaexeq^1$.
    \end{minipage}&
    \begin{minipage}{.25\linewidth}
      $\xymatrix{
        s(s(q_1)) \ar@{=}[r]_{E}\ar[d]^{=}_{\compl(\aaexeq^0)} & s(q_1) \ar[d]^{\compl(\aaexeq^0)}_{=}\\
        q_3 & q_2
      }$
    \end{minipage}
  \end{tabular}
\end{example}



% \begin{example}
%   Let $\A$ be a $\RE$-automaton such that $\Delta=\{f(q_1,q_2) \rw q_0, a \rw
%   q_1, a \rw q_2, g(q_1) \rw q_3\}$, $\Drw=\emptyset$ and $\Deq=\emptyset$.
% \begin{itemize}
%   \item If $E=\{g(x)=a\}$, we have $\sigma=\{x \mapsto q_1\}$ and 
% \[\xymatrix{
% g(q_1) \ar@{=}[r]_{E}\ar[d]_{=} & a \ar[d]^{=}\\
% q_3 & q_2
% }
% \]
% Hence, $\A \simp_E \A'$ where $\A'$ is similar to $\A$ except that $\Deq$
% becomes $\{q_2 \rw q_3, q_3 \rw q_2\}$. 
% % Note that a
% % similar situation could have been found between $g(q_1) \rwAnestar q_3$ and $a
% % \rwAnestar q_1$. We will see in the following that the order used to achieve
% % widening steps does not matter.
%   \item If $E=\{f(x,x)=g(x)\}$ then  there is no substitution $\sigma$ such that $f(x,x)\sigma=f(q_1,q_2)$:  the automaton is unchanged.
% \end{itemize}
% \end{example}


% Consider a $\RE$-automaton $\aaexeq^i = \la \F, \Q^i, \Q_f, \Delta^i
% \cup \Drw^i \cup \Deq^i \ra$, the widening consists in computing a
% $\RE$-automaton $\widen(\aaexeq^i)$ that is obtained from $\aaexeq^i$ by
% using $E$. 
% \begin{tabular}{lc}
%   \hspace{-.3cm}
%   \begin{minipage}{.75\linewidth}
%     For each equation $l = r$ in $E$, we consider all pair
%     $(q, q')$ of distinct states of $\Q^i$ such that there exists a
%     substitution $\sigma$ to have the following diagram.
%     Observe that $\rw^=_{\aaexeq^i}$ is the transitive and reflexive
%     rewriting relation induced by $\Delta^i \cup \Deq^i$.  
%   \end{minipage}&
%   \begin{minipage}{.25\linewidth}
%     $\xymatrix{
%       l\sigma \ar@{=}[r]_{E}\ar[d]^{=}_{\aaexeq^i} & r\sigma \ar[d]^{\aaexeq^i}_{=}\\
%       q & q'
%     }$
%   \end{minipage}
% \end{tabular}

% Intuitively, if we have $u \rw^=_{\aaexeq^i} q$, then we know that there exists a term
% $t$ of $Rep(q)$ such that $t =_E u$.
% We obtain $\Deq^{i+1}$ by adding the transitions $q \rw q'$ and $q'
% \rw q$ to $\Deq^i$, for each pair $(q, q')$. We also keep $\Deq^{i+1}$
% {\em closed by transitivity}, but only for pair of distinct states.
% Roughly, the transitive closure of $\Deq^{i+1}$ corresponds to
% propagate explicitly terms that are equivalent by $=_E$.  As show in
% section~\ref{sec:refinement}, this aspect is important to refine with
% accuracy.  Finally, we have $\widen(\aaexeq^i) = \la \F, \Q^i, \Q_f,
% \Delta^i \cup \Drw^i \cup \Deq^{i+1} \ra$.

% \begin{theorem}
%   \label{thm:W}
%   Assuming that $\aaexeq^i$ is well-defined, we have $\aaexeq^{i} \subseteq \widen(\aaexeq^i)$, and $\widen(\aaexeq^i)$ is well-defined.
% \end{theorem}

% \begin{example}
%   \label{ex:W}
%   Consider the $\RE$-automaton $\aaexeq^1$ in Example~\ref{ex:C}.\\
%   \begin{tabular}{lc}
%     \hspace{-.3cm}
%     \begin{minipage}{.75\linewidth}
%       We compute $\aaexeq^2 = \widen(\aaexeq^1)$ using the equation
%       $s(s(x))=s(x)$.  We have $\sigma=\{x \mapsto q_1\}$ and the following diagram.
%       Then, we obtain $\aaexeq^2 = \la \F, \Q^1, \Q_f, \Delta^1 \cup \Drw^1 \cup \Deq^2\ra$, where $\Deq^2=\{ q_3 \rw q_2,\ q_2 \rw q_3\}$.
%       Observe that $\aaexeq^2$ is a fixpoint: $\compl(\aaexeq^2) = \aaexeq^2$.
%     \end{minipage}&
%     \begin{minipage}{.25\linewidth}
%       $\xymatrix{
%         s(s(q_1)) \ar@{=}[r]_{E}\ar[d]^{=}_{\aaexeq^1} & s(q_1) \ar[d]^{\aaexeq^1}_{=}\\
%         q_3 & q_2
%       }$
%     \end{minipage}
%   \end{tabular}
% \end{example}



% \begin{example}
%   Let $\A$ be a $\RE$-automaton such that $\Delta=\{f(q_1,q_2) \rw q_0, a \rw
%   q_1, a \rw q_2, g(q_1) \rw q_3\}$, $\Drw=\emptyset$ and $\Deq=\emptyset$.
% \begin{itemize}
%   \item If $E=\{g(x)=a\}$, we have $\sigma=\{x \mapsto q_1\}$ and 
% \[\xymatrix{
% g(q_1) \ar@{=}[r]_{E}\ar[d]_{=} & a \ar[d]^{=}\\
% q_3 & q_2
% }
% \]
% Hence, $\A \simp_E \A'$ where $\A'$ is similar to $\A$ except that $\Deq$
% becomes $\{q_2 \rw q_3, q_3 \rw q_2\}$. 
% % Note that a
% % similar situation could have been found between $g(q_1) \rwAnestar q_3$ and $a
% % \rwAnestar q_1$. We will see in the following that the order used to achieve
% % widening steps does not matter.
%   \item If $E=\{f(x,x)=g(x)\}$ then  there is no substitution $\sigma$ such that $f(x,x)\sigma=f(q_1,q_2)$:  the automaton is unchanged.
% \end{itemize}
% \end{example}


%%% Local Variables: 
%%% mode: latex
%%% TeX-PDF-mode: t
%%% TeX-master: "main"
%%% End: 
