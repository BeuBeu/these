\section{Introduction}
\label{sec:introduction}

At the heart of all the techniques that have been proposed for
exploring infinite state spaces, is a symbolic representation that can
finitely represent infinite sets of states.

In this paper, we assume that states of the system are represented by
trees (terms) and set of states by tree automata. In this context, the
transition relation of the system is naturally represented by a set of
rewriting rules. It is well-known that this {\em Tree Regular Model
  Checking framework} is expressive enough to describe communication
protocols\,\cite{ALRd05} as well as a wide range of cryptographic
protocols~\cite{GenetK-CADE00,GenetTTVTT-wits03,avispa-site} and JAVA
applications\,\cite{BoichutGJL-RTA07}. 

In {\em Tree Regular Model Checking}, the main objective is to compute
an automaton representing the set of states of the system. As we are
dealing with infinite-state systems, the problem remains undecidable
and only partial solutions can be proposed. Among theses solutions, we
find the {\em predicate abstraction methodology} that was promoted by
Bouajjani et al.\,\cite{BHRV06a,BHRV06b}. The idea behind abstract
Tree Regular Model Checking consists in computing the automata
obtained after successive applications of the rewriting relation and
then use techniques coming from the predicate abstraction area in
order to over-approximate the set of reachable states. If the property
holds on the abstraction, then it also holds on the concrete
system. If a counter-example is found on the abstraction, then one has
to check if it is indeed a counter-example to the real system. If not,
this spurious counter-example must be used to refine the
abstraction. Bouajjani's algorithm, which may not terminate, proceeds
by successive abstraction/refinement until a decision can be taken.

Independently, Genet et
al.~\cite{Genet-RTA98,FeuilladeGVTT-JAR04,GenetR-JSC10} proposed {\em
  completion} that is another technique to compute an
over-approximation of the set of reachable states. The main difference
with the work in~\cite{BHRV06a} is that completion techniques use
equations to compute the abstraction~\cite{GenetR-JSC10}. Equations
gives a simple and formal semantics to abstractions on
trees~\cite{MeseguerPM-TCS08}.  Contrary to the work in
\cite{BHRV06a,BHRV06b}, completion techniques have been applied to
very complex case studies such as the verification of (industrial)
cryptographic
protocols~\cite{GenetK-CADE00,GenetTTVTT-wits03,avispa-site} and Java
bytecode applications~\cite{BoichutGJL-RTA07}. Unfortunately, these
completion techniques do not embed any notion of counter-example based
refinement.

The objective of this paper is to overcome the above mentioned problem
and propose a {\em CounterExample Guided Abstraction Refinement
  procedure} for completion algorithms. Our contribution is in two
steps. First, we propose {\em $\RE$-automaton}, that is a new
extension of tree automata. A $\RE$-automaton keeps trace of the
equations and rewriting rules applied to the initial automaton. The
automaton can be used to decide whether a term $t$ (or a set of terms)
is reachable from the set of initial states. If the procedure
concludes positively, then the term is indeed reachable. If the
procedure concludes negatively, then one has to refine the
$\RE$-automaton and restart the process. Our second contribution is to
propose a procedure that refines a $\RE$-automaton in an efficient
manner.
%%Our approach is naturally more efficient than the one of
%%Bouajjani et al. as it does not require to store the intermediary
%%computation steps and apply a reverse of the transition relation to
%%conclude whether the term is reachable or not.

\vspace{-.6cm}
\paragraph{Related work.} Regular model checking was first apply to
compute the set of reachable states of systems whose configurations
are represented by words\,\cite{BJNT00,BLW03}. The approach was then
extended to trees and first applied to very simple case
studies\,\cite{ALRd05}. In~\cite{BHV04}, Bouajjani et al. introduced
the first Counterexample abstraction techniques for regular model
checking and extended it to trees in \cite{BHRV06a}.  One of the main
difference with our work is that they have to record all the
intermediary automata to decide whether the counter-example is
spurious, while we avoid this enumeration by synthetizing the
information in a single and hopefully more compact $\RE$-automaton.
Their technique also requires to apply the reverse of the transition
relation in a successive manner. This operation, which involves
determinization and inclusion checks, may be computationally
expensive. Moreover, in their work, they represent the relation with a
tree transducer. This automaton, which encodes the full transition
relation in a whole, may be huge, while rewriting systems are
generally compact. Finally, their abstractions are defined using
automata-based predicates which are less declarative than
equations. In~\cite{BCHK08}, the authors use rewriting rules instead
of transducers, but intermediary steps are still recorded. Moreover,
we shall see that our approach is potentially more general than the
one in~\cite{BCHK08}. Finally, our work extends equational
abstractions~\cite{MeseguerPM-TCS08,Takai-RTA04} with counterexample
detection and refinement.



% The verification of infinite state systems often relies on finite state automata
% to finitely represent the set of states.  When states of a system are more
% accurately represented by trees rather than words, {\em tree automata} can be
% used to finitely represent the infinite set of tree structured states. In this
% case the transition relation between states is usually defined as a tree
% transducer. The regular tree model-checking problem has been intensively studied
% like in~\cite{AbdullaJMd-CAV02,AbdullaLDR-JLAP06}. It consists in showing that a
% set of states can (or cannot) be reached by applying the transition relation on
% a set of initial states.  To deal with systems having a non regular behavior,
% this framework has been extended in~\cite{BouajjaniHRV-ENTCS06}, so as to deal
% with regular abstractions of non regular infinite systems. In this case, since
% an over-approximation of the system behavior is considered, only unreachability
% of states can be proven.

% In parallel, the tree automata completion
% technique~\cite{genet-RTA98,FeuilladeGVTT-JAR04} has been proposed. It also
% tackles the verification of infinite state systems using tree automata. However,
% the transition relation is defined using term rewriting systems. Term rewriting
% systems offer a more concise way to define transitions between states and, as a
% result, scales up more easily than tree transducer based techniques. For
% instance, tree automata completion has been used for the verification of
% (industrial) cryptographic
% protocols~\cite{GenetK-CADE00,GenetTTVTT-wits03,avispa-site} and Java bytecode
% applications~\cite{BoichutGJL-RTA07}. Another interesting point of this
% technique is that abstractions are defined using equations~\cite{GenetR-JSC10},
% %whose expressivity is stronger than predicate abstractions
% %of~\cite{BouajjaniHRV-ENTCS06}. Equations used in completion 
% relying on the equational abstraction framework of~\cite{MeseguerPM-TCS08}.
% %with fewer restrictions though.
% Using equations yields abstractions whose expressivity is very different from
% those produced by predicate abstractions of~\cite{BouajjaniHRV-ENTCS06}.


% On the opposite, with regards to transducer-based techniques, tree automata
% completion has severals weaknesses. When dealing with abstractions, if the
% completion claims that a term is reachable, it is not possible to know if
% the term is actually reachable or if it has been found because of a too coarse
% abstraction. This problem is solved by most of the abstract model-checking
% techniques, like~\cite{BouajjaniHRV-ENTCS06} in the case of tree transducers.

% The first objective of this paper is to equip the tree automata completion
% algorithm with counter-example generation. The aim is here to discriminate, in
% the tree automata, between reachable terms and terms of the abstraction while
% preserving the efficiency of the completion algorithm. When completion stops
% because of a too coarse abstraction, the second objective is to automatically
% refine the abstraction. Here, we aim at taking advantage of the expressivity of
% equations so as to have terminating refinements where predicate abstraction
% of~\cite{BouajjaniHRV-ENTCS06} diverges. 

% Finally, though it will not be discussed in this paper, 
% the objective is to automatically verify Java programs re-using the framework 
% of~\cite{BoichutGJL-RTA07} and the Copster~\cite{copster} tool which compiles
% Java bytecode programs into term rewriting systems.

%%% Local Variables: 
%%% mode: latex
%%% TeX-master: "main"
%%% TeX-PDF-mode: t
%%% End: 
