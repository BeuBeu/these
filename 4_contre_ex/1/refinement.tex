\section{Approximation Refinement}
\label{sec:refinement}

Let $I$ be a set of initial terms characterised by the $\RE-$automaton
$\aaexeq^0$, $\R$ be a TRS, and $Bad$ the set of forbidden terms
represented by $A_{Bad}$ a tree automaton.  The reachability problem
boils down to check $\desc(I) \cap Bad
\stackrel{?}{=}\emptyset$. There are classes of systems for which
$\desc(I)$ is regular and can be computed in a finite amount of time
(see appendix~\ref{sec:exact}) but, in general, the computation does
not terminate. For such cases, our only hope is to work with a
\textit{Counterexample-Guided Abstraction
  Refinement\,\cite{DBLP:conf/time/Clarke03}} procedure that computes
successive abstractions and successively refine them until the
property can be prove correct or not. In the first part of the paper,
we have focused on computing the abstraction. We now propose a
technique that checks whether a term is indeed reachable from the
initial set of terms. If the term is a spurious counterexample, then
it has to be eliminated from the approximation. We then generalize the
operation to (possibly infinite) sets of terms.

%Finally, we develop a complete example of our new approach.

% show how the procedure $P$ handles a term of ment procedure. 
%  In
% Section \ref{subsec:refinementstep}, which exploits the
% well-definition introduced in Section \ref{sec:re-automaton} in order
% to refine a $\RE$-automaton by removing a (possibly infinite) set of
% terms that is characterized by a formula of equations on the states of
% the automaton.


%\Input{ref-step}

% Instead, most automata based techniques compute the
% intersection between $\aaexeq^i$ and $Bad$ and decide its
% emptiness. This cannot be straightforwardly applied here. First,
% designing an intersection algorithm is difficult because our two
% automata do not have a common structure. Second, the emptiness
% algorithm is very specific since emptiness is constrained by formulas
% on $\Drw$ transitions of $\aaexeq^i$.

%%we propose an
%%algorithm that given an approximation $\aaex^*$ and a set $Bad$
%%computes such a formula. If all the terms in $\aaex^*\cap Bad$ are
%%indeed reachable, then this formula reduces to $\top$ , else the
%%formula characterize a set of terms (accepted from the states
%%characterized by the equations) that may or may not be reachable and
%%that must be refined.


% %From a verification point of view, given a
% %reachability property $\phi$, $Bad$ represents the set of
% %configurations satisfying $\neg\phi$.  
% For some classes of $\R$,
% $\desc(\Lang(\A))$ is regular. Thus, the reachability problem,
% i.e. $\desc(\Lang(\A))\cap Bad \stackrel{?}{=}\emptyset$, can be
% decided. For most of those classes, reachability can be decided using
% $\RE$-automata completion since it stops without approximation
% ($E=\emptyset$) on a $\RE$-automaton $\aaex^*$ such that
% $\Lang(\aaex^*)=\desc(\Lang(\A))$ (see Theorem~\ref{thm:regular}).
% When the TRS is outside of those (restricted) classes, approximation
% and refinement are necessary. The principle of our approach is based on CEGAR
% approach. In~\cite{DBLP:conf/time/Clarke03}, the CEGAR
% (\textit{Counterexample-Guided Abstraction Refinement}) paradigm has
% been summarised and a general algorithm consisting in refining an
% abstraction function by analysing a spurious counterexample has been
% given. In section \ref{sec:recompletion}, it has been shown that the
% property of {\it well-definition of $\RE-$automaton}, see Definition
% \ref{def:well-defined}, is preserved by our approach. So, our approach
% fits perfectly with the CEGAR paradigm, since the previous property
% allows the discrimination between terms actually reachable and
% artifacts of the approximations. Moreover, an artifact of the
% approximation is characterized by a logic formula pointing out 
% which approximations done have produced the problematic term. 

% In section \ref{subsec:refinementstep}, we first describe how
% refinement is performed on a $\RE-$automaton $\aaexeq^i$ obtained by
% completion.  Moreover, we observe that $Bad \cap \Lang(\aaexeq^i)$ may
% be infinite. So, we can not iterate on each term in the intersection
% to prune $\aaexeq^i$. Thus, section \ref{subsec:intersectionformula}
% proposes to characterize by a logical formula the set of approximations performed 
% leading to a non-empty intersection. Obtaining the formula $\top$ means that there exists 
% actually a reachable term in the intersection. 


Let $\aaexeq^k = \la \F, \Q^k, \Q_f, \Delta^k \cup \Drw^k \cup \Deq^k
\ra$ be a $\RE$-automaton obtained after $k$ steps of completion and
widening from $\aaexeq^0$ and assume
that %the verification of the property fails for this $\RE$-automaton:
$\Lang(\aaexeq^k) \cap Bad \not=\emptyset$.  Let $t$ be a term of
$\Lang(\aaexeq^k) \cap Bad$.  Then, we know that there exists a run $t
\xrw{\phi}_{\aaexeq^k} q_f$ with $q_f \in \Q_f$.  We know that
$\aaexeq^k$ is well-defined by construction.  It implies that if $\phi
= \top$, we deduce $t \in \desc(I)$.  It means that $t$ is a
counter-example, so from a verification point of view, the property is
broken as formulated in Section~\ref{sec:completion}.  Otherwise, we
have that $\phi = \bigwedge_1^n Eq(q_j, q'_j)$, and $t$ is possibly a
spurious counterexample. We thus decide to remove it from the
approximation. For doing so, we use the {\em pruning
  methodology} that was informally introduced in
Example\,\ref{ex:pruning}.
\vspace{-.6cm}
\paragraph{The pruning step $\prune$.}
\label{subsec:refinementstep}
As we have seen in Example~\ref{ex:pruning}, we compute the pruned
$\RE$-automaton $\prune(\aaexeq^k, \phi)$ in two steps. The first step
consists of removing some transitions of $\Deq^k$ until $\phi$ does
not not hold anymore {\em i.e.} $\phi = \bottom$.  Consider the
formula $\phi$ containing the predicate $Eq(q,q')$: we replace this
predicate by $\bottom$ if we decide to remove the transition $q \rw
q'$ from $\Deq^k$.  The second step consists in propagating the
information. Indeed, we also have to remove all transitions $q
\xrw{\alpha} q' \in \Drw^k$, where the conjunction $\alpha$ contains
some atoms transitions removed from $\Deq^k$.  The procedure is
iterated for each possible reduction of $t$ in $\aaexeq^k$ and thus,
we finally get $\aaexeq^{k+1}$.  It is easy to see that, for any
$\phi$, there exists no run $t \xrw{\phi}_{\aaexeq^{k+1}} q_f$.

Observe that, when removing $t$ from the abstraction, our procedure
may also remove other terms that are generated by $\phi$. We now
briefly show (see the appendix for details) that the {\em possibly
  infinite} set of terms that belong to both the abstraction and the
set of bad states can be removed by exploiting the structure of the
formula. More precisely, we build a set $S$ containing triples of the
form $(q,q',\phi)$ where $q$ is a state of $\aaexeq^k$, $q'$ is a
state of $A_{Bad}$ and $\phi$ is a formula on $\Deq$ transitions of
$\aaexeq^k$.  For all triple $(q,q',\phi)$, the formula $\phi$ holds
if and only if $\Lang(\aaexeq^k,q)\cap \Lang(A_{Bad},q')\not=
\emptyset$. For all triple $(q,q',\phi)$, where $q$ is final in
$\aaexeq^k$, $q'$ is final in $A_{Bad}$ and $\phi = \top$, then some
of the terms recognized by $q'$ in $A_{Bad}$ are reachable.
Otherwise, $\phi$ is the formula to invalidate, i.e. negate some of
its atom so that it becomes $\bot$. Starting from $\phi$, the
refinement process is performed using the technique that presented
above.

\vspace{-.6cm}
%\input{ref-example}
\paragraph{Example.}
Consider the $\RE-$automaton $\aaexeq^1$ given in Example~\ref{ex:W}.
We define $A_{Bad}$ to be the tree automaton whose final state is
$q'_0$ and whose transitions are $a\f q'_1, s(q'_1)\f q'_2, s(q'_2)\f
q'_1$ and $s(q'_2)\f q_0$.  The forbidden terms that belong to the
language of $\Lang(A_{Bad})$ are of the form $f(s^{2k + 1}(a))$.
% Let us recall that the set of transitions of $\aaexeq^1$ is $\Delta^1\cup\Drw^1\cup\Deq^1$ with
% $\Delta^1= \Delta^0 \cup \{s(q_1)\f q_2,s(q_2)\f q_3,f(q_3)\f q_4\}$,
% $\Drw^1=\{q_4\stackrel{\top}{\f}q_0\}$ and $\Deq^1=\{q_2\f q_3, q_3\f
% q_2\}$. 
We observe that $\Lang(\aaexeq^1) \cap \Lang(A_{Bad}) \not=
\emptyset$.  According to the intersection algorithm sketched above,
one can construct a set $S$ of triples $(q_0,q'_0,\phi)$, where $\phi$
is the formula used to prune $\aaexeq^1$ in order to remove those
terms that belong to $\Lang(\aaexeq^1) \cap \Lang(A_{Bad})$.  Here,
$S=\{(q_0,q'_0,Eq(q_2,q_3)\land Eq(q_3,q_2)), (q_0,q'_0,Eq(q_2,q_3)),(q_0,q'_0,Eq(q_3,q_2))\}$.  We apply the pruning step for each
formula $\phi$ in the set.  We compute $\prune(\aaexeq^1,
Eq(q_2,q_3)\land Eq(q_3,q_2))$.  Removing the transition $q_2 \f q_3$
from $\Deq^1$ is sufficient to invalidate $Eq(q_2,q_3)\land
Eq(q_3,q_2)$. Moreover, this removing invalidates
$(q_0,q'_0,Eq(q_2,q_3))$ too. It remains to prune with
$(q_0,q'_0,Eq(q_3,q_2))$. This is done by removing the transition
$q_3\f q_2$ from $\Deq^1$.  At this step, no transition of $\Drw^1$
can be removed anymore. Indeed, all these transitions are all labeled
by $\top$. We thus define
$\aaexeq^2=\prune(\prune(\prune(\aaexeq^1,Eq(q_2,q_3)\land
Eq(q_3,q_2)),Eq(q_2,q_3)),$ $Eq(q_3,q_2))$ with $\Delta^2=\Delta^1$,
$\Drw^2= \Drw^1$ and $\Deq^2=\emptyset$. We observe that $\aaexeq^2$
is not $\R-$closed and should be completed. We thus define
$\aaexeq^3=\widen(\compl(\aaexeq^2))$. We found a new critical pair
for $f(x) \rw f(s(s(x)))$ and we obtain $\Delta^3 = \Delta^2 \cup
\{(q_3)\f q_5, s(q_5)\f q_6, f(q_6) \f q_7$, and $\Drw^3 = \Drw^2 \cup
\{q_7\xrw{\top}q_4\}$.

The interesting point concerns the application of $\widen$.  Observe
that the transitions of $\Deq^3$ directly results from the application
of the equation $s(x) = s(s(x))$ {\em i.e.} transitions $q_2\f q_3,
q_3\f q_2, q_3\f q_5,q_5\f q_3, q_5\f q_6,$ and $q_6\f q_5$.  After
the transitive closure of $\Deq^3$ has been computed, some new
transitions are inferred and added to $\Deq^3$, i.e., $q_2\f q_5,q_5\f
q_2, q_2\f q_6, q_6\f q_2, q_3\f q_6$ and $q_6\f q_3$.  We again check
the emptiness of $\Lang(\aaexeq^3) \cap \Lang(A_{Bad})$.  This
intersection is still not empty and most of the transitions in
$\Deq^3$ can be removed by the pruning operation.  Let $\aaexeq^4$ be
the $\RE-$automaton obtained by applying $\prune$ on $\aaexeq^3$, it
only remains $ q_3\f q_6$ and $q_6\f q_3$ in $\Deq^4$. We also observe
that no transition of $\Drw^3$ are removed: $\Drw^4=\Drw^3$. In fact,
all the transitions in $\Drw^4$ are labeled by $\top$. Then, we
restart the completion process and we observe that
$\aaexeq^4=\compl(\aaexeq^4)$.  We have thus reached a
fix-point. There, we observe that $\Lang(\aaexeq^4)\cap
\Lang(A_{Bad})=\emptyset$ and conclude that $\R^*(I) \cap Bad
=\emptyset$. Observe that our refinement is accurate in this
case. Indeed $\Lang(\aaexeq^4)=f(s^{2*k}(a))$, that is the exact set
of reachable states.

\begin{remark}
The above example cannot be handled with
the approach of~\cite{BCHK08} that also proposes a counterexample
guided abstraction technique. Indeed this technique would prove that a
property where the bad terms are bounded: $Bad_k = \{ f(s^{2*i +1}(a))
\sep i < k\}$ Else the procedure loops, if we consider all the set
$Bad$.
\end{remark}
% compute $\prune(\widen(\compl(\aaexeq^1)))$. According to Section
% \ref{sec:recompletion}, the set of transitions of $\widen(\compl(\aaexeq^1))$ is
% $\Delta^2\cup\Drw^2\cup\Deq^2$ with $\Delta^2=\Delta^1\cup\{s(q_3)\f
% q_5,s(q_5)\f q_6,f(q_6)\f q_7 \}$,
% $\Drw^2=\Drw^1\cup\{q_7\stackrel{\top}{q_4}\}$ and $\Deq^2=\{q_2\f
% q_3, q_3\f q_2, q_3\f q_5,q_5\f q_3, q_5\f q_6,q_6\f q_5, q_2\f
% q_5,q_5\f q_2, q_2\f q_6, q_6\f q_2, q_3\f q_6,q_6\f q_3\}$. Note that
% the transitions $q_2\f q_5,q_5\f q_2, q_2\f q_6, q_6\f q_2, q_3\f
% q_6,q_6\f q_3$ have been inferred from the transitions resulting from
% equation applications i.e. $\{q_2\f q_3, q_3\f q_2, q_3\f q_5,q_5\f
% q_3, q_5\f q_6,q_6\f q_5\}$. Applying the same process as previously,
% one concludes that the transitions $q_2\f q_3, q_3\f q_2, q_3\f
% q_5,q_5\f q_3, q_5\f q_6,q_6\f q_5, q_2\f q_6, q_6\f q_2, q_2\f q_5$
% and $q_5\f q_2$ have to be removed from $\Drw^2$.
 %  Consequently, one obtains $\aaexeq^4$ whose set of transitions is
%   $\Delta^4\cup\Drw^4\cup\Deq^4$ with $\Deq^4=\{q_3\f q_6, q_6\f
%   q_3\}$. 


% all $(\sigma,\alpha)\in S$ where $t\match q \vdash_{\aaexeq^k}S$ and
% $q\in\Q_f$, one can deduce that $\sigma=\emptyset$ and $\alpha
% \not=\top$.  Composing the formula $\alpha_t$ such that
% $\alpha_t=\bigvee_{(\sigma,\alpha)\in S}\alpha$, $\alpha_t$
% characterizes the set of approximations required to make $t$ be a term
% of $\Lang(\aaexeq^k)$.
%
% We propose a refinement operator $P$ such that t
% $\Lang(\prune(W(C(\aaexeq^{k}))))\cap Bad = \emptyset$
%
% We extend our new completion algorithm with 
% a refinement step denoted by $P$. Thus, $\aaexeq^{i+1}$ is defined as 
% $\prune(W(C(\aaexeq^i)))$. 
% Let $\aaexeq^k$ be the current automaton obtained by completion from
% $\aaex^0$ and such that
% $W(C(\aaexeq^k))=\la\F,\Q^k,\Q_f,\Delta^k\cup\Drw^k\cup\Deq^k\ra$ and 
% $\Lang(\aaexeq^{k})\cap Bad =\emptyset$. 
%
% If $\Lang(W(C(\aaexeq^{k})))\cap Bad =\emptyset$ then
% $\prune(W(C(\aaexeq^{k}))) = W(C(\aaexeq^{k}))$.  Assume now that
% $\Lang(W(C(\aaexeq^{k})))\cap Bad \not=\emptyset$ as illustrated in
% Figure \ref{fig:completion_rafinement} (considering that
% $A'=W(C(\aaexeq^{k}))$) and for all $i<k$,
% $\Lang(W(C(\aaexeq^{i})))\cap Bad=\emptyset$.
%
%
% Consequently, if one removes the transitions of $\Deq^k$ involved in
% $\psi$, then the term $t$ not recognized anymore.  This step is called
% {\it approximation refinement}.  The $\RE-$automaton
% $\aaexeq^{k+1}=\prune(W(C(\aaexeq^i)))$ is then obtained by constructing
% the smallest formula $\psi$ as a disjunction of $Eq(q,q')$ atoms with
% $q\f q'\in \Deq^k $ such that $\alpha_t\wedge\neg(\psi)$ can be
% simplified into $\bottom$.  Thus, the transitions of $\Deq^k$ involved
% in $\psi$ are removed from $\Deq^k$. Moreover, for all transitions
% $q''\stackrel{\alpha}{\f}q'\in\Drw^k$, let $\alpha'$ be the formula
% built from $\alpha$ in such a way that all $Eq(q,q')$ occuring in
% $\psi$ are replaced by $\bottom$ in $\alpha$. Consequently, one can
% simplified the formula by applying the rules $x \vee \bottom \f x$ and
% $x\wedge \bottom \f \bottom$ ($\vee$ and $\wedge$ are
% commutative). Let $\beta$ be the normal form after having applying
% these rules on $\alpha'$. If $\beta=\bottom$ then the transition
% $q''\stackrel{\alpha}{\f}q'''$ is removed from $\Drw^k$. Otherwise,
% $q''\stackrel{\alpha}{\f}q'''$ is replaced by
% $q''\stackrel{\beta}{\f}q'''$ in $\Drw^k$. Let $\Drw^{k+1}$ and
% $\Deq^{k+1}$ be respectively the updated sets $\Drw^k$ and
% $\Deq^k$. One can construct $\Delta^{k+1}$ by cleaning $\Delta^k $ of
% useless transitions.
%
%
%
%
%
%%Finally, one obtains
%%$\prune(W(C(\aaexeq^k)=\la\F,\Q^k,\Q_f,\Delta^{k+1}\cup\Drw^{k+1}\cup\Deq^{k+1}\ra
%%$ after having iterating this process on each term $t\in
%%\Lang(\aaexeq^{k})\cap Bad \not=\emptyset$. Note that
%%$\Lang(\aaexeq^{k+1})\cap Bad=\emptyset$.  Completion goes on until
%%finding a fixpoint proving the property $\phi$ or a counterexample of
%%$\phi$.
%
%
% $P$, which can be done at any time during the completion process,
% consists of prunning the current $\RE$-automaton to remove from it some spurious counterexamples.
% Assuming the reachability problem composed by $I$, $\R$, $Bad$, and $E$ a set of equations.
% If $\aaexeq^i$ is the $\RE$-automaton obtained after $i$ steps of completion and
% approximation, we know that $\aaexeq^i$ is well-defined.
% For all term $t$ of $Bad$, if we have $t \xrw{\top} q$, then $t$ is reachable, 
% and $t$ is a counterexample of the expected property. But, if we have $t \xrw{\alpha} q$
% with $\alpha \not\models \top$, then $t$ is a spurious counterexample, and we prune $\aaexeq^i$
% to remove $t$ before resuming the completion. $\alpha$ denotes the transitions of $\Deq$ corresponding
% to the approximation steps used to obtain $t$ by rewriting. Thus, we use $\alpha$ to find the transitions
% of $\aaexeq^i$ to prune.
%
% We observe that $Bad \cap \Lang(\aaexeq^i)$ may be infinite, we can not iterate on each term in the intersection
% to prune $\aaexeq^i$. We propose to a compute a formula that why the $\RE$-automaton 
%
%
%
%
%\begin{figure}
  \centering
  \begin{tikzpicture}[scale=1]
    \pgfdeclareimage[width=1.2cm]{A0}{4_contre_ex/img/A0}
    \pgfdeclareimage[width=2cm]{A1}{4_contre_ex/img/A1}
    \pgfdeclareimage[width=2.2cm]{A2}{4_contre_ex/img/A2}
    \pgfdeclareimage[width=2.2cm]{A3}{4_contre_ex/img/A3}
    \pgfdeclareimage[width=2.4cm]{A4}{4_contre_ex/img/A4}
    % 
    \node (a0) at (0, 0) {\pgfuseimage{A0}};
    \node (a1) at (4, 0) {\pgfuseimage{A1}};
    \node (a2) at (7.5, -.5) {\pgfuseimage{A2}};
    \node (a3) at (11, 0) {\pgfuseimage{A3}};
    \node (a4) at (15, 0) {\pgfuseimage{A4}};
    % virtual nodes
    % \node [circle, minimum size=0.8cm] (v0) at (3, 0) {};\\
    \node [circle, minimum size=2cm] (v1) at (7.6, 0) {};
    % 
    \node (n0) at (0, 0)       {\footnotesize $\aaex^0$};
    \node (n1) at ( 3.9, -.27) {\footnotesize $\aaexeq^{k}$};
    \node (n2) at ( 7.5, -.27) {\footnotesize $\aaexeq^{k+1}$};
    \node (n3) at (11, -.27) {\footnotesize $\aaexeq^{k+2}$};
    \node (n4) at (15  , -.27) {\footnotesize $\aaexeq^*$};
    \node (bad)at ( 7.3, -1.9) {\footnotesize $Bad$};
    % 
    \draw [->, dashed] (a0) to node[above] {\footnotesize $k-1$} node[below] {\footnotesize iter.} (a1);
    \draw [->] (a1) to node[above] {C + W } (v1);
    \draw [->] (v1) to node[above] {P} (a3);
    %\draw [->, dashed] (a3) to node[above] {\footnotesize point-fixe} (a4);
    \draw [->, dashed] (a3) to (a4);
  \end{tikzpicture}
  \caption{Étape de raffinement durant la complétion}
  \label{fig:completion_refinement}
\end{figure}


%%% Local Variables: 
%%% mode: latex
%%% TeX-master: "../../main"
%%% TeX-PDF-mode: t
%%% End: 

%
%
%
%\begin{figure}
%\begin{center}
%\includegraphics[scale=0.7]{img/refine.pdf}
%\end{center}
%\end{figure}
%
% Schéma "à la Bouajjani" qui explique le principe de l'approche à la CEGAR:
% \begin{itemize}
% \item Completion + Etape eventuelle de raffinement ...
%
% \item Dans quel cas l'étape de raffinement est-elle déclenchée???
%
%   \begin{itemize}
%   \item On a calculé $\aaex^{i+1}$
%   \item Un terme de "bad" est reconnu par l'automate : on calule en utilisant le filtrage
%     toutes les formules $\phi$:
%     soit c'est un contre-exemple car il existe $\phi = \top$ (preuve de correction pour le matching alors...)
%     donc le système viole la propriété sinon c'est un contre-exemple 
%     sinon on raffine
%   \item Expliquer que 3 étapes sont nécessaires pour raffiner :
%     \begin{itemize}
%     \item 
%       supprimer toutes les transitions de $\Drw$ telles que
%       toutes les formules deviennent fausses puisque $Eq(q, q') \equiv q \rw q' \in \Drw$
%       si toutes les formules $\phi$ sont fausses alors le terme n'est plus reconnus
%     \item 
%       exliquer le mécanisme pour garder l'information que l'on pouvait inférer
%       par clôture!!! Reprendre l'exemple $f(s(\dots s(a)\dots ))$
%      
%     \item
%       il faut supprimer tous les termes obtenus par réécriture de termes contenus
%       par dans la partie de l'approx que l'on vient de raffiner... 
%       on ne sait plus si ils sont atteignables pour l'instant, 
%       tant que l'on a pas tenter de compléter l'automate obtenu par l'élagage.
%     \end{itemize}
%   \end{itemize}
% \end{itemize}
%
%
%%\subsection{Intersection and emptyness decision}
%%\input{intersection.tex}



%%% Local Variables: 
%%% mode: latex
%%% TeX-PDF-mode: t
%%% TeX-master: "main"
%%% End: 
