
% Section Rappel du principe de la completion...
\section{The Tree Regular Model Checking Problem}
\label{sec:completion}

Our objective is to verify properties of a given system. We will focus
on models of systems whose set of reachable states may be, for
modeling reasons, infinite\,\cite{WB98} -- but our solution also works
for huge finite-state systems. Our first problem is to provide a
symbolic representation to represent and manipulate possibly infinite
sets of states. The problem is undecidable and only partial solutions
exist. Here, we will use {\em Tree Regular Model Checking
  (TRMC)}\,\cite{ALRd05}.
\noindent
In TRMC, a program is a tuple $(\F, I,
Rel)$, where

\begin{itemize}
\item
$\F$ is an alphabet on which a set of terms $\TF$ can be defined;

\item
$I$ is a set of initial configurations represented by a
tree automaton $\A$, i.e. $\Lang(\A)=I$;

\item $Rel$ is a transition relation represented by a set of
  left-linear rewriting rules $\R$.
\end{itemize}

\noindent
In our setting, a program will thus be represented by the tuple $(\F,\A,\R)$.
It has been shown that the above framework can be used to represent a
wide range of applications going from cryptographic protocols to JAVA
applications. We consider reachability problems.

\begin{definition}[Reachability problem]
\label{def:reachability}
Consider a program $(\F,\A,\R)$ and $Bad$ a set of forbidden terms
$Bad$. The reachability problem consists in checking whether there
exists terms of $\desc(\Lang(\A))$ that belong to $Bad$.
\end{definition}

% We will consider the verification of {\em reachability properties},
% i.e., properties that are represented by sets of states. We say that a
% program satisfies a reachability property $\phi$ if all of its
% reachable state is included in the set of states that represent
% $\phi$. Assuming the existence of a tree automaton for representing
% $\phi$, verifying reachability properties reduces to solve the {\em
%   reachability problem}.

% \begin{definition}
% \label{def:reachability}
%   Given a programme $(\F, {\phi}_I, R)$, we consider the reachability
%   problem that consists in computing a tree automaton $\aaex^*$ for
%   the set $R^*(\phi_I)$.
% \end{definition}



\noindent
For finite-state systems, computing the set of reachable terms
($\desc(\Lang(\A))$) reduces to enumerate the terms that can be
reached from the initial set of configurations. Unfortunately, for
infinite-state systems, this enumeration may never terminate.  There
is thus also a need to ``accelerate'' the search through the state
space in order to reach, in a finite amount of time, states at
unbounded depths. Among the existing algorithms used to compute a tree
automaton representing the set of reachable terms of a system, one
finds {\em completion algorithm}. A completion algorithm is a
semi-algorithm that computes an automaton $\aaex^*$ that is possibly
an over-approximation of the set of reachable terms. In the rest of
this section we introduce the principle of completion and point its
current limits.

We say that a tree automaton $\B$ is $\R$-closed if for all terms
$s,t$ such that $s\f_\R$ and $s$ is recognized by $\B$ into state $q$ then so is
$t$. The situation is represented with the following graph.
\begin{tabular}{lc}
  \hspace{-.3cm}
  \begin{minipage}{.8\linewidth}
    It is easy to see that $\Lang(\B) \supseteq \desc(\Lang(\A))$ if
    $\B$ is $\R$-closed and $\Lang(\B)\supseteq
    \Lang(\A)$~\cite{BoyerGJ-IJCAR08}. From an algorithmic point of
    view, building a $\R$-closed $\aaex^*$ from $\A$ consists in {\em
      completing} $\A$ with new transitions.  The completion algorithm
    computes successive automata $\aaex^1,\aaex^2,\ldots$ that
    represent the effect of applying the set of rewriting rules to the
    initial automaton.
  \end{minipage}
  &
  \begin{minipage}{.2\linewidth}
    $
    \xymatrix{
      s \ar[r]_{\R}\ar[d]^{*}_{\B} & t \ar@/^1.2pc/[ld]_{*}^{\B}\\
      q & %\ar[l]^{\A_{i+1}} q'
    }
    $
  \end{minipage}
\end{tabular}

Each application of $\R$ is called a {\em completion step} and
consists in searching for {\em critical pairs} $\langle t,q \rangle$ where the above
diagram is not closed, i.e. $ s\rwR t$, $s \rw_{\A}^* q$ and $t
\not\rw_{\A}^* q$.
% remis car il faut introduire la notation \aaex^i, utilisée dans la suite.
The idea being that the algorithm solves the critical pair by
constructing from $\A$, a new tree automaton $\aaex^1$ with the
additional transitions needed to obtain $t \rw_{\aaex^1}^* q$,
representing the effect of applying $\R$. Then a similar algorithm is
applied on $\aaex^1$ to obtain $\aaex^2$, and so on until a fixpoint
$\aaex^*$ is reached.

As the language recognized by $A$ may be infinite, it is not possible to find
all the critical pairs by enumerating the terms that it recognizes. The solution
that was promoted in \cite{Genet-RTA98} consists in applying sets of
substitutions $\sigma: \X \mapsto \Q$ mapping variables of rewrite rules to
states representing infinite sets of (recognized) terms. Given a tree automaton
$\aaex^i$ and a rewrite rule $l \rw r \in \R$, to find all the critical pairs of
$l \rw r$ on $\aaex^i$, completion uses a {\em matching
  algorithm}~\cite{FeuilladeGVTT-JAR04} that produces the set of substitutions
$\sigma: \X \mapsto \Q$ and states $q\in \Q$ such that $l \sigma \rw_{\aaex^i}^*
q$ and $r\sigma \not\rw_{\aaex^i}^* q$. Solving critical pairs thus consists in
adding new transitions: $r\sigma \rw q'$ and $q' \rw q$. Those transitions may
have to be normalized to respect the definition of transitions of tree
automata. As it was shown in~\cite{Genet-RTA98}, this operation may add not only
new transitions but also new states to the automaton. In the rest of the paper,
the completion-step operation will be represented by $\compl$, i.e., the
automaton obtained by applying the completion step to $\aaex^i$ is denoted
$\compl(\aaex^i)$.

The problem is that, except for particular
classes~\cite{FeuilladeGVTT-JAR04,Genet-Habil}, the automaton
representing the set of reachable terms cannot be obtained from $A$ by
applying a finite number of completion steps and the process thus
needs to be accelerated. For doing so, one can uses an approximation
technique based on a set of equations $E$ and produce an
over-approximation of the set of reachable terms, i.e., a tree
automaton $\aapprox$ such that $\Lang(\aapprox) \supseteq
\desc(\Lang(\A))$.

To produce such an automaton, each automaton $\aaex^i$ obtained by
applying $i$ completion steps to $A$ is approximated using a
widening function $\widen$ parametrized by $E$. An equation $u=v$ is
applied to a tree automaton $\A$ as follows: for all substitution
$\sigma:\X \mapsto \Q$ and distinct states $q_1$ and $q_2$ such that
$u \sigma \rw_{\A}^* q_1$ and $v\sigma \rw_{\A}^* q_2$, states $q_1$
and $q_2$ are merged.  Completion and widening steps can be
linked, i.e. $\aaexeq^0=\A$ and $\aaexeq^{i+1}= \widen(\compl(\aaexeq^i))$, until a
$\R$-closed fixpoint $\aapprox$ is found.  In~\cite{GenetR-JSC10}, it
has been shown that, under some assumptions, the obtained automaton
recognizes no more that terms reachable by rewriting with $\R$ modulo
$E$. As a result, the approximation framework and methodology is close
to equational abstractions of~\cite{MeseguerPM-TCS08}.

% The completion technique has successfully been used for the
% verification of cryptographic
% protocols~\cite{GenetK-CADE00,GenetTTVTT-wits03}.


% Furthermore, the
% tree automata completion implementation, \timbuk, is used as a
% verification backend in AVISPA~\cite{BoichutHKO-AVIS04,avispa}. More
% recently, tree automata completion has been used for fast prototyping
% of static analyzers for Java bytecode
% programs~\cite{BoichutGJL-RTA07}. In this settings, $\R$ encodes the
% system ({\em e.g.} protocol and intruder behavior or Java virtual
% machine semantics), $\aapprox$ represents the over-approximation of
% all states of the system. Then, 


%%the verification consists in showing
%%that the intersection between $\aapprox$ and an automaton $$

\begin{example}
\label{ex:comp}
Let $\R=\{f(x) \rw f(s(s(x)))\}$, $E=\{s(s(x))=s(x)\}$, $\A=\langle \F, \Q, \Q_F,
\Delta\rangle$ be a tree automaton such that $\Q_F=\{q_0\}$ and $\Delta=\{ a \rw
q_1, f(q_1) \rw q_0\}$, i.e. $\Lang(\A)=\{f(a)\}$.
\begin{tabular}{lc}
  \hspace{-.3cm}
  \begin{minipage}{.75\linewidth}
    The first completion step finds the following critical pair: $f(q_1) \rwA^* q_0$
    and $f(s(s(q_1))) \not\rwA^* q_0$. Hence, the completion algorithm produces
    $\aaex^1=\compl(\A)$ having all transitions of $\A$ plus $\{s(q_1) \rw q_2, s(q_2) \rw q_3,
    f(q_3) \rw q_4, q_4 \rw q_0\}$ where $q_2, q_3, q_4$ are new states produced
    by normalization of $f(s(s(q_1))) \rw q_0$. Applying $\widen$ with
    the equation $s(s(x))=s(x)$ on $\aaex^1$ merges the states $q_3$ and $q_2$.
  \end{minipage}&
  \begin{minipage}{.25\linewidth}
    $\xymatrix{
      s(s(q_1)) \ar@{=}[r]\ar[d]_{\aaex^1}^{*} & s(q_1) \ar[d]_{*}^{\aaex^1}\\
      q_3 & q_2
    }
    $
  \end{minipage}
\end{tabular}




  %$s(s(q_1))
  %\rw^*_{\aaex^1} q_3$ and $s(q_1) \rw^*_{\aaex^1} q_2$. 

In~\cite{GenetR-JSC10}, $\A^1_{\R,E}=\widen(\aaex^1)$ is built from $\aaex^1$ by renaming $q_3$ by
$q_2$. The set of transitions of $\A^1_{\R,E}$ is thus $\Delta \cup \{s(q_1) \rw
q_2, s(q_2) \rw q_2, f(q_2) \rw q_4, q_4 \rw q_0\}$.  Completion stops on
$\A^1_{\R,E}$ because it is $\R$-closed, thus $\aapprox=\A^1_{\R,E}$.
Now, let us assume that $Bad=\{f(s(a)), f(s(s(a)))\}$. The first term is not in
$\desc(\Lang(\A))$ but the second is. However, those two terms are 
%However, $f(s(a))$ is not in $\desc(\Lang(\A))$ but $f(s(s(a)))$ is.Those two terms are
recognized by $\aapprox$ and there is no way to distinguish between
the two: no way to detect that the second is {\em really} reachable
nor to automatically refine the abstraction so as to reject the first
one.
\end{example}


If the intersection between $\aapprox$ and $Bad$ is not empty, then it
does not necessarily mean that the system does not satisfy the
property. There is thus the need for techniques to decide whether a
counter-example is indeed a reachable term that does not satisfy the
property or if it is a term added by the abstraction and that cannot
be reached from the set of initial states. If the latter case occurs,
one has to propose a refinement technique that will remove the
false-positive from the abstraction. Studying such techniques for
completion automata is the main objective of this paper.

%%The rest of the paper is organized as follows. In Section
%%\ref{sec:re-automaton}, we introduce $\RE$-automaton that is an
%%extended tree automata. Section \ref{sec:re-automaton} proposes a
%%completion algorithm for $\RE$-automata, while Section \ref{} shows
%%how the extended structure can be used to refine in an efficient
%%manner.




% Tree automata completion~\cite{genet-RTA98,FeuilladeGVTT-JAR04,GenetR-JSC10} is
% an algorithm for computing sets of reachable terms $\desc(I)$, given a regular
% language of initial terms $I=\Lang(\A)$. For most of the known classes of TRS
% $\R$ for which $\desc(\Lang(\A))$ is regular, the output of completion is a tree automaton
% $\aaex^*$ such that $\Lang(\aaex^*)=\desc(\Lang(\A))$. When
% $\desc(\Lang(\A))$ is not regular, it is possible to parameterize the algorithm
% by a set of equations $E$, and to compute a tree automaton $\aapprox$ 
% over-approximating reachable terms,
% i.e. $\Lang(\aapprox) \supseteq \desc(\Lang(\A))$. 



%%% Local Variables: 
%%% mode: latex
%%% TeX-PDF-mode: t
%%% TeX-master: "main"
%%% End: 
