\documentclass{fsttcs}

% Note that \begin{document} comes before title information
\usepackage[utf8]{inputenc}
\usepackage{amsmath, amssymb,graphicx}
\usepackage{url}
%%%%%%%%%%%%%%%%%%%%%%%%%%%%%%%%%%%%%%%%%%%%%%%%%%%%%%%%%%%%%
%%Ces macros sont toutes utilisees par le document
%%%%%%%%%%%%%%%%%%%%%%%%%%%%%%%%%%%%%%%%%%%%%%%%%%%%%%%%%%%%%



% merging of states 
% la commande est deja definie dans le package xypic 
% utilise pour la figure de la paire critique...
% je l'ai remplace de maniere temporaire par \kw{merge}

% \newcommand{\merge}{Merge}
% suggestion de remplacement ?
% \newcommand{\merges}{Merge}


%Insertion de commentaires dans le PDF
%\newcommand{\comments}[1]{\paragraph{\bf Commentaire~: }{\color{blue}#1\newline}}

%%Mots clefs :
\newcommand{\kw}[1]{\mathtt{#1}}
\newcommand{\parts}[1]{\mathcal{P}(#1)}

\newcommand{\bydef}{\stackrel{\triangle}{=}}
\newcommand{\mapswith}[1]{\stackrel{#1}{\longrightarrow}}

\newcommand{\coq}{\textsf{Coq}}
\newcommand{\timbuk}{\textsf{Timbuk}}
\newcommand{\ocaml}{\textsf{OCaml}}
\newcommand{\tom}{\textsf{Tom}}
\newcommand{\scala}{\textsf{Scala}}


%%% A VOIR SI JE GARDE OU PAS
\def \N {\mathbb{N}}
%\def \norm {Norm}
\newcommand{\merge}{Merge}
\newcommand{\E}{\mathcal{E}}
\newcommand{\nr}{N}
\newcommand{\ddom}{\mathcal{D}om}
\newcommand{\dom}{\mathcal{D}om}

%Model Checking

\newcommand{\K}{\mathit K}
\newcommand{\M}{\mathit M}
\newcommand{\trans}{\Delta_\varepsilon^{\leftarrow}}
\newcommand{\Ap}{\mathcal A_p}
\newcommand{\validates}{\models}
\newcommand{\nxt}{\mathbf X}
\newcommand{\fut}{\mathbf F}
\newcommand{\gbl}{\mathbf G}
\newcommand{\unt}{\mathbf U}
\newcommand{\rel}{\mathbf R}


%Maths
\newcommand{\suchthat}{\textrm{ s.t. }}
\newcommand{\Nat}{\mathbb N}
\newcommand{\Natmod}{\Nat[\ ]}
\newfont{\amstoto}{msbm10}
%\newcommand{\NN}{\mbox{\amstoto\char'116}}
\newcommand{\NN}{\N^*}
\newcommand{\ZZ}{\mbox{\amstoto\char'132}}

%Logic
\newcommand{\equ}{\; \Longleftrightarrow \;}
%\newcommand{\validates}{\models}

%Shortcuts
\newcommand{\tn}{T}
\newcommand{\la}{\langle}
\newcommand{\ra}{\rangle}
\newcommand{\eps}{\varepsilon}

\newcommand{\iem}{\text{ième}}
\newcommand{\er}{\text{er}}
\newcommand{\et}{\mbox{ et }}
\newcommand{\st}{\mbox{ t.q. }}
\newcommand{\oute}{\mbox{ ou }}
\newcommand{\ou}{\; \vee \;}

\newcommand{\spf}{\;\; \Longrightarrow \;\;}
\newcommand{\dbf}{\;\; \Longleftrightarrow \;\;}
\newcommand{\imp}{\; \Longrightarrow \;}
\newcommand{\rimp}{\; \Longleftarrow \;}
\newcommand{\notimp}{\; \mathrel{\not\hspace*{-1mm}\Longrightarrow} \;}
\newcommand{\sep}{\; | \;}
\newcommand{\dsep}{\; || \;}


%Term Rewriting Systems
%
\newcommand{\Q}{\mathit Q}
\newcommand{\Qf}{\Q_F}
\newcommand{\arity}{ar}
%\newcommand{\Sub}{{\cal S}}
\newcommand{\F}{\mathcal{F}}
%\newcommand{\Y}{{\cal Y}}
%\newcommand{\C}{{\cal C}}
\newcommand{\D}{{\cal D}}
\newcommand{\TF}{\mathcal{T(F)}}
\newcommand{\TFX}{\mathcal{T(F, X)}}
\newcommand{\TFQ}{\mathcal{ T(F \cup \Q)}}
\newcommand{\TFQp}{\mathcal{ T(F \cup \Q')}}
\newcommand{\TFQX}{\mathcal{ T(F \cup \Q, X)}}
\newcommand{\TFXQ}{\mathcal{ T(F, X \times \Q)}}
\newcommand{\TC}{\mathcal{ T(C)}}
\newcommand{\aut}{\langle \F, \Q, \Qf, \Delta \rangle} 
%\newcommand{\B}{\mathcal{ B}}
\newcommand{\ordlexico}{\prec}
\newcommand{\bottom}{\perp}
%\newcommand{\match}{\unlhd}

\newcommand{\var}{\mathcal{ V}ar}
\newcommand{\pos}{\mathcal{ P}os}
\newcommand{\R}{\mathcal R}
\newcommand{\RE}{{\R_{/E}}}
\newcommand{\Rep}{\R ep}
\newcommand{\desc}{\R^*}
\newcommand{\descE}{\RE^*}
\newcommand{\T}{\mathcal T}
\newcommand{\X}{\mathcal X}
\newcommand{\vars}{\mathcal V}
\newcommand{\rw}{\rightarrow}
\newcommand{\trw}{\rightarrow^{\lambda}}
\newcommand{\lrw}{\longrightarrow}
\newcommand{\xrw}{\xrightarrow}
\newcommand{\rwegal}{\rw^=}

\newcommand{\nrw}{\nrightarrow}
\newcommand{\arw}{\dashrightarrow}
%\newcommand{\rwne}[1]{\rw^*_{#1}}
\newcommand{\rwne}{\rw^{\not\varepsilon}}
\newcommand{\rwned}{\rwne{\Delta}}
\newcommand{\rweq}{\rw^=}
%\newcommand{\rwtag}[1]{\stackrel{#1}{\rw}}
\newcommand{\rwc}{\twoheadrightarrow}
%\newcommand{\rwneq}{\rw^{\not =}}
%\newcommand{\rwe}{\f^\varepsilon}
\newcommand{\rwmod}{\rw_{\R/E}}


%substitutions
\newcommand{\plus}{\sqcup}
\newcommand{\bigplus}{\bigsqcup}
\newcommand{\id}{id}

%completion
\newcommand{\match}{\unlhd}
\newcommand{\matchi}{\lhd}
\newcommand{\matchb}{\dot{\unlhd}}
\newcommand{\matchbi}{\dot{\lhd}}
\newcommand{\Deps}{\upvarepsilon}
\newcommand{\Drw}{\upvarepsilon_{\R}}
\newcommand{\Deq}{\upvarepsilon_{=}}
%\newcommand{\Deps}{\upvarepsilon}
%\newcommand{\norm}{\downarrow}
\newcommand{\norm}{\kw{Norm}}
\newcommand{\slice}{\kw{Slice}}
\newcommand{\compl}{\kw{C}}
\newcommand{\widen}{\kw{W}}
\newcommand{\prune}{\kw{P}}

\newcommand{\simp}{\leadsto}
                          
%Tree Automata
\newcommand{\A}{\mathit A}
\newcommand{\B}{\mathit B}
\newcommand{\C}{\mathit C}
%\newcommand{\F}{\mathcal F}
\newcommand{\Pred}{\mathcal P}
\newcommand{\sub}{\Subset}
\newcommand{\Lang}{\mathcal{L}}
\newcommand{\f}{\rw}
\newcommand{\aaex}{{\mathit A}_{\R}}
\newcommand{\aaexeq}{{\mathit A}_{\R,E}}
\newcommand{\aapprox}{\A^*_{\R,E}}
\newcommand{\automaton}[3]{\la #1, #2, #3 \ra}


\newcommand{\rwA}{\rw_\A}
\newcommand{\rwR}{\rw_\R}
\newcommand{\rwB}{\rw_\B}


%\newcommand{\none}{\kw{none}}
\newcommand{\true}{\mathit{true}}
\newcommand{\false}{\mathit{false}}
\newcommand{\ifte}{\mathit{if}}

% \newcommand{\completion}{\kw{next}}
% \newcommand{\refinedCompletion}{\kw{completion}}
% \newcommand{\equations}{\kw{applyEquations }}
% \newcommand{\refinement}{\kw{refinement}}
% \newcommand{\update}{\kw{updateAndClean}}
% \newcommand{\automatonCleaning}{\kw{ automatonCleaning}}

% \newcommand{\badApproximations}{\kw{badApprox }}
% \newcommand{\badTerms}{\kw{Bad }}
% \newcommand{\moduloTerms}{\kw{moduloTerms }}
% \newcommand{\transToDelete}{\kw{transToDelete }}
% \newcommand{\reachableTerms}{\kw{reachableTerms }}
% \newcommand{\reachableStates}{\kw{reachableStates }}
% \newcommand{\states}{\kw{statesIn }}




%\newcommand{\sun}{\textsc{Sun Microsystem\ }}
%\newcommand{\java}{\textsc{Java\ }}
%\newcommand{\midp}{\textsc{Java MIDP\ }}
%\newcommand{\lande}{\textsc{Lande\ }}
%\newcommand{\danger}{\textsc{\textbf{Danger !}}\normalsize}
%\newcommand{\coq}{{\textit Coq}}

%%%%%%%%%%%%%%%%%%%%%%%%%%%%%%%%%%%%%%%%%%%%%%%%%%%%%%%%%%%%%%%%%%%%
\newcommand{\reff}[1]{[\textsc{Fig}. \ref{#1}]}
\newcommand{\refs}[1]{[sec. \ref{#1}]}
\newcommand{\refa}[1]{[cf. \ref{#1} p. \pageref{#1}]}

%Macros pour construire une regle d'inference small-step semantics
\newcommand{\instr}[1]{instruction_P(m, pc) = \kw{#1}}
\newcommand{\hs}[1]{\langle #1 \rangle}
\newcommand{\infer}[3]
{\dfrac{\instr{#1}}{\hs{#2}::sf \to_{#1}\hs{#3}::sf}}

% #2 = side-conditions : 
\newcommand{\inferi}[5]
{\dfrac{\instr{#1} \quad\ #2 }{#3::sf, \rho \to_{#1} #4::sf, #5}}
%%%%%%%%%%%%%%%%%%%%%%%%%%%%%%%%%%%%%%%%%%%%%%%%%%%%%%%%%%%%%%%%%%%%%

%\newtheorem{property}{Property}
%\newenvironment{property}{\theoremlike{Property}}{\par\medskip}
%\newtheorem{algorithm}[subsection]{Algorithm}
%\newtheorem{example}{Example}
%\newenvironment{example}{\theoremlike{Example}}{\par\medskip}

\newcounter{savetheorem}

%%% Local Variables: 
%%% coding: utf-8
%%% mode: latex
%%% TeX-master: "main"
%%% TeX-PDF-mode: t
%%% ispell-local-dictionary: "french"
%%% End: 


\begin{document}

\title{Nouveau plan du papier}

% These are required for headers and for copyright on title page
\runningtitle{Plan du Nouveau Papier}
\runningauthors{Boichut, Boyer, Genet, Legay}

% Multiple authors, sharing an affiliation, use  \affiliation{...}

%\author[lab2]{B. Boyer}{Benoît Boyer}
%\address[lab2]{IRISA - Université Rennes 1, France}
%\email{Benoit.Boyer@irisa.fr}  %optional

\author{
  Y. Boichut\inst{1}%\thanks{Funded by DOT MOT ROT},
  ,
  B. Boyer\inst{2}%\thanks{Funded by DOT MOT ROT},
  ,
  T. Genet\inst{2}%\thanks{Funded by DOT MOT ROT},
  and
  A. Legay\inst{3}%\thanks{Funded by DOT MOT ROT}
}
  
%   S. Mello, R. Shello\thanks{\ldots but no thanks!}}

\affiliation{
  LIFO - Université Orléans,
  France
}
\affiliation{
  Université Rennes 1,
  France%\\
  % \email{\{Benoit.Boyer, Thomas.Genet\}@irisa.fr}
}
\affiliation{
  INRIA - Rennes,
  France
}
% You can also use 
% 
% \institute{}{
%   Indian Institute of Science\\
%   Bangalore\\
%   \email{\{tweedledum,tweedledee,hohum\}@iisc.ernet.in}
% }  
% 
% which will leave more space between author and affiliation

% No \maketitle required

% \begin{abstract}
% \comments{Un nouvel abstract plus court, et plus lisible ???}
% \end{abstract}

%DECOMMENTER POUR COMPILER l'ABSTRACT SEUL
% \end{document}

\section{Introduction}
\label{sec:introduction}
\comments{Partie refaite par Axel}
\section{Definitions}
\label{sec:definitions}
\comments{
  Toutes les definitions à propos de Réécriture, substitutions ($\Q$-subst inclues) et 
  Automates d'arbres.

  Attention : \\
  
  pas de def independante de normalized transitions et epsilon-transition
  uniquement une def globale pour les automates 
  suivie du langage d'un automate d'arbres sans format def.
}

\section{Tree automata completion}
\label{sec:completion}
Ici une présentation du principe de la technique de vérif basée sur
la complétion d'automate d'arbre à la Genet... schéma ?
et parler des différentes étapes de calcul pour une étape de complétion.
filtrage, normalisation, application des équations...
présentation de l'exemple fil conducteur du papier
et montrer les limitations de la technique en cas d'échec $=>$
ce qui motive la section suivante...


\section{$\RE$-Automaton $+$ contruction}
\label{sec:automaton}
\comments{trouver un vrai titre}
Introduire l'approche choisie :
une nouvelle race d'automate permettant de distinguer un terme dont on est sûr
qu'il est atteignable d'un terme qui ne l'est pas, juste en reconnaissant le
terme...
On se place dans le cadre de la vérification d'une propriété d'atteignabilité
où le problème est formé par $\R$, $I^0$ et $E$.
\begin{itemize}
\item Def de $\RE$-automate
\item Exemple (reprise de l'exemple ???) peut-être pas
  Donner seulement l'intuition qui est derrière juste avec quelques transitions jouet

\item Prop~1 (cf. rta10)
\item Nouvelle sémantique  + Equivalence avec la sémantique standard !
(implique de rester dans la même classe de complexité des automates
meilleures que les automates à contraintes, TAGGED automatas \dots)
\end{itemize}

Ensuite on arrive dans la partie construction d'un RE-automate en utilisant
un schéma similaire à complétion Genet.

\begin{itemize}
\item definition de $\aaex^0$ : limitation dûe à Prop~1 : I doit être finie
  mais expliquer qu'il sera facilement possible de contourner la limitation

\item construction $\aaex^{i+1}$ à partir de $\aaex^i$
  \begin{itemize}
  \item matching + lemme 
  \item réutilisation de la normalisation definie precedemment + nouveau lemme Prop 1.
  \item abstraction liée aux équations
  \item Théroèmes qui assurent la correction de l'approche ! clôture + contre-exemple
  \end{itemize}
\end{itemize}



\comments{Une explication importante à prendre en compte pour expliquer
  comment choisi-t-on de Màj une $\varepsilon$-transtion, lors de la résolution
  de la paire critique...}

Lors de la complétion, dans quel cas un $q_1 \rwtag{\phi} q_2$
devient un $q_1 \rwtag{\phi \vee \phi'} q_2$? J'ai l'impression que notre ordre
**strict** sur les formules est l'implication **stricte**, i.e 
$A > B$ ssi $A \notimp B$ et $A \rimp B$ (et on aura $A=B$ ssi $A \Leftrightarrow B$.).
Par ex, une branche du treilli des formules ($\top$ est en haut et $\bot$ est en bas):
$\top > A \vee B > A > \bot$
car $\top  \rimp  A \vee B  \rimp A  \rimp  \bot$ et $\top  \notimp A \vee  B
\notimp A \notimp \bot$.

Par complétion, on va toujours de la droite vers la gauche, on est strictement
croissant dans l'ordre $>$. Logique, car on augmente le nombre "chances"  d'avoir
la transitions $q_1 \rw q_2$. Initialement ça pourrait être $\bot$ même si on ne le
fait pas en pratique. Puis on augmente le nombre de chemins, jusqu'à atteindre
éventuellement $\top$: on est sûr de l'avoir. Par complétion on ne fait qu'affablir
la formule, croître par rapport à $>$. 

Moralité pour remplacer $q_1 \rwtag{\phi} q_2$ par $q_1 \rwtag{\phi \vee \phi'}
q_2$, il faut que $\phi \vee \phi' > \phi$, c'est à dire:
\begin{itemize}
\item $\phi \vee \phi' \rimp \phi$, ça c'est toujours vrai.
\item $\phi \vee \phi' \notimp \phi$, ça c'est à vérifier à chaque fois.
\end{itemize}


\section{Approximation Refinement}
\label{sec:refinement}

\comments{Il faut réfléchir à la place que l'on réserve au calcul de l'intersection
de RE-Automates}

Schéma "à la Bouajjani" qui explique le principe de l'approche à la CEGAR:
\begin{itemize}
\item Completion + Etape eventuelle de raffinement ...

\item Dans quel cas l'étape de raffinement est-elle déclenchée???

  \begin{itemize}
  \item On a calculé $\aaex^{i+1}$
  \item Un terme de "bad" est reconnu par l'automate : on calule en utilisant le filtrage
    toutes les formules $\phi$:
    soit c'est un contre-exemple car il existe $\phi = \top$ (preuve de correction pour le matching alors...)
    donc le système viole la propriété sinon c'est un contre-exemple 
    sinon on raffine
  \item Expliquer que 3 étapes sont nécessaires pour raffiner :
    \begin{itemize}
    \item 
      supprimer toutes les transitions de $\Deps$ telles que
      toutes les formules deviennent fausses puisque $Eq(q, q') \equiv q \rw q' \in \Deps$
      si toutes les formules $\phi$ sont fausses alors le terme n'est plus reconnus
    \item 
      exliquer le mécanisme pour garder l'information que l'on pouvait inférer
      par clôture!!! Reprendre l'exemple $f(s(\dots s(a)\dots ))$
      
    \item
      il faut supprimer tous les termes obetnus par réécriture de termes contenus
      par dans la partie de l'approx que l'on vient de raffiner... 
      on ne sait plus si ils sont atteignables pour l'instant, 
      tant que l'on a pas tenter de compléter l'automate obtenu par l'élagage.
    \end{itemize}
  \end{itemize}
\end{itemize}


\section{Conclusion}
\label{sec:conclusion}
\comments{à discuter\dots}

\bibliographystyle{abbrv}
%\bibliography{biblio/sabbrev,biblio/eureca,biblio/genet,biblio/mc}
\bibliography{sabbrev,eureca,genet}%,mc}

\end{document}
%%% Local Variables: 
%%% mode: latex
%%% TeX-master: t
%%% End: 
