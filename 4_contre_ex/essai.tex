\chapter{Détection de contre-exemples et Raffinement}
\section{Fausses Alarmes et Contre-Exemples}
Lors de la vérification d'une propriété basée sur le calcul d'une sur-approximation telle
que décrite précédemment, on ne s'est intéressé qu'à la correction de l'approche~: c'est à dire le
cas où la vérification permet de conclure à la préservation de la propriété. Or en pratique, il est rare
de trouver immédiatement une approximation qui contienne l'ensemble des états du système et qui en
même temps soit assez fine pour ne pas produire de fausses alarmes. Les \emph{fausses alarmes} sont
le résultat d'une approximation trop grossière qui contient des états supplémentaires qui violent la
propriété. Puisque ces états n'appartiennent pas au système, le système ne viole pas la propriété.

%TODO introduire un schéma sur les différents cas de 
%\usepackage{tikz}

\begin{center}
  \tikz[thick]{
    \node (approx) at (0, 0)   [draw, rectangle, minimum height=5cm, minimum width=9cm, fill=blue, opacity=.6] {};
    \node (system) at (-.3, 0) [draw, ellipse,   minimum height=4cm, minimum width=7.5cm, rotate=-15, fill=yellow] {};
    \node (prop)   at (5, 0)   [draw, ellipse,   minimum height=4cm, minimum width=6cm, fill=red, opacity=.75, rotate=15] {};
  }
    %\node (l) at (-.8, .8) [] {$l\sigma$};

    % } 
 %\usetikzlibrary{shapes}
    % \begin{scope}%options[]
  %    \clip [rotate=-45] (0,0) ellipse (1.5cm and 0.9cm);
  %    \fill [red, opacity=0.35] (-1, 1) circle (0.7cm);
  %  \end{scope}
  %  }

%     \begin{scope}%options[]
%       \clip [rotate=-45] (0,0) ellipse (1.5cm and 0.9cm);
%       \fill [red, opacity=0.35] (-1, 1) circle (0.7cm);
%     \end{scope}
    

    %\node (l) at (-.8, .8) [] {$l\sigma$};
    %\begin{scope}
    %  \clip [rotate=-45] (0,0) ellipse (1.5cm and 0.9cm);
    %  \fill [red, opacity=0.35] (1, -1) circle (0.7cm);
    %\end{scope}
    %\node (r1) at (.8, -.8) [] {$r\sigma$};
    %\draw [->] (l) to [out=-20,in=110](r1) node[pos=0.9, left] {\small $\R$};
  % }
   

\end{center}

  % \vline
%   \begin{column}{6cm}
%     \begin{center}
%       % Critical Pair 
%       \tikz[node distance=2cm, thick]{
%         \node (l) {$l\sigma$};
%         \node (q) [below of=l]{$q$};
%         \node [right of=l, text=White] {$r\sigma$};
%         \draw [color=Red, ->] (l) -- (q) node[pos=0.9, right]{\small $*$} node[pos=0.9, left] {\small $\resB$};
%         \uncover<2->{
%           \node (r) [right of=l] {$r\sigma$};
%           \draw [->, dashed] (l) -- (r) node[pos=0.9, below]{\small $\TRS$};}
%         \uncover<3->{
%           \draw [->] (l) -- (r) node[pos=0.9, below]{\small $\TRS$};
%           \draw [->] (r) to[out=-90, in=0] (q) node[pos=0.9, below]{\small $\resB$} node[pos=0.9, above]{\small $*$};
%         }
%       }
%     \end{center}
%   \end{column}




%%% Local Variables: 
%%% mode: latex
%%% TeX-master: "../main"
%%% End: 


En résumé, la propriété peut être violée pour 2 raisons~:\\

\begin{itemize}
\item L'exécution du système peut conduire à un état (atteignable) qui ne
  respecte pas la propriété~: cet état constitue un \emph{contre-exemple}.
  
\item Le système respecte la propriété mais l'approximation trop
  grossière introduit des états (inatteignables) qui violent la
  propriété~: c'est une \emph{fausse alarme}.\\
\end{itemize}

Or cette méthode de vérification d'une propriété est semi-décidable: il n'est généralement pas possible
de distinguer un état du système d'un état issu de l'approximation. En résumé, cela signifie que lorsque
la procédure de vérification de la propriété échoue, il n'est pas possible de savoir si le système viole
la propriété.

L'enjeu revient donc à chercher si l'intersection contient des états du système. Alors qu'il est impossible
de déterminer si un état quelconque appartient au système, on peut marquer les états pour lesquels leur
appartenance au système est sûre. Un tel marquage est réalisable itérativement~: il sûr de considérer qu'un état
appartient au système si il peut être produit à partir d'un état pour lequel on a aussi l'assurance qu'il
appartient au système. Dualement, on perd l'assurance pour tout état introduit par une approximation ou
si l'état est obtenu par des transitions issues d'états pour lesquels il n'existe aucune assurance
qu'ils appartiennent au système.


%%%%%%%%%%%%%%%%%%%%% TODO introduire un schéma qui donne les différents cas où l'on sûr ou pas l'on est que des états
%%%%%%%%%%%%%%%%%%%%% appartenant au système ou non.

Ainsi dans le cas où la vérification de la propriété échoue à cause d'un état $s$ dont on est sûr
qu'il appartienne au système, on peut conclure que le système viole la propriété et dans cas 
l'état $s$ constitue un \emph{contre-exemple}, dans le cas contraire on ne sait toujours rien dire,
on est soit dans le cas d'une fausse alarme ou d'un contre-exemple.

\section{Principe de l'approche à la {CEGAR}}
Il est possible d'améliorer l'approche précédente~: lors de la construction de 
la sur-approximation $E$ de l'ensemble des états atteignables du système si la vérification échoue
et on ne sait pas si l'on fait face à un contre-exemple ou une une fausse alarme.
Le principe de l'approche à la {CEGAR} (\emph{Counter-Example Guided Abstraction Refinement})
tel qu'initialement présenté dans \cite{cegar00} consiste à retirer itérativement tous les
états $s_?$ qui conduisent à la mise en échec de la propriété alors qu'il n'est pas possible
de déterminer si ils sont réellement des contre-exemples ou non. On tente alors de terminer le calcul
$E$ privés des états $s_?$. En fonction de la réelle nature de chaque état $s_?$ supprimé on peut différencier
deux cas possibles~:

\begin{itemize}
\item
  Si $s_?$ est atteignable, il existe au moins une exécution du système qui passe par cet état~:
  cela signifie qu'au moment où la propriété a été violée, la trace correspondant à l'exécution
  n'a pas encore été considérée et que donc l'état a été introduit par l'approximation. Ainsi en 
  complétant l'ensemble des états du systèmes, l'état finira par être de nouveau ajouté à l'ensemble
  puisqu'il est réellement atteignable.

\item
  Si $s_?$ n'est pas réellement atteignable, alors il s'agissait d'une fausse alarme qui est supprimée
  par l'étape de raffinement, puisque l'état est retiré de l'approximation.

\end{itemize}

Cependant raffiner approximation de l'ensemble des états peut entrainer une divergence du calcul
puisque la raffinement affaiblit l'approximation~: cela impose de raffiner avec parcimonie, pour
éviter d'accroître ce risque de divergence.

%%%%%%%%%%%%%%%%%%%%%%%%%%%%%%%%%%%%%%%%%%%%%%%%%%%%%%%%%%%%%%%%%%%%%%%%%%%%%%%%%%%%%%%
%%%%%%%%%%%%%%%%%%%% VERSION EXTRAITE DE ../plan/plan_contre_ex.mm %%%%%%%%%%%%%%%%%%%%
%%%%%%%%%%%%%%%%%%%%%%%%%%%%%%%%%%%%%%%%%%%%%%%%%%%%%%%%%%%%

\section{Comment introduire et manipuler la notion de contre-exemple avec les automates d'arbres?}

{Les techniques actuelles se basent sur un historique des automates correspondant à chaque étape de calcul ou du calcul en arrière}

{Bouajjani \cite{artmc06}, Besançon \cite{fib:arra08}}
{Approches intéressantes, mais limitées d'un point de vue complexité et expressivité}
{Utilisation de l'inclusion d'automates d'arbres + intersection, systèmes linéaires à droite et
  à gauche : ca n'est pas le cas du TRS de Java (non linéaire à droite) (pour éviter les calculs d'inclusion, calcul en arrière à partir de la propriété)}
{Autres approches ? voir les papiers de Besançon ???}

\section{Définition d'un $\RE$-automate}

C'est une extension de la sémantique des automates d'arbres qui permettent
de différentier les termes pour lesquels l'atteignabilité est sûre des autres termes,
simplement par reconnaissance.

Principe : repose sur le remplacement de la fusion par des transitions epsilons...
La difficulté rencontrée lors de l'approche jusquà présent repose sur le mécanisme
de fusion tel que présentée \ref{fusion}. Dans l'exemple \ref{ex:toto}....


%%%%%%%%%%%%%%%%%%%% EXEMPLE TOTO pour mettre en évidence la non différentiation du terme et de son abstraction.... %%%%%%%%%%%%%%%%%%%%%%%%%%%%%%


pour différentier les termes exacts (définir la notion de termes exacts ici ou au moment de la def de la complétion???)
des termes issus de l'abstraction. Ainsi on peut savoir si un terme est reconnu simplement en regardant la manière dont 
il est reconnu par l'autom
\bigskip
Un $\RE$-automate est un automate d'arbre -- en fait c'est même une extension des $\R$-automates
\ref{R-automates} -- dans lequel les transitions sont classées en trois catégories.  D'une part les
transitions closes et deux ensembles distincts d'$\varepsilon$-transitions~:
\begin{itemize}
\item les premières caractérisent la relation de réécriture qui peut exister entre les classes
  d'équivalence de l'automate, similairement aux $\R$-automates mais étiquetées par des formules
  logiques.

\item les secondes définissent la relation d'abstraction qui peut exister entre deux classes d'équivalence.
\end{itemize}

Comme pour les $\R$-automates, la sémantique d'un $\RE$-automate n'a de sens que dans le contexte d'un problème d'atteignabilité
constitué d'un ensemble initial, d'un système de réécriture, et d'un ensemble d'équations
\footnote{\footnotesize l'ensemble des équations peut être éventuellement vide, mais on se ramène alors au cas du calcul 
  sans approximation couvert par les $\R$-automates.}.

\subsection{Définitions formelles}
Soit un problème d'atteignabilité défini dans l'ensemble de termes $\TF$ avec 
$I$ comme ensemble fini de termes initiaux $\R$ un ensemble de règles de réécriture
et $E$ un ensemble d'équations.

\begin{definition}{$\RE$-automate}
  \label{def:RE-automaton}
  Un $\RE$-Automaton est un automate d'arbre défini comme un quadruplé de la forme
  $\A= \langle \F, \Q, \Q_F,\Delta \cup \Drw \cup \Deq\rangle$,
  avec $\Q_F \subseteq \Q$ et composé de 3 ensembles de transitions distinctes~:

  \begin{description}
  \item [$\Delta$:] les transitions normaliées
  \item [$\Drw$:] $\varepsilon$-transitions étiquetées par des formules logiques.
  \item [$\Deq$:] $\varepsilon$-transitions.
  \end{description}

  Les formules logiques qui étiquettent les transitions de $\Drw$ sont des formules propositionnelles
  formées par les connecteurs $\land$ et $\lor$ à partir des $\top$, $\bottom$ et $Eq(q, q')$ tels que $q,\ q' \in \Q$.
\end{definition}


\subsection{Sémantique}

The equivalence relation induced by $(Q, \rw^!)$ simply associates to each state sets of terms.
Since the relation $\f^!$ is deterministic, thoses sets are disjoined.

We introduce $\f^=$ which denotes the transitive closure of transitions of $\Delta \cup \Deq$.
The couple $(Q, \f^=)$ induces the equivalence relation such that if terms $u$ and $v$
are in the same equivalence classe, then $u =_E v$.

{Formules propositionnelles en FND}


%%%%%%%%%%%%%%%%%%%% Comparaison avec les TAGED AUTOMATA Ref dans le papier COURBIS HEAM / TAGGED ou les automates à contraintes du TATA...

\subsubsection{Détailler que les étapes de reconnaissance peuvent être permutter tant que la réécriture le permet...}
comme dans le cas des automates d'arbres standards et des $\R$-automates, il est possible de réorganiser les étapes de
réécriture....

\subsubsection{Introduire des sous classes pour le cas de la reconnaissance sans $\Delta_\R$ et sans étapes de transisitons epsilon aux feuilles
  (pas de transition $q -> q'$ sur $f(q)$.)}

\subsection{Exemple de $\RE$-automate}

\section{Equivalence entre la nouvelle sémantique et la sémantique standard}
\section{$\RE$-Automate complet}
\subsection{Clôture par réécriture}


\section{Complétion d'un $\RE$-Automate}
\section{Le problème du filtrage ou comment trouver des ensembles de sous-termes avec les étapes d'approximation nécessaires à les rendre atteignables}
\section{Preuve : quand on a calculé ($\sigma$, $\phi$) on a que tout terme $t --\phi'--> l\sigma --\phi ----> q$ alors le terme est atteignable grâce aux étapes d'approximations $\phi /\ \phi'$}
\section{Dans quelle cas la paire critique n'est pas résolue}
\subsection{indiquer pourquoi il est nécessaire de définir des nouveaux cas pour éviter la divergence tout en restant correct !}
\subsection{La nouvelle technique pour vérifier que l'on est clos}
\subsection{Nouvelle technique de normalisation}
\subsection{Preuve même genre de preuve que pour le filtrage}

\section{Conclusion : preuve que l'étape de complétion est correcte}
\subsection{Clôture sous les contextes}


\section{Test de vacuité de l'intersection}
\subsection{Instrumenter pour obtenir un témoin de non-vacuité + les étapes d'approximation utilisées pour le produire $\phi$}
\subsection{Soit $\phi \equ \top$ donc le terme est un contre-exemple : la propriété est violée}
\subsection{Soit $\phi \not\equ \top$ donc il faut raffiner}

\section{Raffiner un $\RE$-automate}
\subsection{Retirer les transisitions qui ont permis l'introduction du terme dans l'automate ! }
\subsection{Comment raffiner le plus précisément possible ?}
\subsection{Préservement des informations inférées}
\subsection{Nettoyage de l'automate}

\nocite{*}
\bibliographystyle{plain}
\bibliography{4_contre_ex/4_biblio}


%%% Local Variables: 
%%% mode: latex
%%% TeX-master: "../main"
%%% TeX-PDF-mode: t
%%% ispell-local-dictionary: "french"
%%% End: 
